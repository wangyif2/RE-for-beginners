\section{\IFRU{"QR9": Любительская криптосистема вдохновленная кубиком Рубика}
{"QR9": Rubik's cube inspired amateur crypto-algorithm}}

\IFRU{Любительские криптосистемы иногда попадаются довольно странные.}
{Sometimes amateur cryptosystems appear to be pretty bizarre.}

\IFRU{Однажды меня попросили разобраться с одним таким любительским криптоалгоритмом встроенным в 
утилиту для шифрования, исходный код которой был утерян\footnote{Я также получил разрешение от 
клиента на публикацию деталей алгоритма}.}
{I was asked to revese engineer an amateur cryptoalgorithm of some data crypting utility, 
source code of which was lost\footnote{I also got permit from customer to publish the algorithm details}.}

\IFRU{Вот листинг этой утилиты для шифрования, полученный при помощи \IDA}
{Here is also \IDA exported listing from original crypting utility}:

\lstinputlisting{qr9/qr9_original.lst}

\IFRU{Все имена функций и меток даны мною в процессе анализа.}
{All function and label names are given by me while analysis.}

\IFRU{Я начал с самого верха. Вот функция, берущая на вход два имени файла и пароль.}
{I started from top. Here is a function taking two file names and password.}

\begin{lstlisting}
.text:00541320 ; int __cdecl crypt_file(int Str, char *Filename, int password)
.text:00541320 crypt_file      proc near
.text:00541320
.text:00541320 Str             = dword ptr  4
.text:00541320 Filename        = dword ptr  8
.text:00541320 password        = dword ptr  0Ch
.text:00541320
\end{lstlisting}

\IFRU{Открыть файл и сообщить об ошибке в случае ошибки:}{Open file and report error in case of error:}

\begin{lstlisting}
.text:00541320                 mov     eax, [esp+Str]
.text:00541324                 push    ebp
.text:00541325                 push    offset Mode     ; "rb"
.text:0054132A                 push    eax             ; Filename
.text:0054132B                 call    _fopen          ; open file
.text:00541330                 mov     ebp, eax
.text:00541332                 add     esp, 8
.text:00541335                 test    ebp, ebp
.text:00541337                 jnz     short loc_541348
.text:00541339                 push    offset Format   ; "Cannot open input file!\n"
.text:0054133E                 call    _printf
.text:00541343                 add     esp, 4
.text:00541346                 pop     ebp
.text:00541347                 retn
.text:00541348 ; ---------------------------------------------------------------------------
.text:00541348
.text:00541348 loc_541348:
\end{lstlisting}

\IFRU{Узнать размер файла используя}{Get file size via} \TT{fseek()}/\TT{ftell()}:

\lstinputlisting{\IFRU{qr9/1ru.asm}{qr9/1en.asm}}

\IFRU{Этот кусок кода вычисляет длину файла выровненную по 64-байтной границе.
Это потому что этот алгоритм шифрования работает только с блоками размерами 64 байта.
Работает очень просто: разделить длину файла на 64, забыть об остатке, прибавить 1,
умножить на 64.
Следующий код удаляет остаток от деления как если бы это значение уже было разделено 
на 64 и добавляет 64. Это почти то же самое.}
{This piece of code calculates file size aligned to 64-byte border. 
This is because this cryptoalgorithm works with only 64-byte blocks. 
Its operation is pretty simple: divide file size by 64, forget about remainder and add 1, 
then multiple by 64. 
The following code removes remainder as if value was already divided by 64 and adds 64. 
It is almost the same.}

\lstinputlisting{\IFRU{qr9/2ru.asm}{qr9/2en.asm}}

\IFRU{Выделить буфер с выровненным размером:}{Allocate buffer with aligned size:}

\begin{lstlisting}
.text:00541373                 push    esi             ; Size
.text:00541374                 call    _malloc
\end{lstlisting}

\IFRU{Вызвать memset(), т.е., очистить выделенный буфер\footnote{malloc() + memset() можно было бы 
заменить на calloc()}.}{Call memset(), e,g, clear allocated buffer\footnote{malloc() + memset() could 
be replaced by calloc()}.}

\lstinputlisting{\IFRU{qr9/3ru.asm}{qr9/3en.asm}}

\IFRU{Чтение файла используя стандартную функцию Си}{Read file via standard C function} \TT{fread()}.

\begin{lstlisting}
.text:00541392                 mov     eax, [esp+38h+Str]
.text:00541396                 push    eax             ; ElementSize
.text:00541397                 push    ebx             ; DstBuf
.text:00541398                 call    _fread          ; read file
.text:0054139D                 push    ebp             ; File
.text:0054139E                 call    _fclose
\end{lstlisting}

\IFRU{Вызов \TT{crypt()}. Эта функция берет на вход буфер, длину буфера (выровненную) и строку пароля.}
{Call \TT{crypt()}. This function takes buffer, buffer size (aligned) and password string.}

\begin{lstlisting}
.text:005413A3                 mov     ecx, [esp+44h+password]
.text:005413A7                 push    ecx             ; password
.text:005413A8                 push    esi             ; aligned size
.text:005413A9                 push    ebx             ; buffer
.text:005413AA                 call    crypt           ; do crypt
\end{lstlisting}

\IFRU{Создать выходной файл. Кстати, разработчик забыл вставить проверку, создался ли файл успешно!
Результат открытия файла, впрочем, проверяется.}
{Create output file. By the way, developer forgot to check if it is was created correctly! 
File opening result is being checked though.}

\begin{lstlisting}
.text:005413AF                 mov     edx, [esp+50h+Filename]
.text:005413B3                 add     esp, 40h
.text:005413B6                 push    offset aWb      ; "wb"
.text:005413BB                 push    edx             ; Filename
.text:005413BC                 call    _fopen
.text:005413C1                 mov     edi, eax
\end{lstlisting}

\IFRU{Теперь хэндл созданного файла в регистре \EDI. Зписываем сигнатуру "QR9".}
{Newly created file handle is in \EDI register now. Write signature "QR9".}

\begin{lstlisting}
.text:005413C3                 push    edi             ; File
.text:005413C4                 push    1               ; Count
.text:005413C6                 push    3               ; Size
.text:005413C8                 push    offset aQr9     ; "QR9"
.text:005413CD                 call    _fwrite         ; write file signature
\end{lstlisting}

\IFRU{Записываем настоящую длину файла (не выровненную)}{Write actual file size (not aligned)}:

\begin{lstlisting}
.text:005413D2                 push    edi             ; File
.text:005413D3                 push    1               ; Count
.text:005413D5                 lea     eax, [esp+30h+Str]
.text:005413D9                 push    4               ; Size
.text:005413DB                 push    eax             ; Str
.text:005413DC                 call    _fwrite         ; write original file size
\end{lstlisting}

\IFRU{Записываем шифрованный буфер}{Write crypted buffer}:

\begin{lstlisting}
.text:005413E1                 push    edi             ; File
.text:005413E2                 push    1               ; Count
.text:005413E4                 push    esi             ; Size
.text:005413E5                 push    ebx             ; Str
.text:005413E6                 call    _fwrite         ; write crypted file
\end{lstlisting}

\IFRU{Закрыть файл и освободить выделенный буфер}{Close file and free allocated buffer}:

\begin{lstlisting}
.text:005413EB                 push    edi             ; File
.text:005413EC                 call    _fclose
.text:005413F1                 push    ebx             ; Memory
.text:005413F2                 call    _free
.text:005413F7                 add     esp, 40h
.text:005413FA                 pop     edi
.text:005413FB                 pop     esi
.text:005413FC                 pop     ebx
.text:005413FD                 pop     ebp
.text:005413FE                 retn
.text:005413FE crypt_file      endp
\end{lstlisting}

\IFRU{Переписанный на Си код}{Here is reconstructed C-code}:

\begin{lstlisting}
void crypt_file(char *fin, char* fout, char *pw)
{
	FILE *f;
	int flen, flen_aligned;
	BYTE *buf;

	f=fopen(fin, "rb");
	
	if (f==NULL)
	{
		printf ("Cannot open input file!\n");
		return;
	};

	fseek (f, 0, SEEK_END);
	flen=ftell (f);
	fseek (f, 0, SEEK_SET);

	flen_aligned=(flen&0xFFFFFFC0)+0x40;

	buf=(BYTE*)malloc (flen_aligned);
	memset (buf, 0, flen_aligned);

	fread (buf, flen, 1, f);

	fclose (f);

	crypt (buf, flen_aligned, pw);
	
	f=fopen(fout, "wb");

	fwrite ("QR9", 3, 1, f);
	fwrite (&flen, 4, 1, f);
	fwrite (buf, flen_aligned, 1, f);

	fclose (f);

	free (buf);
};
\end{lstlisting}

\IFRU{Процедура дешифрования почти такая же}{Decrypting procedure is almost the same}:

\begin{lstlisting}
.text:00541400 ; int __cdecl decrypt_file(char *Filename, int, void *Src)
.text:00541400 decrypt_file    proc near
.text:00541400
.text:00541400 Filename        = dword ptr  4
.text:00541400 arg_4           = dword ptr  8
.text:00541400 Src             = dword ptr  0Ch
.text:00541400
.text:00541400                 mov     eax, [esp+Filename]
.text:00541404                 push    ebx
.text:00541405                 push    ebp
.text:00541406                 push    esi
.text:00541407                 push    edi
.text:00541408                 push    offset aRb      ; "rb"
.text:0054140D                 push    eax             ; Filename
.text:0054140E                 call    _fopen
.text:00541413                 mov     esi, eax
.text:00541415                 add     esp, 8
.text:00541418                 test    esi, esi
.text:0054141A                 jnz     short loc_54142E
.text:0054141C                 push    offset aCannotOpenIn_0 ; "Cannot open input file!\n"
.text:00541421                 call    _printf
.text:00541426                 add     esp, 4
.text:00541429                 pop     edi
.text:0054142A                 pop     esi
.text:0054142B                 pop     ebp
.text:0054142C                 pop     ebx
.text:0054142D                 retn
.text:0054142E ; ---------------------------------------------------------------------------
.text:0054142E
.text:0054142E loc_54142E:
.text:0054142E                 push    2               ; Origin
.text:00541430                 push    0               ; Offset
.text:00541432                 push    esi             ; File
.text:00541433                 call    _fseek
.text:00541438                 push    esi             ; File
.text:00541439                 call    _ftell
.text:0054143E                 push    0               ; Origin
.text:00541440                 push    0               ; Offset
.text:00541442                 push    esi             ; File
.text:00541443                 mov     ebp, eax
.text:00541445                 call    _fseek
.text:0054144A                 push    ebp             ; Size
.text:0054144B                 call    _malloc
.text:00541450                 push    esi             ; File
.text:00541451                 mov     ebx, eax
.text:00541453                 push    1               ; Count
.text:00541455                 push    ebp             ; ElementSize
.text:00541456                 push    ebx             ; DstBuf
.text:00541457                 call    _fread
.text:0054145C                 push    esi             ; File
.text:0054145D                 call    _fclose
\end{lstlisting}

\IFRU{Проверяем сигнатуру (первые 3 байта)}{Check signature (first 3 bytes)}:

\begin{lstlisting}
.text:00541462                 add     esp, 34h
.text:00541465                 mov     ecx, 3
.text:0054146A                 mov     edi, offset aQr9_0 ; "QR9"
.text:0054146F                 mov     esi, ebx
.text:00541471                 xor     edx, edx
.text:00541473                 repe cmpsb
.text:00541475                 jz      short loc_541489
\end{lstlisting}

\IFRU{Сообщить об ошибке если сигнатура отсутствует}{Report an error if signature is absent}:

\begin{lstlisting}
.text:00541477                 push    offset aFileIsNotCrypt ; "File is not crypted!\n"
.text:0054147C                 call    _printf
.text:00541481                 add     esp, 4
.text:00541484                 pop     edi
.text:00541485                 pop     esi
.text:00541486                 pop     ebp
.text:00541487                 pop     ebx
.text:00541488                 retn
.text:00541489 ; ---------------------------------------------------------------------------
.text:00541489
.text:00541489 loc_541489:
\end{lstlisting}

\IFRU{Вызвать}{Call} \TT{decrypt()}.

\begin{lstlisting}
.text:00541489                 mov     eax, [esp+10h+Src]
.text:0054148D                 mov     edi, [ebx+3]
.text:00541490                 add     ebp, 0FFFFFFF9h
.text:00541493                 lea     esi, [ebx+7]
.text:00541496                 push    eax             ; Src
.text:00541497                 push    ebp             ; int
.text:00541498                 push    esi             ; int
.text:00541499                 call    decrypt
.text:0054149E                 mov     ecx, [esp+1Ch+arg_4]
.text:005414A2                 push    offset aWb_0    ; "wb"
.text:005414A7                 push    ecx             ; Filename
.text:005414A8                 call    _fopen
.text:005414AD                 mov     ebp, eax
.text:005414AF                 push    ebp             ; File
.text:005414B0                 push    1               ; Count
.text:005414B2                 push    edi             ; Size
.text:005414B3                 push    esi             ; Str
.text:005414B4                 call    _fwrite
.text:005414B9                 push    ebp             ; File
.text:005414BA                 call    _fclose
.text:005414BF                 push    ebx             ; Memory
.text:005414C0                 call    _free
.text:005414C5                 add     esp, 2Ch
.text:005414C8                 pop     edi
.text:005414C9                 pop     esi
.text:005414CA                 pop     ebp
.text:005414CB                 pop     ebx
.text:005414CC                 retn
.text:005414CC decrypt_file    endp
\end{lstlisting}

\IFRU{Переписанный на Си код}{Here is reconstructed C-code}:

\begin{lstlisting}
void decrypt_file(char *fin, char* fout, char *pw)
{
	FILE *f;
	int real_flen, flen;
	BYTE *buf;

	f=fopen(fin, "rb");
	
	if (f==NULL)
	{
		printf ("Cannot open input file!\n");
		return;
	};

	fseek (f, 0, SEEK_END);
	flen=ftell (f);
	fseek (f, 0, SEEK_SET);

	buf=(BYTE*)malloc (flen);

	fread (buf, flen, 1, f);

	fclose (f);

	if (memcmp (buf, "QR9", 3)!=0)
	{
		printf ("File is not crypted!\n");
		return;
	};

	memcpy (&real_flen, buf+3, 4);

	decrypt (buf+(3+4), flen-(3+4), pw);
	
	f=fopen(fout, "wb");

	fwrite (buf+(3+4), real_flen, 1, f);

	fclose (f);

	free (buf);
};
\end{lstlisting}

\IFRU{OK, посмотрим глубже}{OK, now let's go deeper}.

\IFRU{Функция}{Function} \TT{crypt()}:

\begin{lstlisting}
.text:00541260 crypt           proc near
.text:00541260
.text:00541260 arg_0           = dword ptr  4
.text:00541260 arg_4           = dword ptr  8
.text:00541260 arg_8           = dword ptr  0Ch
.text:00541260
.text:00541260                 push    ebx
.text:00541261                 mov     ebx, [esp+4+arg_0]
.text:00541265                 push    ebp
.text:00541266                 push    esi
.text:00541267                 push    edi
.text:00541268                 xor     ebp, ebp
.text:0054126A
.text:0054126A loc_54126A:
\end{lstlisting}

\IFRU{Этот кусок кода копирует часть входного буфера во внутренний буфер, который я поже назвал "cube64".}
{This piece of code copies part of input buffer to internal array I named later "cube64".}
\IFRU{Длина в регистре \ECX. \TT{MOVSD} означает "скопировать 32-битное слово", так что, 16 32-битных слов
это как раз 64 байта.}{The size is in \ECX register. \TT{MOVSD} means "move 32-bit dword", so, 
16 of 32-bit dwords are exactly 64 bytes.}

\begin{lstlisting}
.text:0054126A                 mov     eax, [esp+10h+arg_8]
.text:0054126E                 mov     ecx, 10h
.text:00541273                 mov     esi, ebx   ; EBX is pointer within input buffer
.text:00541275                 mov     edi, offset cube64
.text:0054127A                 push    1
.text:0054127C                 push    eax
.text:0054127D                 rep movsd
\end{lstlisting}

\IFRU{Вызвать}{Call} \TT{rotate\_all\_with\_password()}:

\begin{lstlisting}
.text:0054127F                 call    rotate_all_with_password
\end{lstlisting}

\IFRU{Скопировать зашифрованное содержимое из "cube64" назад в буфер}
{Copy crypted contents back from "cube64" to buffer}:

\begin{lstlisting}
.text:00541284                 mov     eax, [esp+18h+arg_4]
.text:00541288                 mov     edi, ebx
.text:0054128A                 add     ebp, 40h
.text:0054128D                 add     esp, 8
.text:00541290                 mov     ecx, 10h
.text:00541295                 mov     esi, offset cube64
.text:0054129A                 add     ebx, 40h  ; add 64 to input buffer pointer
.text:0054129D                 cmp     ebp, eax  ; EBP contain ammount of crypted data.
.text:0054129F                 rep movsd
\end{lstlisting}

\IFRU{Если \EBP не больше чем длина во входном аргументе, тогда переходим к следующему блоку.}
{If \EBP is not bigger that input argument size, then continue to next block.}

\begin{lstlisting}
.text:005412A1                 jl      short loc_54126A
.text:005412A3                 pop     edi
.text:005412A4                 pop     esi
.text:005412A5                 pop     ebp
.text:005412A6                 pop     ebx
.text:005412A7                 retn
.text:005412A7 crypt           endp
\end{lstlisting}

\IFRU{Реконструированная функция \TT{crypt()}}{Reconstructed \TT{crypt()} function}:

\begin{lstlisting}
void crypt (BYTE *buf, int sz, char *pw)
{
	int i=0;
	
	do
	{
		memcpy (cube, buf+i, 8*8);
		rotate_all (pw, 1);
		memcpy (buf+i, cube, 8*8);
		i+=64;
	}
	while (i<sz);
};
\end{lstlisting}

\IFRU{OK, углубимся в функцию \TT{rotate\_all\_with\_password()}. Она берет на вход два аргумента: 
строку пароля и число.}{OK, now let's go deeper into function \TT{rotate\_all\_with\_password()}. 
It takes two arguments: password string and number.}
\IFRU{В функции \TT{crypt()}, число 1 используется и в \TT{decrypt()} (где \TT{rotate\_all\_with\_password()}
функция вызывается также), число 3.}
{In \TT{crypt()}, number 1 is used, and in \TT{decrypt()} (where \TT{rotate\_all\_with\_password()} function 
is called too), number is 3.}

\begin{lstlisting}
.text:005411B0 rotate_all_with_password proc near
.text:005411B0
.text:005411B0 arg_0           = dword ptr  4
.text:005411B0 arg_4           = dword ptr  8
.text:005411B0
.text:005411B0                 mov     eax, [esp+arg_0]
.text:005411B4                 push    ebp
.text:005411B5                 mov     ebp, eax
\end{lstlisting}

\IFRU{Проверяем символы в пароле. Если это ноль, выходим:}{Check for character in password. If it is zero, exit:}

\begin{lstlisting}
.text:005411B7                 cmp     byte ptr [eax], 0
.text:005411BA                 jz      exit
.text:005411C0                 push    ebx
.text:005411C1                 mov     ebx, [esp+8+arg_4]
.text:005411C5                 push    esi
.text:005411C6                 push    edi
.text:005411C7
.text:005411C7 loop_begin:
\end{lstlisting}

\IFRU{Вызываем \TT{tolower()}, стандартную функцию Си.}{Call \TT{tolower()}, standard C function.}

\begin{lstlisting}
.text:005411C7                 movsx   eax, byte ptr [ebp+0]
.text:005411CB                 push    eax             ; C
.text:005411CC                 call    _tolower
.text:005411D1                 add     esp, 4
\end{lstlisting}

\IFRU{Хмм, если пароль содержит символ не из латинского алфавита, он пропускается!
Действительно, если мы запускаем утилиту для шифрования используя символы не латинского алфавита, 
похоже, они просто игнорируются.}
{Hmm, if password contains non-alphabetical latin character, it is skipped! 
Indeed, if we run crypting utility and try non-alphabetical latin characters in password, 
they seem to be ignored.}

\begin{lstlisting}
.text:005411D4                 cmp     al, 'a'
.text:005411D6                 jl      short next_character_in_password
.text:005411D8                 cmp     al, 'z'
.text:005411DA                 jg      short next_character_in_password
.text:005411DC                 movsx   ecx, al
\end{lstlisting}

\IFRU{Отнимем значение "a" (97) от символа.}{Subtract "a" value (97) from character.}

\begin{lstlisting}
.text:005411DF                 sub     ecx, 'a'  ; 97
\end{lstlisting}

\IFRU{После вычитания, тут будет 0 для "a", 1 для "b", и так далее. И 25 для "z".}
{After subtracting, we'll get 0 for "a" here, 1 for "b", etc. And 25 for "z".}

\begin{lstlisting}
.text:005411E2                 cmp     ecx, 24
.text:005411E5                 jle     short skip_subtracting
.text:005411E7                 sub     ecx, 24
\end{lstlisting}

\IFRU{Похоже, символы "y" и "z" также исключительные.
После этого куска кода, "y" становится 0, а "z" ~--- 1.
Это значит что 26 латинских букв становятся значениями в интервале 0..23, (всего 24).}
{It seems, "y" and "z" are exceptional characters too. 
After that piece of code, "y" becomes 0 and "z" ~--- 1. 
This means, 26 latin alphabet symbols will become values in range 0..23, (24 in total).}

\begin{lstlisting}
.text:005411EA
.text:005411EA skip_subtracting:                       ; CODE XREF: rotate_all_with_password+35
\end{lstlisting}

\IFRU{Это, на самом деле, деление через умножение.
Читайте об этом больше в секции "\DivisionByNineSectionName"~\ref{sec:divisionbynine}.}
{This is actually division via multiplication. 
Read more about it in "\DivisionByNineSectionName" section~\ref{sec:divisionbynine}.}

\IFRU{Это код, на самом деле, делит значение символа пароля на 3.}
{The code actually divides password character value by 3.}

\begin{lstlisting}
.text:005411EA                 mov     eax, 55555556h
.text:005411EF                 imul    ecx
.text:005411F1                 mov     eax, edx
.text:005411F3                 shr     eax, 1Fh
.text:005411F6                 add     edx, eax
.text:005411F8                 mov     eax, ecx
.text:005411FA                 mov     esi, edx
.text:005411FC                 mov     ecx, 3
.text:00541201                 cdq
.text:00541202                 idiv    ecx
\end{lstlisting}

\IFRU{\EDX ~--- остаток от деления.}{\EDX is the remainder of division.}

\lstinputlisting{\IFRU{qr9/4ru.asm}{qr9/4en.asm}}

\IFRU{Если остаток 2, вызываем \TT{rotate3()}. 
\EDX это второй аргумент функции \TT{rotate\_all\_with\_password()}. 
Как я уже писал, 1 это для шифрования, 3 для дешифрования.
Так что здесь цикл, функции rotate1/2/3 будут вызываться столько же раз, сколько значение переменной
в первом аргументе.}
{If remainder is 2, call \TT{rotate3()}. 
\EDI is a second argument of \TT{rotate\_all\_with\_password()}. 
As I already wrote, 1 is for crypting operations and 3 is for decrypting. 
So, here is a loop. When crypting, rotate1/2/3 will be called the same number of times as 
given in the first argument.}

\begin{lstlisting}
.text:00541215 call_rotate3:
.text:00541215                 push    esi
.text:00541216                 call    rotate3
.text:0054121B                 add     esp, 4
.text:0054121E                 dec     edi
.text:0054121F                 jnz     short call_rotate3
.text:00541221                 jmp     short next_character_in_password
.text:00541223
.text:00541223 call_rotate2:
.text:00541223                 test    ebx, ebx
.text:00541225                 jle     short next_character_in_password
.text:00541227                 mov     edi, ebx
.text:00541229
.text:00541229 loc_541229:
.text:00541229                 push    esi
.text:0054122A                 call    rotate2
.text:0054122F                 add     esp, 4
.text:00541232                 dec     edi
.text:00541233                 jnz     short loc_541229
.text:00541235                 jmp     short next_character_in_password
.text:00541237
.text:00541237 call_rotate1:
.text:00541237                 test    ebx, ebx
.text:00541239                 jle     short next_character_in_password
.text:0054123B                 mov     edi, ebx
.text:0054123D
.text:0054123D loc_54123D:
.text:0054123D                 push    esi
.text:0054123E                 call    rotate1
.text:00541243                 add     esp, 4
.text:00541246                 dec     edi
.text:00541247                 jnz     short loc_54123D
.text:00541249
\end{lstlisting}

\IFRU{Достать следующий символ из строки пароля.}{Fetch next character from password string.}

\begin{lstlisting}
.text:00541249 next_character_in_password:
.text:00541249                 mov     al, [ebp+1]
\end{lstlisting}

\IFRU{Инкремент указателя на символ в строке пароля:}{Increment character pointer within password string:}

\begin{lstlisting}
.text:0054124C                 inc     ebp
.text:0054124D                 test    al, al
.text:0054124F                 jnz     loop_begin
.text:00541255                 pop     edi
.text:00541256                 pop     esi
.text:00541257                 pop     ebx
.text:00541258
.text:00541258 exit:
.text:00541258                 pop     ebp
.text:00541259                 retn
.text:00541259 rotate_all_with_password endp
\end{lstlisting}

\IFRU{Реконструированный код на Си:}{Here is reconstructed C code:}

\begin{lstlisting}
void rotate_all (char *pwd, int v)
{
	char *p=pwd;

	while (*p)
	{
		char c=*p;
		int q;

		c=tolower (c);

		if (c>='a' && c<='z')
		{
			q=c-'a';
			if (q>24)
				q-=24;

			int quotient=q/3;
			int remainder=q % 3;

			switch (remainder)
			{
			case 0: for (int i=0; i<v; i++) rotate1 (quotient); break;
			case 1: for (int i=0; i<v; i++) rotate2 (quotient); break;
			case 2: for (int i=0; i<v; i++) rotate3 (quotient); break;
			};
		};

		p++;
	};
};
\end{lstlisting}

\IFRU{Углубимся еще дальше и исследуем функции rotate1/2/3.
Каждая функция вызывает еще две.
В итоге я назвал их \TT{set\_bit()} и \TT{get\_bit()}.}
{Now let's go deeper and investigate rotate1/2/3 functions. 
Each function calls two another functions. 
I eventually gave them names \TT{set\_bit()} and \TT{get\_bit()}.}

\IFRU{Начнем с \TT{get\_bit()}:}{Let's start with \TT{get\_bit()}:}

\begin{lstlisting}
.text:00541050 get_bit         proc near
.text:00541050
.text:00541050 arg_0           = dword ptr  4
.text:00541050 arg_4           = dword ptr  8
.text:00541050 arg_8           = byte ptr  0Ch
.text:00541050
.text:00541050                 mov     eax, [esp+arg_4]
.text:00541054                 mov     ecx, [esp+arg_0]
.text:00541058                 mov     al, cube64[eax+ecx*8]
.text:0054105F                 mov     cl, [esp+arg_8]
.text:00541063                 shr     al, cl
.text:00541065                 and     al, 1
.text:00541067                 retn
.text:00541067 get_bit         endp
\end{lstlisting}

\IFRU{... иными словами: подсчитать индекс в массиве cube64}{... in other words: calculate an index in 
the array cube64}: \IT{arg\_4 + arg\_0 * 8}.
\IFRU{Затем сдвинуть байт из массива вправо на количество бит заданных в arg\_8. 
Изолировать самый младший бит и вернуть его}{Then shift a byte from an array by arg\_8 bits right. 
Isolate lowest bit and return it.}

\IFRU{Посмотрим другую функцию}{Let's see another function}, \TT{set\_bit()}:

\begin{lstlisting}
.text:00541000 set_bit         proc near
.text:00541000
.text:00541000 arg_0           = dword ptr  4
.text:00541000 arg_4           = dword ptr  8
.text:00541000 arg_8           = dword ptr  0Ch
.text:00541000 arg_C           = byte ptr  10h
.text:00541000
.text:00541000                 mov     al, [esp+arg_C]
.text:00541004                 mov     ecx, [esp+arg_8]
.text:00541008                 push    esi
.text:00541009                 mov     esi, [esp+4+arg_0]
.text:0054100D                 test    al, al
.text:0054100F                 mov     eax, [esp+4+arg_4]
.text:00541013                 mov     dl, 1
.text:00541015                 jz      short loc_54102B
\end{lstlisting}

\IFRU{DL тут равно 1. Сдвигаем эту единицу на количество указанное в arg\_8. Например, если в arg\_8 число 4,
тогда значение в DL станет 0x10 или 1000 в двоичной системе счисления.}
{DL is 1 here. Shift left it by arg\_8. For example, if arg\_8 is 4, DL register value became 
0x10 or 1000 in binary form.}

\begin{lstlisting}
.text:00541017                 shl     dl, cl
.text:00541019                 mov     cl, cube64[eax+esi*8]
\end{lstlisting}

\IFRU{Вытащить бит из массива и явно выставить его.}{Get bit from array and explicitly set one.} % TODO: rewrite

\begin{lstlisting}
.text:00541020                 or      cl, dl
\end{lstlisting}

\IFRU{Сохранить его назад:}{Store it back:} % TODO: rewrite

\begin{lstlisting}
.text:00541022                 mov     cube64[eax+esi*8], cl
.text:00541029                 pop     esi
.text:0054102A                 retn
.text:0054102B ; ---------------------------------------------------------------------------
.text:0054102B
.text:0054102B loc_54102B:
.text:0054102B                 shl     dl, cl
\end{lstlisting}

\IFRU{Если arg\_C не ноль...}{If arg\_C is not zero...}

\begin{lstlisting}
.text:0054102D                 mov     cl, cube64[eax+esi*8]
\end{lstlisting}

\IFRU{... инвертировать DL. Например, если состояние DL после сдвига 0x10 или 1000 в двоичной системе,
здесь будет 0xEF после инструкции \NOT или 11101111 в двоичной системе.}
{... invert DL. For example, if DL state after shift was 0x10 or 1000 in binary form, 
there will be 0xEF after \NOT instruction or 11101111 in binary form.}

\begin{lstlisting}
.text:00541034                 not     dl
\end{lstlisting}

\IFRU{Эта инструкция сбрасывает бит, иными словами, она сохраняет все биты в CL которые так же
выставлены в DL кроме тех в DL, что были сброшены. Это значит что если в DL, например,
11101111 в двоичной системе, все биты будут сохранены кроме пятого (считая с младшего бита).}
{This instruction clears bit, in other words, it saves all bits in CL which are also set in 
DL except those in DL which are cleared. This means that if DL is, for example, 
11101111 in binary form, all bits will be saved except 5th (counting from lowest bit).}

\begin{lstlisting}
.text:00541036                 and     cl, dl
\end{lstlisting}

\IFRU{Сохранить его назад}{Store it back:}

\begin{lstlisting}
.text:00541038                 mov     cube64[eax+esi*8], cl
.text:0054103F                 pop     esi
.text:00541040                 retn
.text:00541040 set_bit         endp
\end{lstlisting}

\IFRU{Это почти то же самое что и \TT{get\_bit()} кроме того что если arg\_C ноль, тогда функция сбрасывает
указанный бит в массиве, либо же, в противном случае, выставляет его в 1.}
{It is almost the same as \TT{get\_bit()}, except, if arg\_C is zero, the function clears specific bit in array, 
or sets it otherwise.}

\IFRU{Мы так же знаем что размер массива 64. Первые два аргумента и у \TT{set\_bit()} и у \TT{get\_bit()}
могут быть представлены как двумерные координаты. Таким образом, массив это матрица 8*8.}
{We also know that array size is 64. First two arguments both in \TT{set\_bit()} and \TT{get\_bit()} 
could be seen as 2D cooridnates. Then array will be 8*8 matrix.}

\IFRU{Представление на Си всего того, что мы уже знаем:}{Here is C representation of what we already know:}

\begin{lstlisting}
#define IS_SET(flag, bit)       ((flag) & (bit))
#define SET_BIT(var, bit)       ((var) |= (bit))
#define REMOVE_BIT(var, bit)    ((var) &= ~(bit))

char cube[8][8];

void set_bit (int x, int y, int shift, int bit)
{
	if (bit)
		SET_BIT (cube[x][y], 1<<shift);
	else
		REMOVE_BIT (cube[x][y], 1<<shift);
};

int get_bit (int x, int y, int shift)
{
	if ((cube[x][y]>>shift)&1==1)
		return 1;
	return 0;
};
\end{lstlisting}

\IFRU{Теперь вернемся к функциям rotate1/2/3.}{Now let's get back to rotate1/2/3 functions.}

\begin{lstlisting}
.text:00541070 rotate1         proc near
.text:00541070
\end{lstlisting}

\IFRU{Выделение внутреннего массива размером 64 байта в локальном стеке:}
{Internal array allocation in local stack, its size 64 bytes:}

\begin{lstlisting}
.text:00541070 internal_array_64= byte ptr -40h
.text:00541070 arg_0           = dword ptr  4
.text:00541070
.text:00541070                 sub     esp, 40h
.text:00541073                 push    ebx
.text:00541074                 push    ebp
.text:00541075                 mov     ebp, [esp+48h+arg_0]
.text:00541079                 push    esi
.text:0054107A                 push    edi
.text:0054107B                 xor     edi, edi        ; EDI is loop1 counter
\end{lstlisting}

\EBX \IFRU{указывает на внутренний массив}{is a pointer to internal array:}

\begin{lstlisting}
.text:0054107D                 lea     ebx, [esp+50h+internal_array_64]
.text:00541081
\end{lstlisting}

\IFRU{Здесь два вложенных цикла:}{Two nested loops are here:}

\lstinputlisting{\IFRU{qr9/5ru.asm}{qr9/5ru.asm}}

\IFRU{Мы видим что оба счетчика циклов в интервале 0..7. 
Также, они используются как первый и второй аргумент \TT{get\_bit()}.
Третий аргумент \TT{get\_bit()} это единственный аргумент \TT{rotate1()}. 
То что возвращает \TT{get\_bit()} будет сохранено во внутреннем массиве.}
{... we see that both loop counters are in range 0..7. 
Also, they are used as first and second arguments of \TT{get\_bit()}. 
Third argument of \TT{get\_bit()} is the only argument of \TT{rotate1()}. 
What \TT{get\_bit()} returns, is being placed into internal array.}

\IFRU{Снова приготовить указатель на внутренний массив:}{Prepare pointer to internal array again:}

\lstinputlisting{\IFRU{qr9/6ru.asm}{qr9/6en.asm}}

\IFRU{... этот код кладет содержимое из внутреннего массива в глобальный массив cube используя функцию 
\TT{set\_bit()}, \IT{но}, в обратном порядке!
Теперь счетчик первого цикла в интервале 7 до 0, уменьшается на 1 на каждой итерации!}
{... this code is placing contents from internal array to cube global array via \TT{set\_bit()} function, 
\IT{but}, in different order!
Now loop 1 counter is in range 7 to 0, decrementing at each iteration!}

\IFRU{Представление кода на Си выглядит так:}{C code representation looks like:}

\begin{lstlisting}
void rotate1 (int v)
{
	bool tmp[8][8]; // internal array
	int i, j;

	for (i=0; i<8; i++)
		for (j=0; j<8; j++)
			tmp[i][j]=get_bit (i, j, v);

	for (i=0; i<8; i++)
		for (j=0; j<8; j++)
			set_bit (j, 7-i, v, tmp[x][y]);
};
\end{lstlisting}

\IFRU{Не очень понятно, но если мы посмотрим в функцию \TT{rotate2()}:}
{Not very understandable, but if we will take a look at \TT{rotate2()} function:}

\lstinputlisting{\IFRU{qr9/7ru.asm}{qr9/7en.asm}}

\IFRU{\IT{Почти} то же самое, за исключением порядка аргументов в \TT{get\_bit()} и \TT{set\_bit()}.
Перепишем это на Си-подобный код:}
{It is \IT{almost} the same, except of order of arguments of \TT{get\_bit()} and \TT{set\_bit()}. 
Let's rewrite it in C-like code:}

\begin{lstlisting}
void rotate2 (int v)
{
	bool tmp[8][8]; // internal array
	int i, j;

	for (i=0; i<8; i++)
		for (j=0; j<8; j++)
			tmp[i][j]=get_bit (v, i, j);

	for (i=0; i<8; i++)
		for (j=0; j<8; j++)
			set_bit (v, j, 7-i, tmp[i][j]);
};
\end{lstlisting}

\IFRU{Перепишем также функцию \TT{rotate3()}:}{Let's also rewrite \TT{rotate3()} function:}

\begin{lstlisting}
void rotate3 (int v)
{
	bool tmp[8][8];
	int i, j;

	for (i=0; i<8; i++)
		for (j=0; j<8; j++)
			tmp[i][j]=get_bit (i, v, j);

	for (i=0; i<8; i++)
		for (j=0; j<8; j++)
			set_bit (7-j, v, i, tmp[i][j]);
};
\end{lstlisting}

\IFRU{Теперь всё проще. Если мы представим cube64 как трехмерный куб 8*8*8, где каждый элемент это бит,
то \TT{get\_bit()} и \TT{set\_bit()} просто берут на вход координаты бита.}
{Well, now things are simpler. If we consider cube64 as 3D cube 8*8*8, where each element is bit, 
\TT{get\_bit()} and \TT{set\_bit()} take just coordinates of bit on input.}

\IFRU{Функции rotate1/2/3 просто поворачивают все биты на определенной плоскости.
Три функции, каждая на каждую сторону куба и аргумент v выставляет плоскость в интервале 0..7}
{rotate1/2/3 functions are in fact rotating all bits in specific plane. 
Three functions are each for each cube side and v argument is setting plane in range 0..7.}

\newcommand{\URLWPRU}{http://en.wikipedia.org/wiki/Rubik's_Cube}

\IFRU{Может быть, автор алгоритма думал о \href{\URLWPRU}{кубике Рубика} 8*8*8?!}
{Maybe, algorithm's author was thinking of 8*8*8 \href{\URLWPRU}{Rubik's cube}?!}

\IFRU{Да, действительно.}{Yes, indeed.}

\IFRU{Рассмотрим функцию \TT{decrypt()}, я переписал её:}{Let's get closer into \TT{decrypt()} function, 
I rewrote it here:}

\begin{lstlisting}
void decrypt (BYTE *buf, int sz, char *pw)
{
	char *p=strdup (pw);
	strrev (p);
	int i=0;

	do
	{
		memcpy (cube, buf+i, 8*8);
		rotate_all (p, 3);
		memcpy (buf+i, cube, 8*8);
		i+=64;
	}
	while (i<sz);
	
	free (p);
};
\end{lstlisting}

\newcommand{\URLMSDNSTRREV}{http://msdn.microsoft.com/en-us/library/9hby7w40(VS.80).aspx}

\IFRU{Почти то же самое что и crypt(), \IT{но} строка пароля разворачивается стандартной функцией Си
\href{\URLMSDNSTRREV}{strrev()} и \TT{rotate\_all()} вызывается с аргументом 3.}
{It is almost the same excert of \TT{crypt()}, \IT{but} password string is reversed by 
\href{\URLMSDNSTRREV}{strrev()} standard C function and \TT{rotate\_all()} is called with argument 3.} 

\IFRU{Это значит что, в случае дешифровки, rotate1/2/3 будут вызываться трижды.}
{This means, that in case of decryption, each corresponding rotate1/2/3 call will be performed thrice.}

\IFRU{Это почти кубик Рубика!
Если вы хотите вернуть его состояние назад, делайте то же самое в обратном порядке и направлении!
Чтобы вернуть эффект от поворота плоскости по часовой стрелке, нужно повернуть её же против 
часовой стрелки трижды.}
{This is almost as in Rubik'c cube! 
If you want to get back, do the same in reverse order and direction! 
If you need to undo effect of rotating one place in clockwise direction, rotate it thrice
in counter-clockwise direction.}

\IFRU{\TT{rotate1()}, вероятно, поворот "лицевой" плоскости. \TT{rotate2()}, вероятно, поворот "верхней" плоскости.
\TT{rotate3()}, вероятно, поворот "левой" плоскости.}
{\TT{rotate1()} is probably for rotating "front" plane. \TT{rotate2()} is probably for rotating "top" plane. 
\TT{rotate3()} is probably for rotating "left" plane.}

\IFRU{Вернемся к ядру функции \TT{rotate\_all()}}{Let's get back to core of \TT{rotate\_all()} function:}

\begin{lstlisting}
q=c-'a';
if (q>24)
	q-=24;

int quotient=q/3; // in range 0..7
int remainder=q % 3;

switch (remainder)
{
    case 0: for (int i=0; i<v; i++) rotate1 (quotient); break; // front
    case 1: for (int i=0; i<v; i++) rotate2 (quotient); break; // top
    case 2: for (int i=0; i<v; i++) rotate3 (quotient); break; // left
};
\end{lstlisting}

\IFRU{Так понять проще: каждый символ пароля определяет сторону (одну из трех) и плоскость (одну из восьми).
3*8 = 24, вот почему два последних символа латинского алфавита переопределяются так чтобы алфавит состоял
из 24-х элементов.}
{Now it is much simpler to understand: each password character defines side (one of three) and plane (one of 8). 
3*8 = 24, that's why two last characters of latin alphabet are remapped to fit an alphabet of exactly 
24 elements.}

\IFRU{Алгоритм очевидно слаб: в случае коротких паролей, в бинарном редакторе файлов можно будет увидеть, 
что в зашифрованных файлах остались незашифрованные символы.}
{The algorithm is clearly weak: in case of short passwords, one can see, that in crypted file there are 
some original bytes of original file in binary files editor.}

\IFRU{Весь исходный код в реконструированном виде:}{Here is reconstructed whole source code:}

\lstinputlisting{qr9/qr9.cpp}


