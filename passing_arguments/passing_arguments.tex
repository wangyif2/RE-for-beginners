% done

\section{\IFRU{Передача параметров через стек}{Passing arguments via stack}}

\IFRU
{Как мы уже успели заметить, вызывающая функция передает аргументы для вызываемой через стек. 
А как вызываемая функция имеет к ним доступ?}
{Now we figured out that caller function passing arguments to callee via stack. 
But how callee\footnote{function being called} access them?}

\begin{lstlisting}
#include <stdio.h>

int f (int a, int b, int c)
{
	return a*b+c;
};

int main() 
{
	printf ("%d\n", f(1, 2, 3));
	return 0;
};
\end{lstlisting}

\IFRU{Имеем в итоге}{What we have after compilation} (MSVC 2010 Express):

\begin{lstlisting}
_TEXT	SEGMENT
_a$ = 8							; size = 4
_b$ = 12						; size = 4
_c$ = 16						; size = 4
_f	PROC
; File c:\...\1.c
; Line 4
	push	ebp
	mov	ebp, esp
; Line 5
	mov	eax, DWORD PTR _a$[ebp]
	imul	eax, DWORD PTR _b$[ebp]
	add	eax, DWORD PTR _c$[ebp]
; Line 6
	pop	ebp
	ret	0
_f	ENDP

_main	PROC
; Line 9
	push	ebp
	mov	ebp, esp
; Line 10
	push	3
	push	2
	push	1
	call	_f
	add	esp, 12					; 0000000cH
	push	eax
	push	OFFSET $SG2463 ; '%d', 0aH, 00H
	call	_printf
	add	esp, 8
; Line 11
	xor	eax, eax
; Line 12
	pop	ebp
	ret	0
_main	ENDP
\end{lstlisting}

\IFRU{Итак, здесь видно: в функции \TT{\_main} заталкиваются три числа в стек и вызывается 
функиця \TT{f(int,int,int)}.}
{What we see is that 3 numbers are pushing to stack in function \TT{\_main} and \TT{f(int,int,int)} is called then.}
\IFRU{Внутри \TT{f()}, доступ к аргументам, также как и к локальным переменным, происходит через макросы: 
\TT{\_a\$ = 8}, но разница в том, что эти смещения со знаком \IT{плюс}, 
таким образом если прибавить макрос \TT{\_a\$} к указателю на \EBP, то адресуется \IT{внешняя} 
часть стека относительно \EBP.}
{Argument access inside \TT{f()} is organized with help of macros like: \TT{\_a\$ = 8}, 
in the same way as local variables accessed, but difference in that these offsets are positive 
(addressed with \IT{plus} sign).
So, adding \TT{\_a\$} macro to \EBP register value, \IT{outer} side of stack frame is addressed.}

\IFRU{Далее все более-менее просто: значение a кладется в \EAX. Далее \EAX умножается при помощи инструкции \IMUL на то что лежит в \TT{\_b}, так в \EAX остается произведение\footnote{результат умножения} этих двух значений.}
{Then \TT{a} value is stored into \EAX. After \IMUL instruction execution, \EAX value is 
product\footnote{result of multiplication} of \EAX and what is stored in \TT{\_b}.}
\IFRU{Далее к регистру \EAX прибавляется то что лежит в \TT{\_c}.}{After \IMUL execution, \ADD is 
summing \EAX and what is stored in \TT{\_c}.}
\IFRU
{Значение из \EAX никуда не нужно перекладывать, оно уже лежит где надо. Возвращаем управление вызываемой 
функции ~--- она возьмет значение из \EAX и отправит его в \printf.}
{Value in \EAX is not needed to be moved: it is already in place it need. Now return to caller ~--- it 
will take value from \EAX and used it as \printf argument.}
\IFRU{Скомпилируем то же в GCC 4.4.1 и посмотрим результат в IDA:}{Let's compile the same in GCC 4.4.1:}

\begin{lstlisting}
                public f
f               proc near               ; CODE XREF: main+20

arg_0           = dword ptr  8
arg_4           = dword ptr  0Ch
arg_8           = dword ptr  10h

                push    ebp
                mov     ebp, esp
                mov     eax, [ebp+arg_0]
                imul    eax, [ebp+arg_4]
                add     eax, [ebp+arg_8]
                pop     ebp
                retn
f               endp

                public main
main            proc near               ; DATA XREF: _start+17

var_10          = dword ptr -10h
var_C           = dword ptr -0Ch
var_8           = dword ptr -8

                push    ebp
                mov     ebp, esp
                and     esp, 0FFFFFFF0h
                sub     esp, 10h        ; char *
                mov     [esp+10h+var_8], 3
                mov     [esp+10h+var_C], 2
                mov     [esp+10h+var_10], 1
                call    f
                mov     edx, offset aD  ; "%d\n"
                mov     [esp+10h+var_C], eax
                mov     [esp+10h+var_10], edx
                call    _printf
                mov     eax, 0
                leave
                retn
main            endp
\end{lstlisting}

\IFRU{Практически то же самое, если не считать мелких отличий описанных раннее.}{Almost the same result.}
