% done

\documentclass[11pt,a4paper,oneside]{book}

\usepackage{cmap}

\ifdefined\RUSSIAN
\usepackage[english,russian]{babel}
\else
\usepackage[russian,english]{babel}
\fi

\usepackage[T2A]{fontenc}
\usepackage[utf8]{inputenc}
\usepackage{listings}
\usepackage{url}
\usepackage{graphicx}
\usepackage{listingsutf8}
\usepackage[cm]{fullpage}
\usepackage{color}
\usepackage{fancyvrb}
\usepackage{xspace}
\usepackage{framed}
\usepackage{ccicons}
\usepackage{amsmath}
\usepackage[]{hyperref} % should be last

\definecolor{lstbgcolor}{rgb}{0.94,0.94,0.94}

\newcommand{\TT}[1]{\texttt{#1}}
\newcommand{\IT}[1]{\textit{#1}}
\newcommand{\IFRU}[2]{\iflanguage{russian}{#1}{#2}}

\newcommand{\IDA}{IDA~\ref{IDA}\xspace}

\newcommand{\Task}{\IFRU{Задача}{Task}\xspace}

\newcommand{\Tchar}{\IT{char}\xspace} 
\newcommand{\Tint}{\IT{int}\xspace}
\newcommand{\Tbool}{\IT{bool}\xspace}
\newcommand{\Tfloat}{\IT{float}\xspace}
\newcommand{\Tdouble}{\IT{double}\xspace}

\newcommand{\CCpp}{\IFRU{Си/Си++}{C/C++}\xspace}

\newcommand{\Ox}{\TT{/Ox}\xspace}

% common C functions
\newcommand{\printf}{\TT{printf()}\xspace} 
\newcommand{\main}{\TT{main()}\xspace} 
\newcommand{\qsort}{\TT{qsort()}\xspace} 
\newcommand{\strlen}{\TT{strlen()}\xspace} 
\newcommand{\scanf}{\TT{scanf()}\xspace} 
\newcommand{\rand}{\TT{rand()}\xspace} 

% instructions
\newcommand{\ADD}{\TT{ADD}\xspace} 
\newcommand{\AND}{\TT{AND}\xspace} 
\newcommand{\CALL}{\TT{CALL}\xspace} 
\newcommand{\CPUID}{\TT{CPUID}\xspace} 
\newcommand{\CMP}{\TT{CMP}\xspace} 
\newcommand{\DEC}{\TT{DEC}\xspace} 
\newcommand{\FADDP}{\TT{FADDP}\xspace} 
\newcommand{\FCOM}{\TT{FCOM}\xspace}
\newcommand{\FCOMP}{\TT{FCOMP}\xspace}
\newcommand{\FCOMI}{\TT{FCOMI}\xspace}
\newcommand{\FCOMIP}{\TT{FCOMIP}\xspace}
\newcommand{\FUCOM}{\TT{FUCOM}\xspace}
\newcommand{\FUCOMI}{\TT{FUCOMI}\xspace}
\newcommand{\FUCOMIP}{\TT{FUCOMIP}\xspace}
\newcommand{\FUCOMPP}{\TT{FUCOMPP}\xspace}
\newcommand{\FDIVR}{\TT{FDIVR}\xspace} 
\newcommand{\FDIV}{\TT{FDIV}\xspace} 
\newcommand{\FLD}{\TT{FLD}\xspace} 
\newcommand{\FMUL}{\TT{FMUL}\xspace} 
\newcommand{\FSTP}{\TT{FSTP}\xspace} 
\newcommand{\FDIVP}{\TT{FDIVP}\xspace}
\newcommand{\IDIV}{\TT{IDIV}\xspace} 
\newcommand{\IMUL}{\TT{IMUL}\xspace} 
\newcommand{\INC}{\TT{INC}\xspace} 
\newcommand{\JAE}{\TT{JAE}\xspace} 
\newcommand{\JA}{\TT{JA}\xspace} 
\newcommand{\JBE}{\TT{JBE}\xspace} 
\newcommand{\JB}{\TT{JBE}\xspace} 
\newcommand{\JE}{\TT{JE}\xspace} 
\newcommand{\JGE}{\TT{JGE}\xspace} 
\newcommand{\JG}{\TT{JG}\xspace} 
\newcommand{\JLE}{\TT{JLE}\xspace} 
\newcommand{\JL}{\TT{JL}\xspace} 
\newcommand{\JMP}{\TT{JMP}\xspace} 
\newcommand{\JNE}{\TT{JNE}\xspace} 
\newcommand{\JNZ}{\TT{JNZ}\xspace} 
\newcommand{\JNA}{\TT{JNA}\xspace} 
\newcommand{\JNAE}{\TT{JNAE}\xspace} 
\newcommand{\JNB}{\TT{JNB}\xspace} 
\newcommand{\JNBE}{\TT{JNBE}\xspace} 
\newcommand{\JZ}{\TT{JZ}\xspace} 
\newcommand{\JP}{\TT{JP}\xspace} 
\newcommand{\Jcc}{\TT{Jcc}\xspace} 
\newcommand{\SETcc}{\TT{SETcc}\xspace} 
\newcommand{\LEA}{\TT{LEA}\xspace} 
\newcommand{\LOOP}{\TT{LOOP}\xspace}
\newcommand{\MOVSX}{\TT{MOVSX}~\ref{MOVSX}\xspace} 
\newcommand{\MOVZX}{\TT{MOVZX}\xspace} 
\newcommand{\MOV}{\TT{MOV}\xspace} 
\newcommand{\NOP}{\TT{NOP}\xspace} 
\newcommand{\POP}{\TT{POP}\xspace} 
\newcommand{\PUSH}{\TT{PUSH}\xspace} 
\newcommand{\NOT}{\TT{NOT}\xspace} 
\newcommand{\RET}{\TT{RET}\xspace} 
\newcommand{\SETNZ}{\TT{SETNZ}\xspace} 
\newcommand{\SETBE}{\TT{SETBE}\xspace} 
\newcommand{\SETNBE}{\TT{SETNBE}\xspace} 
\newcommand{\SUB}{\TT{SUB}\xspace} 
\newcommand{\TEST}{\TT{TEST}\xspace} 
\newcommand{\FNSTSW}{\TT{FNSTSW}\xspace}
\newcommand{\SAHF}{\TT{SAHF}\xspace}
\newcommand{\XOR}{\TT{XOR}\xspace} 
\newcommand{\OR}{\TT{OR}\xspace} 
\newcommand{\LEAVE}{\TT{LEAVE}\xspace} 
\newcommand{\MOVDQA}{\TT{MOVDQA}\xspace} 
\newcommand{\MOVDQU}{\TT{MOVDQU}\xspace} 
\newcommand{\PADDD}{\TT{PADDD}\xspace} 
\newcommand{\PCMPEQB}{\TT{PCMPEQB}\xspace} 

% flags

\newcommand{\ZF}{\TT{ZF}\xspace} 
\newcommand{\CF}{\TT{CF}\xspace} 
\newcommand{\PF}{\TT{PF}\xspace} 

% registers

\newcommand{\AL}{\TT{AL}\xspace} 
\newcommand{\AH}{\TT{AH}\xspace} 
\newcommand{\AX}{\TT{AX}\xspace} 
\newcommand{\EAX}{\TT{EAX}\xspace} 
\newcommand{\EBX}{\TT{EBX}\xspace} 
\newcommand{\ECX}{\TT{ECX}\xspace} 
\newcommand{\EDX}{\TT{EDX}\xspace} 
\newcommand{\DL}{\TT{DL}\xspace} 
\newcommand{\ESI}{\TT{ESI}\xspace} 
\newcommand{\EDI}{\TT{EDI}\xspace} 
\newcommand{\STZERO}{\TT{ST(0)}\xspace} 
\newcommand{\STONE}{\TT{ST(1)}\xspace} 
\newcommand{\EBP}{\TT{EBP}\xspace} 
\newcommand{\ESP}{\TT{ESP}\xspace} 
\newcommand{\XMMONE}{\TT{XMM1}\xspace}
\newcommand{\XMMZERO}{\TT{XMM0}\xspace}


\include{common_URLS}

\newcommand{\TITLE}{\IFRU{Краткое введение в reverse engineering для начинающих}
{Quick introduction to reverse engineering for beginners}}
\newcommand{\AUTHOR}{\IFRU{Денис Юричев}{Dennis Yurichev}}
\newcommand{\EMAIL}{dennis@yurichev.com}
\newcommand{\VERSION}{0.6} % version WHICH IS PREPARED NOW. 0.2 mean 0.19999(9)

\hypersetup{
    pdftex,
    colorlinks=true,
    allcolors=blue,
    pdfauthor={\AUTHOR},
    pdftitle={\TITLE}
    }

\selectlanguage{english}

\lstset{
    backgroundcolor=\color{lstbgcolor},
    basicstyle=\footnotesize\ttfamily,
    breaklines=true,
    frame=single,
    inputencoding=cp1251,
    basicstyle=\ttfamily,
    columns=fullflexible,keepspaces,
}
\renewcommand{\ttdefault}{cmtt} % need it here?

\newcommand{\SignedNumbersSectionName}{\IFRU{Представление знака в числах}
{Signed number representations}}

\newcommand{\DivisionByNineSectionName}{\IFRU{Деление на 9}{Division by 9}}

\newcommand{\WorkingWithFloatAsWithStructSubSubSectionName}{\IFRU
{Работа с типом float как со структурой}{Working with the float type as with a structure}}

\begin{document}

\VerbatimFootnotes

\frontmatter

\begin{titlepage}
\begin{center}
\vspace*{\fill}
\LARGE \TITLE

\vspace*{\fill}

\large \AUTHOR

\large \TT{<\EMAIL>}
\vspace*{\fill}
\vfill

\ccbyncnd

\copyright 2013, \AUTHOR. 

\IFRU{Это произведение доступно по лицензии Creative Commons «Attribution-NonCommercial-NoDerivs» 
(«Атрибуция — Некоммерческое использование — Без производных произведений») 3.0 Непортированная. 
Чтобы увидеть копию этой лицензии, посетите}
{This work is licensed under the Creative Commons Attribution-NonCommercial-NoDerivs 3.0 Unported License. 
To view a copy of this license, visit} \url{http://creativecommons.org/licenses/by-nc-nd/3.0/}.

\IFRU{Версия этого текста}{Text version} \VERSION{} ({\large \today}).   
\end{center}
\end{titlepage}

\tableofcontents
\cleardoublepage

% done

\chapter{\IFRU{Введение}{Preface}}

\IFRU
{Здесь (будет) немного моих заметок о reverse engineering на русском языке для начинающих, 
для тех кто хочет научиться понимать создаваемый \CCpp компиляторами код, коего, 
практически, больше всего остального.}
{Here (will be) some of my notes about reverse engineering in English language for 
those beginners who like to learn to understand x86 code created by \CCpp compilers 
(which is a most large mass of all executable software in the world).}

\IFRU{Наиболее используемых компилятора два: MSVC и GCC, на них и будем ставить эксперименты.}
{There are two most used compilers: MSVC and GCC, these we will use for experiments.}

\IFRU
{Имеется два основных синтаксиса X86 ассемблера: Intel (больше распространенный в DOS/Windows) и 
AT\&T (распространен в *NIX)}
{There are two most used x86 assembler syntax: Intel (most used in DOS/Windows) and AT\&T (used in *NIX)}
\footnote{\url{http://en.wikipedia.org/wiki/X86_assembly_language\#Syntax}}. 
\IFRU
{Здесь принят Intel-овский синтаксис. \IDA также выдает Intel-овский.}
{Here we use Intel syntax. \IDA produce Intel syntax listings too.}


\mainmatter

\chapter{\IFRU{Паттерны компиляторов}{Compiler's patterns}}

\IFRU
{Когда я учил Си, а затем Си++, я просто писал небольшие куски кода, компилировал и смотрел что 
получилось на ассемблере, так мне понять было намного проще. Я делал это такое количество раз, 
что связь между кодом на \CCpp и тем что генерирует компилятор вбилась мне в подсознание достаточно 
глубоко, поэтому я могу глядя на код на ассемблере сразу понимать, в общих чертах, что там было написано 
на Си. Возможно это поможет кому-то еще, попробую описать некоторые примеры.}
{When I first learn C and then C++, I was just writing small pieces of code, compiling it and see what 
is producing in assembler, that was an easy way to me. I did it a lot times and relation 
between \CCpp code and what compiler produce was imprinted in my mind so deep so that 
I can quickly understand what was in C code when I look at produced x86 code. 
Perhaps, this method may be helpful for anyone else, so I'll try to describe here some examples.}

% TODO: english/russian version is also present: URL

% done

\section{Hello, world!}
\label{sec:helloworld}

\IFRU{Начнем с знаменитого примера из книги}{Let's start with famous example from the book} "The C programming Language"\footnote{\url{http://en.wikipedia.org/wiki/The_C_Programming_Language}}:

\lstinputlisting{helloworld/1_1.c}

\IFRU{Компилируем в}{Let's compile it in} MSVC 2010: \TT{cl 1.cpp /Fa1.asm}

\IFRU
{(Ключ /Fa означает сгенерировать листинг на ассемблере)}
{(/Fa option mean generate assembly listing file)}

\begin{lstlisting}
CONST	SEGMENT
$SG3830	DB	'hello, world', 00H
CONST	ENDS
PUBLIC	_main
EXTRN	_printf:PROC
; Function compile flags: /Odtp
_TEXT	SEGMENT
_main	PROC
	push	ebp
	mov	ebp, esp
	push	OFFSET $SG3830
	call	_printf
	add	esp, 4
	xor	eax, eax
	pop	ebp
	ret	0
_main	ENDP
_TEXT	ENDS
\end{lstlisting}

\IFRU{Компилятор сгенерировал файл \TT{1.obj}, который впоследствии будет слинкован линкером в \TT{1.exe}.} 
{Compiler generated \TT{1.obj} file which will be linked into \TT{1.exe}.}

\IFRU{В нашем случае, этот файл состоит из двух сегментов: \TT{CONST} (для данных-констант) и \TT{\_TEXT} (для кода).}
{In our case, the file contain two segments: \TT{CONST} (for data constants) and \TT{\_TEXT} (for code).} 

\IFRU{Строка \TT{"hello, world"} в \CCpp имеет тип \TT{const char*}, однако не имеет имени.}
{The string \TT{"hello, world"} in \CCpp has type \TT{const char*}, however hasn't its own name.}

\IFRU{Но компилятору нужно как-то с ней работать, так что он дает ей внутреннее имя \TT{\$SG3830}.}
{But compiler need to work with this string somehow, so it define internal name \TT{\$SG3830} for it.}

\IFRU{Как видно, строка заканчивается нулевым байтом ~--- это требования стандарта \CCpp насчет строк.}
{As we can see, the string is terminated by zero byte ~--- this is \CCpp standard of strings.}

\IFRU{В сегменте кода \TT{\_TEXT} находится пока только одна функция ~--- \TT{\_main}.}
{In the code segment \TT{\_TEXT} there are only one function so far ~--- \TT{\_main}.}

\IFRU{Функция \TT{\_main}, как и практически все функции, начинается с пролога и заканчивается эпилогом.}
{Function \TT{\_main} starting with prologue code and ending with epilogue code, like almost any function.}

\IFRU{Об этом смотрите подробнее в разделе о прологе и эпилоге функции}
{Read more about it in section about function prolog and epilog}
~\ref{sec:prologepilog}.

\IFRU{Далее следует вызов функции \printf}
{After function prologue we see a function \printf call}: \TT{CALL \_printf}. 

\IFRU
{Перед этим вызовом, адрес строки (или указатель на нее) с нашим приветствием при помощи инструкции \PUSH помещается в стек.}
{Before the call, string address (or pointer to it) containing our greeting is placed into stack with help of \PUSH instruction.}

\IFRU{После того как функция \printf возвращает управление в функцию \main, адрес строки (или указатель на нее) все еще лежит в стеке.}
{When \printf function returning control flow to \main function, string address (or pointer to it) is still in stack.}

\IFRU{Так как он больше не нужен, то указатель стека (регистр \ESP) корректируется.} 
{Because we do not need it anymore, stack pointer (\ESP register) is to be corrected.}

\TT{ADD ESP, 4} \IFRU{означает прибавить 4 к значению в регистре \ESP.}
{mean add 4 to the value in \ESP register.}

\IFRU
{Так как, это 32-битный код, для передачи адреса нужно аккурат 4 байта. В x64-коде это 8 байт.}
{Since it is 32-bit code, we need exactly 4 bytes for address passing through the stack. 
It's 8 bytes in x64-code}

\IFRU
{Некоторые компиляторы, например Intel C++ Compiler, в этой же ситуации, могут вместо 
\ADD сгенерировать \TT{POP ECX} (это можно встретить например в коде Oracle RDBMS, им скомпилированном), 
что почти то же самое, только портится значение в регистре \ECX.}
{Some compilers like Intel C++ Compiler, at the same point, could emit \TT{POP ECX} 
instead of \ADD (for example, this can be observed in Oracle RDBMS code, compiled by Intel compiler), 
and this instruction has almost the same effect, but \ECX register contents will be rewritten.}

\IFRU
{Возможно, компилятор применяет \TT{POP ECX} потому что эта инструкция короче (1 байт против 3).}
{Probably, Intel compiler using \TT{POP ECX} because this instruction's opcode is shorter then 
\TT{ADD ESP, x} (1 byte against 3).}

\IFRU{О стеке можно прочитать в соответствующем разделе}{More about stack in relevant section}~\ref{sec:stack}.

\IFRU{После вызова \printf, в оригинальном коде на \CCpp было \TT{return 0} 
- вернуть 0 в качестве результата функции \main.} 
{After \printf call, in original \CCpp code was \TT{return 0} ~--- return zero as a \main function result.} 

\IFRU{В сгенерированном коде это обеспечивается инструкцией}
{In the generated code this is implemented by instruction} \TT{XOR EAX, EAX} 

\IFRU
{\XOR, на самом деле, как легко догадаться, "исключающее ИЛИ"}
{\XOR, in fact, just "eXclusive OR"}
\footnote{\url{http://en.wikipedia.org/wiki/Exclusive_or}}, 
\IFRU
{но компиляторы часто используют его вместо простого}
{but compilers using it often instead of}
\TT{MOV EAX, 0} ~--- 
\IFRU
{снова опкод немного короче (2 байта против 5).}
{slightly shorter opcode again (2 bytes against 5).}

\IFRU{Бывает так, что некоторые компиляторы генерируют}{Some compilers emitting} 
\TT{SUB EAX, EAX}, 
\IFRU
{что значит, \IT{отнять значение \EAX от \EAX}, в любом случае это даст 0 в результате.}
{which mean \IT{SUBtract \EAX value from \EAX}, which is in any case will result zero.}

\IFRU{Самая последняя инструкция \RET возвращает управление в вызывающую функцию.
Обычно, это код \CCpp CRT\footnote{C Run-Time Code}, который, в свою очередь, 
вернет управление операционной системе.}
{Last instruction \RET returning control flow to calling function.
Usually, it's \CCpp CRT\footnote{C Run-Time Code} code, which, in turn, 
return control to operation system.}

\IFRU{Теперь скомпилируем то же самое компилятором GCC 4.4.1 в Linux}
{Now let's try to compile the same \CCpp code in GCC 4.4.1 compiler in Linux}: \TT{gcc 1.c -o 1}

\IFRU{Затем при помощи \IDA. посмотрим как создалась функция \main.}
{After, with the \IDA disassembler assistance, let's see how \main function was created.} 

\IFRU{С другой стороны, мы можем посмотреть результат работы GCC при помощи ключа}
{Note: we could also see GCC assembler result applying option} \TT{-S -masm=intel})

\begin{lstlisting}
main            proc near

var_10          = dword ptr -10h

                push    ebp
                mov     ebp, esp
                and     esp, 0FFFFFFF0h
                sub     esp, 10h
                mov     eax, offset aHelloWorld ; "hello, world"
                mov     [esp+10h+var_10], eax
                call    _printf
                mov     eax, 0
                leave
                retn
main            endp
\end{lstlisting}

\IFRU{Почти то же самое. 
Адрес строки "hello, world" лежащей в сегменте данных, в начале сохраняется в \EAX, затем записывается в стек.
А еще в прологе функции мы видим \TT{AND ESP, 0FFFFFFF0h} ~--- 
эта инструкция выравнивает значение в \ESP по 16-байтной границе, делая некоторые значения 
в стеке также выровненными по этой границе (процессор более эффективно работает с переменными расположенными
в памяти по адресам кратным 4 или 16)\footnote{\URLWPDA}.}
{Almost the same.
Address of "hello world" string (stored in data segment) is saved in \EAX register first, then it stored into stack.
Also, in function prologue we see \TT{AND ESP, 0FFFFFFF0h} ~--- 
this instruction aligning \ESP value on 16-byte border, resulting some values in stack aligned too
(CPU performing better if values it working with are located in memory at addresses aligned by 
4 or 16 byte border)\URLWPDA.}

\TT{SUB ESP, 10h} \IFRU{выделяет в стеке 16 байт, хотя, как будет видно далее, нам достаточно только 4.}
{allocate 16 bytes in stack, although, as we could see below, we need only 4.} 

\IFRU{Это происходит потому что количество выделяемого места в локальном стеке тоже выровнено по 
16-байтной границе.}{This is because the size of allocated stack is also aligned on 16-byte border.}

\IFRU{Адрес строки (или указатель на строку) затем записывается прямо в стек без помощи инструкции \PUSH.
\IT{var\_10} по совместительству ~--- и локальная переменная и одновременно аргумент для \printf{}. Подробнее об этом будет ниже.}
{String address (or pointer to string) is then writing directly into stack space without \PUSH instruction use.
\IT{var\_10} ~--- is local variable, but also argument for \printf(). Read below about it.}

\IFRU{Затем вызывается \printf.}{Then \printf function is called.}

\IFRU{В отличие от MSVC, GCC в компиляции без включенной оптимизации генерирует \TT{MOV EAX, 0} вместо 
более короткого опкода.}{Unlike MSVC, GCC while compiling without optimization turned on, 
emitting \TT{MOV EAX, 0} instead of shorter opcode.}

\IFRU{Последняя инструкция \LEAVE ~--- это аналог команд \TT{MOV ESP, EBP} и \TT{POP EBP} ~--- 
то есть возврат указателя стека и регистра \EBP в первоначальное состояние.} 
{The last instruction \LEAVE ~--- is \TT{MOV ESP, EBP} and \TT{POP EBP} instructions pair equivalent ~--- 
in other words, this instruction setting back stack pointer (\ESP) and \EBP register to its initial state.} 

\IFRU{Это необходимо, т.к., в начале функции мы модифицировали регистры \ESP и \EBP (при помощи}
{This is necessary because we modified these register values (\ESP and \EBP) at the function start (executing}
\TT{\MOV EBP, ESP} / \TT{AND ESP, ...}).



% done

\section{\IFRU{Стек}{Stack}}
\label{sec:stack}

\IFRU{Стек в компьютерных науках ~--- это одна из наиболее фундаментальных вещей}
{Stack ~--- is one of the most fundamental things in computer science.}\footnote{\url{http://en.wikipedia.org/wiki/Call_stack}}.

\IFRU{Технически, это просто блок памяти в памяти процесса + регистр \ESP который указывает где-то в пределах этого блока.}{Technically, this is just memory block in process memory + \ESP register as a pointer within this block.}

\IFRU
{Часто используемые инструкции для работы со стеком это \PUSH и \POP. 
\PUSH уменьшает \ESP на 4, затем записывает по адресу на который указывает \ESP содержимое своего единственного операнда.}
{Most frequently used stack access instructions are \PUSH and \POP. 
\PUSH subtracting \ESP by 4 and then writing contents of its sole operand to the memory address pointing by \ESP.} 

\IFRU{\POP это обратная операция ~--- сначала достает из \ESP значение и кладет его в операнд 
(который очень часто является регистром) и затем увеличивает \ESP на 4. 
Конечно, это для 32-битной среды. В x64-среде это будет 8 а не 4.}
{\POP is reverse operation: get a data from memory pointing by \ESP and then add 4 to \ESP. Of course, 
this is for 32-bit environment. 8 will be here instead of 4 in x64 environment.}

\IFRU{В самом начале, \ESP указывает на конец стека.}{After stack allocation, \ESP pointing to the end of stack.}
\IFRU{\PUSH уменьшает \ESP, а \POP ~--- увеличивает.}{\PUSH increasing \ESP, and \POP decreasing.}
\IFRU{Конец стека находится в начале блока памяти выделенного под стек. Это странно, но это так.}
{The end of stack is actually at the beginning of allocated for stack memory block. 
It seems strange, but it is so.}

\IFRU{Для чего используется стек?}{What stack is used for?}

\subsection{\IFRU{Сохранение адреса куда должно вернуться управление после вызова функции}
{Save return address where function should return control after execution}}

\IFRU
{При вызове другой функции через \CALL, сначала в стек записывается адрес указывающий на место аккурат после 
инструкции \CALL, затем делается безусловный переход (\TT{JMP}) на адрес указанный в операнде.} 
{While calling another function by \CALL instruction, the address of point exactly after \CALL instruction is saved 
to stack, and then unconditional jump to the address from CALL operand is executed.} 

\IFRU{\CALL это аналог пары инструкций \TT{PUSH address\_after\_call / JMP}.}
{\CALL is \TT{PUSH address\_after\_call / JMP operand} instructions pair equivalent}.

\IFRU{\RET вытаскивает из стека значение и передает управление по этому адресу ~--- 
это аналог пары инструкций \TT{POP tmp / JMP tmp}.}
{\RET is fetching value from stack and jump to it ~--- it is \TT{POP tmp / JMP tmp} instructions pair equivalent.}

\IFRU{Крайне легко устроить переполнение стека запустив бесконечную рекурсию:}
{Stack overflow is simple, just run eternal recursion:}

\begin{lstlisting}
void f()
{
	f();
};
\end{lstlisting}

\IFRU{MSVC 2008 предупреждает о проблеме:}{MSVC 2008 reporting about problem:}

\begin{lstlisting}
c:\tmp6>cl ss.cpp /Fass.asm
Microsoft (R) 32-bit C/C++ Optimizing Compiler Version 15.00.21022.08 for 80x86
Copyright (C) Microsoft Corporation.  All rights reserved.

ss.cpp
c:\tmp6\ss.cpp(4) : warning C4717: 'f' : recursive on all control paths, function will cause runtime stack overflow
\end{lstlisting}

\IFRU{... но тем не менее создает нужный код:}{... but generates right code anyway:}

\begin{lstlisting}
?f@@YAXXZ PROC						; f
; File c:\tmp6\ss.cpp
; Line 2
	push	ebp
	mov	ebp, esp
; Line 3
	call	?f@@YAXXZ				; f
; Line 4
	pop	ebp
	ret	0
?f@@YAXXZ ENDP						; f
\end{lstlisting}

\IFRU
{... причем, если включить оптимизацию (\Ox), то будет даже интереснее, без переполнения стека, 
но работать будет \IT{корректно}\footnote{здесь ирония}:}
{... Also, if we turn on optimization (\Ox option), the optimized code will not overflow stack, 
but will work \IT{correctly}\footnote{irony here}:}

\begin{lstlisting}
?f@@YAXXZ PROC						; f
; File c:\tmp6\ss.cpp
; Line 2
$LL3@f:
; Line 3
	jmp	SHORT $LL3@f
?f@@YAXXZ ENDP						; f
\end{lstlisting}

\IFRU{GCC 4.4.1 генерирует точно такой же код в обоих случаях, хотя и не предупреждает о проблеме.}
{GCC 4.4.1 generating the same code in both cases, although not warning about problem.}

\subsection{\IFRU{Передача параметров для функции}{Function arguments passing}}

\begin{lstlisting}
push arg3
push arg2
push arg1
call f
add esp, 4*3
\end{lstlisting}

\IFRU{Вызываемая функция получает свои параметры также через указатель \ESP.}
{Callee{\footnote{Function being called}} function get its arguments via \ESP ponter.}

\IFRU{См.также в соответствующем разделе о способах передачи аргументов через стек}
{See also section about calling conventions}~\ref{sec:callingconventions}.

\IFRU{Важно отметить, что, в общем, никто не заставляет программистов передавать параметры именно через стек,
это не является требованием к исполняемому коду.}
{It is important to note that no one oblige programmers to pass arguments through stack, it is not prerequisite.}

\IFRU{Вы можете делать это совершенно иначе, не используя стек.}
{One could implement any other method not using stack.}

\IFRU{К примеру, можно выделять в куче\footnote{heap в англоязычной литературе} место для аргументов, 
заполнять их и передавать в функцию указатель на это место через \EAX. И это вполне будет работать.}
{For example, it is possible to allocate a place for arguments in heap, fill it and pass to a function 
via pointer to this pack in \EAX register. And this will work} 

\IFRU{Однако, так традиционно сложилось, что передача аргументов происходит именно через стек.}
{However, it is convenient tradition to use stack for this.}

\subsection{\IFRU{Хранение локальных переменных}{Local variable storage}}

\IFRU{Функция может выделить для себя некоторое место в стеке для локальных переменных просто отодвинув 
\ESP глубже к концу стека.}
{A function could allocate some space in stack for its local variables just shifting 
\ESP pointer deeply enough to stack bottom.}

\IFRU{Это снова не является необходимым требованием. Вы можете хранить локальные переменные где угодно. 
Но по традиции всё сложилось так.}
{It is also not prerequisite. You could store local variables wherever you like. 
But traditionally it is so.}

\subsubsection{\IFRU{Функция alloca()}{alloca() function}}

\IFRU{Интересен случай с функцией \TT{alloca()}}
{It is worth noting \TT{alloca()} function.}\footnote{
\IFRU
{В MSVC, реализацию функции можно посмотреть в файлах}
{As of MSVC, function implementation can be found in} 
  \TT{alloca16.asm} 
  \IFRU{и}{and} 
  \TT{chkstk.asm} 
  \IFRU{в}{in} 
  \TT{C:\textbackslash{}Program Files (x86)\textbackslash{}Microsoft Visual Studio 10.0\textbackslash{}VC\textbackslash{}crt\textbackslash{}src\textbackslash{}intel}}. 

\IFRU{Эта функция работает как \TT{malloc()}, но выделяет память прямо в стеке.} 
{This function works like \TT{malloc()}, but allocate memory just in stack.}

\IFRU{Память освобождать через \TT{free()} не нужно, так как эпилог функции~\ref{sec:prologepilog} 
вернет \ESP назад в изначальное состояние и выделенная память просто аyнулируется.}
{Allocated memory chunk is not needed to be freed via \TT{free()} function call because 
function epilogue~\ref{sec:prologepilog} will return \ESP back to initial state and 
allocated memory will be just annuled.} 

\IFRU{Интересна реализация функции \TT{alloca()}.}
{It is worth noting how \TT{alloca()} implemented.}

\IFRU{Эта функция, если упрощенно, просто сдвигает \ESP вглубь стека 
на столько байт сколько вам нужно и возвращает \ESP в качестве указателя на выделенный блок.}
{This function, if to simplify, just shifting \ESP deeply to stack bottom so much bytes you 
need and set \ESP as a pointer to that \IT{allocated} block.}
\IFRU{Попробуем:}{Let's try:}

\lstinputlisting{stack/2_1.c}

\IFRU{(Функция \TT{\_snprintf()} работает так же как и \printf, только вместо выдачи результата в 
stdout (т.е., на терминал или в консоль),
записывает его в буфер \TT{buf}. \TT{puts()} выдает содержимое буфера \TT{buf} в stdout. Конечно, можно было бы
заменить оба этих вызова на один \printf, но мне нужно проиллюстрировать использование небольшого буфера.)}
{(\TT{\_snprintf()} function works just like \printf, but instead dumping result into stdout (e.g., to terminal or 
console), write it to \TT{buf} buffer. \TT{puts()} copies \TT{buf} contents to stdout. Of course, these two
function calls might be replaced by one \printf call, but I would like to illustrate small buffer usage.)}

\IFRU{Компилируем}{Let's compile} (MSVC 2010):

\lstinputlisting{stack/2_2_msvc.asm}

\IFRU
{Единственный параметр в \TT{alloca()} передается через \EAX, а не как обычно через стек.}
{The sole \TT{alloca()} argument passed via \EAX (instead of pushing into stack).}

\IFRU{После вызова \TT{alloca()}, \ESP теперь указывает на блок в 600 байт который 
мы можем использовать под \TT{buf}.}
{After \TT{alloca()} call, \ESP is now pointing to the block of 600 bytes and we can 
use it as memory for \TT{buf} array.}

\IFRU{А GCC 4.4.1 обходится без вызова других функций:}
{GCC 4.4.1 can do the same without calling external functions:}

\lstinputlisting{\IFRU{stack/2_2_gcc_ru.asm}{stack/2_2_gcc_en.asm}}

\subsection{(Windows) SEH}

\IFRU{В стеке хранятся записи SEH (\IT{Structured Exception Handling}) для функции (если имеются)}
{SEH (\IT{Structured Exception Handling}) records are also stored in stack (if needed).}
\footnote{
\IFRU{О SEH: классическая статья Мэтта Питрека}{Classic Matt Pietrek article about SEH}: 
\url{http://www.microsoft.com/msj/0197/Exception/Exception.aspx}}.

\subsection{\IFRU{Защита от переполнений буфера}{Buffer overflow protection}}

\IFRU{Здесь больше об этом}{More about it here}~\ref{subsec:bufferoverflow}.



% done

\section{\IFRU{printf() с несколькими агрументами}{
printf() with several arguments}}

\IFRU{Попробуем теперь немного расширить пример \IT{Hello, world!}~\ref{sec:helloworld}, 
написав в теле функции \main:}
{Now let's extend \IT{Hello, world!}~\ref{sec:helloworld} example, replacing \printf in 
\main function body by this:}

\begin{lstlisting}
printf("a=%d; b=%d; c=%d", 1, 2, 3);
\end{lstlisting}

\IFRU{Компилируем при помощи MSVC 2010 Express, и в итоге получим:}
{Let's compile it by MSVC 2010 and we got:}

\begin{lstlisting}
$SG3830	DB	'a=%d; b=%d; c=%d', 00H

...

	push	3
	push	2
	push	1
	push	OFFSET $SG3830
	call	_printf
	add	esp, 16					; 00000010H
\end{lstlisting}

\IFRU{Все почти то же, за исключением того, что теперь видно, что аргументы для \printf заталкиваются в стек в обратном порядке: самый первый аргумент заталкивается последним.}
{Almost the same, but now we can see that \printf arguments are pushing into stack in reverse order: and the first argument is pushing in as the last one.}

\IFRU
{Всего 4 аргумента. 4*4 = 16 ~--- именно 16 байт занимают в стеке указатель на строку плюс еще 
3 числа типа \Tint.}
{Here 4 arguments. 4*4 = 16 ~--- exactly 16 byte they occupy in stack: 32-bit pointer to string and 
3 number of \Tint type.}

\IFRU{Переменные типа \Tint в 32-битной системе, как известно, имеет ширину 32 бита.}
{Variables of \Tint type in 32-bit environment has 32-bit width.}

\IFRU{Когда при помощи инструкции \TT{ADD ESP, X} корректируется указатель стека \ESP 
после вызова какой-либо функции, зачастую можно сделать вывод о том, сколько аргументов 
у вызываемой функции было, разделив X на 4.}
{When stack pointer (\ESP register) is corrected by \TT{ADD ESP, X} instruction after some function 
call, often, the number of function arguments could be deduced here: just divide X by 4.}

\IFRU{Конечно, это относится только к cdecl-методу передачи аргументов через стек.}
{Of course, this is related only to \IT{cdecl} calling convention.}

\IFRU{См.также в соответствующем разделе о способах передачи аргументов через стек}
{See also section about calling conventions}~\ref{sec:callingconventions}.

\IFRU{Иногда бывает так, что подряд идут несколько вызовов разных функций, 
но стек корректируется только один раз, после последнего вызова:}
{It is also possible for compiler to merge several \TT{ADD ESP, X} instructions into one, after last call:}

\begin{lstlisting}
push a1
push a2
call ...
...
push a1
call ...
...
push a1
push a2
push a3
call ...
add esp, 24
\end{lstlisting}

\IFRU{Скомпилируем то же самое в Linux при помощи GCC 4.4.1 и посмотрим в \IDA что вышло:}
{Now let's compile the same in Linux by GCC 4.4.1 and take a look in \IDA what we got:}

\begin{lstlisting}
main            proc near

var_10          = dword ptr -10h
var_C           = dword ptr -0Ch
var_8           = dword ptr -8
var_4           = dword ptr -4

                push    ebp
                mov     ebp, esp
                and     esp, 0FFFFFFF0h
                sub     esp, 10h
                mov     eax, offset aADBDCD ; "a=%d; b=%d; c=%d"
                mov     [esp+10h+var_4], 3
                mov     [esp+10h+var_8], 2
                mov     [esp+10h+var_C], 1
                mov     [esp+10h+var_10], eax
                call    _printf
                mov     eax, 0
                leave
                retn
main            endp
\end{lstlisting}

\IFRU{Можно сказать, что этот короткий код созданный GCC отличается от кода MSVC только способом помещения 
значений в стек.
Здесь GCC снова работает со стеком напрямую без \PUSH/\POP.}
{It can be said, the difference between code by MSVC and GCC is only in method of placing arguments into stack. 
Here GCC working directly with stack without \PUSH/\POP.}

\IFRU{Кстати, эта разница неплохо иллюстрирует тот важный момент, что процессору, в общем, все равно как будут 
передаваться параметры функций. Можно создать гипотетический компилятор, который будет передавать их при 
помощи указателя на структуру с параметрами, не пользуясь стеком вообще.}
{By the way, this difference is a good illustration that CPU is not aware of how arguments is passed to functions. 
It is also possible to create hypothetical compiler which is able to pass arguments 
via some special structure not using stack at all.}


% done

\section{scanf()}

\IFRU{Теперь попробуем использовать scanf().}{Now let's use scanf().}

\begin{lstlisting}
int main() 
{
	int x;
	printf ("Enter X:\n");

	scanf ("%d", &x);

	printf ("You entered %d...\n", x);

	return 0;
};
\end{lstlisting}

\IFRU
{Да, согласен, использовать \scanf в наши времена для того чтобы спросить у юзера что-то: не самая хорошая идея.
Но я хотел проиллюстрировать передачу указателя на \Tint.}
{OK, I agree, it is not clever to use \scanf today. But I wanted to illustrate passing pointer to \Tint.}

\IFRU{Что получаем на ассемблере компилируя MSVC 2010:}
{What we got after compiling in MSVC 2010:}

\lstinputlisting{scanf/4_1_msvc.asm}

\IFRU{Переменная \TT{x} является локальной.}{Variable \TT{x} is local.} 

\IFRU{По стандарту \CCpp она доступна только из этой же функции и ни откуда более. 
Так получилось, что локальные переменные располагаются в стеке. 
Может быть, можно было бы использовать и другие варианты, но в x86 это традиционно так.}
{\CCpp standard tell us it must be visible only in this function and not from any other place. 
Traditionally, local variables are placed in the stack. 
Probably, there could be other ways, but in x86 it is so.}

\IFRU{Следующая после пролога инструкция \TT{PUSH ECX} не ставит своей целью сохранить 
значение регистра \ECX. 
(Заметьте отсутствие сооветствующей инструкции \TT{POP ECX} в конце функции)}
{Next after function prologue instruction \TT{PUSH ECX} is not for saving \ECX state 
(notice absence of corresponding \TT{POP ECX} at the function end).}

\IFRU{Она на самом деле выделяет в стеке 4 байта для хранения \TT{x} в будущем.} 
{In fact, this instruction just allocate 4 bytes in stack for \TT{x} variable storage.} 

\IFRU{Доступ к \TT{x} будет осуществляться при помощи объявленного макроса \TT{\_x\$} 
(он равен -4) и регистра \EBP указывающего на текущий фрейм.}
{\TT{x} will be accessed with the assistance of \TT{\_x\$} macro 
(it equals to -4) and \EBP register pointing to current frame.}

\IFRU{Вообще, во все время исполнения функции, \EBP указывает на текущий фрейм и через \TT{EBP+смещение}
можно иметь доступ как к локальным переменным функции, так и аргументам функции.} 
{Over a span of function execution, \EBP is pointing to current stack frame and it is possible 
to have an access to local variables and function arguments via \TT{EBP+offset}.}

\IFRU
{Можно было бы использовать \ESP, но он во время исполнения функции постоянно меняется.}
{It is also possible to use \ESP, but it's often changing and not very handy.}

\IFRU
{У функции \scanf в нашем примере два аргумента.}{Function \scanf in our example has two arguments.}

\IFRU
{Первый ~--- указатель на строку содержащую \TT{"\%d"} и второй ~--- адрес переменной \TT{x}.} 
{First is pointer to the string containing \TT{"\%d"} and second ~--- address of variable \TT{x}.} 

\IFRU{Вначале адрес \TT{x} помещается в регистр \EAX при помощи инструкции \TT{lea eax, DWORD PTR \_x\$[ebp]}. 
{First of all, address of \TT{x} is placed into \EAX register by \TT{lea eax, DWORD PTR \_x\$[ebp]} instruction}

\IFRU{Инструкция \LEA означает \IT{load effective address}, но со временем она изменила свою функцию}
{\LEA meaning \IT{load effective address}, but over a time it changed its primary application}
~\ref{sec:LEA}.

\IFRU{Можно сказать что в данном случае \LEA просто помещает в \EAX результат суммы значения в регистре 
\EBP и макроса \TT{\_x\$}.}
{It can be said, \LEA here just placing to \EAX sum of \EBP value and \TT{\_x\$} macro.}

\IFRU{Это тоже что и}{It is the same as} \TT{lea eax, [ebp-4]}.

\IFRU{Итак, от значения \EBP отнимается 4 и помещается в \EAX.
Далее значение \EAX заталкивается в стек и вызывается \scanf.}
{So, 4 subtracting from \EBP value and result is placed to \EAX. 
And then value in \EAX is pushing into stack and \scanf is called.}

\IFRU{После этого вызывается \printf. Первый аргумент вызова которого, строка:} 
{After that, \printf is called. First argument is pointer to string: }
\TT{"You entered \%d...\textbackslash{}n"}.}

\IFRU{Второй аргумент: \TT{mov ecx, [ebp-4]}, эта инструкция кладет в \ECX не адрес переменной \TT{x}, 
а его значение, что там сейчас находится.}
{Second argument is prepared as: \TT{mov ecx, [ebp-4]},
this instruction placing to \ECX not address of \TT(x) variable but its contents.}

\IFRU{Далее значение \ECX заталкивается в стек и вызывается последний \printf.}
{After, \ECX value is placing into stack and last \printf called.}

\IFRU{Попробуем тоже самое скомпилировать в Linux при помощи GCC 4.4.1:}
{Let's try to compile this code in GCC 4.4.1 under Linux:}

\lstinputlisting{scanf/4_1_gcc.asm}

\label{puts}
\IFRU{GCC заменил первый вызов \printf на \TT{puts()}. 
Действительно, \printf с одним агрументом это почти аналог \TT{puts()}.}
{GCC replaced first \printf call to \TT{puts()}. 
Indeed: \printf with only sole argument is almost analogous to \TT{puts()}.} 

\IFRU
{\IT{Почти}, если принять условие что в строке не будет управляющих символов \printf 
начинающихся со знака процента. Тогда эффект от работы этих двух функций будет разным.}
{\IT{Almost}, because we need to be sure that this string will not contain printf-control 
statements starting with \IT{\%}: then effect of these two functions will be different.}

\IFRU{Why GCC replaced \printf to \TT{puts}? Because \TT(puts()) work faster}
{Зачем GCC заменил один вызов на другой? Потому что \TT{puts()} работает быстрее}
\footnote{\url{http://www.ciselant.de/projects/gcc_printf/gcc_printf.html}}. 

\IFRU
{Видимо потому, что просто проталкивает символы в stdout не сравнивая каждый со знаком процента.}
{It working faster because just passes characters to stdout not comparing each with \IT{\%} symbol.}

\IFRU
{Почему \scanf переименовали в \TT{\_\_\_isoc99\_scanf}, я честно говоря, пока не знаю.}
{Why \scanf is renamed to \TT{\_\_\_isoc99\_scanf}, I do not know yet.}

\IFRU{Далее все как и прежде ~--- параметры заталкиваются через стек при помощи \MOV.}
{As before ~--- arguments are placed into stack by \MOV instruction.}

\subsection{\IFRU{Глобальные переменные}{Global variables}}

\IFRU
{А что если переменная \TT{x} из предыдущего примера будет глобальной переменной а не локальной? 
Ну, тогда к ней смогут обращаться из любого другого места, а не только из тела функции. 
Это снова не очень хорошая практика программирования, но ради примера мы можем себе это позволить.}
{What if \TT{x} variable from previous example will not local but global variable? 
Then it will be accessible from any place but not only from function body. 
It is not very good programming practice, but for the sake of experiment we could do this.}

\lstinputlisting{scanf/4_2_msvc.asm}

\IFRU
{Ничего особенного, в целом. Теперь \TT{x} объявлена в сегменте \TT{\_DATA}. 
Память для нее в стеке более не выделяется. Все обращения к ней происходит не через стек, а уже напрямую. 
Её значение неопределено. 
Это означает, что память под нее будет выделена, но ни компилятор, ни ОС не будет заботиться о том, 
что там будет лежать на момент старта функции \TT{\_main}.
В качестве домашнего задания, попробуйте объявить большой неопределенный массив и посмотреть 
что там будет лежать после загрузки.}
{Now \TT{x} variable is defined in \TT{\_DATA} segment. 
Memory in local stack is not allocated anymore. 
All accesses to it are not via stack but directly to process memory. 
Its value is not defined. 
This mean that memory will be allocated by operation system, but not compiler, 
neither operation system will not take care about its initial value at the moment of 
\main function start.
As experiment, try to declare large array and see what will it contain after 
program loading.}

\IFRU{Попробуем изменить объявление этой переменной:}{Now let's assign value to variable explicitly:}

\begin{lstlisting}
int x=10; // default value
\end{lstlisting}

\IFRU{Выйдет в итоге:}{We got:}

\begin{lstlisting}
_DATA	SEGMENT
_x	DD	0aH

...
\end{lstlisting}

\IFRU{Здесь уже по месту этой переменной записано \TT{0xA} с типом DD (dword = 32 бита).}
{Here we see value 0xA typed as DWORD (DD meaning DWORD = 32 bit).}

\IFRU{Если вы откроете скомпилированный .exe-файл в \IDA, то увидите что \IT{x} 
находится аккурат в начале сегмента \TT{\_DATA}, после этой переменной будут текстовые строки.}
{If you will open compiled .exe in \IDA, you will see \IT{x} placed at the beginning of 
\TT{\_DATA} segment, and after you'll see text strings.}

\IFRU{А вот если вы откроете в \IDA, .exe скомплированный в прошлом примере, 
где значение \IT{x} неопределено, то в IDA вы увидите:}
{If you will open compiled .exe in \IDA from previous example where \IT{x} value is not defined, 
you'll see something like this:}

\begin{lstlisting}
.data:0040FA80 _x              dd ?                    ; DATA XREF: _main+10
.data:0040FA80                                         ; _main+22
.data:0040FA84 dword_40FA84    dd ?                    ; DATA XREF: _memset+1E
.data:0040FA84                                         ; unknown_libname_1+28
.data:0040FA88 dword_40FA88    dd ?                    ; DATA XREF: ___sbh_find_block+5
.data:0040FA88                                         ; ___sbh_free_block+2BC
.data:0040FA8C ; LPVOID lpMem
.data:0040FA8C lpMem           dd ?                    ; DATA XREF: ___sbh_find_block+B
.data:0040FA8C                                         ; ___sbh_free_block+2CA
.data:0040FA90 dword_40FA90    dd ?                    ; DATA XREF: _V6_HeapAlloc+13
.data:0040FA90                                         ; __calloc_impl+72
.data:0040FA94 dword_40FA94    dd ?                    ; DATA XREF: ___sbh_free_block+2FE
\end{lstlisting}

\IFRU{\TT{\_x} обозначен как \TT{?}, наряду с другими переменными не требующими инициализции. 
Это означает, что при загрузке .exe в память, место под все это выделено будет. 
Но в самом .exe ничего этого нет. Неинициализированные переменные не занимают места в исполняемых файлах. Удобно для больших массивов, например.}
{\TT{\_x} marked as \TT{?} among another variables not required to be initialized. 
This mean that after loading .exe to memory, place for all these variables will be 
allocated and some random garbage will be here. 
But in .exe file these not initialized variables are not occupy anything. 
It is suitable for large arrays, for example.}

\IFRU{В Linux все также почти. За исключением того что если значение \TT{x} не определено, 
то эта переменная будет находится в сегменте \TT{\_bss}. В ELF\footnote{Формат исполняемых файлов, использующийся в Linux и некоторых других *NIX} этот сегмент имеет такие аттрибуты:}
{It is almost the same in Linux, except segment names and properties: 
not initialized variables are located in \TT{\_bss} segment. 
In ELF\footnote{Executable file format widely used in *NIX system including Linux} 
file format this segment has such attributes:}

\begin{lstlisting}
; Segment type: Uninitialized
; Segment permissions: Read/Write
\end{lstlisting}

\IFRU{Ну а если сделать присвоение этой переменной значения 10, то она будет находится в сегменте \TT{\_data},
это сегмент с такими аттрибутами:}
{If to assign some value to variable, it will be placed in \TT{\_data} segment, 
this is segment with such attributes:}

\begin{lstlisting}
; Segment type: Pure data
; Segment permissions: Read/Write
\end{lstlisting}

\subsection{\IFRU{Проверка результата scanf()}{scanf() result checking}}

\IFRU
{Как я уже говорил, использовать \scanf в наше время это слегка старомодно. 
Но если уж жизнь заставила этим заниматься, нужно хотя бы проверять, сработал ли \scanf 
правильно или пользователь ввел вместо числа что-то другое, что \scanf не смог трактовать как число.}
{As I said before, it is not very clever to use \scanf today. 
But if we have to, we need at least check if \scanf finished correctly without error.}

\begin{lstlisting}
int main() 
{
	int x;
	printf ("Enter X:\n");

	if (scanf ("%d", &x)==1)
		printf ("You entered %d...\n", x);
	else
		printf ("What you entered? Huh?\n");

	return 0;
};
\end{lstlisting}

\IFRU{По стандарту, }{By standard, } \scanf\footnote{\href{http://msdn.microsoft.com/en-us/library/9y6s16x1(VS.71).aspx}{MSDN: scanf, wscanf}} \IFRU{возвращает количество успешно полученных значений.}{function returning number of fields it successfully read.}

\IFRU{В нашем случае, если все успешно и пользователь ввел таки некое число, \scanf вернет 1. 
А если нет, то 0 или EOF.} 
{In our case, if everything went fine and user entered some number, 
\scanf will return 1 or 0 or EOF in case of error.}

\IFRU{Мы добавили код, который проверяет результат \scanf и в случае ошибки, говорит пользователю что-то другое.}
{We added C code for \scanf result checking and printing error message in case of error.}

\IFRU{Вот, что выходит на ассемблере}{What we got in assembler} (MSVC 2010):

\begin{lstlisting}
; Line 8
	lea	eax, DWORD PTR _x$[ebp]
	push	eax
	push	OFFSET $SG3833 ; '%d', 00H
	call	_scanf
	add	esp, 8
	cmp	eax, 1
	jne	SHORT $LN2@main
; Line 9
	mov	ecx, DWORD PTR _x$[ebp]
	push	ecx
	push	OFFSET $SG3834 ; 'You entered %d...', 0aH, 00H
	call	_printf
	add	esp, 8
; Line 10
	jmp	SHORT $LN1@main
$LN2@main:
; Line 11
	push	OFFSET $SG3836 ; 'What you entered? Huh?', 0aH, 00H
	call	_printf
	add	esp, 4
$LN1@main:
; Line 13
	xor	eax, eax
\end{lstlisting}

\IFRU{Для того чтобы вызывающая функция имела доступ к результату вызываемой функции, 
вызываемая функция (в нашем случае \scanf) оставляет это значение в регистре \EAX.}
{Caller function (\main) must have access to result of callee function (\scanf), 
so callee leave this value in \EAX register.}

\IFRU{Мы проверяем его инструкцией \TT{CMP EAX, 1} (\IT{CoMPare}), то есть, 
сравниваем значение в \EAX с 1.}
{After, we check it using instruction \TT{CMP EAX, 1} (\IT{CoMPare}), 
in other words, we compare value in \EAX with 1.} 

\IFRU{Следующий за инструкцией \CMP: условный переход \JNE. 
Это означает \IT{Jump if Not Equal}, то есть, условный переход \IT{если не равно}.}
{\JNE conditional jump follows \CMP instruction. \JNE mean \IT{Jump if Not Equal}.}

\IFRU{Итак, если \EAX не равен 1, то \JNE заставит перейти процессор 
по адресу указанном в операнде \JNE, у нас это \TT{\$LN2@main}.}
{So, if \EAX value not equals to 1, then the processor will pass execution to the 
address mentioned in operand of \JNE, in our case it is \TT{\$LN2@main}.}
\IFRU
{Передав управление по этому адресу, процессор как раз начнет исполнять вызов \printf с 
аргументом \TT{"What you entered? Huh?"}.}
{Passing control to this address, microprocesor will execute function \printf 
with argument \TT{"What you entered? Huh?"}.}
\IFRU
{Но если все нормально, перехода не случится, и исполнится другой \printf с двумя аргументами: 
\TT{'You entered \%d...'} и значением переменной \TT{x}.}
{But if everything is fine, conditional jump will not be taken, and another \printf call 
will be executed, with two arguments: \TT{'You entered \%d...'} and value of variable \TT{x}. }

\IFRU
{А для того чтобы после этого вызова не исполнился сразу второй вызов \printf, 
после него имеется инструкция \JMP, безусловный переход, он отправит процессор на место аккурат 
после второго \printf и перед инструкцией \TT{XOR EAX, EAX}, которая собственно \TT{return 0}.}
{Because second subsequent \printf not needed to be executed, there are \JMP after (unconditional jump) 
it will pass control to the place after second \printf and before \TT{XOR EAX, EAX} instruction, 
which implement \TT{return 0}.}

\IFRU{Итак, можно сказать, что в подавляющем случае сравнение какой либо переменной с чем-то другим 
происходит при помощи пары инструкций \CMP и \Jcc, где \IT{cc} это \IT{condition code}.}
{So, it can be said that most often, comparing some value with another is implemented 
by \CMP/\Jcc instructions pair, where \IT{cc} is \IT{condition code}.}
\IFRU{\CMP сравнивает два значения и выставляет 
флаги процессора\footnote{См.также о флагах x86-процессора: \url{http://en.wikipedia.org/wiki/FLAGS_register_(computing)}.}.}
{\CMP comparing two values and set 
processor flags\footnote{About x86 flags, see also: \url{http://en.wikipedia.org/wiki/FLAGS_register_(computing)}.}.}
\IFRU
{\Jcc проверяет нужные ему флаги и выполняет переход по указанному адресу (или не выполняет).}
{\Jcc check flags needed to be checked and pass control to mentioned address (or not pass).}

\label{CMPandSUB}
\IFRU{Но на самом деле, как это не парадоксально поначалу звучит, \CMP это почти то же самое что и 
инструкция \SUB, которая отнимает числа одно от другого.}
{But in fact, this could be perceived paradoxial, but CMP instruction is in fact SUB (subtract).}
\IFRU{Все арифметические инструкции также выставляют флаги в соответствии с результатом, не только \CMP.}
{All arithmetic instructions set processor flags too, not only \CMP.}
\IFRU{Если мы сравним 1 и 1, от еденицы отнимется единица, получится ноль, и выставится флаг 
\ZF (\IT{zero flag}), означающий что последний полученный результат является нулем.}
{If we compare 1 and 1, 1-1 will result zero, \ZF flag will be set (meaning that last result was zero).}
\IFRU{Ни при каких других значениях \EAX, флаг \ZF выставлен не будет, кроме тех, когда операнды равны друг другу.}
{There are no any other circumstances when it's possible except when operands are equal.}
\IFRU{Инструкция \JNE проверяет только флаг \ZF, и совершает переход только если флаг не поднят. 
Фактически, \JNE это синоним инструкции \JNZ (\IT{Jump if Not Zero}).}
{\JNE checks only \ZF flag and jumping only if it is not set. 
\JNE is in fact a synonym of \JNZ (\IT{Jump if Not Zero}) instruction.}
\IFRU{Ассемблер транслирует обе инструкции в один и тот же опкод.}
{Assembler translating both \JNE and \JNZ instructions into one single opcode.}
\IFRU
{Таким образом, можно \CMP заменить на \SUB и все будет работать также, но разница в том что \SUB 
все-таки испортит значение в первом операнде. \CMP это \IT{SUB без сохранения результата}.}
{So, \CMP can be replaced to \SUB and almost everything will be fine, but the difference is in 
that \SUB alter value at first operand. \CMP is \IT{"SUB without saving result"}.}

\IFRU
{Код созданный при помощи GCC 4.4.1 в Linux практически такой же, если не считать мелких отличий, 
которые мы уже рассмотрели раннее.}
{Code generated by GCC 4.4.1 in Linux is almost the same, except differences we already considered.}


% done

\section{\IFRU{Передача параметров через стек}{Passing arguments via stack}}

\IFRU
{Как мы уже успели заметить, вызывающая функция передает аргументы для вызываемой через стек. 
А как вызываемая функция имеет к ним доступ?}
{Now we figured out that caller function passing arguments to callee via stack. 
But how callee\footnote{function being called} access them?}

\begin{lstlisting}
#include <stdio.h>

int f (int a, int b, int c)
{
	return a*b+c;
};

int main() 
{
	printf ("%d\n", f(1, 2, 3));
	return 0;
};
\end{lstlisting}

\IFRU{Имеем в итоге}{What we have after compilation} (MSVC 2010 Express):

\begin{lstlisting}
_TEXT	SEGMENT
_a$ = 8							; size = 4
_b$ = 12						; size = 4
_c$ = 16						; size = 4
_f	PROC
; File c:\...\1.c
; Line 4
	push	ebp
	mov	ebp, esp
; Line 5
	mov	eax, DWORD PTR _a$[ebp]
	imul	eax, DWORD PTR _b$[ebp]
	add	eax, DWORD PTR _c$[ebp]
; Line 6
	pop	ebp
	ret	0
_f	ENDP

_main	PROC
; Line 9
	push	ebp
	mov	ebp, esp
; Line 10
	push	3
	push	2
	push	1
	call	_f
	add	esp, 12					; 0000000cH
	push	eax
	push	OFFSET $SG2463 ; '%d', 0aH, 00H
	call	_printf
	add	esp, 8
; Line 11
	xor	eax, eax
; Line 12
	pop	ebp
	ret	0
_main	ENDP
\end{lstlisting}

\IFRU{Итак, здесь видно: в функции \TT{\_main} заталкиваются три числа в стек и вызывается 
функиця \TT{f(int,int,int)}.}
{What we see is that 3 numbers are pushing to stack in function \TT{\_main} and \TT{f(int,int,int)} is called then.}
\IFRU{Внутри \TT{f()}, доступ к аргументам, также как и к локальным переменным, происходит через макросы: 
\TT{\_a\$ = 8}, но разница в том, что эти смещения со знаком \IT{плюс}, 
таким образом если прибавить макрос \TT{\_a\$} к указателю на \EBP, то адресуется \IT{внешняя} 
часть стека относительно \EBP.}
{Argument access inside \TT{f()} is organized with help of macros like: \TT{\_a\$ = 8}, 
in the same way as local variables accessed, but difference in that these offsets are positive 
(addressed with \IT{plus} sign).
So, adding \TT{\_a\$} macro to \EBP register value, \IT{outer} side of stack frame is addressed.}

\IFRU{Далее все более-менее просто: значение a кладется в \EAX. Далее \EAX умножается при помощи инструкции \IMUL на то что лежит в \TT{\_b}, так в \EAX остается произведение\footnote{результат умножения} этих двух значений.}
{Then \TT{a} value is stored into \EAX. After \IMUL instruction execution, \EAX value is 
product\footnote{result of multiplication} of \EAX and what is stored in \TT{\_b}.}
\IFRU{Далее к регистру \EAX прибавляется то что лежит в \TT{\_c}.}{After \IMUL execution, \ADD is 
summing \EAX and what is stored in \TT{\_c}.}
\IFRU
{Значение из \EAX никуда не нужно перекладывать, оно уже лежит где надо. Возвращаем управление вызываемой 
функции ~--- она возьмет значение из \EAX и отправит его в \printf.}
{Value in \EAX is not needed to be moved: it is already in place it need. Now return to caller ~--- it 
will take value from \EAX and used it as \printf argument.}
\IFRU{Скомпилируем то же в GCC 4.4.1 и посмотрим результат в IDA:}{Let's compile the same in GCC 4.4.1:}

\begin{lstlisting}
                public f
f               proc near               ; CODE XREF: main+20

arg_0           = dword ptr  8
arg_4           = dword ptr  0Ch
arg_8           = dword ptr  10h

                push    ebp
                mov     ebp, esp
                mov     eax, [ebp+arg_0]
                imul    eax, [ebp+arg_4]
                add     eax, [ebp+arg_8]
                pop     ebp
                retn
f               endp

                public main
main            proc near               ; DATA XREF: _start+17

var_10          = dword ptr -10h
var_C           = dword ptr -0Ch
var_8           = dword ptr -8

                push    ebp
                mov     ebp, esp
                and     esp, 0FFFFFFF0h
                sub     esp, 10h        ; char *
                mov     [esp+10h+var_8], 3
                mov     [esp+10h+var_C], 2
                mov     [esp+10h+var_10], 1
                call    f
                mov     edx, offset aD  ; "%d\n"
                mov     [esp+10h+var_C], eax
                mov     [esp+10h+var_10], edx
                call    _printf
                mov     eax, 0
                leave
                retn
main            endp
\end{lstlisting}

\IFRU{Практически то же самое, если не считать мелких отличий описанных раннее.}{Almost the same result.}


% done

\section{\IFRU{И еще немного о возвращаемых результатах}{One more word about results returning.}}

\newcommand{\MSDNURL}{\href{http://msdn.microsoft.com/en-us/library/7572ztz4.aspx}{MSDN: Return Values (C++)}}

\IFRU
{Резльутат выполнения функции возвращается\footnote{См.также: \MSDNURL} через регистр \EAX, а если результат типа байт, 
то в самой младшей части \EAX ~--- \AL. Если функция возвращает число с плавающей запятой, 
то регистр FPU \STZERO будет использован.}
{Function execution result is often returned\footnote{See also: \MSDNURL} in \EAX register. 
If it's byte type ~--- then in lowest register \EAX part ~--- \AL. If function returning \Tfloat number, FPU register 
\STZERO will be used instead.}

\IFRU{Вот почему старые компиляторы Си не способны создавать функции возвращающие нечто большее нежели помещается 
в один регистр (обычно, тип \Tint), а когда нужно, приходится возвращать через указатели, указываемые 
в аргументах.}
{That is why old C compilers can't create functions capable of returning something not fitting in one 
register (usually type \Tint), but if one need it, one should return information via pointers passed 
in function arguments.}
\IFRU{Хотя, позже и стало возможным, вернуть, скажем, целую структуру, но этот метод до сих пор не очень популярен. 
Если функция должна вернуть структуру, вызывающая функция должна сама, скрыто и прозрачно для программиста, 
выделить место и передать указатель на него в качестве первого аргумента. Это почти то же самое 
что и сделать это вручную, но компилятор прячет это.

Небольшой пример:}
{Now it is possible, to return, let's say, whole structure, but its still not very popular. 
If function should return a large structure, caller must allocate it and pass pointer to it via first argument, 
hiddenly and transparently for programmer. 
That is almost the same as to pass pointer in first argument manually, but compiler hide this.

Small example:}

\lstinputlisting{return_results/6_1.c}

\IFRU{... получим}{... what we got} (MSVC 2010 \Ox):

\lstinputlisting{return_results/6_1.asm}

\IFRU{Имя внутреннего макроса для передачи указателя на структуру здесь это \TT{\$T3853}.}
{Macro name for internal variable passing pointer to structure is \TT{\$T3853} here.}



% done

\section{\IFRU{Условные переходы}{Conditional jumps}}
\label{sec:Jcc}

\IFRU{Об условных переходах.}{Now about conditional jumps.}

\lstinputlisting{jcc/7_1.c}

\IFRU{Имеем в итоге функцию \TT{f\_signed()}:}{What we have in \TT{f\_signed()} function:}

\lstinputlisting{jcc/7_2.asm}

\IFRU
{Первая инструкция \JLE значит \IT{Jump if Larger or Equal}. То есть, если второй операнд больше первого или 
равен ему, произойдет переход туда, где будет следующая проверка.}
{First instruction \JLE mean \IT{Jump if Larger or Equal}. In other words, if second operand is 
larger than first or equal, control flow will be passed to address or label mentioned in instruction.}
\IFRU
{А если это условие не срабатывает, то есть второй операнд меньше первого, то перехода не будет, 
и сработает первый \printf.}
{But if this condition will not trigger (second operand less than first), control flow will 
not be altered and first \printf will be called.}
\IFRU
{Вторая проверка это \JNE: \IT{Jump if Not Equal}. Переход не произойдет, если операнды равны.}
{The second check is \JNE: \IT{Jump if Not Equal}. Control flow will not altered if operands are 
equals to each other.}
\IFRU
{Третья проверка \JGE: \IT{Jump if Greater or Equal} ~--- переход если второй операнд больше 
первого или равен ему.}
{The third check is \JGE: \IT{Jump if Greater or Equal} ~--- jump if second operand is larger 
than first or they are equals to each other.}
\IFRU
{Кстати, если все три условных перехода сработают, ни один \printf не вызовется. 
Но, без внешнего вмешательства, это, пожалуй, невозможно.}
{By the way, if all three conditional jumps are triggered, no \printf will be called at all. 
But, without special intervention, it is nearly impossible.}

GCC 4.4.1 \IFRU{производит почти такой же код, за исключением}
{produce almost the same code, but with} \TT{puts()}~\ref{puts} \IFRU{вместо}{instead of} \printf.

\IFRU{Далее функция \TT{f\_unsigned()} скомпилированная GCC:}
{Now let's take a look of \TT{f\_unsigned()} produced by GCC:}

\lstinputlisting{jcc/7_3_gcc.asm}

\IFRU{Здесь все то же самое, только инструкции условных переходов немного другие: 
\JBE ~--- \IT{Jump if Below or Equal} и \JAE ~--- \IT{Jump if Above or Equal}.}
{Almost the same, with exception of instructions: \JBE ~--- \IT{Jump if Below or Equal} 
and \JAE ~--- \IT{Jump if Above or Equal}.}
\IFRU{Эти инструкции (\JA/\JAE/\JB/\JBE) отличаются от \JG/\JGE/\JL/\JLE тем, 
что работают с беззнаковыми переменными.}
{These instructions (\JA/\JAE/\JB/\JBE) is different from \JG/\JGE/\JL/\JLE in that way, 
it works with unsigned numbers.}

\IFRU{Отступление: представление знака в числах}
{See also section about signed number representations}~\ref{sec:signednumbers}.
\IFRU{Таким образом, увидев где используется \JG/\JL вместо \JA/\JB и наоборот, 
можно сказать почти уверенно насчет того, 
является ли тип переменной знаковым (signed) или беззнаковым (unsigned).}
{So, where we see usage of \JG/\JL instead of \JA/\JB or otherwise, 
we can almost be sure about signed or unsigned type of variable.}

\IFRU{Далее функция \main, где ничего нового для нас нет:}
{Here is also \main function, where nothing new to us:}

\lstinputlisting{jcc/7_4.asm}



% done

\section{switch()/case/default}

\subsection{\IFRU{Если вариантов мало}{Few number of cases}}

\lstinputlisting{switch/8_1.c}

\IFRU{Это дает в итоге}{Result} (MSVC 2010):

\lstinputlisting{switch/8_2_msvc.asm}

\IFRU{В принципе, ничего особо нового для нас здесь, за исключением того, что компилятор зачем-то 
перекладывает входящую переменную \TT{a} во временную в локальном стеке \TT{v64}.}
{Nothing specially new to us, with the exception that compiler moving input variable 
\TT{a} to temporary local variable \TT{tv64}.}

\IFRU{Если скомпилировать это при помощи GCC 4.4.1, то будет почти то же самое, даже с максимальной оптимизацией 
(ключ \TT{-O3}).}
{If to compile the same in GCC 4.4.1, we'll get alsmost the same, even with maximal optimization 
turned on (\TT{-O3} option).}

\IFRU{Попробуем, включить оптимизацию кодегенератора}
{Now let's turn on optimization in} MSVC (\Ox): \TT{cl 1.c /Fa1.asm /Ox}

\lstinputlisting{switch/8_3_msvc.asm}

\IFRU{Вот здесь уже все немного по-другому, причем не без грязных хаков.}
{Here we can see even dirty hacks.}

\IFRU
{Первое: \TT{а} ложится в \EAX и от него отнимается 0. Звучит абсурдно, но нужно это для того, чтобы проверить, 
0 ли в \EAX был до этого? Если да, то выставится флаг \ZF (что означает что результат отнимания нуля от числа 
стал нулем) и первый условный переход \JE (\IT{Jump if Equal} или его синоним \JZ ~--- \IT{Jump if Zero}) 
сработает на метку \TT{\$LN4@f}, где выводится сообщение \TT{'zero'}.
Если первый переход не сработал, от значения отнимается по единице, 
и если на какой-то стадии образуется в результате 0, то сработает соответствующий переход.}
{First: \TT{a} is placed into \EAX and 0 subtracted from it. Sounds absurdly, but it may need to check if 
0 was in \EAX before? If yes, flag \ZF will be set (this also mean that subtracting from zero is zero) 
and first conditional jump \JE (\IT{Jump if Equal} or synonym \JZ ~--- \IT{Jump if Zero}) will be triggered 
and control flow passed to \TT{\$LN4@f} label, where \TT{'zero'} message is begin printed. 
If first jump was not triggered, 1 subtracted from input value and if at some stage 0 will be resulted, 
corresponding jump will be triggered.}

\IFRU{И в конце концов, если ни один из условных переходов не сработал, управление передается \printf
с агрументом \TT{'something unknown'}.}
{And if no jump triggered at all, control flow passed to \printf with argument \TT{'something unknown'}.}

\IFRU
{Второе: мы видим две, мягко говоря, необычные вещи: указатель на сообщение помещается в переменную \TT{a}, 
и затем \printf вызывается не через \CALL, а через \JMP. Объяснение этому простое. 
Вызывающая функция заталкивает в стек некоторое значение и через \CALL вызывает нашу функцию. 
\CALL в свою очередь затакливает в стек адрес возврата и делает безусловный переход на адрес нашей функции. 
Наша функция в самом начале (да и в любом её месте, потому что в теле функции нет ни одной инструкции, 
которая меняет что-то в стеке или в \ESP) имеет следующую разметку стека:}
{Second: we see unusual thing for us: string pointer is placed into \TT{a} variable, and 
then \printf is called not via \CALL, but via \JMP. This could be explained simply. 
Caller pushing to stack some value and via \CALL calling our function. 
\CALL itself pushing returning address to stack and do unconditional jump to our function address. 
Our function at any place of its execution (since it do not contain any instruction moving stack 
pointer) has the following stack layout:}

\begin{itemize}
\item\ESP ~--- \IFRU{хранится адрес возврата}{pointing to return address} 
\item\TT{ESP+4} ~--- \IFRU{хранится значение \TT{a}}{pointing to \TT{a} variable} 
\end{itemize}

\IFRU{С другой стороны, чтобы вызвать \printf нам нужна почти такая же разметка стека, 
только в первом аргументе нужен указатель на строку. Что, собственно, этот код и делает.}
{On the other side, when we need to call \printf here, we need exactly the same stack 
layout, except of first \printf argument pointing to string. 
And that is what our code does.}

\IFRU{Он заменяет свой первый аргумент на другой и затем передает управление \printf, как если бы вызвали не 
нашу функцию \TT{f()}, а сразу \printf. 
\printf выводит некую строку на \TT{stdout}, затем исполняет инструкцию \RET, 
которая из стека достает адрес возврата и управление передается в ту функцию, 
которая вызывала \TT{f()}, минуя при этом саму \TT{f()}.}
{It replaces function's first argument to different and 
jumping to \printf, as if not our function \TT{f()} was called firstly, but immediately \printf.
\printf printing some string to \TT{stdout} and then execute \RET instruction, which POPping 
return address from stack and control flow is returned not to \TT{f()}, but to \TT{f()}'s callee, 
escaping \TT{f()}.}

\newcommand{\URLSJ}{\url{http://en.wikipedia.org/wiki/Setjmp.h}}
\IFRU{Все это возможно потому что \printf вызывается в \TT{f()} в самом конце. 
Все это чем-то даже похоже на \TT{longjmp()}\footnote{\URLSJ}.
И все это, разумеется, сделано для экономии времени исполнения.}
{All it's possible because \printf is called right at the end of \TT{f()} in any case. 
In some way, it's all similar to \TT{longjmp()}\footnote{\URLSJ}. 
And of course, it's all done for the sake of speed.}

\subsection{\IFRU{И если много}{A lot of cases}}

\IFRU{А если вветвлений слишком много, то конечно генерировать слишком длинный код с многочисленными \JE/\JNE 
уже не так удобно.}
{If \TT{switch()} statement contain a lot of case's, it is not very handy for compiler to emit too large code
with a lot \JE/\JNE instructions.}

\lstinputlisting{switch/8_4.c}

\IFRU{Имеем в итоге}{We got} (MSVC 2010):

\lstinputlisting{switch/8_5_msvc.asm}

\IFRU{Здесь происходит следующее: в теле функции есть набор вызовов \printf с разными аргументами. 
Все они имеют, конечно же, адреса, а также внутренние символические метки, которые придумал им компилятор.
Все эти метки складываются во внутреннюю таблицу \TT{\$LN11@f}.}
{OK, what we see here is: there are a set of \printf calls with various arguments. 
All them has not only addresses in process memory, but also internal symbolic labels generated 
by compiler. 
All these labels are also places into \TT{\$LN11@f} internal table.}

\IFRU{В начале функции, если \TT{a} больше 4, то сразу происходит переход на метку \TT{\$LN1@f}, 
где вызывается \printf с аргументом \TT{'something unknown'}.}
{At the function beginning, if \TT{a} is greater than 4, control flow is passed to label 
\TT{\$LN1@f}, where \printf with argument \TT{'something unknown'} is called.}

\IFRU{А если \TT{a} меньше или равно 4, то это значение умножается на 4 и прибавляется адрес таблицы с переходами. 
Таким образом, получается адрес внутри таблицы, где лежит нужный адрес внутри тела функции. 
Например, возьмем \TT{a} равным 2. 2*4 = 8 (ведь все элементы таблицы это адреса внутри 32-битного процесса, 
таким образом, каждый элемент занимает 4 байта). 8 прибавить к \TT{\$LN11@f} ~--- это будет элемент таблицы,
где лежит \TT{\$LN4@f}. \JMP вытаскивает из таблицы адрес \TT{\$LN4@f} и делает безусловный переход туда.}
{And if \TT{a} value is less or equals to 4, let's multiply it by 4 and add \TT{\$LN1@f} 
table address. That is how address inside of table is constructed, pointing exactly to the 
element we need. For example, let's say \TT{a} is equal to 2. 2*4 = 8 (all table elements 
are addresses within 32-bit process, that is why all elements contain 4 bytes). 
Address of \TT{\$LN11@f} table + 8 = it will be table element where \TT{\$LN4@f} label is stored. 
\JMP fetch \TT{\$LN4@f} address from the table and jump to it.}

\IFRU{А там вызывается \printf с агрументом \TT{'two'}. 
Дословно, инструкция \TT{jmp DWORD PTR \$LN11@f[ecx*4]} 
означает \IT{перейти по DWORD, который лежит по адресу} \TT{\$LN11@f + ecx * 4}.}
{Then corresponding \printf is called with argument \TT{'two'}. 
Literally, \TT{jmp DWORD PTR \$LN11@f[ecx*4]} instruction mean 
\IT{jump to DWORD, which is stored at address} \TT{\$LN11@f + ecx * 4}.}

\TT{npad}~\ref{sec:npad} 
\IFRU{это команда ассемблеру немного выровнять начало таблицы, дабы она располагалась по 
адресу кратному 4 (или 16). Это нужно для того чтобы процессор мог эффективнее загружать 32-битное 
значения из памяти, через шину с памятью, кеш-память, итд.}
{is assembler macro, aligning next label so that it will be stored at address aligned by 4 bytes (or 16). 
This is very suitable for processor, because it can fetch 32-bit values from memory through memory bus, 
cache memory, etc, in much effective way if it is aligned.}

\IFRU{Посмотрим что сгенерирует GCC 4.4.1:}{Let's see what GCC 4.4.1 generate:}

\lstinputlisting{switch/8_6_gcc.asm}

\IFRU{Практически то же самое, за исключением мелкого нюанса: аргумент из \TT{arg\_0} умножается на 4 
при помощи сдвига влево на 2 бита (это почти то же самое что и умножение на 4)~\ref{SHR}. 
Затем адрес метки внутри функции берется из массива \TT{off\_804855C} и адресуется при помощи 
вычисленного индекса.}
{It is almost the same, except little nuance: argument \TT{arg\_0} is multiplied by 4 with by
shifting it to left by 2 bits (it is almost the same as multiplication by 4)~\ref{SHR}. 
Then label address is taken from \TT{off\_804855C} array, address calculated and stored into 
\EAX, then \TT{JMP EAX} do actual jump.}



% done

\section{\IFRU{Циклы}{Loops}}

\IFRU
{Для организации циклов, в архитектуре x86 есть старая инструкция \LOOP, 
она проверяет значение регистра \ECX и если оно не ноль, делает декремент \ECX и переход 
по метке указанной в операнде. 
Возможно, эта инструкция не слишком удобная, поэтому я не видел современных компиляторов, 
которые использовали бы её. Так что, если вы видите где-то \LOOP, то это, с большой вероятностью, 
написанный руками код на ассемблере.}
{There is a special \LOOP instruction in x86 instruction set, it is checking \ECX register value and 
if it is not zero, do \ECX decrement and pass control flow to label mentioned in \LOOP operand. 
Probably, this instruction is not very handy, so, I didn't ever see any modern compiler emitting it automatically. 
So, if you see the instruction somewhere in code, it is most likely this is hand-written piece of assembler code.}

\IFRU{Кстати, в качестве домашнего задания, вы можете попытаться объяснить, чем именно эта инструкция неудобна.}
{By the way, as home exercise, you could try to explain, why this instruction is not very handy.}

\IFRU{Циклы на \CCpp описываются при помощи \TT{for()}, \TT{while()}, \TT{do/while()}.}
{In \CCpp loops are defined by \TT{for()}, \TT{while()}, \TT{do/while()} statements.}

\IFRU{Начнем с}{Let's start with} \TT{for()}.

\IFRU{Это выражение описывает инициализацию, условие, что делать после каждой итерации (инкремент/декремент) 
и тело цикла.}
{This statement defines loop initialization (set loop counter to initial value), 
loop condition (is counter is bigger than a limit?), what is done at each iteration (increment/decrement) 
and of course loop body.}

\IFRU{\lstinputlisting{loops/loops_1_ru.c}}{\lstinputlisting{loops/loops_1_en.c}}

\IFRU{Примерно также, генерируемый код и будет состоять из этих четырех частей.}
{So, generated code will be consisted of four parts too.}

\IFRU{Возьмем пример:}{Let's start with simple example:}

\begin{lstlisting}
int main()
{
	int i;

	for (i=2; i<10; i++)
		f(i);

	return 0;
};
\end{lstlisting}

\IFRU{Имеем в итоге}{Result} (MSVC 2010):

\lstinputlisting{\IFRU{loops/9_1_ru_msvc.asm}{loops/9_1_en_msvc.asm}}

\IFRU{В принципе, ничего необычного.}{Nothing very special, as we see.}

\IFRU{GCC 4.4.1 выдает примерно такой же код, с небольшой разницей:}
{GCC 4.4.1 emitting almost the same code, with small difference:}

\lstinputlisting{loops/9_1_en_gcc.asm}

\IFRU{Интересно становится, если скомпилируем этот же код при помощи MSVC 2010 с включенной оптимизацией}
{Now let's see what we will get if optimization is turned on} (\Ox):

\lstinputlisting{loops/9_1_msvc_Ox.asm}

\IFRU{Здесь происходит следующее: переменную \IT{i} компилятор не выделяет в локальном стеке, 
а выделяет целый регистр под нее: \ESI.}
{What is going on here is: place for \IT{i} variable is not allocated in local stack anymore, 
but even individual register: \ESI. This is possible in such small functions where not so 
much local variables are used.}

\IFRU{В принципе, все то же самое, только теперь одна важная особенность: \TT{f()} не должна менять значение \ESI. 
Наш компилятор уверен в этом, а если бы и была необходимость использовать регистр \ESI в функции \TT{f()}, 
то её значение сохранялось бы в стеке. Примерно также как и в нашем листинге: 
обратите внимание на \TT{PUSH ESI/POP ESI} в начале и конце функции.}
{One very important property is that \TT{f()} function should not change \ESI value, in fact, it should 
not use that register at all. Our compiler is sure here. 
And if compiler decided to use \ESI in \TT{f()} too, it would be saved then at \TT{f()} function 
prologue and restored at \TT{f()} epilogue. Almost like in our listing: please note \TT{PUSH ESI/POP ESI}
at the function begin and end.}

\IFRU
{Попробуем GCC 4.4.1 с максимальной оптимизацией (\TT{-O3}):}
{Let's try GCC 4.4.1 with maximal optimization turned on (\TT{-O3} option):}

\lstinputlisting{loops/9_1_en_gcc_O3.asm}

\IFRU
{Однако, GCC просто \IT{развернул} цикл\footnote{loop unwinding в англоязчной литературе}.}
{Huh, GCC just unwind our loop.}

\IFRU{Делается это в тех случаях, когда итераций не слишком много, как в нашем примере, 
и можно немного сэкономить время, убрав все инструкции обеспечивающие цикл. 
В качестве обратной стороны медали, размер кода увеличился.}
{Loop unwinding has advantage in that cases when there are not so much iterations and 
we could economy some execution speed by removing all loop supporting instructions. 
On the other side, resulting code is obviously larger.}

\IFRU{OK, увеличим максимальное значение \IT{i} в цикле до 100 и попробуем снова. GCC выдаст подобное:}
{OK, let's increase maximal value of \IT{i} to 100 and try again. GCC resulting:}

\lstinputlisting{\IFRU{loops/9_2_gcc_ru.asm}{loops/9_2_gcc_en.asm}}

\IFRU{Это уже похоже на то что сделал MSVC 2010 в режиме оптимизации (\Ox). 
За исключением того, что под переменную \TT{i} будет выделен регистр \EBX.}
{It's quite similar to what MSVC 2010 with optimization (\Ox) produce. 
With the exception that \EBX register will be fixed to \TT{i} variable. }
\IFRU{GCC уверен что этот регистр не будет 
модифицироваться внутри \TT{f()}, а если вдруг это и прийдется там сделать, то его значение будет сохранено 
в начале функции, прямо как в \main здесь.}
{GCC is sure this register will not be modified inside of \TT{f()} function, 
and if it will, it will be saved at the function prologue and restored at epilogue, 
just like here in \main.}



% done

\section{strlen()}

\IFRU{Еще немного о циклах. Часто, функция \TT{strlen()}\footnote{подсчет длины строки в Си} 
реализуется при помощи \TT{while()}.}
{Now let's talk about loops one more time. Often, \TT{strlen()} 
function\footnote{counting characters in string in C language} is implemented using \TT{while()} statement.}
\IFRU{Например, как это сделано в стандартных библиотеках MSVC:}
{Here is how it done in MSVC standard libraries:}

\begin{lstlisting}
int strlen (const char * str)
{
        const char *eos = str;

        while( *eos++ ) ;

        return( eos - str - 1 );
}
\end{lstlisting}

\IFRU{Итак, компилируем:}{Let's compile:}

\lstinputlisting{\IFRU{strlen/10_1_msvc_ru.asm}{strlen/10_1_msvc_en.asm}}

\IFRU{Здесь две новых инструкции: \MOVSX и \TEST.}
{Two new instructions here: \MOVSX and \TEST.}

\label{MOVSX}
\IFRU{О первой: \MOVSX предназначен для того чтобы взять байт из какого-либо места в памяти и положить его, 
в нашем случае, в регистр \EDX. 
Но регистр \EDX ~--- 32-битный. \MOVSX означает \IT{MOV with Sign-Extent}. 
Оставшиеся биты с 8-го по 31-й \MOVSX сделает еденицей, если исходный байт в памяти имеет знак \IT{минус}, 
или заполнит нулями, если знак \IT{плюс}.}
{About first: \MOVSX is intended to take byte from some place and store value in 32-bit register. 
\MOVSX meaning \IT{MOV with Sign-Extent}. 
Rest bits starting at 8th till 31th \MOVSX will set to 1 if source byte in memory has \IT{minus} 
sign or to 0 if \IT{plus}.}

\IFRU{И вот зачем все это.}{And here is why all this.}

\IFRU{По стандарту \CCpp, тип \Tchar ~--- знаковый. Если у нас есть две переменные, одна \Tchar, а другая \Tint 
(\Tint тоже знаковый), и если в первой переменной лежит -2 (что кодируется как 0xFE) и мы просто 
переложим это в \Tint, 
то там будет 0x000000FE, а это, с точки зрения \Tint, даже знакового, будет 254, но никак не -2. 
-2 в переменной \Tint кодируется как 0xFFFFFFFE. И для того чтобы значение 0xFE из переменной типа 
\Tchar переложить 
в знаковый \Tint с сохранением всего, нужно узнать его знак, и затем заполнить остальные биты. 
Это делает \MOVSX.}
{\CCpp standard defines \Tchar type as signed. If we have two values, one is \Tchar 
and another is \Tint, (\Tint is signed too), and if first value contain -2 (it is coded as 0xFE) 
and we just copying this byte into \Tint container, there will be 0x000000FE, and this, 
from the point of signed \Tint view is 254, but not -2. In signed int, -2 is coded as 0xFFFFFFFE. 
So if we need to transfer 0xFE value from variable of \Tchar type to \Tint, 
we need to identify its sign and extend it. That is what \MOVSX does.}

\IFRU{См.также об этом раздел}
{See also more in section} \IT{\SignedNumbersSectionName}~\ref{sec:signednumbers}.

\IFRU{Хотя, конкретно здесь, компилятору врядли была особая надобность хранить значение \Tchar в регистре \EDX 
а не его восьмибитной части, скажем, \DL. Но получилось как получилось: должно быть, register allocator
компилятора сработал именно так.}
{I'm not sure if compiler need to store \Tchar variable in \EDX, it could take 8-bit register part 
(let's say \DL). But probably, compiler's register allocator works like that.}

\IFRU{Позже выполняется \TT{TEST EDX, EDX}. 
Об инструкции \TEST читайте в разделе о битовых полях~\ref{sec:bitfields}.
Но конкретно здесь, эта инструкция просто проверяет состояние регистра \EDX на 0.}
{Then we see \TT{TEST EDX, EDX}. 
About \TEST instruction, read more in section about bit fields~\ref{sec:bitfields}.
But here, this instruction just checking \EDX value, if it is equals to 0.}

\IFRU{Попробуем GCC 4.4.1:}{Let's try GCC 4.4.1:}

\lstinputlisting{strlen/10_3_gcc.asm}

\IFRU{Результат очень похож на MSVC, вот только здесь используется \MOVZX а не \MOVSX. 
\MOVZX означает \IT{MOV with Zero-Extent}. Эта инструкция перекладывает какое-либо значение 
в регистр и остальное добивает нулями. 
Фактически, преимущество этой инструкции только в том, что она позволяет 
заменить две инструкции сразу: \TT{xor eax, eax / mov al, [...]}.}
{The result almost the same as MSVC did, but here we see \MOVZX isntead of \MOVSX. 
\MOVZX mean \IT{MOV with Zero-Extent}. 
This instruction place 8-bit or 16-bit value into 32-bit register and set the rest bits to zero. 
In fact, this instruction is handy only because it is able to replace two instructions at once: 
\TT{xor eax, eax / mov al, [...]}.}

\IFRU{С другой стороны, нам очевидно, что здесь можно было бы написать вот так: 
\TT{mov al, byte ptr [eax] / test al, al} ~--- это тоже самое, хотя старшие биты \EAX будут "замусорены". 
Но, будем считать, что это погрешность компилятора ~--- он не смог сделать код более экономным или более понятным. 
Строго говоря, компилятор вообще не нацелен на то чтобы генерировать понятный (для человека) код.}
{On other hand, it is obvious to us that compiler could produce that code: 
\TT{mov al, byte ptr [eax] / test al, al} ~--- it is almost the same, however, 
highest \EAX register bits will contain random noise. 
But let's think it is compiler's drawback ~--- it can't produce more understandable code. 
Strictly speaking, compiler is not obliged to emit understandable (to humans) code at all.}

\IFRU{Следующая новая инструкция для нас ~--- \SETNZ. В данном случае, если в \AL был не ноль, 
то \TT{test al, al} выставит флаг \ZF в 0, а \SETNZ, если \TT{ZF==0} 
(\IT{NZ} значит \IT{not zero}) выставит еденицу в \AL. 
Смысл этой процедуры в том, что, если говорить человеческим языком, 
\IT{если AL не ноль, то выполнить переход на} \TT{loc\_80483F0}.
Компилятор выдал немного избыточный код, но не будем забывать что оптимизация выключена.}
{Next new instruction for us is \SETNZ. Here, if \AL contain not zero, \TT{test al, al} 
will set zero to \ZF flag, but \SETNZ, if \TT{ZF==0} (\IT{NZ} mean \IT{not zero}) will set 1 to \AL. 
Speaking in natural language, \IT{if AL is not zero, let's jump to loc\_80483F0}. 
Compiler emitted slightly redundant code, but let's not forget that optimization is turned off.}

\IFRU{Теперь скомпилируем все то же самое в MSVC 2010, но с включенной оптимизацией (\Ox)}
{Now let's compile all this in MSVC 2010, with optimization turned on (\Ox)}:

\lstinputlisting{strlen/10_2_en.asm}

\IFRU{Здесь все попроще стало. Но следует отметить, что компилятор обычно может так хорошо использовать регистры 
только на не очень больших функций с не очень большим количеством локальных переменных.}
{Now it's all simpler. But it is needless to say that compiler could use registers such efficiently 
only in small functions with small number of local variables.}

\INC/\DEC ~--- \IFRU{это инструкции инкремента-декремента, попросту говоря: 
увеличить на еденицу или уменьшить.}
{are increment/decrement instruction, in other words: add 1 to variable or subtract.}

\IFRU{Попробуем GCC 4.4.1 с влюченной оптимизацией (ключ \TT{-O3}):}
{Let's check GCC 4.4.1 with optimization turned on (\TT{-O3} key):}

\lstinputlisting{strlen/10_3_gcc_O3.asm}

\IFRU{Здесь GCC не очень отстает от MSVC за исключением наличия \MOVZX.} 
{Here GCC is almost the same as MSVC, except of \MOVZX presence.}

\IFRU
{Впрочем, только кроме того что почему-то используется \MOVZX, который явно можно заменить на}
{However, \MOVZX could be replaced here to} \TT{mov dl, byte ptr [eax]}.

\IFRU{Но, возможно, компилятору GCC просто проще помнить что у него под переменную типа \Tchar отведен целый 
32-битный регистр и быть уверенным в том что старшие биты регистра не будут замусорены.}
{Probably, it is simpler for GCC compiler's code generator to \IT{remember} that whole register 
is allocated for \Tchar variable and it can be sure that highest bits will not contain noise 
at any point.}

\IFRU{Далее мы видим новую для нас инструкцию \NOT. Эта инструкция инвертирует все биты в операнде. 
Можно сказать что здесь это синонимично инструкции \TT{XOR ECX, 0ffffffffh}. 
\NOT и следующая за ней инструкция \ADD вычисляют разницу указателей и отнимают от результата единицу. 
Только происходит это слегка по-другому. Сначала \ECX, где хранится указатель на \IT{str}, 
инвертируется и от него отнимается единица.}
{After, we also see new instruction \NOT. This instruction inverts all bits in operand. 
It can be said, it is synonym to \TT{XOR ECX, 0ffffffffh} instruction. 
\NOT and following \ADD calculating pointer difference and subtracting 1. 
At the beginning \ECX, where pointer to str is stored, inverted and 1 is subtracted from it.}

\IFRU{См. также раздел:}{See also:} \SignedNumbersSectionName~\ref{sec:signednumbers}.

\IFRU{Иными словами, в конце функции, после цикла, происходит примерно следующее:} 
{In other words, at the end of function, just after loop body, these operations are executed:}

\begin{lstlisting}
ecx=str;
eax=eos;
ecx=(-ecx)-1; 
eax=eax+ecx
return eax
\end{lstlisting}

\IFRU{... что эквивалентно:}{... and this is equivalent to:}

\begin{lstlisting}
ecx=str;
eax=eos;
eax=eax-ecx;
eax=eax-1;
return eax
\end{lstlisting}

\IFRU
{Но почему GCC решил что так будет лучше? Снова не берусь сказать. Но я не сомневаюсь, 
что эти оба варианта работают примерно равноценно в плане эффективности и скорости.}
{Why GCC decided it would be better? I cannot be sure. 
But I'm assure that both variants are equivalent in efficiency sense.}


% done

\section{\DivisionByNineSectionName}
\label{sec:divisionbynine}

\IFRU{Простая функция:}{Very simple function:}

\begin{lstlisting}
int f(int a)
{
	return a/9;
};
\end{lstlisting}

\IFRU{Компилируется вполне предсказуемо:}{Is compiled in a very predictable way:}

\lstinputlisting{\IFRU{division_by_9/11_1_msvc_ru.asm}{division_by_9/11_1_msvc_en.asm}}

\IFRU{\IDIV делит 64-битное число хранящееся в паре регистров \TT{EDX:EAX} на значение в \ECX. 
В результате, \EAX будет содержать частное\footnote{результат деления}, а \EDX ~--- остаток от деления. 
Результат возвращается из функции через \EAX, так что после операции деления, 
это значение не перекладывается больше никуда, 
оно уже там где надо.}
{\IDIV divides 64-bit number stored in \TT{EDX:EAX} register pair by value in \ECX register. 
As a result, \EAX will contain quotient\footnote{result of division}, and \EDX ~--- remainder.
Result is returning from f() function in \EAX register, so, that value is not moved anymore after division 
operation, it is in right place already.}
\IFRU
{Из-за того что \IDIV требует пару регистров \TT{EDX:EAX}, то перед этим инструкция \TT{CDQ} 
расширяет \EAX до 64-битного значения учитывая знак, также как это делает \MOVSX.}
{Because \IDIV require value in \TT{EDX:EAX} register pair, \TT{CDQ} instruction (before \IDIV) extending 
\EAX value to 64-bit value taking value sign into account, just as \MOVSX does.}
\IFRU{Со включенной оптимизацией (\Ox) получается:}
{If we turn optimization on (\Ox), we got:}

\lstinputlisting{division_by_9/11_1_msvc_Ox.asm}

\newcommand{\URLMSDN}{\href{http://blogs.msdn.com/b/devdev/archive/2005/12/12/502980.aspx}
{MSDN: Integer division by constants}}
\newcommand{\URLN}{http://www.nynaeve.net/?p=115}

\IFRU{Это ~--- деление через умножение. Умножение конечно быстрее работает. 
Поэтому можно используя этот трюк
\footnote{Подробнее о делении через умножение: \URLMSDN, \url{\URLN}} 
создать код эквивалентный тому что мы хотим и работающий быстрее.}
{This is ~--- division using multiplication. Multiplication operation working much faster. 
And it is possible to use that trick
\footnote{More about division by multiplication: \URLMSDN, \url{\URLN}} 
to produce a code which is equivalent and faster.}
\IFRU
{GCC 4.4.1 даже без включенной оптимизации генерит примерно такой же код как и MSVC с оптимизацией:}
{GCC 4.4.1 even without optimization turned on, generate almost the same code as MSVC with optimization turned on:}

\lstinputlisting{division_by_9/11_2_gcc.asm}



% done

\section{\IFRU{Работа с FPU}{Work with FPU}}
\label{sec:FPU}

\newcommand{\FNURLSTACK}{\footnote{\url{http://en.wikipedia.org/wiki/Stack_machine}}}
\newcommand{\FNURLFORTH}{\footnote{\url{http://en.wikipedia.org/wiki/Forth_(programming_language)}}}
\newcommand{\FNURLIEEE}{\footnote{\url{http://en.wikipedia.org/wiki/IEEE_754-2008}}}
\newcommand{\FNURLSP}{\footnote{\url{http://en.wikipedia.org/wiki/Single-precision_floating-point_format}}}
\newcommand{\FNURLDP}{\footnote{\url{http://en.wikipedia.org/wiki/Double-precision_floating-point_format}}}
\newcommand{\FNURLEP}{\footnote{\url{http://en.wikipedia.org/wiki/Extended_precision}}}

FPU (\IT{Floating-point unit}) ~--- \IFRU{девайс в процессоре работающий с числами с плавающей запятой.}
{is a device within main CPU specially designed to work with floating point numbers.}

\IFRU{Раньше он назывался сопроцессором. Он немного похож на программируемый калькулятор и 
стоит немного в стороне от основного процессора.}
{It was called coprocessor in past. It looks like programmable calculator in some way and 
stay aside of main processor.}

\IFRU{Перед изучением FPU полезно ознакомиться с тем как работают стековые машины\FNURLSTACK, 
или ознакомиться с основами языка Forth\FNURLFORTH.}
{It is worth to study stack machines\FNURLSTACK before FPU studying, or learn Forth language basics\FNURLFORTH.}

\IFRU{Интересен факт, что в свое время (до 80486) сопроцессор был отдельным чипом на материнской плате, 
и вследствии его высокой цены, он стоял не всегда. Его можно было докупить отдельно и поставить.}
{It is interesting to know that in past (before 80486 CPU) coprocessor was a separate chip 
and it was not always settled on motherboard. It was possible to buy it separately and install.}

\IFRU{Начиная с процессора 80486, FPU уже всегда входит в его состав.}
{From the 80486 CPU, FPU is always present in it.}

\IFRU{FPU имеет стек из восьми 80-битных регистров, каждый может содержать число в формате IEEE 754\FNURLIEEE.}
{FPU has a stack capable to hold 8 80-bit registers, each register can hold a number 
in IEEE 754\FNURLIEEE format.}

\IFRU{В \CCpp типы имеются два типа для работы с числами с плавающей запятой, 
это \Tfloat (\IT{число одинарной точности}\FNURLSP, 32 бита)
\footnote{Формат представления float-чисел затрагивается в разделе 
\IT{\WorkingWithFloatAsWithStructSubSubSectionName}~\ref{sec:floatasstruct}.}
и \Tdouble (\IT{число двойной точности}\FNURLDP, 64 бита).}
{\CCpp language offer at least two floating number types, \Tfloat (\IT{single-precision}\FNURLSP, 32 bits)
\footnote{single precision float numbers format is also addressed in 
\IT{\WorkingWithFloatAsWithStructSubSubSectionName}~\ref{sec:floatasstruct} section}
and \Tdouble (\IT{double-precision}\FNURLDP, 64 bits).}

\IFRU{GCC поддерживает тип \IT{long double} (\IT{extended precision}\FNURLEP, 80 бит), но MSVC ~--- нет.}
{GCC also supports \IT{long double} type (\IT{extended precision}\FNURLEP, 80 bit) but MSVC is not.}

\IFRU{Не смотря на то что \Tfloat занимает столько же места сколько \Tint на 32-битной архитектуре, 
представление чисел, разумеется, совершенно другое.}
{\Tfloat type require the same number of bits as \Tint type in 32-bit environment, 
but number representation is completely different.}

\IFRU{Число с плавающей точкой состоит из знака, мантиссы\footnote{\IT{significand} или \IT{fraction} 
в анлоязчной литературе} и экспоненты.}
{Number consisting of sign, significand (also called \IT{fraction}) and exponent.}

\IFRU{Функция, имеющая \Tfloat или \Tdouble среди аргументов, получает эти значения через стек. 
Если функция возвращает \Tfloat или \Tdouble, она оставляет значение в регистре \STZERO ~--- то есть, 
на вершине FPU-стека.}
{Function having \Tfloat or \Tdouble among argument list is getting that value via stack. 
If function is returning \Tfloat or \Tdouble value, it leave that value in \STZERO 
register ~--- at top of FPU stack.}

% done

\subsection{\IFRU{Простой пример}{Simple example}}

\IFRU{Рассмотрим простой пример:}{Let's consider simple example:}

\begin{lstlisting}
double f (double a, double b)
{
	return a/3.14 + b*4.1;
};
\end{lstlisting}

\IFRU{Компилим в}{Compile it in} MSVC 2010:

\lstinputlisting{\IFRU{FPU/12_1_ru.asm}{FPU/12_1_en.asm}}

\IFRU{\FLD берет 8 байт из стека и загружает из в регистр \STZERO, автоматически конвертируя во внутренний 
80-битный формат (\IT{extended precision}).}
{\FLD takes 8 bytes from stack and load the number into \STZERO register, automatically converting 
it into internal 80-bit format \IT{extended precision}).}

\IFRU{\FDIV делит содержимое регистра \STZERO на число лежащее по адресу \TT{\_\_real@40091eb851eb851f} ~--- 
там закодировано значение 3.14. Синтаксис ассемблера не поддерживает подобные числа, 
так что то что мы там видим, это шестандцатиричное представление числа \IT{3.14} в формате IEEE 754.}
{\FDIV divide value in register \STZERO by number storing at address \TT{\_\_real@40091eb851eb851f} ~--- 
3.14 value is coded there. Assembler syntax missing floating point numbers, so, 
what we see here is hexadecimal representation of \IT{3.14} number in 64-bit IEEE 754 encoded.}

\IFRU{После выполнения \FDIV, в \STZERO остается результат деления.}
{After \FDIV execution, \STZERO will hold quotient\footnote{division result}.}

\IFRU{Кстати, есть еще инструкция \FDIVP, которая делит \STONE на \STZERO, 
выталкивает эти числа из стека и заталкивает результат. 
Если вы знаете язык Forth\FNURLFORTH, то это как раз оно и есть ~--- стековая машина\FNURLSTACK.}
{By the way, there are also \FDIVP instruction, which divide \STONE by \STZERO, 
popping both these values from stack and then pushing result. 
If you know Forth language\FNURLFORTH, you will quickly understand that this is stack machine\FNURLSTACK.}

\IFRU{Следующая \FLD заталкивает в стек значение \IT{b}.}
{Next \FLD instruction pushing \IT{b} value into stack.}

\IFRU{После этого, в \STONE перемещается результат деления, а в \STZERO теперь будет \IT{b}.}
{After that, quotient is placed to \STONE, and \STZERO will hold \IT{b} value.}

\IFRU{Следующий \FMUL умножает \IT{b} из \STZERO на значение \TT{\_\_real@4010666666666666} ~--- 
там лежит число 4.1, и оставляет результат в \STZERO.}
{Next \FMUL instruction do multiplication: \IT{b} from \STZERO register by value at 
\TT{\_\_real@4010666666666666} (4.1 number is there) and leaves result in \STZERO.}

\IFRU{Самая последняя инструкция \FADDP складывает два значения из вершины стека, 
в \STONE и затем выталкивает значение лежащее в \STZERO, 
таким образом результат сложения остается на вершине стека в \STZERO.}
{Very last \FADDP instruction adds two values at top of stack, placing result at \STONE 
register and then popping value at \STONE, hereby leaving result at top of stack in \STZERO.}

\IFRU{Функция должна вернуть результат в \STZERO, так что больше ничего здесь не производится, 
кроме эпилога функции.}
{The function must return result in \STZERO register, so, after \FADDP there are no any 
other code except of function epilogue.}

\IFRU{GCC 4.4.1 (с опцией \TT{-O3}) генерирует похожий код, хотя и с некоторой разницей:}
{GCC 4.4.1 (with \TT{-O3} option) emitting the same code, however, slightly different:}

\lstinputlisting{\IFRU{FPU/12_2_ru.asm}{FPU/12_2_en.asm}}

\IFRU{Разница в том, что в стек сначала заталкивается 3.14 (в \STZERO), а затем значение 
из \TT{arg\_0} делится на то что лежит в регистре \STZERO.}
{The difference is that, first of all, 3.14 is pushing to stack (into \STZERO), and then value 
in \TT{arg\_0} is dividing by what is in \STZERO register.}

\IFRU{\FDIVR означает \IT{Reverse Divide} ~--- делить поменяв делитель и делимое местами. 
Точно такой же инструкции для умножения нет, потому она была бы бессмыслена (ведь умножение ~--- 
операция коммутативная), так что там остается только \FMUL без соответсвующей ей \TT{-R} инструкции.}
{\FDIVR meaning \IT{Reverse Divide} ~--- to divide with divisor and dividend swapped with each other. 
There are no such instruction for multiplication, because multiplication is 
commutative operation, so we have just \FMUL without its \TT{-R} counterpart.}

\IFRU{\FADDP не только складывает два значения, но также и выталкивает из стека одно значение. 
После этого, в \STZERO остается только результат сложения.}
{\FADDP adding two values but also popping one value from stack. 
After that operation, \STZERO will contain sum.}

\IFRU{Этот кусок кода получен при помощи \IDA, которая регистр \STZERO называет для краткости просто \TT{ST}.}
{This piece of disassembled code was produced using \IDA which named \STZERO register as \TT{ST} for short.}



% done

\subsection{\IFRU{Передача чисел с плавающей запятой в аргументах}{Passing floating point number via arguments}}

\begin{lstlisting}
int main ()
{
	printf ("32.01 ^ 1.54 = %lf\n", pow (32.01,1.54));

	return 0;
}
\end{lstlisting}

\IFRU{Посмотрим что у нас вышло}{Let's see what we got in} (MSVC 2010):

\lstinputlisting{\IFRU{FPU/12_3_ru.asm}{FPU/12_3_en.asm}}

\IFRU{\FLD и \FSTP перемешают переменные из/в сегмента данных в FPU-стек. 
\TT{pow()}\footnote{стандартная функция Си, возводящая число в степень} достает оба значения из FPU-стека и 
возвращает результат в \STZERO. 
\printf берет 8 байт из стека и трактует их как переменную типа \Tdouble.}
{\FLD and \FSTP are moving variables from/to data segment to FPU stack. 
\TT{pow()}\footnote{standard C function, raises a number to the given power} taking both values from FPU-stack and 
returning result in \STZERO. 
\printf takes 8 bytes from local stack and interpret them as \Tdouble type variable.}


% done

\subsection{\IFRU{Пример с сравнением}{Comparison example}}

\IFRU{Попробуем теперь вот это:}{Let's try this:}

\begin{lstlisting}
double d_max (double a, double b)
{
	if (a>b)
		return a;

	return b;
};
\end{lstlisting}

\IFRU{Несмотря на кажущуюся простоту этой функции, понять как она работает будет чуть сложнее.}
{Despite simplicity of that function, it will be harder to understand how it works.}

\IFRU{Вот что выдал MSVC 2010:}{MSVC 2010 generated:}

\lstinputlisting{\IFRU{FPU/12_4_ru.asm}{FPU/12_4_en.asm}}

\IFRU{Итак, \FLD загружает \TT{\_b} в регистр \STZERO.}
{So, \FLD loading \TT{\_b} into \STZERO register.}

\newcommand{\Czero}{\TT{C0}\xspace}
\newcommand{\Ctwo}{\TT{C2}\xspace}
\newcommand{\Cthree}{\TT{C3}\xspace}
\newcommand{\CThreeBits}{\Cthree/\Ctwo/\Czero}

\IFRU{\FCOMP сравнивает содержимое \STZERO с тем что лежит в \TT{\_a} и выставляет биты \CThreeBits в 
регистре статуса FPU. Это 16-битный регистр отражающий текущее состояние FPU.} 
{\FCOMP compares \STZERO register state with what is in \TT{\_a} value and set \CThreeBits bits in FPU 
status word register. This is 16-bit register reflecting current state of FPU.} 

\IFRU{Итак, биты \CThreeBits выставлены, но, к сожалению, у процессоров до Intel P6 
\footnote{Intel P6 это Pentium Pro, Pentium II, и далее} нет инструкций условного перехода,
проверяющих эти биты. 
Возможно это исторически так сложилось (вспомните о том что FPU когда-то был вообще отдельным чипом). 
А у Intel P6 появились инструкции \FCOMI/\FCOMIP/\FUCOMI/\FUCOMIP ~--- делающие тоже самое, 
только напрямую модифицирующие флаги \ZF/\PF/\CF.}
{For now \CThreeBits bits are set, but unfortunately, CPU before Intel P6
\footnote{Intel P6 is Pentium Pro, Pentium II, etc} have not any conditional 
jumps instructions which are checking these bits. 
Probably, it is a matter of history (remember: FPU was separate chip in past). 
Modern CPU starting at Intel P6 has \FCOMI/\FCOMIP/\FUCOMI/\FUCOMIP instructions ~--- 
which does that same, but modifies CPU flags \ZF/\PF/\CF.}

\IFRU{После этого, инструкция \FCOMP выдергивает одно значение из стека. 
Это отличает её от \FCOM, которая просто сравнивает значения, оставляя стек в таком же состоянии.}
{After bits are set, the \FCOMP instruction popping one variable from stack. 
This is what differentiate if from \FCOM, which is just comparing values, leaving the stack at the same state.}

\IFRU{\FNSTSW копирует содержимое регистра статуса в \AX. Биты \CThreeBits занимают позиции, 
соответственно, 14, 10, 8, в этих позициях они и остаются в регистре \AX, 
и все они расположены в старшей части регистра ~--- \AH.}
{\FNSTSW copies FPU status word register to \AX. Bits \CThreeBits are placed at positions 14/10/8, 
they will be at the same positions in \AX registers and all them are placed in high part of \AX ~--- \AH.}

\begin{itemize}
\item
\IFRU{Если b>a в нашем случае, то биты \CThreeBits должны быть выставлены так:}
{If b>a in our example, then \CThreeBits bits will be set as following:} 0, 0, 0.
\item
\IFRU{Если a>b, то биты будут выставлены:}{If a>b, then bits will be set:} 0, 0, 1.
\item
\IFRU{Если a=b, то биты будут выставлены так:}{If a=b, then bits will be set:} 1, 0, 0.
\end{itemize}
% TODO: table here?

\IFRU{После исполнения \TT{test ah, 5}, бит \Cthree и \TT{C1} сбросится в ноль, 
на позициях 0 и 2 (внутри регистра \AH) 
останутся соответственно \Czero и \Ctwo.}
{After \TT{test ah, 5} execution, bits \Cthree and \TT{C1} will be set to 0, 
but at positions 0 и 2 (in \AH registers) 
\Czero and \Ctwo bits will be leaved.}

\IFRU{Теперь немного о \IT{parity flag}\footnote{флаг четности}. Еще один замечательный рудимент:} 
{Now let's talk about parity flag. Yet another notable epoch rudiment.}

\begin{framed}
\begin{quotation}
One common reason to test the parity flag actually has nothing to do with parity. The FPU has four condition flags 
(C0 to C3), but they can not be tested directly, and must instead be first copied to the flags register. 
When this happens, C0 is placed in the carry flag, C2 in the parity flag and C3 in the zero flag. 
The C2 flag is set when e.g. incomparable floating point values (NaN or unsupported format) are compared 
with the FUCOM instructions.\footnote{\href{http://en.wikipedia.org/wiki/Parity_flag}{wikipedia: parity flag}}
\end{quotation}
\end{framed}

\IFRU
{Этот флаг выставляется в 1 если количество едениц в последнем результате ~--- четно. И в ноль если ~--- нечетно.}
{This flag is to be set to 1 if ones number is even. And to zero if odd.}

\IFRU{Таким образом, что мы имеем, флаг \PF будет выставлен в еденицу, если \Czero и \Ctwo 
оба нули или оба единицы. 
И тогда сработает последующий \JP (\IT{jump if PF==1}). 
Если мы вернемся чуть назад и посмотрим значения \CThreeBits 
для разных вариантов, то увидим, что условный переход \JP сработает в двух случаях: если b>a или если a==b 
(ведь бит \Cthree уже \IT{вылетел} после исполнения \TT{test ah, 5}).}
{Thus, \PF flag will be set to 1 if both \Czero and \Ctwo are set to zero or both are ones. 
And then following \JP (\IT{jump if PF==1}) will be triggered. 
If we remembered values of \CThreeBits for different cases, we will see that conditional jump 
\JP will be triggered in two cases: if b>a or a==b 
(\Cthree bit is already not considering here, because it was cleared while execution of 
\TT{test ah, 5} instruction).}

\IFRU{Дальше все просто. Если условный переход сработал, то \FLD загрузит значение \TT{\_b} в \STZERO, 
а если не сработал, то загрузится \TT{\_a} и произойдет выход из функции.}
{It is all simple thereafter. If conditional jump was triggered, \FLD will load \TT{\_b} value 
to \STZERO, and if it's not triggered, \TT{\_a} will be loaded.}

\IFRU{Но это еще не все!}{But it is not over yet!}

\subsubsection{\IFRU{А теперь скомпилим все это в MSVC 2010 с опцией \Ox}
{Now let's compile it with MSVC 2010 with optimization option \Ox}}

\lstinputlisting{\IFRU{FPU/12_5_ru.asm}{FPU/12_5_en.asm}}

\IFRU{\FCOM отличается от \FCOMP тем что просто сравнивает значения и оставляет стек в том же состоянии. 
В отличие от предыдущего примера, операнды здесь в другом порядке. 
Поэтому и результат сравнения в \CThreeBits будет другим чем раньше:}
{\FCOM is different from \FCOMP is that sense that it just comparing values and leave FPU stack 
in the same state. 
Unlike previous example, operands here in reversed order. 
And that is why result of comparision in \CThreeBits will be different:}

\begin{itemize}
\item
\IFRU{Если a>b в нашем случае, то биты \CThreeBits должны быть выставлены так:}
{If a>b in our example, then \CThreeBits bits will be set as:} 0, 0, 0.
\item
\IFRU{Если b>a, то биты будут выставлены:}{If b>a, then bits will be set as:} 0, 0, 1.
\item
\IFRU{Если a=b, то биты будут выставлены так:}{If a=b, then bits will be set as:} 1, 0, 0.
\end{itemize}
% TODO: table?

\IFRU{Инструкция \TT{test ah, 65} как бы оставляет только два бита ~--- \Cthree и \Czero. 
Они оба будут нулями, если a>b: в таком случае переход \JNE не сработает. 
Далее имеется инструкция \TT{FSTP ST(1)} ~--- эта инструкция копирует 
значение \STZERO в указанный операнд и выдергивает одно значение из стека. В данном случае, 
она копирует \STZERO 
(где сейчас лежит \TT{\_a}) в \STONE. 
После этого на вершине стека два раза лежат \TT{\_a}. Затем одно значение выдергивается. 
После этого в \STZERO остается \TT{\_a} и функция завершается.}
{It can be said, \TT{test ah, 65} instruction just leave two bits ~--- \Cthree и \Czero. 
Both will be zeroes if a>b: in that case \JNE jump will not be triggered. 
Then \TT{FSTP ST(1)} is following ~--- this instruction copies \STZERO value into operand and 
popping one value from FPU stack. 
In other words, that instruction copies \STZERO (where \TT{\_a} value now) into \STONE. 
After that, two values of {\_a} are at the top of stack now. 
After that, one value is popping.
After that, \STZERO will contain {\_a} and function is finished.}

\IFRU{Условный переход \JNE сработает в двух других случаях: если b>a или a==b. 
\STZERO скопируется в \STZERO, что как бы холостая операция, 
затем одно значение из стека вылетит и на вершине стека останется то что 
до этого лежало в \STONE (то есть, \TT{\_b}). И функция завершится. 
Эта инструкция используется здесь видимо потому что в FPU нет инструкции которая просто выдергивает 
значение из стека и больше ничего.}
{Conditional jump \JNE is triggered in two cases: of b>a or a==b. 
\STZERO into \STZERO will be copied, it is just like idle (\NOP) operation, then one value 
is popping from stack and top of stack (\STZERO) will contain what was in \STONE before 
(that is {\_b}). Then function finishes. 
That instruction used here probably because FPU has no instruction to pop value from stack and 
not to store it anywhere.}

\IFRU{Но и это еще не все.}{Well, but it is still not over.}

\subsubsection{GCC 4.4.1}

\lstinputlisting{\IFRU{FPU/12_6_ru.asm}{FPU/12_6_en.asm}}

\IFRU{\FUCOMPP ~--- это почти то же что и \FCOM, только выкидывает из стека оба значения после сравнения, 
а также несколько иначе реагирует на "не-числа".}
{\FUCOMPP ~--- is almost like \FCOM, but popping both values from stack and handling 
"not-a-numbers" differently.}

\IFRU{Немного о \IT{не-числах}}{More about \IT{not-a-numbers}}:

\newcommand{\NANFN}{\IFRU{\footnote{\url{http://ru.wikipedia.org/wiki/NaN}}}
{\footnote{\url{http://en.wikipedia.org/wiki/NaN}}}}

\IFRU
{FPU умеет работать со специальными переменными, которые числами не являются и называются "не числа" или 
NaN\NANFN{}. 
Это бесконечность, результат деления на ноль, и так далее. Нечисла бывают "тихие" и "сигнализирующие". 
С первыми можно продолжать работать и далее, а вот если вы попытаетесь совершить какую-то операцию 
с сигнализирующим нечислом, то сработает исключение.}
{FPU is able to work with special values which are \IT{not-a-numbers} or 
NaNs\NANFN{}. 
These are infinity, result of dividing by zero, etc. 
Not-a-numbers can be "quiet" and "signalling". It is possible to continue to work with "quiet" NaNs, 
but if one try to do some operation with "signalling" NaNs ~--- an exception will be raised.}

\IFRU{Так вот, \FCOM вызовет исключение если любой из операндов ~--- какое-либо нечисло.
\FUCOM же вызовет исключение только если один из операндов именно "сигнализирующее нечисло".}
{\FCOM will raise exception if any operand ~--- NaN. 
\FUCOM will raise exception only if any operand ~--- signalling NaN (SNAN).}

\IFRU{Далее мы видим \SAHF ~--- это довольно редкая инструкция в коде не использущим FPU. 
8 бит из \AH перекладываются в младшие 8 бит регистра статуса процессора в таком порядке: 
\TT{SF:ZF:-:AF:-:PF:-:CF <- AH}.}
{The following instruction is \SAHF ~--- this is rare instruction in the code which is not use FPU. 
8 bits from AH is movinto into lower 8 bits of CPU flags in the following order: 
\TT{SF:ZF:-:AF:-:PF:-:CF <- AH}.}

\IFRU{Вспомним, что \FNSTSW перегружает интересующие нас биты \CThreeBits в \AH, 
и соответственно они будут в позициях 6, 2, 0 в регистре \AH.}
{Let's remember that \FNSTSW is moving interesting for us bits \CThreeBits into \AH 
and they will be in positions 6, 2, 0 in \AH register.}

\IFRU{Иными словами, пара инструкций \TT{fnstsw  ax / sahf} перекладывает биты \CThreeBits в флаги \ZF, \PF, \CF.}
{In other words, \TT{fnstsw  ax / sahf} instruction pair is moving \CThreeBits into \ZF, \PF, \CF CPU flags.}

\IFRU{Теперь снова вспомним, какие значения бит \CThreeBits будут при каких результатах сравнения:}
{Now let's also remember, what values of \CThreeBits bits will be set:}

\begin{itemize}
\item
\IFRU{Если a больше b в нашем случае, то биты \CThreeBits должны быть выставлены так:}
{If a is greater than b in our example, then \CThreeBits bits will be set as:} 0, 0, 0.
\item
\IFRU{Если a меньше b, то биты будут выставлены:}{if a is less than b, then bits will be set as:} 0, 0, 1.
\item
\IFRU{Если a=b, то биты будут выставлены так:}{If a=b, then bits will be set:} 1, 0, 0.
\end{itemize}
% TODO: table?

\IFRU{Иными словами, после инструкций \FUCOMPP/\FNSTSW/\SAHF, мы получим такое состояние флагов:}
{In other words, after \FUCOMPP/\FNSTSW/\SAHF instructions, we will have these CPU flags states:}

\begin{itemize}
\item
\IFRU{Если a>b в нашем случае, то флаги будут выставлены так:}
{If a>b, CPU flags will be set as:} \TT{ZF=0, PF=0, CF=0}.
\item
\IFRU{Если a<b, то флаги будут выставлены:}{If a<b, then CPU flags will be set as:} \TT{ZF=0, PF=0, CF=1}.
\item
\IFRU{Если a=b, то флаги будут выставлены так:}{If a=b, then CPU flags will be set as:} \TT{ZF=1, PF=0, CF=0}.
\end{itemize}
% TODO: table?

\IFRU{Инструкция \SETNBE выставит в \AL единицу или ноль, в зависимости от флагов и условий. 
Это почти аналог \JNBE, за тем лишь исключением, что \SETcc
\footnote{\IT{cc} это \IT{condition code}}
выставляет 1 или 0 в \AL, а \Jcc делает переход или нет. 
\SETNBE запишет 1 если только \TT{CF=0} и \TT{ZF=0}. Если это не так, то запишет 0 в \AL.}
{How \SETNBE instruction will store 1 or 0 to AL: it is depends of CPU flags. 
It is almost \JNBE instruction counterpart, with exception that что \SETcc 
\footnote{\IT{cc} is \IT{condition code}} is storing 1 or 0 to \AL, but \Jcc do actual jump or not. 
\SETNBE store 1 only if \TT{CF=0} and \TT{ZF=0}. If it is not true, zero will be stored into \AL.}

\IFRU{\CF будет 0 и \ZF будет 0 одновременно только в одном случае: если a>b.}
{Both \CF is 0 and \ZF is 0 simultaneously only in one case: if a>b.}

\IFRU{Тогда в \AL будет записана единица, последующий условный переход \JZ взят не будет, 
и функция вернет \TT{\_a}. 
В остальных случаях, функция вернет \TT{\_b}.}
{Then one will be stored to \AL and the following \JZ will not be triggered and function will 
return {\_a}. On all other cases, {\_b} will be returned.}

\IFRU{Но и это еще не конец.}{But it is still not over.}

\subsubsection{GCC 4.4.1 \IFRU{с оптимизацией \TT{-O3}}{with \TT{-O3} optimization turned on}}

\lstinputlisting{\IFRU{FPU/12_7_ru.asm}{FPU/12_7_en.asm}}

\IFRU{Почти все что здесь есть уже описано мною, кроме одного: использование \JA после \SAHF. 
Действительно, инструкции условных переходов "больше", "меньше", "равно" для сравнения беззнаковых чисел 
(\JA, \JAE, \JB, \JBE, \JE/\JZ, \JNA, \JNAE, \JNB, \JNBE, \JNE/\JNZ) проверяют только флаги \CF и \ZF. 
И биты \CThreeBits после сравнения перекладываются в эти флаги аккурат так, 
чтобы перечисленные инструкции переходов могли работать. \JA сработает если \CF и \ZF обнулены.}
{It is almost the same except one: \JA usage instead of \SAHF. 
Actually, conditional jump instructions checking "larger", "lesser" or "equal" for unsigned number comparison 
(\JA, \JAE, \JB, \JBE, \JE/\JZ, \JNA, \JNAE, \JNB, \JNBE, \JNE/\JNZ) are checking only \CF and \ZF flags. 
And \CThreeBits bits after comparison are moving into these flags exactly in the same fashion 
so conditional jumps will work here. \JA will work if both \CF are \ZF zero.}

\IFRU{Таким образом, перечисленные инструкции условного перехода можно использовать после инструкций \FNSTSW/\SAHF.}
{Thereby, conditional jumps instructions listed here can be used after \FNSTSW/\SAHF instructions pair.}

\IFRU{Вполне возможно что биты статуса FPU \CThreeBits преднамерено были размещены таким образом, 
чтобы переноситься на базовые флаги процессора без перестановок.}
{It seems, FPU \CThreeBits status bits was placed there deliberately so to map them to base CPU flags 
without additional permutations.}



% done

\section{\IFRU{Массивы}{Arrays}}
\label{arrays}

\IFRU{Массив это просто набор переменных в памяти, обязательно лежащих рядом, и обязательно одного типа.}
{Array is just a set of variables in memory, always lying next to each other, always has same type.}

\begin{lstlisting}
#include <stdio.h>

int main() 
{
	int a[20];
	int i;

	for (i=0; i<20; i++)
		a[i]=i*2;

	for (i=0; i<20; i++)
		printf ("a[%d]=%d\n", i, a[i]);

	return 0;
};
\end{lstlisting}

\IFRU{Компилируем:}{Let's compile:}

\lstinputlisting{arrays/13_2_msvc.asm}

\IFRU{Однако, ничего особенного, просто два цикла, один заполняет цикл, второй печатает его содержимое. 
Команда \TT{shl ecx, 1} используется для умножения \ECX на 2, об этом ниже~\ref{SHR}.}
{Nothing very special, just two loops: first is filling loop and second is printing loop.
\TT{shl ecx, 1} instruction is used for \ECX value multiplication by 2, more about below~\ref{SHR}.}

\IFRU{Под массив выделено в стеке 80 байт, это 20 элементов по 4 байта.}
{80 bytes are allocated in stack for array, that's 20 elements of 4 bytes.}

\IFRU{То что делает GCC 4.4.1:}{Here is what GCC 4.4.1 does:}

\lstinputlisting{arrays/13_2_gcc.asm}

\subsection{\IFRU{Переполнение буфера}{Buffer overflow}}
\label{subsec:bufferoverflow}

\IFRU{Итак, индексация массива это просто \IT{массив\lbrack{}индекс\rbrack}.  % TODO как-то плохо отображаются []
Если вы присмотритесь к коду, в цикле печати значений массива через \printf вы 
не увидите проверок индекса, \IT{меньше ли он двадцати?} 
А что будет если он будет больше двадцати? 
Эта одна из особенностей \CCpp, за которую их, собственно, и ругают.}
{So, array indexing is just \IT{array\lbrack{}index\rbrack}.
If you study generated code closely, you'll probably note missing index bounds checking,
which could check index, \IT{if it is less than 20}.
What if index will be greater than 20?
That's the one \CCpp feature it's often blamed for.}

\IFRU{Вот код который и компилится и работает:}
{Here is a code successfully compiling and working:}

\begin{lstlisting}
#include <stdio.h>

int main() 
{
	int a[20];
	int i;

	for (i=0; i<20; i++)
		a[i]=i*2;

	printf ("a[100]=%d\n", a[100]);

	return 0;
};
\end{lstlisting}

\IFRU{Вот в это}{Compilation results} (MSVC 2010):

\lstinputlisting{arrays/13_3_msvc.asm}

\IFRU{У меня оно при запуске выдало вот это:}{I'm running it, and I got:}

\begin{lstlisting}
a[100]=760826203
\end{lstlisting}

\IFRU{Это просто \IT{что-то}, что волею случая лежало в стеке рядом с массивом, 
через 400 байт от его первого элемента.}
{It is just \IT{something}, lying in the stack near to array, 400 bytes from its first element.}

\IFRU{Действительно, а как могло бы быть иначе? Компилятор мог бы встроить какой-то код, 
каждый раз проверяющий индекс на соответствие пределам массива, как в языках программирования 
более высокого уровня\footnote{Java, Python, итд}, что делало бы запускаемый код медленнее.}
{Indeed, how it could be done differently? Compiler may incorporate some code, checking index value to be always
in array's bound, like in higher-level programming languages\footnote{Java, Python, etc}, 
but this makes running code slower.}

\IFRU{Итак, мы прочитали какое-то число из стека явно \IT{нелегально}, а что если мы запишем?}
{OK, we read some values in stack \IT{illegally}, but what if we could write something to it?}

\IFRU{Вот что мы пишем:}{Here is what we will write:}

\begin{lstlisting}
#include <stdio.h>

int main() 
{
	int a[20];
	int i;

	for (i=0; i<30; i++)
		a[i]=i;

	return 0;
};
\end{lstlisting}

\IFRU{И вот что имеем на ассемблере:}{And what we've got:}

\lstinputlisting{\IFRU{arrays/13_1_ru.asm}{arrays/13_1_en.asm}}

\IFRU{Запускаете скомпилированную программу, и она падает. Немудрено. Но давайте теперь узнаем, где именно.}
{Run compiled program and its crashing. No wonder. Let's see, where exactly it's crashing.}

\IFRU{Отладчик я уже давно не использую, так как надоело для всяких мелких задач вроде подсмотреть состояние 
регистров, запускать что-то, двигать мышью, итд. 
Поэтому я написал очень минималистическую утилиту для себя, \IT{tracer}~\ref{tracer}, коей обхожусь.}
{I'm not using debugger anymore, because I tired to run it each time, move mouse, etc, when I need just to
spot some register's state at specific point.
That's why I wrote very minimalistic tool for myself, \IT{tracer}~\ref{tracer}, which is enough for my tasks.}

\IFRU{Помимо всего прочего, я могу использовать мою утилиту просто чтобы посмотреть 
где и какое исключение произошло. 
Итак, пробую:}
{I can also use it just to see, where debuggee is crashed.
So let's see:}

\begin{lstlisting}
generic tracer 0.4 (WIN32), http://conus.info/gt

New process: C:\PRJ\...\1.exe, PID=7988
EXCEPTION_ACCESS_VIOLATION: 0x15 (<symbol (0x15) is in unknown module>), ExceptionInformation[0]=8
EAX=0x00000000 EBX=0x7EFDE000 ECX=0x0000001D EDX=0x0000001D
ESI=0x00000000 EDI=0x00000000 EBP=0x00000014 ESP=0x0018FF48
EIP=0x00000015
FLAGS=PF ZF IF RF
PID=7988|Process exit, return code -1073740791
\end{lstlisting}

\IFRU{Итак, следите внимательно за регистрами.}
{So, please keep an eye on registers.}

\IFRU{Исключение произошло по адресу 0x15. Это явно нелегальный адрес для кода ~--- по крайней мере, win32-кода! 
Мы там как-то очутились, причем, сами того не хотели. Интересен также тот факт что в \EBP хранится 0x14, 
а в \ECX и \EDX ~--- 0x1D.}
{Exception occured at address 0x15. It's not legal address for code ~--- at least for win32 code!
We trapped there somehow against our will. It's also interesting fact that \EBP register contain 0x14,
\ECX and \EDX ~--- 0x1D.}

\IFRU{И еще немного изучим разметку стека.}{Let's study stack layout more.}

\IFRU{После того как управление передалось в \TT{\_main}, в стек было сохранено значение \EBP. 
Затем, для массива + переменной \IT{i} было выделено 84 байта. Это \TT{(20+1)*sizeof(int)}. 
\ESP сейчас указывает на переменную \TT{\_i} в локальном стеке и при исполнении следующего \TT{PUSH что-либо}, 
\IT{что-либо} появится рядом с \TT{\_i}.}
{After control flow was passed into \TT{\main}, \EBP register value was saved into stack.
Then, 84 bytes was allocated for array and \IT{i} variable. That's \TT{(20+1)*sizeof(int)}.
\ESP pointing now to \TT{\_i} variable in local stack and after execution of next \TT{PUSH something},
\IT{something} will be appeared next to \TT{\_i}.}

\IFRU{Вот так выглядит разметка стека пока управление находится внутри}
{That's stack layout while control is inside} \TT{\_main}:

\begin{center}
\begin{tabular}{ | l | l | }
\hline
  \TT{ESP}    & \IFRU{4 байта для \IT{i}}{4 bytes for \IT{i}} \\
  \TT{ESP+4}  & \IFRU{80 байт для массива \TT{a[20]}}{80 bytes for \TT{a[20]} array} \\
  \TT{ESP+84} & \IFRU{сохраненное значение \EBP}{saved \EBP value} \\
  \TT{ESP+88} & \IFRU{адрес возврата}{returning address} \\
\hline
\end{tabular}
\end{center}

\IFRU{Команда \TT{a[19]=чего\_нибудь} записывает последний \Tint в пределах массива (пока что в пределах!)}
{Instruction \TT{a[19]=something} writes last \Tint in array bounds (in bounds yet!)}

\IFRU{Команда \TT{a[20]=чего\_нибудь} записывает \IT{чего\_нибудь} на место где сохранено значение \EBP.}
{Instruction \TT{a[20]=something} writes \IT{something} to the place where \EBP value is saved.}

\IFRU{Обратите внимание на состояние регистров на момент падения процесса. В нашем случае, 
в 20-й элемент записалось значение 20. 
И вот все дело в том, что заканчиваясь, эпилог функции восстанавливал значение \EBP. 
(20 в десятичной системе это как раз 0x14 в шестнадцетиричной). 
Далее выполнилась инструкция \RET, которая на самом деле эквивалентна \TT{POP EIP}.}
{Please take a look at registers state at the crash moment. In our case,
number 20 was written to 20th element. 
By the function ending, function epilogue restore \EBP value.
(20 in decimal system is 0x14 in hexadecimal).
Then, \RET instruction was executed, which is equivalent to \TT{POP EIP} instruction.}

\IFRU{Инструкция \RET вытащила из стека адрес возврата (это адрес в какой-то CRT\footnote{C Run-Time}-функции, 
которая вызвала \TT{\_main}), 
а там было записано 21 в десятичной системе, то есть 0x15 в шестнадцетиричной. 
И вот процессор оказался по адресу 0x15, но исполняемого кода там нет, так что случилось исключение.}
{\RET instruction taking returning adddress from stack (that's address in some CRT\footnote{C Run-Time}-function,
which was called \TT{\_main}),
and 21 was stored there (0x15 in hexadecimal).
The CPU trapped at the address 0x15, but there are no executable code, so exception was raised.}

\IFRU{Добро пожаловать! Это называется}
{Welcome! It's called} \IT{buffer overflow}\footnote{\url{http://en.wikipedia.org/wiki/Stack_buffer_overflow}}.

\newcommand{\URLPHRACK}{\href{http://www.phrack.com/issues.html?issue=49&id=14}
{Smashing The Stack For Fun And Profit}}

\IFRU{Замените массив \Tint на строку (массив \Tchar), нарочно создайте слишком длинную строку, 
просуньте её в ту программу, 
в ту функцию, которая не проверяя длину строки скопирует её в слишком короткий буфер, 
и вы сможете указать программе, по какому именно адресу перейти. 
Не все так просто в реальности, конечно, но началось все с этого\footnote{Классическая статья об этом: \URLPHRACK}.}
{Replace \Tint array by string (\Tchar array), create a long string deliberately,
pass it to the program, to the function which is not checking string length and copies it to short buffer,
and you'll able to point to a program an address to which it should jump.
Not that simple in reality, but that's how it was started\footnote{Classic article about it: \URLPHRACK}.}

\newcommand{\URLWPB}{\href{http://en.wikipedia.org/wiki/Buffer_overflow_protection}
{Wikipedia: \IFRU{описания защит, которые компилятор может вставлять в код}
{compiler-side buffer overflow protection methods}}}

\IFRU{В наше время пытаются бороться с этой напастью, не взирая на халатность программистов на \CCpp. 
В MSVC есть опции вроде\footnote{\URLWPB}:}
{There are several methods to protect against it, regardless of \CCpp programmers' negligence.
MSVC has options like\footnote{\URLWPB}:}

\begin{verbatim}
 /RTCs Stack Frame runtime checking
 /GZ Enable stack checks (/RTCs)
\end{verbatim}

\IFRU{Один из методов, это в прологе функции вставлять в область локальных переменных 
некоторое случайное значение 
и в эпилоге функции, перед выходом, это число проверять. 
И если проверка не прошла, то не выполнять инструкцию \RET а остановиться (или зависнуть). 
Процесс зависнет, но это лучше чем удаленная атака на ваш хост.}
{One of the methods is to write random value among local variables to stack at function prologue 
and to check it in function epilogue before function exiting.
And if value is not the same, do not execute last instruction \RET, but halt (or hang).
Process will hang, but that's much better then remote attack to your host.}

\IFRU{Попробуем то же самое в GCC 4.4.1. У нас выходит такое:}{Let's try the same code in GCC 4.4.1. We got:}

\lstinputlisting{arrays/13_4_gcc.asm}

\IFRU{Запуск этого в Linux выдаст:}{Running this in Linux will produce:} \TT{Segmentation fault}.

\IFRU{Если запустить полученное в отладчике GDB, получим:}
{If we run this in GDB debugger, we getting this:}

\begin{lstlisting}
(gdb) r
Starting program: /home/dennis/RE/1 

Program received signal SIGSEGV, Segmentation fault.
0x00000016 in ?? ()
(gdb) info registers
eax            0x0	0
ecx            0xd2f96388	-755407992
edx            0x1d	29
ebx            0x26eff4	2551796
esp            0xbffff4b0	0xbffff4b0
ebp            0x15	0x15
esi            0x0	0
edi            0x0	0
eip            0x16	0x16
eflags         0x10202	[ IF RF ]
cs             0x73	115
ss             0x7b	123
ds             0x7b	123
es             0x7b	123
fs             0x0	0
gs             0x33	51
(gdb) 
\end{lstlisting}

\IFRU{Значения регистров немного другие чем в примере win32, это потому что разметка стека чуть другая.}
{Register values are slightly different then in win32 example, 
that's because stack layout is slightly different too.}

\subsection{\IFRU{Еще немного о массивах}{One more word about arrays}}

\IFRU{Теперь понятно, почему нельзя написать в исходном коде на \CCpp что-то вроде:
\footnote{GCC способен это сделать выделяя место под массив динамически в стеке (как alloca()), 
но это расширение не является частью стандарта}}
{Now we understand, why it's not possible to write something like that in \CCpp code
\footnote{GCC can actually do this by allocating array dynammically in stack (like alloca()), 
but it's not standard langauge extension}:}

\begin{lstlisting}
void f(int size)
{
    int a[size];
...
};
\end{lstlisting}

\IFRU{Все просто потому, чтобы выделять место под массив в локальном стеке или же сегменте данных 
(если массив глобальный), компилятору нужно знать его размер, чего он, на стадии компиляции, 
разумеется знать не может.}
{That's just because compiler should know exact array size to allocate place for it in local stack layout or
in data segment (in case of global variable) on compiling stage.}

\IFRU{Если вам нужен массив произвольной длины, то выделите столько, сколько нужно, через \TT{malloc()}, 
затем обращайтесь к выделенному блоку байт как к массиву того типа, который вам нужен.}
{If you need array of arbitrary size, allocate it by \TT{malloc()}, then access allocated memory block
as array of variables of type you need.}

\subsection{\IFRU{Многомерные массивы}{Multidimensional arrays}}

\IFRU{Многомерный массив выглядит внутри так же как и линейный.}
{Internally, multidimensional array is essentially the same thing as linear array.}

\IFRU{Попробуем:}{Let's see:}

\begin{lstlisting}
#include <stdio.h>

int a[10][20][30];

void insert(int x, int y, int z, int value)
{
	a[x][y][z]=value;
};
\end{lstlisting}

\IFRU{В итоге}{We got} (MSVC 2010):

\lstinputlisting{arrays/13_5_msvc.asm}

\IFRU{В принципе, ничего удивительного. В \TT{insert()} для индексирования нужного элемента массива, 
три входных аргумента перемножаются так, чтобы представить массив трехмерным.}
{Nothing special. For index calculation, three input arguments are multiplying 
in such way to represent array as multidimensional.}

GCC 4.4.1:

\lstinputlisting{arrays/13_5_gcc.asm}



% done

\section{\IFRU{Битовые поля}{Bit fields}}
\label{sec:bitfields}

\IFRU{Немало функций задают различные флаги в аргументах при помощи битовых 
полей\footnote{bit fields в анлоязычной литературе}.}
{A lot of functions defining input flags in arguments using bit fields.}
\IFRU{Наверное, вместо этого, можно было бы использовать набор переменных типа \IT{bool}, но это было бы 
не очень экономно.}
{Of course, it could be substituted by \IT{bool}-typed variables set, but it's not frugally.}

\subsection{\IFRU{Проверка какого-либо бита}{Specific bit checking}}

\IFRU{Например в Win32 API:}{Win32 API example:}

\begin{lstlisting}
	HANDLE fh;

	fh=CreateFile ("file", GENERIC_WRITE | GENERIC_READ, FILE_SHARE_READ, NULL, OPEN_ALWAYS, FILE_ATTRIBUTE_NORMAL, NULL);
\end{lstlisting}

\IFRU{Получаем}{We got} (MSVC 2010):

\begin{lstlisting}
	push	0
	push	128					; 00000080H
	push	4
	push	0
	push	1
	push	-1073741824				; c0000000H
	push	OFFSET $SG78813
	call	DWORD PTR __imp__CreateFileA@28
	mov	DWORD PTR _fh$[ebp], eax
\end{lstlisting}

\IFRU{Заглянем в файл}{Let's take a look into} WinNT.h:

\begin{lstlisting}
#define GENERIC_READ                     (0x80000000L)
#define GENERIC_WRITE                    (0x40000000L)
#define GENERIC_EXECUTE                  (0x20000000L)
#define GENERIC_ALL                      (0x10000000L)
\end{lstlisting}

\IFRU{Все ясно}{Everything is clear}, 
\TT{GENERIC\_READ | GENERIC\_WRITE = 0x80000000 | 0x40000000 = 0xC0000000}, 
\IFRU{и это значение используется как второй аргумент для}
{and that's value is used as second argument for} \TT{CreateFile()}\footnote{\href{http://msdn.microsoft.com/en-us/library/aa363858(VS.85).aspx}{MSDN: CreateFile function}} function.

\IFRU{Как \TT{CreateFile()} будет проверять флаги?}{How \TT{CreateFile()} will check flags?}

\IFRU{Заглянем в KERNEL32.DLL от Windows XP SP3 x86 и найдем в функции \TT{CreateFileW()} в том числе и 
такой кусок кода:}
{Let's take a look into KERNEL32.DLL in Windows XP SP3 x86 and we'll find
this piece of code in the function \TT{CreateFileW}:}

\begin{lstlisting}
.text:7C83D429                 test    byte ptr [ebp+dwDesiredAccess+3], 40h
.text:7C83D42D                 mov     [ebp+var_8], 1
.text:7C83D434                 jz      short loc_7C83D417
.text:7C83D436                 jmp     loc_7C810817
\end{lstlisting}

\IFRU{Здесь мы видим инструкцию \TEST, впрочем, она берет не весь второй аргумент функции, 
но только его самый старший байт (\TT{ebp+dwDesiredAccess+3}) и проверяет его на флаг 0x40 
(имеется ввиду флаг \TT{GENERIC\_WRITE}).}
{Here we see \TEST instruction, it takes, however, not the whole second argument,
but only most significant byte (\TT{ebp+dwDesiredAccess+3}) and checks it for 0x40 flag
(meaning \TT{GENERIC\_WRITE} flag here)}

\IFRU{\TEST это то же что и \AND, только без сохранения результата 
(вспомните что \CMP это то же что и \SUB, только без сохранения результатов}
{\TEST is merely the same instruction as \AND, but without result saving 
(recall the fact \CMP instruction is merely the same as \SUB, but without result saving}~\ref{CMPandSUB}).

\IFRU{Логика данного куска кода примерно такая:}{This piece of code logic is as follows:}

\begin{lstlisting}
if ((dwDesiredAccess&0x40000000) == 0) goto loc_7C83D417
\end{lstlisting}

\IFRU{Если после операции \AND останется этот бит, то флаг \ZF не будет поднят и условный переход \JZ не сработает. 
Переход возможен только если в переменной \TT{dwDesiredAccess} отсутствует бит 0x40000000 ~--- тогда результат \AND будет 0, флаг \ZF будет поднят и переход сработает.}
{If \AND instruction leaving this bit, \ZF flag will be cleared and \JZ conditional jump will not be triggered.
Conditional jump is possible only if 0x40000000 bit is absent in \TT{dwDesiredAccess} variable ~--- then \AND result will be 0, \ZF flag will be set and conditional jump is to be triggered.}

\IFRU{Попробуем GCC 4.4.1 и Linux:}{Let's try GCC 4.4.1 and Linux:}

\begin{lstlisting}
#include <stdio.h>
#include <fcntl.h>

void main()
{
	int handle;

	handle=open ("file", O_RDWR | O_CREAT);
};
\end{lstlisting}

\IFRU{Получим}{We got}:

\lstinputlisting{bitfields/14_3.asm}

\IFRU{Заглянем в реализацию функции \TT{open()} в библиотеке \TT{libc.so.6}, но обнаружим что там только вызов сисколла:}
{Let's take a look into \TT{open()} function in \TT{libc.so.6} library, but there is only syscall calling:}

\begin{lstlisting}
.text:000BE69B                 mov     edx, [esp+4+mode] ; mode
.text:000BE69F                 mov     ecx, [esp+4+flags] ; flags
.text:000BE6A3                 mov     ebx, [esp+4+filename] ; filename
.text:000BE6A7                 mov     eax, 5
.text:000BE6AC                 int     80h             ; LINUX - sys_open
\end{lstlisting}

\IFRU{Значит, битовые поля флагов \TT{open()} вероятно проверяются где-то в ядре Linux.}
{So, \TT{open()} bit fields are probably checked somewhere in Linux kernel.}

\IFRU{Разумеется, и стандартные библиотеки Linux и ядро Linux можно получить в виде исходников, 
но нам интересно попробовать разобраться без них.}
{Of course, it is easily to download both Glibc and Linux kernel source code, 
but we are interesting to understand the matter without it.}

\IFRU{Итак, при вызове сисколла \TT{sys\_open}, управление в конечном итоге передается в \TT{do\_sys\_open} в ядре Linux 2.6. 
Оттуда ~--- в \TT{do\_filp\_open()} (эта функция находится в исходниках ядра в файле \TT{fs/namei.c}).}
{So, as of Linux 2.6, when \TT{sys\_open} syscall is called, control eventually passed into \TT{do\_sys\_open} kernel function.
From there ~--- to \TT{do\_filp\_open()} function (this function located in kernel source tree in the file \TT{fs/namei.c}).}

\newcommand{\URLREGPARM}{\url{http://ohse.de/uwe/articles/gcc-attributes.html\#func-regparm}}

\IFRU{Важное отступление. Помимо передачи параметров функции через стек, существует также возможность передавать 
некоторые из них через регистры. Это называется в том числе fastcall~\ref{fastcall}. 
Это работает немного быстрее, так как процессору не нужно обращаться к стеку лежащему в памяти для чтения 
аргументов. 
В GCC есть опция \IT{regparm}\footnote{\URLREGPARM}, 
и с её помощью можно задать, сколько аргументов можно передать через регистры.}
{Important note. Aside from usual passing arguments via stack, there are also method to pass some of them
via registers. This is also called fastcall~\ref{fastcall}.
This works faster, because CPU not needed to access a stack in memory to read argument values.
GCC has option \IT{regparm}\footnote{\URLREGPARM},
and it's possible to set a number of arguments which might be passed via registers.}

\newcommand{\URLKERNELNEWB}{\url{http://kernelnewbies.org/Linux_2_6_20\#head-042c62f290834eb1fe0a1942bbf5bb9a4accbc8f}}
\newcommand{\CALLINGHFILE}{arch\textbackslash{}x86\textbackslash{}include\textbackslash{}asm\textbackslash{}calling.h}

\IFRU{Ядро Linux 2.6 собирается с опцией \TT{-mregparm=3}~\footnote{\URLKERNELNEWB}
\footnote{См. также файл \TT{\CALLINGHFILE} в исходниках ядра}.}
{Linux 2.6 kernel compiled with \TT{-mregparm=3} option~\footnote{\URLKERNELNEWB}
\footnote{See also \TT{\CALLINGHFILE} file in kernel tree}.}

\IFRU{И для нас это означает, что первые три аргумента функции будут передаваться через регистры \EAX, \EDX и \ECX, 
а остальные через стек. Разумеется, если аргументов у функции меньше трех, то будет задействована только часть регистров.}
{What it means to us, the first 3 arguments will be passed via \EAX, \EDX and \ECX registers, the other ones via stack. Of course, if arguments number is less than 3, only part of registers will be used.}

\IFRU{Итак, качаем ядро 2.6.31, собираем его в Ubuntu: \TT{make vmlinux}, открываем в \IDA, 
находим функцию \TT{do\_filp\_open()}. В начале мы увидим подобное (комментарии мои):}
{So, let's download Linux Kernel 2.6.31, compile it in Ubuntu: \TT{make vmlinux}, open it in \IDA, 
find the \TT{do\_filp\_open()} function. At the beginning, we will see (comments are mine):}

\lstinputlisting{\IFRU{bitfields/14_4_ru.asm}{bitfields/14_4_en.asm}}

\IFRU{GCC сохраняет значения первых трех аргументов в локальном стеке. Иначе, если эти три регистра не трогать вообще, то функции компилятора, распределяющей переменные по регистрам (так называемый \IT{register allocator}), 
будет очень тесно.}
{GCC saves first 3 arguments values in local stack. Otherwise, if compiler would not touch these registers, 
it would be too tight environment for compiler's register allocator}.

\IFRU{Далее находим примерно такой кусок:}{Let's find this piece of code:}

\lstinputlisting{bitfields/14_5.asm}

\IFRU{0x40 ~--- это то чему равен макрос \TT{O\_CREAT}. 
\TT{open\_flag} проверяется на наличие бита 0x40 и если бит равен 1, то выполняется следующие за \JNZ инструкции.}
{0x40 ~--- is what \TT{O\_CREAT} macro equals to.
\TT{open\_flag} checked for 0x40 bit presence, and if this bit is 1, next \JNZ instruction is triggered.}

\subsection{\IFRU{Установка/сброс отдельного бита}{Specific bit setting/clearing}}

\IFRU{Например:}{For example:}

\lstinputlisting{bitfields/14_6.c}

\IFRU{Имеем в итоге}{We got} (MSVC 2010):

\lstinputlisting{bitfields/14_6_msvc.asm}

\IFRU{Инструкция \OR здесь добавляет в переменную еще один бит, игнорируя остальные.}
{\OR instruction adding one more bit to value, ignoring others.}

\IFRU{А \AND сбрасывает некий бит. Можно также сказать, что \AND здесь копирует все биты, кроме одного. 
Действительно, во втором операнде \AND выставлены в еденицу те биты, которые нужно сохранить, 
кроме одного, копировать который мы не хотим (и который 0 в битовой маске).
Так проще понять и запомнить.}
{\AND resetting one bit. It can be said, \AND just copies all bits except one.
Indeed, in the second \AND operand only those bits are set, which are needed to be saved,
except one bit we wouldn't like to copy (which is 0 in bitmask).
It's easier way to memorize the logic.}

\IFRU{Если скомпилировать в MSVC с оптимизацией (\Ox), то код будет еще короче:}
{If we compile it in MSVC with optimization turned on (\Ox), the code will be even shorter:}

\lstinputlisting{bitfields/14_6_msvc_Ox.asm}

\IFRU{Попробуем GCC 4.4.1 без оптимизации:}{Let's try GCC 4.4.1 without optimization:}

\lstinputlisting{bitfields/14_6_gcc.asm}

\IFRU{Также избыточный код, хотя короче чем у MSVC без оптимизации.}
{There are some redundant code present, however, it's shorter then MSVC version without optimization.}

\IFRU{Попробуем теперь GCC с оптимизацией}{Now let's try GCC with optimization turned on} \TT{-O3}:

\lstinputlisting{bitfields/14_6_gcc_O3.asm}

\IFRU{Уже короче. Важно отметить что через регистр \AH, компилятор работает с частью регистра \EAX, 
эта его часть от 8-го до 15-го бита включительно.}
{That's shorter. It is important to note that compiler works with \EAX register part via \AH register ~--- 
that's \EAX register part from 8th to 15th bits inclusive.}

\IFRU{Важное отступление: в 16-битном процессоре 8086 аккумулятор имел название \AX и состоял из двух 8-битных половин ~--- \AL (младшая часть) и \AH (старшая). 
В 80386 регистры были расширены до 32-бит, 
аккумулятор стал называться \EAX, но в целях совместимости, к его \IT{более старым} частям все еще можно 
обращаться как к \AX/\AH/\AL.}
{Important note: 16-bit CPU 8086 accumulator was named \AX and consisted of two 8-bit halves ~--- \AL (lower byte) and \AH (higher byte).
In 80386 almost all regsiters were extended to 32-bit, accumulator was named \EAX, 
but for the sake of compatibility,
its \IT{older parts} may be still accessed as \AX/\AH/\AL registers.}

\IFRU{Из-за того что все x86 процессоры ~--- наследники 16-битного 8086, эти \IT{старые} 16-битные опкоды короче 
нежели более новые 32-битные. 
Поэтому, инструкция \TT{or ah, 40h} занимает только 3 байта. 
Было бы логичнее сгенерировать здесь \TT{or eax, 04000h}, но это уже 5 байт, или даже 6 
(если регистр в первом операнде не \EAX).}
{Because all x86 CPUs are 16-bit 8086 CPU successors, these \IT{older} 16-bit opcodes are shorter than newer 32-bit opcodes.
That's why \TT{or ah, 40h} instruction occupying only 3 bytes.
It would be more logical way to emit here \TT{or eax, 04000h}, but that's 5 bytes, or even 6
(if register in first operand is not \EAX).}

\IFRU{Если мы скомпилируем этот же пример не только с включенной оптимизацией \TT{-O3}, 
но еще и с опцией \TT{regparm=3}, о которой я писал немного выше, то получится еще короче:}
{It would be even shorter if to turn on \TT{-O3} optimization flag and also set \TT{regparm=3}.}

\lstinputlisting{bitfields/14_6_gcc_O3_regparm3.asm}

\newcommand{\URLINL}{\url{http://en.wikipedia.org/wiki/Inline_function}}

\IFRU{Действительно ~--- первый аргумент уже загружен в \EAX, и прямо здесь можно начинать с ним работать. 
Интересно, что и пролог функции (\TT{push ebp / mov ebp,esp}) и эпилог (\TT{pop ebp}) функции можно смело выкинуть
за ненадобностью, 
но возможно GCC еще не так хорош для подобных оптимизаций по размеру кода. 
Впрочем, в реальной жизни, подобные короткие функции лучше всего автоматически делать в виде \IT{inline-функций}\footnote{\URLINL}.}
{Indeed ~--- first argument is already loaded into \EAX, so it's possible to work with it in-place.
It's worth noting that both function prologue (\TT{push ebp / mov ebp,esp}) and epilogue can easily be omitted
here, but GCC probably isn't good enough for such code size optimizations.
However, such short functions are better to be \IT{inlined functions}\footnote{\URLINL}.}

\subsection{\IFRU{Сдвиги}{Shifts}}

\IFRU{Битовые сдвиги в \CCpp реализованы при помощи операторов \TT{<<} и \TT{>>}.}
{Bit shifts in \CCpp are implemented via \TT{<<} and \TT{>>} operators.}

\IFRU{Вот этот несложный пример иллюстрирует функцию, считающую количество бит-единиц во входной переменной:}
{Here is a simple example of function, calculating number of 1 bits in input variable:}

\lstinputlisting{bitfields/14_7.c}

\IFRU{В этом цикле, счетчик итераций \IT{i} считает от 0 до 31, а \TT{1<<i} будет от 1 до 0x80000000. 
Описывая это словами, можно сказать 
\IT{сдвинуть единицу на n бит влево}.
Т.е., в некотором смысле, выражение \TT{1<<i} последовательно выдаст все возможные позиции бит в 32-битном числе. 
Кстати, освободившийся бит справа всегда обнуляется. Макрос \TT{IS\_SET} проверяет наличие этого бита в \TT{a}.}
{In this loop, iteration count value \IT{i} counting from 0 to 31, \TT{1<<i} statement will be counting from 1 to 0x80000000. 
Describing this operation in naturaul language, we would say \IT{shift 1 by n bits left}.
In other words, \TT{1<<i} statement will consequentially produce all possible bit positions in 32-bit number.
By the way, freed bit at right is always cleared. \TT{IS\_SET} macro is checking bit presence in \TT{a}.}

\IFRU{Макрос \TT{IS\_SET} на самом деле это операция логического И (\IT{AND}) 
и она возвращает ноль если бита там нет, 
либо эту же битовую маску, если бит там есть. 
В \CCpp, конструкция \TT{if()} срабатывает, если выражение внутри её не ноль, пусть хоть 123, 
поэтому все будет работать.}
{The \TT{IS\_SET} macro is in fact logical and operation (\IT{AND}) 
and it returns zero if specific bit is absent there,
or bit mask, if the bit is present.
\IT{if()} operator triggered in \CCpp if expression in it isn't zero, it might be even 123, that's why
it always working correctly.}

\IFRU{Компилируем}{Let's compile} (MSVC 2010):

\lstinputlisting{bitfields/14_1_en.asm}

\IFRU{Вот так работает SHL (\IT{SHift Left})}{That's how SHL (\IT{SHift Left}) working}.

\IFRU{Скомпилим то же и в}{Let's compile it in} GCC 4.4.1:

\lstinputlisting{bitfields/14_7_gcc.asm}

\IFRU{Инструкции сдвига также активно применяются при делении или умножении 
на числа-степени двойки (1, 2, 4, 8, итд).}
{Shift instructions are often used in division and multiplications by power of two numbers (1, 2, 4, 8, etc).}

\IFRU{Например:}{For example:}

\begin{lstlisting}
unsigned int f(unsigned int a)
{
	return a/4;
};
\end{lstlisting}

\IFRU{Имеем в итоге}{We got} (MSVC 2010):

\begin{lstlisting}
_a$ = 8							; size = 4
_f	PROC
	mov	eax, DWORD PTR _a$[esp-4]
	shr	eax, 2
	ret	0
_f	ENDP
\end{lstlisting}

\label{SHR}
\IFRU{Инструкция SHR (\IT{SHift Right}) сдвигает число на 2 бита вправо. 
При этом освободившиеся два бита слева (т.е., самые 
старшие разряды) выставляются в нули. А самые младшие 2 бита выкидываются. 
Фактически, эти два выкинутых бита ~--- остаток от деления.}
{SHR (\IT{SHift Right}) instruction is shifting a number by 2 bits right.
Two freed bits at left (e.g., two most significant bits) are set to zero.
Two least significant bits are dropped.
In fact, these two dropped bits ~--- division operation remainder.}

\IFRU{Для того, чтобы это проще понять, представьте себе десятичную систему счисления и число 23. 
23 можно разделить на 10 просто выкинув последний разряд (3 ~--- это остаток от деления). 
После этой операции останется 2 как частное
\footnote{результат деления}.}
{It can be easily understood if to imagine decimal numeral system and number 23.
23 can be easily divided by 10 just by dropping last digit (3 ~--- is division remainder). 
2 is leaving after operation as a quotient
\footnote{division result}.}

\IFRU{Так и с умножением. Умножить на 4 это просто сдвинуть число на 2 бита влево, 
вставив 2 нулевых бита справа (как два самых младших бита).}
{The same story about multiplication. Multiplication by 4 is just shifting the number to the left by 2 bits,
inserting 2 zero bits at right (as the last two bits).}

\subsection{\IFRU{Пример вычисления CRC32}{CRC32 calculation example}}

\newcommand{\URLCRCSRC}{\url{http://burtleburtle.net/bob/c/crc.c}}

\IFRU{Это распространенный табличный способ вычисления хеша алгоритмом 
CRC32\footnote{Исходник взят тут: \URLCRCSRC}.}
{This is very popular table-based CRC32 hash calculation 
method\footnote{Source code was taken here: \URLCRCSRC}.}

\lstinputlisting{bitfields/14_CRC.c}

\IFRU{Нас интересует функция \TT{crc()}. 
Кстати, обратите внимание, автор указал два инициализатора в выражении \TT{for()}: \TT{hash=len, i=0}. 
Стандарт \CCpp, конечно, допускает это. А в итоговом коде, вместо одной операции инициализации цикла, будет две.}
{We are interesting in \TT{crc()} function only.
By the way, please note: programmer used two loop initializers in \TT{for()} statement: \TT{hash=len, i=0}.
\CCpp standard allows this, of course. Emited code will contain two operations in loop initialization part
instead of usual one.}

\IFRU{Компилируем в MSVC с оптимизацией (\Ox). 
Для краткости, я приведу только функцию \TT{crc()}, с некоторыми комментариями.}
{Let's compile it in MSVC with optimization (\Ox).
For the sake of brevity, only \TT{crc()} function is listed here, with my comments.}

\lstinputlisting{\IFRU{bitfields/14_2_ru.asm}{bitfields/14_2_en.asm}}

\IFRU{Попробуем то же самое в GCC 4.4.1 с опцией \TT{-O3}:}
{Let's try the same in GCC 4.4.1 with \TT{-O3} option:}

\lstinputlisting{\IFRU{bitfields/14_CRC_gcc_O3_ru.asm}{bitfields/14_CRC_gcc_O3_en.asm}}

\IFRU{GCC немного выровнял начало тела цикла по 8-байтной границе, для этого добавил 
\NOP и \TT{lea esi, [esi+0]} (что тоже \IT{холостая операция}). 
Подробнее об этом смотрите в разделе о npad~\ref{sec:npad}.}
{GCC aligned loop start by 8-byte border by adding \NOP and \TT{lea esi, [esi+0]} (that's \IT{idle operation} too).
Read more about it in npad section~\ref{sec:npad}.}


% done 

\section{\IFRU{Структуры}{Structures}}

\IFRU{В принципе, структура в \CCpp это, с некоторыми допущениями, просто всегда лежащий рядом, 
и в той же последовательности, набор переменных, не обязательно одного типа.}
{It can be defined that \CCpp structure, with some assumptions, just a set of variables, always stored
in memory together, not necessary of the same type.}

\subsection{\IFRU{Пример SYSTEMTIME}{SYSTEMTIME example}}

\newcommand{\FNSYSTEMTIME}{\footnote{\href{http://msdn.microsoft.com/en-us/library/ms724950(VS.85).aspx}{MSDN: SYSTEMTIME structure}}}

\IFRU{Возьмем, к примеру, структуру SYSTEMTIME\FNSYSTEMTIME из win32 описывающую время.}
{Let's take SYSTEMTIME\FNSYSTEMTIME win32 structure describing time.}

\IFRU{Она объявлена так:}{That's how it's defined:}

\begin{lstlisting}
typedef struct _SYSTEMTIME {
  WORD wYear;
  WORD wMonth;
  WORD wDayOfWeek;
  WORD wDay;
  WORD wHour;
  WORD wMinute;
  WORD wSecond;
  WORD wMilliseconds;
} SYSTEMTIME, *PSYSTEMTIME;
\end{lstlisting}

\IFRU{Пишем на Си функцию для получения текущего системного времени:}
{Let's write a C function to get current time:}

\lstinputlisting{structs/15_0.c}

\IFRU{Что в итоге}{We got} (MSVC 2010):

\lstinputlisting{structs/15_0.asm}

\IFRU{Под структуру в стеке выделено 16 байт ~--- именно столько будет \TT{sizeof(WORD)*8}
(в структуре 8 переменных с типом WORD).}
{16 bytes are allocated for this structure in local stack ~--- that's exactly \TT{sizeof(WORD)*8}
(there are 8 WORD variables in the structure).}

\newcommand{\FNMSDNGST}{\footnote{\href{http://msdn.microsoft.com/en-us/library/ms724390(VS.85).aspx}{MSDN: GetSystemTime function}}}

\IFRU{Обратите внимание: структура начинается с поля \TT{wYear}. 
Можно сказать что в качестве аргумента для \TT{GetSystemTime()}\FNMSDNGST передается указатель на структуру 
SYSTEMTIME, но можно также сказать, что передается указатель на поле \TT{wYear}, 
что одно и тоже! 
\TT{GetSystemTime()} пишет текущий год в тот WORD на который указывает переданный указатель, 
затем сдвигается на 2 байта вправо, пишет текущий месяц, итд, итд.}
{Take a note: the structure beginning with \TT{wYear} field.
It can be said, an pointer to SYSTEMTIME structure is passed to \TT{GetSystemTime()}\FNSYSTEMTIME,
but it's also can be said, pointer to \TT{wYear} field is passed, and that's the same!
\TT{GetSystemTime()} writting current year to the WORD pointer pointing to, then shifting 2 bytes
ahead, then writting current month, etc, etc.}

\subsection{\IFRU{Выделяем место для структуры через malloc()}{Let's allocate place for structure using malloc()}}

\IFRU{Однако, бывает и так, что проще хранить структуры не в стеке а в куче\footnote{heap}:}
{However, sometimes it's simpler to place structures not in local stack, but in heap:}

\lstinputlisting{structs/15_4.c}

\IFRU{Скомпилируем на этот раз с оптимизацией (\Ox) чтобы было проще увидеть то, что нам нужно.}
{Let's compile it now with optimization (\Ox) so to easily see what we need.}

\lstinputlisting{structs/15_4.asm}

\IFRU{Итак, \TT{sizeof(SYSTEMTIME) = 16}, именно столько байт выделяется при помощи \TT{malloc()}. 
Она возвращает указатель на только что выделенный блок памяти в \EAX, который копируется в \ESI. 
Win32 функция \TT{GetSystemTime()} обязуется сохранить состояние \ESI, 
поэтому здесь оно нигде не сохраняется и продолжает использоваться после вызова \TT{GetSystemTime()}.}
{So, \TT{sizeof(SYSTEMTIME) = 16}, that's exact number of bytes to be allocated by \TT{malloc()}.
It return the pointer to freshly allocated memory block in \EAX, which is then moved into \ESI.
\TT{GetSystemTime()} win32 function undertake to save \ESI value, 
and that's why it is not saved here and continue to be used after \TT{GetSystemTime()} call.}

\IFRU{
Новая инструкция ~--- \MOVZX (\IT{Move with Zero eXtent}). 
Она нужна почти там же где и \MOVSX, 
только всегда очищает остальные биты в 0. Дело в том что \printf требует 32-битный тип \Tint, 
а в структуре лежит WORD ~--- это 16-битный беззнаковый тип. Поэтому копируя значение из WORD в \Tint, 
нужно очистить биты от 16 до 31, иначе там будет просто случайный мусор, оставшийся от предыдущих действий 
с регистрами.}
{New instruction ~--- \MOVZX (\IT{Move with Zero eXtent}).
It may be used almost in those cases as \MOVSX, but, it clearing other bits to 0.
That's because \printf require 32-bit \Tint, but we got WORD in structure ~--- that's 16-bit unsigned type.
That's why by copying value from WORD into \Tint{}, bits from 16 to 31 should be cleared, 
because there will be random noise otherwise, leaved from previous operations on registers.}

\subsection{Linux}

\IFRU{В Линуксе, для примера, возьем структуру \TT{tm} из \TT{time.h}:}
{As of Linux, let's take \TT{tm} structure from \TT{time.h} for example:}

\lstinputlisting{structs/15_1.c}

\IFRU{Компилируем при помощи}{Let's compile it in} GCC 4.4.1:

\IFRU{\lstinputlisting{structs/15_1_ru.asm}}{\lstinputlisting{structs/15_1_en.asm}}

\IFRU{К сожалению, по какой-то причине, \IDA не сформировала названия локальных переменных в стеке. 
Но так как мы уже опытные реверсеры :-) то можем обойтись и без этого в таком простом примере.}
{Somehow, \IDA didn't created local variables names in local stack.
But since we already experienced reverse engineers :-) we may do it without this information in 
this simple example.}

\IFRU{Обратите внимание на \TT{lea edx, [eax+76Ch]} ~--- эта инструкция прибавляет 0x76C к \EAX, 
но не модифицирует флаги. См. также соответствующий раздел об инструкции \LEA{}~\ref{sec:LEA}.}
{Please also pay attention to \TT{lea edx, [eax+76Ch]} ~--- this instruction just adding 0x76C to \EAX,
but not modify any flags. See also relevant section about \LEA{}~\ref{sec:LEA}.}

\subsection{\IFRU{Упаковка полей в структуре}{Fields packing in structure}}

\IFRU{Достаточно немаловажный момент, это упаковка полей в структурах\footnote{См.также: \URLWPDA}.}
{One important thing is fields packing in structures\footnote{See also: \URLWPDA}.}

\IFRU{Возьмем простой пример:}{Let's take a simple example:}

\lstinputlisting{structs/15_5.c}

\IFRU{Как видно, мы имеем два поля \Tchar (занимающий один байт) и еще два ~--- \Tint (по 4 байта).}
{As we see, we have two \Tchar fields (each is exactly one byte) and two more ~--- \Tint (each - 4 bytes).}

\IFRU{Компилируется это все в:}{That's all compiling into:}

\lstinputlisting{structs/15_5.asm}

\IFRU{Мы видим здесь что адрес каждого поля в структуре выравнивается по 4-байтной границе. 
Так что каждый \Tchar здесь занимает те же 4 байта что и \Tint. Зачем? 
Затем что процессору удобнее обращаться по таким адресам и кешировать данные из памяти.}
{As we can see, each field's address is aligned by 4-bytes border.
That's why each \Tchar using 4 bytes here, like \Tint. Why?
Thus it's easier for CPU to access memory at aligned addresses and to cache data from it.}

\IFRU{Но это не экономично по размеру данных.}{However, it's not very economical in size sense.}

\IFRU{Попробуем скомпилировать тот же исходник с опцией}{Let's try to compile it with option} (\TT{/Zp1}) 
(\IT{/Zp[n] pack structs on n-byte boundary}).

\lstinputlisting{structs/15_5_msvc_Zp1.asm}

\IFRU{Теперь структура занимает 10 байт и все \Tchar занимают по байту. Что это дает? 
Экономию места. Недостаток ~--- процессор будет обращаться к этим полям не так эффективно 
по скорости как мог бы.}
{Now the structure takes only 10 bytes and each \Tchar value takes 1 byte. What it give to us?
Size economy. And as drawback ~--- CPU will access these fields without maximal performance it can.}

\IFRU{Как нетрудно догадаться, если структура используется много в каких исходниках и объектных файлах, 
все они должны быть откомпилированы с одним и тем же соглашением об упаковке структур.}
{As it can be easily guessed, if the structure is used in many source and object files,
all these should be compiled with the same convention about structures packing.}

\newcommand{\FNURLMSDNZP}{\footnote{\href{http://msdn.microsoft.com/en-us/library/ms253935.aspx}
{MSDN: Working with Packing Structures}}}
\newcommand{\FNURLGCCPC}{\footnote{\href{http://gcc.gnu.org/onlinedocs/gcc/Structure_002dPacking-Pragmas.html}
{Structure-Packing Pragmas}}}

\IFRU{Помимо ключа MSVC \TT{/Zp}, указывающего, по какой границе упаковывать поля структур, есть также 
опция компилятора \TT{\#pragma pack}, её можно указывать прямо в исходнике. 
Это справедливо и для MSVC\FNURLMSDNZP и GCC\FNURLGCCPC{}.}
{Aside from MSVC \TT{/Zp} option which set how to align each structure field, here is also
\TT{\#pragma pack} compiler option, it can be defined right in source code.
It's available in both MSVC\FNURLMSDNZP and GCC\FNURLGCCPC{}.}

\IFRU{Давайте теперь вернемся к \TT{SYSTEMTIME}, которая состоит из 16-битных полей. 
Откуда наш компилятор знает что их надо паковать по однобайтной границе?}
{Let's back to \TT{SYSTEMTIME} structure consisting in 16-bit fields.
How our compiler know to pack them on 1-byte alignment method?}

\IFRU{В файле \TT{WinNT.h} попадается такое:}{\TT{WinNT.h} file has this:}

\begin{lstlisting}
#include "pshpack1.h"
\end{lstlisting}

\IFRU{И такое:}{And this:}

\begin{lstlisting}
#include "pshpack4.h"                   // 4 byte packing is the default
\end{lstlisting}

\IFRU{Сам файл PshPack1.h выглядит так:}{The file PshPack1.h looks like:}

\begin{lstlisting}
#if ! (defined(lint) || defined(RC_INVOKED))
#if ( _MSC_VER >= 800 && !defined(_M_I86)) || defined(_PUSHPOP_SUPPORTED)
#pragma warning(disable:4103)
#if !(defined( MIDL_PASS )) || defined( __midl )
#pragma pack(push,1)
#else
#pragma pack(1)
#endif
#else
#pragma pack(1)
#endif
#endif /* ! (defined(lint) || defined(RC_INVOKED)) */
\end{lstlisting}

\IFRU{Собственно, так и задается компилятору, как паковать объявленные после \TT{\#pragma pack} структуры.}
{That's how compiler will pack structures defined after \TT{\#pragma pack}.}

\subsection{\IFRU{Вложенные структуры}{Nested structures}}

\IFRU{Теперь, как насчет ситуаций, когда одна структура определяет внутри себя еще одну структуру?}
{Now what about situations when one structure define another structure inside?}

\lstinputlisting{structs/15_6.c}

\IFRU{... в этом случае, оба поля \TT{inner\_struct} просто будут располагаться между полями a,b и d,e в 
\TT{outer\_struct}.}
{... in this case, both \TT{inner\_struct} fields will be placed between a,b and d,e fields of
\TT{outer\_struct}.}

\IFRU{Компилируем}{Let's compile} (MSVC 2010):

\lstinputlisting{structs/15_6_msvc.asm}

\IFRU{Очень любопытный момент в том, что глядя на этот код на ассемблере, мы даже не видим, 
что была использована какая-то еще другая структура внутри этой!
Так что, пожалуй, можно сказать, что все вложенные структуры в итоге разворачиваются в одну, \IT{линейную} 
или \IT{одномерную} структуру.}
{One curious point here is that by looking onto this assembler code, we do not even see that
another structure was used inside of it!
Thus, we would say, nested structures are finally unfolds into \IT{linear} or \IT{one-dimensional} structure.}

\IFRU{Конечно, если заменить объявление \TT{struct inner\_struct c;} на \TT{struct inner\_struct *c;} 
(объявляя таким образом указатель), ситауция будет совсем иная.}
{Of course, if to replace \TT{struct inner\_struct c;} declaration to \TT{struct inner\_struct *c;} 
(thus making a pointer here) situation will be significally different.}

\subsection{\IFRU{Работа с битовыми полями в структуре}{Bit fields in structure}}

\subsubsection{\IFRU{Пример CPUID}{CPUID example}}

\IFRU{Язык \CCpp позволяет укзывать, сколько именно бит отвести для каждого поля структуры. 
Это удобно если нужно экономить место в памяти. К примеру, для переменной типа \Tbool достаточно одного бита.
Но, это не очень удобно, если нужна скорость.}
{\CCpp language allow to define exact number of bits for each structure fields.
It's very useful if one need to save memory space. 
For example, one bit is enough for variable of \Tbool type.
But of course, it's not rational if speed is important.}

\newcommand{\FNCPUID}{\footnote{\url{http://en.wikipedia.org/wiki/CPUID}}}

\IFRU{Рассмотрим пример с инструкцией \CPUID\FNCPUID. 
Эта инструкция возвращает информацию о том, какой процессор имеется в наличии и какие фичи он имеет.}
{Let's consider \CPUID\FNCPUID instruction example.
This instruction return information about current CPU and its features.}

\IFRU{Если перед исполнением инструкции в \EAX будет 1, 
то \CPUID вернет упакованную в \EAX такую информацию о процессоре:}
{If \EAX is set to 1 before instruction execution, \CPUID will return this information packed into \EAX
register:}

\begin{tabular}{ | l | l | }
3:0 & Stepping \\
7:4 & Model \\
11:8 & Family \\
13:12 & Processor Type \\
19:16 & Extended Model \\
27:20 & Extended Family \\
\end{tabular}

\newcommand{\FNGCCAS}{\footnote{\href{http://www.ibiblio.org/gferg/ldp/GCC-Inline-Assembly-HOWTO.html}
{\IFRU{Подробнее о встроенном ассемблере GCC}{More about internal GCC assembler}}}}

\IFRU{MSVC 2010 имеет макрос для \CPUID, а GCC 4.4.1 ~--- нет. 
Поэтому для GCC сделаем эту функцию сами, используя его встроенный ассемблер\FNGCCAS.}
{MSVC 2010 has \CPUID macro, but GCC 4.4.1 ~--- hasn't.
So let's make this function by yourself for GCC, using its built-in assembler\FNGCCAS.}

\lstinputlisting{structs/15_7.c}

\IFRU{После того как \CPUID заполнит \EAX/\EBX/\ECX/\EDX, у нас они отразятся в массиве \TT{b[]}. 
Затем, мы имеем указатель на структуру \TT{CPUID\_1\_EAX}, и мы указываем его на значение 
\EAX из массива \TT{b[]}.}
{After \CPUID will fill \EAX/\EBX/\ECX/\EDX, these registers will be reflected in \TT{b[]} array.
Then, we have a pointer to \TT{CPUID\_1\_EAX} structure and we point it to \EAX value from \TT{b[]} array.}

\IFRU{Иными словами, мы трактуем 32-битный \Tint как структуру.}
{In other words, we treat 32-bit \Tint value as a structure.}

\IFRU{Затем мы читаем из структуры.}{Then we read from the stucture.}

\IFRU{Компилируем в MSVC 2008 с опцией \Ox}{Let's compile it in MSVC 2008 with \Ox option}:

\lstinputlisting{structs/15_7_msvc_Ox.asm}

\IFRU{Инструкция \TT{SHR} сдвигает значение из \EAX на то количество бит, 
которое нужно \IT{пропустить}, то есть, мы игнорируем некоторые биты \IT{справа}.}
{\TT{SHR} instruction shifting value in \EAX by number of bits should be
\IT{skipped}, e.g., we ignore some bits \IT{at right}.}

\IFRU{А инструкция \AND очищает биты \IT{слева} которые нам не нужны, или же, говоря иначе, 
она оставляет по маске только те биты в \EAX, которые нам сейчас нужны.}
{\AND instruction clearing not needed bits \IT{at left}, or, in other words, 
leave only those bits in \EAX we need now.}

\IFRU{Попробуем GCC 4.4.1 с опцией \TT{-O3}.}{Let's try GCC 4.4.1 with \TT{-O3} option.}

\lstinputlisting{structs/15_7_gcc_O3.asm}

\IFRU{Практически, то же самое. Единственное что стоит отметить это то, что GCC решил зачем-то объеденить 
вычисление \TT{extended\_model\_id} и \TT{extended\_family\_id} в один блок, 
вместо того чтобы вычислять их перед соответствующим вызовом \printf.}
{Almost the same. The only thing to note is that GCC somehow united calculation of
\TT{extended\_model\_id} and \TT{extended\_family\_id} into one block,
instead of calculating them separately, before corresponding each \printf call.}

\subsubsection{\WorkingWithFloatAsWithStructSubSubSectionName}
\label{sec:floatasstruct}

\IFRU{Как уже раннее указывалось в секции о FPU~\ref{sec:FPU}, и \Tfloat и \Tdouble содержат в себе знак, 
мантиссу и экспоненту. 
Однако, можем ли мы работать с этими полями напрямую? Попробуем с \Tfloat.}
{As it was already noted in section about FPU~\ref{sec:FPU}, both \Tfloat and \Tdouble types consisted of sign,
significand (or fraction) and exponent.
But will we able to work with these fields directly? Let's try with \Tfloat.}

\begin{figure}[ht!]
\centering
\includegraphics[scale=0.66]{structs/500px-Float_example.png}
\caption{\IFRU{Формат значения float (иллюстрация взята из wikipedia)}
{float value format (illustration taken from wikipedia)}}
\end{figure}

\lstinputlisting{structs/15_2_en.c}

\IFRU{Структура \TT{float\_as\_struct} занимает в памяти столько же места сколько и \Tfloat, 
то есть 4 байта или 32 бита.}
{\TT{float\_as\_struct} structure occupies as much space is memory as \Tfloat, e.g., 4 bytes or 32 bits.}

\IFRU{Далее мы выставляем во входящем значении отрицательный знак, 
а также прибавляя двойку к экспоненте, мы тем 
самым умножаем всё значение на \TT{$2^2$}, то есть на 4.}
{Now we setting negative sign in input value and also by addding 2 to exponent we thereby multiplicating
the whole number by \TT{$2^2$}, e.g., by 4.}

\IFRU{Компилируем в MSVC 2008 без оптимизации:}{Let's compile in MSVC 2008 without optimization:}

\IFRU{\lstinputlisting{structs/15_2_msvc_ru.asm}}{\lstinputlisting{structs/15_2_msvc_en.asm}}

\IFRU{Слекга избыточно. В версии скомпиленной с флагом \Ox нет вызовов \TT{memcpy()}, 
там работа происходит сразу с переменной f. Но по неоптимизированной версии будет проще понять.}
{Redundant for a bit. If it compiled with \Ox flag there are no \TT{memcpy()} call,
f variable is used directly. But it's easier to understand it all considering unoptimized version.}

\IFRU{А что сделает GCC 4.4.1 с опцией \TT{-O3}?}{What GCC 4.4.1 with \TT{-O3} will do?}

\lstinputlisting{structs/15_2_gcc_O3_en.asm}

\IFRU{Да, функция \TT{f()} в целом понятна. Однако, что интересно, еще при компиляции, 
не взирая на мешанину с полями структуры, GCC умудрился вычислить результат функции \TT{f(1.234)} и 
сразу подставить его в аргумент для \printf{}!}
{The \TT{f()} function is almost understandable. However, what is interesting, GCC was able to calculate
\TT{f(1.234)} result during compilation stage despite all this hodge-podge with structure fields
and prepared this argument to \printf{} as precalculated!}



% done

\section{\IFRU{Классы в Си++}{C++ classes}}

\IFRU{Я намеренно расположил описание классов здесь сразу за структурами, 
потому что внутреннее представление классов в Си++ почти такое же как и представление структур.}
{I placed a C++ classes description here intentionally after structures description,
because internally, C++ classes representation is almost the same as structures representation.}

\IFRU{Давайте попробуем простой пример с парой переменных, парой конструкторов и одним методом:}
{Let's try an example with two variables, couple of constructors and one method:}

\lstinputlisting{classes/16_1.cpp}

\IFRU{Вот как выглядит \main на ассемблере:}{Here is how \main function looks like translated into assembler:}

\lstinputlisting{classes/16_1_msvc.asm}

\IFRU{Вот что происходит. 
Под каждый экземпляр класса \IT{c} выделяется по 8 байт, столько же, сколько нужно 
для хранения двух переменных.}
{So what's going on.
For each object (instance of class \IT{c}) 8 bytes allocated, that's exactly size of 2 variables storage.}

\IFRU{Для \IT{c1} вызывается конструктор по умолчанию без аргументов \TT{??0c@@QAE@XZ}. 
Для \IT{c2} вызывается другой конструктор \TT{??0c@@QAE@HH@Z} и передаются два числа в качестве аргументов.}
{For \IT{c1} a default argumentless constructor \TT{??0c@@QAE@XZ} is called.
For \IT{c2} another constructor \TT{??0c@@QAE@HH@Z} is called and two numbers are passed as arguments.}

\IFRU{А указатель на объект (\IT{this} в терминологии Си++) передается в регистре \ECX. 
Это называется thiscall~\ref{thiscall} ~--- метод передачи указателя на объект.}
{A pointer to object (\IT{this} in C++ terminology) is passed in \ECX register.
This is called thiscall~\ref{thiscall} ~--- a pointer to object passing method.}

\IFRU{В данном случае, MSVC делает это через \ECX. Необходимо помнить, что это не стандартизированный метод, 
и другие компиляторы могут делать это иначе, например через первый аргумент функции (как GCC).}
{MSVC doing it using \ECX register. Needless to say, it's not a standardized method, other compilers could do it
differently, for example, via first function argument (like GCC).}

%\newcommand{\URLNM}{\href{http://en.wikipedia.org/wiki/Visual_C\%2B\%2B_name_mangling}{Wikipedia: Visual C++ name mangling}}
\newcommand{\URLNM}{\href{http://en.wikipedia.org/wiki/Name_mangling}{Wikipedia: Name mangling}}

\IFRU{Почему у имен функций такие странные имена? Это \IT{name mangling}\footnote{\URLNM}.}
{Why these functions has so odd names? That's \IT{name mangling}\footnote{\URLNM}.}

\IFRU{В Си++, у класса, может иметься несколько методов с одинаковыми именами 
но аргументами разных типов ~--- это полиморфизм. 
Ну и конечно, у разных классов могут быть методы с одинаковыми именами.}
{C++ class may contain several methods sharing the same name but having different arguments ~--- 
that's polymorphism.
And of course, different classes may own methods sharing the same name.}

\IFRU{\IT{Name mangling} позволяет закодировать имя класса + имя метода + типы всех аргументов метода 
в одной ASCII-строке, которая затем используется как внутреннее имя функции. 
Это все потому что ни линкер, ни загрузчик DLL операционной системы 
(мангленные имена могут быть среди экспортов/импортов в DLL), 
ничего не знают о Си++ или ООП.}
{\IT{Name mangling} allows to encode class name + method name + all method argument types 
in one ASCII-string, which will be used as internal function name.
That's all because neither linker, nor DLL operation system loader (mangled names may be among 
DLL exports as well) knows nothing about C++ or OOP.}

\IFRU{Далее вызывается два раза \TT{dump()}.}{\TT{dump()} function called two times after.}

\IFRU{Теперь смотрим на код в конструкторах:}{Now let's see constructors' code:}

\lstinputlisting{classes/16_2_msvc.asm}

\IFRU{Конструкторы это просто функции, они используют указатель на структуру в \ECX, 
перекладывают его себе в локальную переменную, хотя это и не обязательно.}
{Constructors are just functions, they use pointer to structure in \ECX,
moving the pointer into own local variable, however, it's not necessary.}

\IFRU{И еще метод \TT{dump()}:}{Now \TT{dump()} method:}

\lstinputlisting{classes/16_3_msvc.asm}

\IFRU{Все очень просто, \TT{dump()} берет указатель на структуру состоящую из двух \Tint через \ECX, 
выдергивает оттуда две переменные и передает их в \printf.}
{Simple enough: \TT{dump()} taking pointer to the structure containing two \Tint's in \ECX,
takes two values from it and passing it into \printf.}

\IFRU{А если скомпилировать с оптимизацией (\Ox), то будет намного меньше всего:}
{The code is much shorter if compiled with optimization (\Ox):}

\lstinputlisting{classes/16_4_msvc_Ox.asm}

\IFRU{Вот и все. Единственное о чем еще нужно сказать, это о том что в функции \main, 
когда вызывался второй конструктор с двумя аргументами, за ним не корректировался стек при помощи 
\TT{add esp, X}. В то же время, в конце у конструктора вместо \RET имеется \TT{RET 8}.}
{That's all. One more thing to say is that stack pointer after second constructor calling wasn't corrected
with \TT{add esp, X}. Please also note that, constructor has \TT{ret 8} instead of \RET at the end.}

\IFRU{Это потому что здесь используется thiscall~\ref{thiscall}, который, вместе с stdcall~\ref{stdcall} 
(все это ~--- методы передачи аргументов через стек), предлагает вызываемой функции корректировать стек. 
Инструкция \TT{ret X} сначала прибавляет \TT{X} к \ESP, затем передает управление вызывающей функции.}
{That's all because here used thiscall~\ref{thiscall} calling convention, the method of passing values through the
stack, which is, together with stdcall~\ref{stdcall} method, offers to correct stack to callee 
rather then to caller.
\TT{ret x} instruction adding \TT{X} to \ESP, then passes control to caller function.}

\IFRU{См.также в соответствующем разделе о способах передачи аргументов через стек}
{See also section about calling conventions}~\ref{sec:callingconventions}.

\IFRU{Еще, кстати, нужно отметить, что именно компилятор решает, когда вызывать конструктор и деструктор ~--- 
но это итак известно из основ языка Си++.}
{It's also should be noted that compiler deciding when to call constructor and destructor ~--- but that's 
we already know from C++ language basics.}

\subsection{GCC}

\IFRU{В GCC 4.4.1 все почти так же, за исключением некоторых различий.}
{It's almost the same situation in GCC 4.4.1, with few exceptions.}

\lstinputlisting{classes/16_5_gcc.asm}

\newcommand{\URLAGNER}{\url{http://www.agner.org/optimize/calling_conventions.pdf}}

\IFRU{Здесь мы видим что применяется иной \IT{name mangling} характерный для стандартов 
GNU\footnote{Еще о name mangling разных компиляторов: \URLAGNER}. Во-вторых, указатель на экземпляр передается как первый аргумент функции ~--- конечно же, скрыто от программиста.}
{Here we see another \IT{name mangling} style, specific to GNU\footnote{One more document about different compilers name mangling types: \URLAGNER} standards. It's also can be noted that pointer to object is passed as first function argument ~--- hiddenly from programmer, of course.}

\IFRU{Это первый конструктор:}{First constructor:}

\begin{lstlisting}
                public _ZN1cC1Ev ; weak
_ZN1cC1Ev       proc near               ; CODE XREF: main+10

arg_0           = dword ptr  8

                push    ebp
                mov     ebp, esp
                mov     eax, [ebp+arg_0]
                mov     dword ptr [eax], 667
                mov     eax, [ebp+arg_0]
                mov     dword ptr [eax+4], 999
                pop     ebp
                retn
_ZN1cC1Ev       endp
\end{lstlisting}

\IFRU{Он просто записывает два числа по указателю переданному в первом (и единственном) аргументе.}
{What it does is just writes two numbers using pointer passed in first (and sole) argument.}

\IFRU{Второй конструктор:}{Second constructor:}

\begin{lstlisting}
                public _ZN1cC1Eii
_ZN1cC1Eii      proc near

arg_0           = dword ptr  8
arg_4           = dword ptr  0Ch
arg_8           = dword ptr  10h

                push    ebp
                mov     ebp, esp
                mov     eax, [ebp+arg_0]
                mov     edx, [ebp+arg_4]
                mov     [eax], edx
                mov     eax, [ebp+arg_0]
                mov     edx, [ebp+arg_8]
                mov     [eax+4], edx
                pop     ebp
                retn
_ZN1cC1Eii      endp
\end{lstlisting}

\IFRU{Это функция, аналог которой мог бы выглядеть так:}{This is a function, analog of which could be looks like:}

\begin{lstlisting}
void ZN1cC1Eii (int *obj, int a, int b)
{
  *obj=a;
  *(obj+1)=b;
};
\end{lstlisting}

\IFRU{... что, в общем, предсказуемо.}{... and that's completely predictable.}

\IFRU{И функция \TT{dump()}:}{Now \TT{dump()} function:}

\begin{lstlisting}
                public _ZN1c4dumpEv
_ZN1c4dumpEv    proc near

var_18          = dword ptr -18h
var_14          = dword ptr -14h
var_10          = dword ptr -10h
arg_0           = dword ptr  8

                push    ebp
                mov     ebp, esp
                sub     esp, 18h
                mov     eax, [ebp+arg_0]
                mov     edx, [eax+4]
                mov     eax, [ebp+arg_0]
                mov     eax, [eax]
                mov     [esp+18h+var_10], edx
                mov     [esp+18h+var_14], eax
                mov     [esp+18h+var_18], offset aDD ; "%d; %d\n"
                call    _printf
                leave
                retn
_ZN1c4dumpEv    endp
\end{lstlisting}

\IFRU{Эта функция \IT{во внутреннем представлении} имеет один аргумент, через который передается указатель на 
объект\footnote{экземпляр класса} (\IT{this}).}
{This function in its \IT{internal representation} has sole argument, 
used as pointer to the object (\IT{this}).}

\IFRU{Таким образом, если брать в учет только эти простые примеры, разница между MSVC и GCC 
в способе кодирования имен функций (\IT{name mangling}) и передаче указателя на экземпляр класса 
(через \ECX или через первый аргумент).}
{Thus, if to base our judgment on these simple examples, the difference between MSVC and GCC
is style of function names encoding (\IT{name mangling}) and passing pointer to object
(via \ECX register or via first argument).}

% TODO: RTTI

\section{\IFRU{Объединения (union)}{Unions}}

% done

\subsection{\IFRU{Пример генератора случайных чисел}{Pseudo-random number generator example}}

\IFRU{Если нам нужны случайные значения с плавающей запятой в интервале от 0 до 1, самое простое это взять
генератор ПСЧ вроде Mersenne twister выдающий случайные 32-битные числа в виде DWORD, преобразовать
это число в \Tfloat и затем разделить на \TT{RAND\_MAX} (\IT{0xffffffff} в данном случае) ~--- 
полученное число будет в интервале от 0 до 1.}
{If we need float random numbers from 0 to 1, the most simplest thing is to use random numbers generator like
Mersenne twister producing random 32-bit values in DWORD form, transform this value to \Tfloat and then
divide it by \TT{RAND\_MAX} (\IT{0xffffffff} in our case) ~--- value we got will be in 0..1 interval.}

\IFRU{Но как известно, операция деления это медленная операция почти всегда. 
Сможем ли мы избежать её, как в случае с делением через умножение?}
{But as we know, division operation is almost always very slow.
Will it be possible to get rid of it, as in case of division by multiplication?}
~\ref{sec:divisionbynine}

\IFRU{Вспомним состав числа с плавающей запятой: это бит знака, биты мантиссы и биты экпоненты. 
Для получения случайного числа, нам нужно просто заполнить случайными битами все биты мантиссы!}
{Let's remember what float number consisted of: sign bit, significand bits and exponent bits.
We need just to store random bits to significand bits for getting float number!}

\IFRU{Экспонента не может быть нулевой (иначе число будет денормализованным), 
так что в эти биты мы запишем \IT{01111111} ~--- 
это будет означать что экспонента равна единице. Далее заполняем мантиссу случайными битами, 
знак оставляем в виде 0 (что значит наше число положительное), и вуаля. 
Генерируемые числа будут в интервале от 1 до 2, так что нам еще нужно будет отнять единицу.}
{Exponent cannot be zero (number will be denormalized in this case), so we will store \IT{01111111} 
to exponent ~--- this mean exponent will be 1. Then fill significand with random bits, set sign bit to
0 (which mean positive number) and voilà.
Generated numbers will be in 1 to 2 interval, so we also should subtract 1 from it.}

\newcommand{\URLXOR}{\url{http://xor0110.wordpress.com/2010/09/24/how-to-generate-floating-point-random-numbers-efficiently}}

\IFRU{В моем примере\footnote{идея взята здесь: \URLXOR} 
применяется очень простой линейный конгруэнтный генератор случайных чисел, выдающий 32-битные числа.
Генератор инициализируется текущим временем в стиле UNIX.}
{Very simple linear congruential random numbers generator is used in my 
example\footnote{idea was taken from: \URLXOR}, producing 32-bit numbers. 
The PRNG initializing by current time in UNIX-style.}

\IFRU{Далее, тип \Tfloat представляется в виде \IT{union} ~--- это конструкция \CCpp позволяющая 
интерпретировать часть памти по-разному. В нашем случае, мы можем создать переменную типа \TT{union} 
и затем обращаться к ней как к \Tfloat или как к \IT{uint32\_t}. Можно сказать что это хак, причем грязный.}
{Then, \Tfloat type represented as \IT{union} ~--- that is the \CCpp construction allowing us
to interpret piece of memory differently typed. In our case, we are able to create a variable
of \TT{union} type and then access to it as it's \Tfloat or as it's \IT{uint32\_t}. 
It can be said, it's just a hack. A dirty one.}

\lstinputlisting{unions/FPU_PRNG.cpp}

MSVC 2010 (\Ox): 

\lstinputlisting{\IFRU{unions/FPU_PRNG_msvc_2010_Ox_ru.asm}{unions/FPU_PRNG_msvc_2010_Ox_en.asm}}

\IFRU{А результат GCC будет почти таким же.}{GCC producing very same code.}




% done

\section{\IFRU{Указатели на функции}{Pointers to functions}}
\label{sec:pointerstofunctions}

\IFRU{Указатель на функцию, в целом, как и любой другой указатель, просто адрес указывающий на начало функции 
в сегменте кода.}
{Pointer to function, as any other pointer, is just an address of function beginning in its code segment.}

\IFRU{Это применяется часто в т.н. callback-ах}{It is often used in callbacks}
\footnote{\url{http://en.wikipedia.org/wiki/Callback_(computer_science)}}.

\IFRU{Известные примеры:}{Well-known examples are:}

\begin{itemize}
\item
\qsort\footnote{\url{http://en.wikipedia.org/wiki/Qsort_(C_standard_library)}},
{\TT{atexit()}}\footnote{\url{http://www.opengroup.org/onlinepubs/009695399/functions/atexit.html}} \IFRU{из стандартной библиотеки Си}{from the standard C library}; 
\item
\IFRU{сигналы в *NIX ОС}{signals in *NIX OS}\footnote{\url{http://en.wikipedia.org/wiki/Signal.h}};
\item
\IFRU{запуск тредов}{thread starting}: \TT{CreateThread()} (win32), \TT{pthread\_create()} (POSIX);
\item
\IFRU{множество функций win32, например}{a lot of win32 functions, for example} \TT{EnumChildWindows()}\footnote{\url{http://msdn.microsoft.com/en-us/library/ms633494(VS.85).aspx}}.
\end{itemize}

\IFRU{Итак, функция \qsort это реализация алгоритма ``быстрой сортировки''. 
Функция может сортировать что угодно, 
любые типы данных, но при условии что вы имеете функцию сравнения двух элементов данных и 
\qsort может вызывать её.}
{So, \qsort function is a \CCpp standard library quicksort implemenation. The functions is able to sort
anything, any types of data, if you have a function for two elements comparison and \qsort is able
to call it.}

\IFRU{Эта функция сравнения может определяться так:}{The comparison function can be defined as:}

\begin{lstlisting}
int (*compare)(const void *, const void *)
\end{lstlisting}

\IFRU{Воспользуемся немного модифицированным примером, который я нашел вот}
{Let's use slightly modified example I found} \href{http://cplus.about.com/od/learningc/ss/pointers2_8.htm}
{\IFRU{здесь}{here}}:

\lstinputlisting{pointers_to_functions/17_1.c}

\IFRU{Компилируем в MSVC 2010 (я убрал некоторые части для краткости) с опцией \Ox}
{Let's compile it in MSVC 2010 (I omitted some parts for the sake of brefity) with \Ox option}:

\lstinputlisting{pointers_to_functions/17_2_msvc_Ox.asm}

\IFRU{Ничего особо удивительного здесь мы не видим. В качестве четвертого аргумента, 
в \qsort просто передается адрес метки \TT{\_comp}, где собственно и располагается функция \TT{comp()}.}
{Nothing surprising so far. As a fourth argument, an address of label \TT{\_comp} is passed, that's just a place
where function \TT{comp()} located.}

\IFRU{Как \qsort вызывает её?}{How \qsort calling it?}

\IFRU{Посмотрим в MSVCR80.DLL (эта DLL куда в MSVC вынесены функции из стандартных библиотек Си):}
{Let's take a look into this function located in MSVCR80.DLL (a MSVC DLL module with C standard library functions):}

\lstinputlisting{pointers_to_functions/17_3_MSVCR.lst}

\IFRU{\TT{comp} ~--- это четвертый аргумент функции. 
Здесь просто передается управление по адресу указанному в \TT{comp}. 
Перед этим подготавливается два аргумента для функции \TT{comp()}. Далее, проверяется результат её выполнения.}
{\TT{comp} ~--- is fourth function argument.
Here the control is just passed to the address in \TT{comp}.
Before it, two arguments prepared for \TT{comp()}. Its result is checked after its execution.}

\IFRU{Вот почему использование указателей на функции ~--- это опасно. 
Во-первых, если вызвать \qsort с неправильным указателем на функцию, 
то \qsort, дойдя до этого вызова, может передать управление неизвестно куда, 
процесс упадет, и эту ошибку можно будет найти не сразу.}
{That's why it's dangerous to use pointers to functions.
First of all, if you call \qsort with incorrect pointer to function, \qsort may pass control
to incorrect place, process may crash and this bug will be hard to find.}

\IFRU{Во-вторых, типизация callback-функции должна строго соблюдаться, 
вызов не той функции с не теми аргументами не того типа, 
может привести к плачевным результатам, 
хотя падение процесса это и не проблема ~--- а проблема это найти ошибку ~--- ведь компилятор 
на стадии компиляции может вас и не предупредить о потенциальных неприятностях.}
{Second reason is that callback function types should comply strictly, calling wrong function
with wrong arguments of wrong types may lead to serious problems, however, process crashing is not a 
big problem ~--- big problem is to determine a reason of crashing ~--- because compiler may be 
silent about potential trouble while compiling.}

\subsection{GCC}

\IFRU{Не слишком большая разница:}{Not a big difference:}

\begin{lstlisting}
                lea     eax, [esp+40h+var_28]
                mov     [esp+40h+var_40], eax
                mov     [esp+40h+var_28], 764h
                mov     [esp+40h+var_24], 2Dh
                mov     [esp+40h+var_20], 0C8h
                mov     [esp+40h+var_1C], 0FFFFFF9Eh
                mov     [esp+40h+var_18], 0FF7h
                mov     [esp+40h+var_14], 5
                mov     [esp+40h+var_10], 0FFFFCFC7h
                mov     [esp+40h+var_C], 43Fh
                mov     [esp+40h+var_8], 58h
                mov     [esp+40h+var_4], 0FFFE7960h
                mov     [esp+40h+var_34], offset comp
                mov     [esp+40h+var_38], 4
                mov     [esp+40h+var_3C], 0Ah
                call    _qsort
\end{lstlisting}

\IFRU{Функция \TT{comp()}}{\TT{comp()} function}:

\begin{lstlisting}
                public comp
comp            proc near

arg_0           = dword ptr  8
arg_4           = dword ptr  0Ch

                push    ebp
                mov     ebp, esp
                mov     eax, [ebp+arg_4]
                mov     ecx, [ebp+arg_0]
                mov     edx, [eax]
                xor     eax, eax
                cmp     [ecx], edx
                jnz     short loc_8048458
                pop     ebp
                retn
loc_8048458:
                setnl   al
                movzx   eax, al
                lea     eax, [eax+eax-1]
                pop     ebp
                retn
comp            endp
\end{lstlisting}

\IFRU{Реализация \qsort находится в \TT{libc.so.6}, и представляет собой просто враппер для \TT{qsort\_r()}.}
{\qsort implementation is located in \TT{libc.so.6} and it is in fact just a wrapper for \TT{qsort\_r()}.}

\IFRU{Она, в свою очередь, вызывает \TT{quicksort()}, где есть вызовы определенной нами функции через 
переданный указатель:}
{It will call then \TT{quicksort()}, where our defined function will be called via passed pointer:}

\IFRU{(файл libc.so.6, версия glibc ~--- 2.10.1)}{(File libc.so.6, glibc version ~--- 2.10.1)}

\begin{lstlisting}
.text:0002DDF6                 mov     edx, [ebp+arg_10]
.text:0002DDF9                 mov     [esp+4], esi
.text:0002DDFD                 mov     [esp], edi
.text:0002DE00                 mov     [esp+8], edx
.text:0002DE04                 call    [ebp+arg_C]
...
\end{lstlisting}


% done

\section{SIMD}

SIMD \IFRU{это акроним:}{is just acronym:} \IT{Single Instruction, Multiple Data}.

\IFRU{Как можно судить по названию, это обработка множества данных исполняя только одну инструкцию.}
{As it's said, it's multiple data processing using only one instruction.}

\IFRU{Как и FPU, эта подсистема процессора выглядит также отдельным процессором внутри x86.}
{Just as FPU, that CPU subsystem looks like separate processor inside x86.}

\IFRU{SIMD в x86 начался с MMX. Появилось 8 64-битных регистров MM0-MM7.}
{SIMD began as MMX in x86. 8 new 64-bit registers appeared: MM0-MM7.}

\IFRU{Каждый MMX-регистр может содержать 2 32-битных значения, 4 16-битных или же 8 байт. 
Например, складывая значения двух MMX-регистров, можно складывать одновременно 8 8-битных значений.}
{Each MMX register may hold 2 32-bit values, 4 16-bit values or 8 bytes.
For example, it is possible to add 8 8-bit values (bytes) simultaneously by adding two values in MMX-registers.}

\IFRU{Простой пример, это некий графический редактор, который хранит открытое изображение как двумерный массив. 
Когда пользователь меняет яркость изображения, редактору нужно, например, прибавить некий коэффициент 
ко всем пикселям, или отнять. 
Для простоты можно представить, что изображение у нас бело-серо-черное и каждый пиксель занимает один байт, 
то с помощью MMX можно менять яркость сразу у восьми пикселей.}
{One simple example is graphics editor, representing image as a two dimensional array.
When user change image brightness, the editor should add some coefficient to each pixel value, or to subtract.
For the sake of brevity, our image may be grayscale and each pixel defined by one 8-bit byte, then it's possible
to change brightness of 8 pixels simultaneously.}

\IFRU{Когда MMX только появилось, эти регистры на самом деле распологались в FPU-регистрах. 
Можно было использовать 
либо FPU либо MMX в одно и то же время. Можно подумать что Intel решило немного сэкономить на транзисторах, 
но на самом деле причина такого симбиоза проще ~--- более старая операционная система не знающая о дополнительных 
регистрах процессора не будет сохранять их во время переключения задач, а вот регистры FPU сохранять будет. 
Таким образом, процессор с MMX + старая операционная система + задача использующая MMX = все 
это может работать вместе.}
{When MMX appeared, these registers was actually located in FPU registers. 
It was possible to use either FPU or MMX at the same time. One might think, Intel saved on transistors,
but in fact, the reason of such symbiosis is simpler ~--- older operation system may not aware 
of additional CPU registers wouldn't save them at the context switching, but will save FPU registers.
Thus, MMX-enabled CPU + old operation system + process using MMX = that all will work together.}

SSE ~--- \IFRU{это расширение регистров до 128 бит, теперь уже отдельно от FPU.}{is extension of SIMD registers up to 128 bits, now separately from FPU.}

AVX ~--- \IFRU{расширение регистров до 256 бит.}{another extension to 256 bits.}

\IFRU{Немного о практическом применении.}{Now about practical usage.}

\IFRU{Конечно же, копирование блоков в памяти (\TT{memcpy}), сравнение (\TT{memcmp}), и подобное.}
{Of course, memory copying (\TT{memcpy}), memory comparing (\TT{memcmp}) and so on.}

\IFRU{Еще пример: имеется алгоритм шифрования DES, который берет 64-битный блок, 56-битный ключ, 
шифрует блок с ключем и образуется 64-битный результат.
Алгоритм DES можно легко представить в виде очень большой электронной цифровой схемы, 
с проводами, элементами И, ИЛИ, НЕ.}
{One more example: we got DES encryption algorithm, it takes 64-bit block, 56-bit key, encrypt block and produce 64-bit result.
DES algorithm may be considered as a very large electronic circuit, with wires and AND/OR/NOT gates.}

\label{bitslicedes}
\newcommand{\URLBS}{\url{http://www.darkside.com.au/bitslice/}}

\IFRU{Идея bitslice DES\footnote{\URLBS} ~--- это обработка сразу группы блоков и ключей одновременно. 
Скажем, на x86 перменная типа \IT{unsigned int} вмещает в себе 32 бита, так что там можно хранить 
промежуточные результаты сразу для 32-х блоков-ключей, используя 64+56 переменных типа \IT{unsigned int}.}
{Bitslice DES\footnote{\URLBS} ~--- is an idea of processing group of blocks and keys simultaneously.
Let's say, variable of type \IT{unsigned int} on x86 may hold up to 32 bits, so, it's possible to store there
intermediate results for 32 blocks-keys pairs simultaneously, using 64+56 variables of \IT{unsigned int} type.}

\IFRU{Я написал утилиту для перебора паролей/хешей Oracle RDBMS (которые основаны на алгоритме DES), 
переделав алгоритм bitslice DES для SSE2 и AVX ~--- и теперь возможно шифровать одновременно 
128 или 256 блоков-ключей:}
{I wrote an utility to brute-force Oracle RDBMS passwords/hashes (ones based on DES),
slightly modified bitslice DES algorithm for SSE2 and AVX ~--- now it's possible to encrypt 128 
or 256 block-keys pairs simultaneously.}

\url{http://conus.info/utils/ops_SIMD/}
 
\subsection{\IFRU{Векторизация}{Vectorization}}

\newcommand{\URLVEC}{\href{http://en.wikipedia.org/wiki/Vectorization_(computer_science)}{Wikipedia: vectorization}}

\IFRU{Векторизация\footnote{\URLVEC} это когда у вас есть цикл, который берет на вход несколько массивов и выдает, 
например, один массив данных. 
Тело цикла берет некоторые элементы из входных массивов, что-то делает с ними и кладет в выходной. 
Важно что операция применяемая ко всем элементам одна и та же. 
Векторизация ~--- это обрабатывать несколько элементов одновременно.}
{Vectorization\footnote{\URLVEC}, for example, is when you have a loop taking couple of arrays at input and producing one array.
Loop body takes values from input arrays, do something and put result into output array.
It's important that there is only one single operation applied to each element.
Vectorization ~--- is to process several elements simultaneously.}

/IFRU{Например:}{For example:}

\begin{lstlisting}
for (i = 0; i < 1024; i++)
{
    C[i] = A[i]*B[i];
}
\end{lstlisting}

\IFRU{Этот кусок кода берет элементы из A и B, перемножает и сохраняет результат в C.}
{This piece of code takes elements from A and B, multiplies them and save result into C.}

\newcommand{\PMULLD}{\IT{PMULLD} (\IT{\IFRU{Multiply Packed Signed Dword Integers and Store Low Result}
{Перемножить запакованные знаковые DWORD и сохранить младшую часть результата}})}
\newcommand{\PMULHW}{\TT{PMULHW} (\IT{\IFRU{Multiply Packed Signed Integers and Store High Result}
{Перемножить апакованные знаковые DWORD и сохранить старшую часть результата}})}

\IFRU{Если представить что каждый элемент массива ~--- это 32-битный \Tint, то их можно загружать сразу 
по 4 из А в 128-битный XMM-регистр, 
из B в другой XMM-регистр и выполнив инстукцию \PMULLD и \PMULHW, можно получить 4 64-битных 
произведения\footnote{результат умножения} сразу.}
{If each array element we have is 32-bit \Tint, then it's possible to load 4 elements from A into 128-bit 
XMM-register, from B to another XMM-registers, and by executing \PMULLD and \PMULHW, 
it's possible to get 4 64-bit products\footnote{multiplication result} at once.}

\IFRU{Таким образом, тело цикла исполняется 1024/4 раза вместо 1024, что в 4 раза меньше, и, конечно, быстрее.}
{Thus, loop body count is 1024/4 instead of 1024, that's 4 times less and, of course, faster.}

\newcommand{\URLINTELVEC}{\href{http://www.intel.com/intelpress/sum_vmmx.htm}{Excerpt: Effective Automatic Vectorization}}

\IFRU{Некоторые компиляторы умеют делать автоматическую векторизацию в простых случаях, 
например Intel C++\footnote{Еще о том как Intel C++ умеет автоматически векторизировать циклы: \URLINTELVEC}.}
{Some compilers can do vectorization automatically in some simple cases, 
for example, Intel C++\footnote{More about Intel C++ automatic vectorization: \URLINTELVEC}.}

\IFRU{Я написал очень простую функцию:}{I wrote tiny function:}

\begin{lstlisting}
int f (int sz, int *ar1, int *ar2, int *ar3)
{
	for (int i=0; i<sz; i++)
		ar3[i]=ar1[i]+ar2[i];

	return 0;
};
\end{lstlisting}

\subsubsection{Intel C++}

\IFRU{Компилирую при помощи}{Let's compile it with} Intel C++ 11.1.051 win32:

\begin{verbatim}
icl intel.cpp /QaxSSE2 /Faintel.asm /Ox
\end{verbatim}

\IFRU{Имеем такое (в \IDA):}{We got (in \IDA):}

\lstinputlisting{SIMD/18_1_en.asm}

\IFRU{Инструкции имеющие отношение к SSE2 это:}{SSE2-related instructions are:}

\MOVDQU (\IT{Move Unaligned Double Quadword}) ~--- \IFRU{она просто загружает 16 байт из памяти в XMM-регистр}
{it just load 16 bytes from memory into XMM-register}.

\newcommand{\URLDSA}{\href{http://en.wikipedia.org/wiki/Data_structure_alignment}{Wikipedia: data structure alignment}}

\PADDD (\IT{Add Packed Integers}) ~--- 
\IFRU{складывает сразу 4 пары чисел и оставляет в первом операнде результат. 
Кстати, если произойдет переполнение, то исключения не произойдет и никакие флаги не установятся, 
запишутся просто младшие 32 бита результата. 
Если один из операндов \PADDD ~--- адрес значения в памяти, 
то требуется чтобы адрес был выровнен по 16-байтной границе. Если он не выровнен, произойдет исключение
\footnote{О выравнивании данных см. также: \URLDSA}.}
{adding 4 pairs of numbers and leaving result in first operand.
By the way, no exception raised in case of overflow and no flags will be set, just low 32-bit of result will
be stored.
If one of \PADDD operands ~--- address of value in memory,
address should be aligned by 16-byte border. If it's not aligned, exception will be raised
\footnote{More about data aligning: \URLDSA}.}

\MOVDQA (\IT{Move Aligned Double Quadword}) ~--- \IFRU{тоже что и \MOVDQU, только подразумевает 
что адрес в памяти выровнен по 16-байтной границе. 
Если он не выровнен, произойдет исключение. 
\MOVDQA работает быстрее чем \MOVDQU, но требует вышеозначенного.}
{the same as \MOVDQU, but requires address of value in memory to be aligned by 16-bit border.
If it's not aligned, exception will be raised.
\MOVDQA works faster than \MOVDQU, but requires aforesaid.}

\IFRU{Итак, эти SSE2-инструкции исполнятся только в том случае если еще осталось просуммировать 
4 пары переменных типа \Tint плюс если указатель \TT{ar3} выровнен по 16-байтной границе.}
{So, these SSE2-instructions will be executed only in case if there are more 4 pairs to work on
plus pointer \TT{ar3} is aligned on 16-byte border.}

\IFRU{Более того, если еще и \TT{ar2} выровнен по 16-байтной границе, то будет выполняться этот кусок:}
{More than that, if \TT{ar2} is aligned on 16-byte border too, this piece of code will be executed:}

\begin{lstlisting}
                movdqu  xmm0, xmmword ptr [ebx+edi*4] ; ar1+i*4
                paddd   xmm0, xmmword ptr [esi+edi*4] ; ar2+i*4
                movdqa  xmmword ptr [eax+edi*4], xmm0 ; ar3+i*4
\end{lstlisting}

\IFRU{А иначе, значение из \TT{ar2} загрузится в \XMMZERO используя инструкцию \MOVDQU, 
которая не требует выровненного указателя, зато может работать чуть медленнее:}
{Otherwise, value from \TT{ar2} will be loaded to \XMMZERO using \MOVDQU,
it doesn't require aligned pointer, but may work slower:}

\begin{lstlisting}
                movdqu  xmm1, xmmword ptr [ebx+edi*4] ; ar1+i*4
                movdqu  xmm0, xmmword ptr [esi+edi*4] ; ar2+i*4 is not 16-byte aligned, so load it to xmm0
                paddd   xmm1, xmm0
                movdqa  xmmword ptr [eax+edi*4], xmm1 ; ar3+i*4
\end{lstlisting}

\IFRU{А во всех остальных случаях, будет исполняться код, который был бы как если бы не была 
включена поддержка SSE2.}
{In all other cases, non-SSE2 code will be executed.}

\subsubsection{GCC}

\newcommand{\URLGCCVEC}{\url{http://gcc.gnu.org/projects/tree-ssa/vectorization.html}}

\IFRU{Но и GCC умеет кое-что векторизировать\footnote{Подробнее о векторизации в GCC: \URLGCCVEC}, 
если компилировать с опциями \TT{-O3} и включить поддержку SSE2: \TT{-msse2}.}
{GCC may also vectorize in some simple cases\footnote{More about GCC vectorization support: \URLGCCVEC},
if to use \TT{-O3} option and to turn on SSE2 support: \TT{-msse2}.}

\IFRU{Вот что вышло}{What we got} (GCC 4.4.1):

\lstinputlisting{SIMD/18_2_gcc_O3.asm}

\IFRU{Почти то же самое, хотя и не так дотошно как Intel C++.}
{Almost the same, however, not as meticulously as Intel C++ doing it.}

\subsection{\IFRU{Реализация \strlen() при помощи SIMD}{SIMD \strlen() implementation}}

\newcommand{\URLMSDNSSE}{\href{http://msdn.microsoft.com/en-us/library/y0dh78ez(VS.80).aspx}{MSDN: MMX, SSE, and SSE2 Intrinsics}}

\IFRU{Прежде всего, следует заметить, что SIMD-инструкции можно вставлять в \CCpp код при помощи специальных 
макросов\footnote{\URLMSDNSSE}. В MSVC, часть находится в файле \TT{intrin.h}.}
\IFRU{It should be noted that SIMD-instructions may be inserted into \CCpp code via 
special macros\footnote{\URLMSDNSSE}.
As of MSVC, some of them are located in \TT{intrin.h} file.}

\IFRU{Имеется возможность реализовать функцию \strlen\footnote{strlen() ~--- стандартная функция Си 
для подсчета длины строки} при помощи SIMD-инструкций, работающий в 2-2.5 раза быстрее обычной реализации. 
Эта функция будет загружать в XMM-регистр сразу 16 байт и проверять каждый на ноль.}
{It is possible to implement \strlen function\footnote{strlen() ~--- standard C library function for calculating
string length} using SIMD-instructions, working 2-2.5 times faster than usual implementation.
This function will load 16 characters into XMM-register and check each against zero.}

\lstinputlisting{SIMD/18_3.c}

\newcommand{\URLSTRLEN}{http://www.strchr.com/sse2\_optimised\_strlen}

\IFRU{(пример базируется на исходнике \href{\URLSTRLEN}{отсюда}).}
{(the example is based on source code from \href{\URLSTRLEN}{there}).}

\IFRU{Компилируем в MSVC 2010 с опцией \Ox:}{Let's compile in MSVC 2010 with \Ox option:}

\lstinputlisting{SIMD/18_4_msvc_Ox.asm}

\IFRU{Итак, прежде всего, мы проверяем указатель \TT{str}, выровнен ли он по 16-байтной границе. 
Если нет, то мы вызовем обычную реализацию \strlen.}
{First of all, we check \TT{str} pointer, if it's aligned by 16-byte border.
If not, let's call usual \strlen implementation.}

\IFRU{Далее мы загружаем по 16 байт в регистр \XMMONE при помощи команды \MOVDQA.}
{Then, load next 16 bytes into \XMMONE register using \MOVDQA instruction.}

\IFRU{Наблюдательный читатель может спросить, почему в этом месте мы не можем использовать \MOVDQU, 
которая может загружать откуда угодно не взирая на факт, выровнен ли указатель?}
{Observant reader might ask, why \MOVDQU cannot be used here, because it can load data from the memory
regardless the fact if the pointer aligned or not.}

\IFRU{Да, можно было бы сделать вот как: если указатель выровнен, загружаем используя \MOVDQA, 
иначе используем работающую чуть медленнее \MOVDQU.}
{Yes, it might be done in this way: if pointer is aligned, load data using \MOVDQA,
if not ~--- use slower \MOVDQU.}

\IFRU{Однако здесь кроется не сразу заметная проблема, которая проявляется вот в чем:}
{But here we are may stick into hard to notice problem:}

\newcommand{\URLPAGE}{\url{http://en.wikipedia.org/wiki/Page_(computer_memory)}}

\IFRU{В ОС линии Windows NT, и не только, память выделяется страницами по 4 KiB (4096 байт). 
Каждый win32-процесс якобы имеет в наличии 4 GiB, но на самом деле, 
только некоторые части этого адресного пространства присоеденены к реальной физической памяти. 
Если процесс обратится к блоку памяти, которого не существует, сработает исключение. 
Так работает виртуальная память\footnote{\URLPAGE}.}
{In Windows NT line of operation systems, but not limited to it, memory allocated by pages of 4 KiB (4096 bytes).
Each win32-process have ostensibly 4 GiB, but in fact, only some parts
of address space are connected to real physical memory.
If the process accessing to the absent memory block, exception will be raised.
That's how virtual memory works\footnote{\URLPAGE}.}

\IFRU{Так вот, функция, читающая сразу по 16 байт, имеет возможность нечаянно вылезти за границу 
выделенного блока памяти. 
Предположим, ОС выделила программе 8192 (0x2000) байт по адресу 0x008c0000. 
Таким образом, блок занимает байты с адреса 0x008c0000 по 0x008c1fff включительно.}
{So, some function loading 16 bytes at once, may step over a border of allocated memory block.
Let's consider, OS allocated 8192 (0x2000) bytes at the address 0x008c0000.
Thus, the block is the bytes starting from address 0x008c0000 to 0x008c1fff inclusive.}

\IFRU{За этим блоком, то есть начиная с адреса 0x008c2000 нет вообще ничего, т.е., ОС не выделяла там память. 
Обращение к памяти начиная с этого адреса вызовет исключение.}
{After that block, that is, starting from address 0x008c2008 there are nothing at all, e.g., OS not allocated
any memory there. Attempt to access a memory starting from that address will raise exception.}

\IFRU{И предположим, что программа хранит некую строку из, скажем, пяти символов почти в самом конце блока, 
что не является преступлением:}
{And let's consider, the program holding some string containing 5 characters almost at the end of block,
and that's not a crime.}

\begin{center}
  \begin{tabular}{ | l | l | }
    \hline
        0x008c1ff8 & 'h' \\
        0x008c1ff9 & 'e' \\
        0x008c1ffa & 'l' \\
        0x008c1ffb & 'l' \\
        0x008c1ffc & 'o' \\
        0x008c1ffd & '\textbackslash{}x00' \\
        0x008c1ffe & \IFRU{здесь случайный мусор}{random noise} \\
        0x008c1fff & \IFRU{здесь случайный мусор}{random noise} \\
    \hline
  \end{tabular}
\end{center}

\IFRU{В обычных условиях, программа вызвает \strlen передав ей указатель на строку \TT{'hello'} 
лежащую по адресу 0x008c1ff8. 
\strlen будет читать по одному байту до 0x008c1ffd, где ноль, и здесь она закончит работу.}
{So, in usual conditions the program calling \strlen passing it a pointer to string \TT{'hello'} 
lying in memory at address 0x008c1ff8.
\strlen will read one byte at a time until 0x008c1ffd, where zero-byte, and so here it will stop working.}

\IFRU{Теперь, если мы напишем свою реализацию \strlen читающую сразу по 16 байт, с любого адреса, 
будь он выровнен по 16-байтной границе или нет, 
\MOVDQU попытается загрузить 16 байт с адреса 0x008c1ff8 по 0x008c2008, и произойдет исключение. 
Это ситуация которой, конечно, хочется избежать.}
{Now if we implement own \strlen reading 16 byte at once, starting at any address, will it be alligned or not,
\MOVDQU may attempt to load 16 bytes at once at address 0x008c1ff8 up to 0x008c2008, 
and then exception will be raised.
That's the situation to be avoided, of course.}

\IFRU{Поэтому мы будем работать только с адресами выровненными по 16 байт, что в сочетании со знанием 
что размер страницы также как правило выровнен по 16 байт, 
даст некоторую гарантию что наша функция не будет пытаться читать из мест в невыделенной памяти.}
{So then we'll work only with the addresses aligned by 16 byte border, what in combination with a knowledge
of operation system page size is usually aligned by 16 byte too, give us some warranty our function will not
read from unallocated memory.}

\IFRU{Вернемся к нашей функции}{Let's back to our function}.

\verb|_mm_setzero_si128()| ~--- \IFRU{это макрос, генерирующий \TT{pxor xmm0, xmm0} ~--- инструкция просто обнуляет регистр \XMMZERO.}
{is a macro generating \TT{pxor xmm0, xmm0} ~--- instruction just clear /XMMZERO register}

\verb|_mm_load_si128()| ~--- \IFRU{это макрос для \MOVDQA, он просто загружает 16 байт по адресу из указателя в \XMMONE.}
{is a macro for \MOVDQA, it just loading 16 bytes from the address in \XMMONE.}

\verb|_mm_cmpeq_epi8()| ~--- \IFRU{это макрос для \PCMPEQB, это инструкция которая 
побайтово сравнивает значения из двух XMM регистров.} 
{is a macro for \PCMPEQB, is an instruction comparing two XMM-registers bytewise.}

\IFRU{И если какой-то из байт равен другому, то в результирующем значении будет выставлено на месте этого 
байта 0xff, либо 0, если байты не были равны.}
{And if some byte was equals to other, there will be 0xff at this place in result or 0 if otherwise.}

\IFRU{Например.}{For example.}

\begin{verbatim}
XMM1: 11223344556677880000000000000000
XMM0: 11ab3444007877881111111111111111
\end{verbatim}

\IFRU{После исполнения \TT{pcmpeqb xmm1, xmm0}, регистр \XMMONE будет содержать:}
{After \TT{pcmpeqb xmm1, xmm0} execution, \XMMONE register will contain:}

\begin{verbatim}
XMM1: ff0000ff0000ffff0000000000000000
\end{verbatim}

\IFRU{Эта инструкция в нашем случае, сравнивает каждый 16-байтный блок с блоком состоящим из 16-и нулевых байт, 
выставленным в \XMMZERO при помощи \TT{pxor xmm0, xmm0}.}
{In our case, this instruction comparing each 16-byte block with the block of 16 zero-bytes,
was set in \XMMZERO by \TT{pxor xmm0, xmm0}.}

\IFRU{Следующий макрос \TT{\_mm\_movemask\_epi8()} ~--- это инструкция \TT{PMOVMSKB}.}
{The next macro is \TT{\_mm\_movemask\_epi8()} ~--- that is \TT{PMOVMSKB} instruction.}

\IFRU{Она очень удобна как раз для использования в паре с \PCMPEQB.}
{It is very useful if to use it with \PCMPEQB.}

\TT{pmovmskb eax, xmm1}

\IFRU{Эта инструкция выставит самый первый бит \EAX в еденицу, если старший бит первого байта в 
регистре \XMMONE является единицей. 
Иными словами, если первый байт в регистре \XMMONE является 0xff, то первый бит в \EAX будет также единицей, 
иначе нулем.}
{This instruction will set first \EAX bit into 1 if most significant bit of the first byte in \XMMONE is 1.
In other words, if first byte of \XMMONE register is 0xff, first \EAX bit will be set to 1 too.}

\IFRU{Если второй байт в регистре \XMMONE является 0xff, то второй бит в \EAX также будет единицей. 
Иными словами, инструкция отвечает на вопрос, \IT{какие из байт в \XMMONE являются 0xff?}
В результате приготовит 16 бит и запишет в \EAX. Остальные биты в \EAX обнулятся.}
{If second byte in \XMMONE register is 0xff, then second \EAX bit will be set to 1 too.
In other words, the instruction is answer to the question \IT{which bytes in \XMMONE are 0xff?}
And will prepare 16 bits in \EAX. Other \EAX bits will be cleared.}

\IFRU{Кстати, не забывайте также вот о какой особенности нашего алгоритма:}
{By the way, do not forget about this feature of our algorithm:}

\IFRU{На вход может прийти 16 байт вроде}{There might be 16 bytes on input like} \TT{hello\textbackslash{}x00garbage\textbackslash{}x00ab}

\IFRU{Это строка \TT{'hello'}, после нее терминирующий ноль, затем немного мусора в памяти.}
{It's a \TT{'hello'} string, terminating zero after and some random noise in memory.}

\newcommand{\MSBFOOTNOTE}{\footnote{most significant bit}}
\newcommand{\LSBFOOTNOTE}{\footnote{least significant bit}}

\IFRU{Если мы загрузим эти 16 байт в \XMMONE и сравним с нулевым \XMMZERO, то в итоге получим такое 
(я использую здесь порядок с MSB\MSBFOOTNOTE до LSB\LSBFOOTNOTE):}
{If we load these 16 bytes into \XMMONE and compare them with zeroed \XMMZERO, we will get something like
(I use here order from MSB\MSBFOOTNOTE to LSB\LSBFOOTNOTE):}

\begin{verbatim}
XMM1: 0000ff00000000000000ff0000000000
\end{verbatim}

\IFRU{Это означает что инструкция сравнения обнаружила два нулевых байта, что и не удивительно.}
{This mean, the instruction found two zero bytes, and that's not surprising.}

\IFRU{\TT{PMOVMSKB} в нашем случае подготовит \EAX вот так (в двоичном представлении):} 
{\TT{PMOVMSKB} in our case will prepare \EAX like (in binary representation):} \IT{0010000000100000b}.

\IFRU{Совершенно очевидно что далее наша функция должна учитывать только первый встретившийся ноль 
и игнорировать все остальное.}
{Obviously, our function should consider only first zero and ignore others.}

\IFRU{Следующая инструкция}{The next instruction} ~--- \TT{BSF} (\IT{Bit Scan Forward}). 
\IFRU{Это инструкция находит самый младший бит во втором операнде и записывает его позицию в первый операнд.}
{This instruction find first bit set to 1 and stores its position into first operand.}

\begin{verbatim}
EAX=0010000000100000b
\end{verbatim}


\IFRU{После исполнения этой инструкции \TT{bsf eax, eax}, в \EAX будет 5, что означает, 
что единица найдена в пятой позиции (считая с нуля).}
{After \TT{bsf eax, eax} instruction execution, \EAX will contain 5, this mean, 
1 found at 5th bit position (starting from zero).}

\IFRU{Для использования этой инструкции, в MSVC также имеется макрос}
{MSVC has a macro for this instruction:}\TT{\_BitScanForward}.

\IFRU{А дальше все просто. Если нулевой байт найден, его позиция прибавляется к тому что 
мы уже насчитали и возвращается результат.}
{Now it's simple. If zero byte found, its position added to what we already counted and now we have 
ready to return result.}

\IFRU{Почти всё.}{Almost all.}

\IFRU{Кстати, следует также отметить, что компилятор MSVC сгенерировал два тела цикла сразу, для оптимизации.}
{By the way, it's also should be noted, MSVC compiler emitted two loop bodies side by side, for optimization.}

\IFRU{Кстати, в SSE 4.2 (который появился в Intel Core i7) все эти манипуляции со строками могут быть еще проще:}
{By the way, SSE 4.2 (appeared in Intel Core i7) offers more instructions where these string manipulations might be
even easier:}\url{http://www.strchr.com/strcmp\_and\_strlen\_using\_sse\_4.2}



% done

\section{x86-64}

\IFRU{Это расширение x86-архитуктуры до 64 бит.}{It's a 64-bit extension to x86-architecture.}

\IFRU{С точки зрения начинающего reverse engineer-а, наиболее важные отличия от 32-битного x86 это:}
{From the reverse engineer's perspective, most important differences are:}

\begin{itemize}

\item
\IFRU{Почти все регистры (кроме FPU и SIMD) расширены до 64-бит и получили префикс r-. 
И еще 8 регистров добавлено. 
В итоге имеются эти регистры общего пользования:}
{Almost all registers (except FPU and SIMD) are extended to 64 bits and got r- prefix.
8 additional registers added.
Now general purpose registers are:} \TT{rax}, \TT{rbx}, \TT{rcx}, \TT{rdx}, 
\TT{rbp}, \TT{rsp}, \TT{rsi}, \TT{rdi}, \TT{r8}, \TT{r9}, \TT{r10}, 
\TT{r11}, \TT{r12}, \TT{r13}, \TT{r14}, \TT{r15}. 

\IFRU{К ним также можно обращаться так же как и прежде. Например, для доступа к младшим 32 битам \TT{RAX} 
можно использовать \EAX.}
{It's still possible to access to \IT{older} register parts as usual. 
For example, it's possible to access lower 32-bit part of \TT{RAX} using \EAX.}

\IFRU{У новых регистров \TT{r8-r15} также имеются их \IT{младшие части}: \TT{r8d-r15d} 
(младшие 32-битные части), 
\TT{r8w-r15w} (младшие 16-битные части), \TT{r8b-r15b} (младшие 8-битные части).}
{New \TT{r8-r15} registers also has its \IT{lower parts}: \TT{r8d-r15d} (lower 32-bit parts),
\TT{r8w-r15w} (lower 16-bit parts), \TT{r8b-r15b} (lower 8-bit parts).}

\IFRU{Удвоено количество SIMD-регистров: с 8 до 16:}
{SIMD-registers number are doubled: from 8 to 16:} \TT{XMM0-XMM15}.

\item
\IFRU{В win64 передача всех параметров немного иная, это немного похоже на fastcall~\ref{fastcall}. 
Первые 4 аргумента записываются в регистры \TT{RCX}, \TT{RDX}, \TT{R8}, \TT{R9}, а остальные ~--- в стек. 
Вызывающая функция также должна подготовить место из 32 байт чтобы вызываемая функция могла сохранить 
там первые 4 аргумента и использовать эти регистры по своему усмотрению. 
Короткие функции могут использовать аргументы прямо из регистров, но б\'{о}льшие функции могут сохранять 
их значения на будущее.}
{In Win64, function calling convention is slightly different, somewhat resembling fastcall~\ref{fastcall}.
First 4 arguments stored in \TT{RCX}, \TT{RDX}, \TT{R8}, \TT{R9} registers, others ~--- in stack.
Caller function should also allocate 32 bytes so the callee may save there 4 first arguments and use these 
registers for own needs.
Short functions may use arguments just from registers, but larger may save their values into stack.}

\IFRU{См.также в соответствующем разделе о способах передачи аргументов через стек}
{See also section about calling conventions}~\ref{sec:callingconventions}.

\item
\IFRU{Сишный \Tint остается 32-битным для совместимости.}{C \Tint type is still 32-bit for compatibility.}

\item
\IFRU{Все указатели теперь 64-битные.}{All pointers are 64-bit now.}

\end{itemize}

\IFRU{Из-за того что регистров общего пользования теперь вдвое больше, у компиляторов теперь больше 
свободного места для маневра называемого \IT{register allocation}\footnote{распределение переменных по регистрам}. 
Для нас это означает, что в итоговом коде будет меньше локальных переменных.}
{Since now registers number are doubled, compilers has more space now for maneuvering calling 
\IT{register allocation}\footnote{assigning variables to registers}.
What it meanings for us, emitted code will contain less local variables.}

\IFRU{Для примера, функция вычисляющая первый S-бокс алгоритма шифрования DES, 
она обрабатывает сразу 32/64/128/256 значений, в зависимости от типа \TT{DES\_type} (uint32, uint64, SSE2 или AVX), 
методом bitslice DES (больше об этом методе читайте здесь~\ref{bitslicedes}):}
{For example, function calculating first S-box of DES encryption algorithm, it processing
32/64/128/256 values at once (depending on \TT{DES\_type} type (uint32, uint64, SSE2 or AVX)) 
using bitslice DES method
(read more about this method here ~\ref{bitslicedes}):}

\lstinputlisting{x64/19_1.c}

\IFRU{Здесь много локальных переменных. Конечно, далеко не все они будут в локальном стеке. 
Компилируем обычным MSVC 2008 с опцией \Ox:}
{There is a lot of local variables. Of course, not them all will be in local stack.
Let's compile it with MSVC 2008 with \Ox option:}

\lstinputlisting{x64/19_2_msvc_Ox.asm}

\IFRU{5 переменных компилятору пришлось разместить в локальном стеке.}
{5 variables was allocated in local stack by compiler.}

\IFRU{Теперь попробуем то же самое только в 64-битной версии MSVC 2008:}
{Now let's try the same thing in 64-bit version of MSVC 2008:}

\lstinputlisting{x64/19_3_msvc_x64.asm}

\IFRU{Компилятор ничего не выделил в локальном стеке, а \TT{x36} это синоним для \TT{a5}.}
{Nothing allocated in local stack by compiler, \TT{x36} is synonym for \TT{a5}.}

\IFRU{Кстати, видно что функция сохраняет регистры \TT{RCX}, \TT{RDX} в отведенных для 
этого вызываемой функцией местах, 
а \TT{R8} и \TT{R9} не сохраняет, а начинает использовать их сразу.}
{By the way, we can see here, the function saved \TT{RCX} and \TT{RDX} registers in allocated by caller space,
but \TT{R8} and \TT{R9} are not saved but used from the beginning.}

\IFRU{Кстати, существуют процессоры с еще большим количеством регистров общего использования, например, 
Itanium ~--- 128 регистров.}
{By the way, there are CPUs with much more general purpose registers, Itanium, for example ~---
128 registers.}




\chapter{\IFRU{Еще кое-что}{Couple things to add}}

% done

\section{\IFRU{Инструкция LEA}{LEA instruction}}
\label{sec:LEA}

\newcommand{\URLAM}{\url{http://en.wikipedia.org/wiki/Addressing_mode}}

\IFRU
{\LEA (\IT{Load Effective Address}) это инструкция которая задумывалась вовсе не для складывания чисел, 
а для формирования адреса например из указателя на массив и прибавления индекса к нему\footnote{См. также: \URLAM}.}
{\LEA (\IT{Load Effective Address}) is instruction intended not for values summing but for address forming, 
for example, for forming address of array element by adding array address, element index, with 
multiplication of element size\footnote{See also: \URLAM}.}

\IFRU{Важная особенность \LEA в том что производимые ею вычисления не модифицируют флаги.}
{Important property of \LEA instruction is that it do not alter processor flags.}

% TODO: дописать про x*n+m

\begin{lstlisting}
int f(int a, int b)
{
	return a*8+b;
};
\end{lstlisting}

\IFRU{Компилируем в MSVC 2010 с \Ox:}{MSVC 2010 with \Ox option:}

\begin{lstlisting}
_a$ = 8							; size = 4
_b$ = 12						; size = 4
_f	PROC
	mov	eax, DWORD PTR _b$[esp-4]
	mov	ecx, DWORD PTR _a$[esp-4]
	lea	eax, DWORD PTR [eax+ecx*8]
	ret	0
_f	ENDP
\end{lstlisting}

% done

\section{\IFRU{Пролог и эпилог в функции}{Function prologue and epilogue}}
\label{sec:prologepilog}

\IFRU{Пролог функции это инструкции в самом начале функции. Как правило это что-то вроде:}
{Function prologue is instructions at function start. It is often something like this:}

\begin{lstlisting}
    push    ebp
    mov     ebp, esp
    sub     esp, X
\end{lstlisting}

\IFRU
{Эти инструкции делают следующее: сохраняют значение регистра \EBP на будущее, выставляют \EBP равным \ESP, 
затем подготавливают место в стеке для хранения локальных переменных.}
{What these instruction do: save \EBP register value, set \EBP to \ESP and then allocate space in stack 
for local variables.}

\IFRU{\EBP сохраняет свое значение на протяжении всей функции, он будет использоваться здесь для доступа 
к локальным переменным и аргументам. Можно было бы использовать и \ESP, но он постоянно меняется и 
это не очень удобно.}
{\EBP value is fixed over a period of function execution and it will be used for local variables and 
arguments access. 
One can use \ESP, but it changing over time and it is not handy.}

\IFRU{Эпилог функции анулирует выделенное место в стеке, возвращает значение \EBP на то что было и возвращает 
управление в вызывающую функцию:}
{Function epilogue annuled allocated space in stack, returning \EBP value to initial state 
and returning control flow to callee:}

\begin{lstlisting}
    mov    esp, ebp
    pop    ebp
    ret    0
\end{lstlisting}

\IFRU{Наличие эпилога и пролога может несколько ухудшить эффективность рекурсии.

Например, однажды я написал функцию для поиска нужного узла дерева. 
Рекурсивно она выглядела очень красиво, но из-за того что при каждом вызове тратилось время на эпилог и пролог, 
все это работало в несколько раз медленнее чем та же функция но без рекурсии.}
{Epilogue and prologue can make recursion performance worse.

For example, once upon a time I wrote a function to seek right tree node. 
As a recursive function it would look stylish but because some time was spent at each function call
for prologue/epilogue, it was working couple of times slower than the implementation 
without recursion.}

\newcommand{\URLT}{\url{http://en.wikipedia.org/wiki/Tail_call}}
\IFRU
{Кстати, поэтому есть такая вещь как хвостовая рекурсия\footnote{\URLT}: 
когда компилятор или интерпретатор превращает рекурсию (с которой возможно это проделать: 
\IT{хвостовую}) в итерацию для эффективности.}
{By the way, that is the reason of tail call\footnote{\URLT} existence: when compiler (or interpreter) 
transforms recursion (with which it's possible: \IT{tail recursion}) into iteration for efficiency.}

% done

\section{npad}
\label{sec:npad}

\IFRU{Это макрос в ассемблере, для выравнивания некоторой метки по некоторой границе.}
{It's an assembler macro for label aligning by some specific border.}

\IFRU{Это нужно для тех \IT{нагруженных} меток, куда чаще всего передается управление, например, начало цикла. 
Для того чтобы процессор мог эффективнее вытягивать данные или код из памяти, через шину с памятью, 
кеширование, итд.}
{That's often need for the busy labels to where control flow is often passed, for example, loop begin.
So the CPU will effectively load data or code from the memory, through memory bus, cache lines, etc.}

\IFRU{Взято из}{Taken from} \TT{listing.inc} (MSVC):

\IFRU{Это, кстати, любопытный пример различных вариантов \NOP{}-ов. 
Все эти инструкции не дают никакого эффекта, но отличаются разной длиной.}
{By the way, it's curious example of different \NOP variations.
All these instructions has no effects at all, but has different size.}

\begin{lstlisting}
;; LISTING.INC
;;
;; This file contains assembler macros and is included by the files created
;; with the -FA compiler switch to be assembled by MASM (Microsoft Macro
;; Assembler).
;;
;; Copyright (c) 1993-2003, Microsoft Corporation. All rights reserved.

;; non destructive nops
npad macro size
if size eq 1
  nop
else
 if size eq 2
   mov edi, edi
 else
  if size eq 3
    ; lea ecx, [ecx+00]
    DB 8DH, 49H, 00H
  else
   if size eq 4
     ; lea esp, [esp+00]
     DB 8DH, 64H, 24H, 00H
   else
    if size eq 5
      add eax, DWORD PTR 0
    else
     if size eq 6
       ; lea ebx, [ebx+00000000]
       DB 8DH, 9BH, 00H, 00H, 00H, 00H
     else
      if size eq 7
	; lea esp, [esp+00000000]
	DB 8DH, 0A4H, 24H, 00H, 00H, 00H, 00H 
      else
       if size eq 8
        ; jmp .+8; .npad 6
	DB 0EBH, 06H, 8DH, 9BH, 00H, 00H, 00H, 00H
       else
        if size eq 9
         ; jmp .+9; .npad 7
         DB 0EBH, 07H, 8DH, 0A4H, 24H, 00H, 00H, 00H, 00H
        else
         if size eq 10
          ; jmp .+A; .npad 7; .npad 1
          DB 0EBH, 08H, 8DH, 0A4H, 24H, 00H, 00H, 00H, 00H, 90H
         else
          if size eq 11
           ; jmp .+B; .npad 7; .npad 2
           DB 0EBH, 09H, 8DH, 0A4H, 24H, 00H, 00H, 00H, 00H, 8BH, 0FFH
          else
           if size eq 12
            ; jmp .+C; .npad 7; .npad 3
            DB 0EBH, 0AH, 8DH, 0A4H, 24H, 00H, 00H, 00H, 00H, 8DH, 49H, 00H
           else
            if size eq 13
             ; jmp .+D; .npad 7; .npad 4
             DB 0EBH, 0BH, 8DH, 0A4H, 24H, 00H, 00H, 00H, 00H, 8DH, 64H, 24H, 00H
            else
             if size eq 14
              ; jmp .+E; .npad 7; .npad 5
              DB 0EBH, 0CH, 8DH, 0A4H, 24H, 00H, 00H, 00H, 00H, 05H, 00H, 00H, 00H, 00H
             else
              if size eq 15
               ; jmp .+F; .npad 7; .npad 6
               DB 0EBH, 0DH, 8DH, 0A4H, 24H, 00H, 00H, 00H, 00H, 8DH, 9BH, 00H, 00H, 00H, 00H
              else
	       %out error: unsupported npad size
               .err
              endif
             endif
            endif
           endif
          endif
         endif
        endif
       endif
      endif
     endif
    endif
   endif
  endif
 endif
endif
endm
\end{lstlisting}

% done

\section{\SignedNumbersSectionName}
\label{sec:signednumbers}

\newcommand{\URLS}{\url{http://en.wikipedia.org/wiki/Signed_number_representations}}

\IFRU
{Методов представления чисел с знаком ``плюс'' или ``минус'' несколько\footnote{\URLS}, 
а в x86 применяется метод ``дополнительный код'' или ``two's complement''.}
{There are several methods of representing signed numbers\footnote{\URLS}, 
but in x86 architecture used ``two's complement''.}

\IFRU{Разница в подходе к знаковым/беззнаковым числам, собственно, нужна потому что, например, 
если представить 0xFFFFFFFE и 0x0000002 как беззнаковое, то первое число (4294967294) больше второго (2). 
Если их оба представить как знаковые, то первое будет -2, которое, разумеется, меньше чем второе (2).
Вот почему инструкции для условных переходов~\ref{sec:Jcc} представлены в обоих версиях ~--- 
и для знаковых сравнений (например \JG, \JL) и для беззнаковых (\JA, \JB).}
{Difference between signed and unsigned numbers is that if we represent 0xFFFFFFFE and 0x0000002 
as unsigned, then first number (4294967294) is bigger than second (2). 
If to represent them both as signed, first will be -2, and it is lesser than second (2). 
That is the reason why conditional jumps~\ref{sec:Jcc} are present both for signed (for example, \JG, \JL) 
and unsigned (\JA, \JB) operations.}

\subsection{\IFRU{Переполнение integer}{Integer overflow}}

\IFRU{Бывает так, что ошибки представления знаковых/беззнаковых могут привести к уязвимости 
\IT{переполнение integer}.}
{It is worth noting that incorrect representation of number can lead integer overflow vulnerability.}

\IFRU{Например, есть некий сервис, который принимает по сети некие пакеты. 
В пакете есть заголовок где указана длина пакета. Это 32-битное значение. 
В процессе приема пакета, 
сервис проверяет это значение и сверяет, больше ли оно чем максимальный размер пакета, скажем, константа
\TT{MAX\_PACKET\_SIZE} (например, 10 килобайт). 
Сравнение знаковое. Злоумышленник подставляет значение 0xFFFFFFFF. Это число трактуется как знаковое -1 
и оно меньше чем 10000. Проверка проходит. Продолжаем дальше и копируем этот пакет куда-нибудь себе 
в сегмент данных... вызов функции \TT{memcpy (dst, src, 0xFFFFFFFF)} скорее всего, 
затрет много чего внутри процесса.}
{For example, we have some network service, it receives network packets. 
In that packets there are also field where subpacket length is coded. 
It is 32-bit value. 
After network packet received, service checking that field, and if it is larger than, 
for example, some \TT{MAX\_PACKET\_SIZE} (let's say, 10 kilobytes), packet ignored as incorrect. 
Comparison is signed. Intruder set this value to 0xFFFFFFFF. 
While comparison, this number is considered as signed -1 and it's lesser than 10 kilobytes. 
No error here. 
Service would like to copy that subpacket to another place in memory and call 
\TT{memcpy (dst, src, 0xFFFFFFFF)} function: this operation, rapidly scratching a lot of 
inside of process memory.}

\IFRU{Немного подробнее}{More about it}: \url{http://www.phrack.org/issues.html?issue=60&id=10}


% done

\section{\IFRU{Способы передачи аргументов при вызове функций}{Arguments passing methods (calling conventions)}}
\label{sec:callingconventions}

\subsection{cdecl}

\IFRU{Этот способ передачи аргументов через стек чаще всего используется в языках \CCpp.}
{This is the most popular method for arguments passing to functions in \CCpp languages.}

\IFRU{Вызывающая функция заталкивает в стек аргументы в обратном порядке: сначала последний аргумент в стек, 
затем предпоследний, и в самом конце ~--- первый аргумент. 
Вызывающая функция должна также затем вернуть указатель \ESP в нормальное состояние, 
после возврата вызываемой функции.}
{Caller pushing arguments to stack in reverse order: last argument, then penultimate element 
and finally ~--- first argument.
Caller also should return back \ESP to its initial state after callee function exit.}

\begin{lstlisting}
push arg3
push arg2
push arg3
call function
add esp, 12 ; return ESP
\end{lstlisting}

\subsection{stdcall}
\label{stdcall}

\newcommand{\SIZEOFINT}{\IFRU{Размер переменной типа \Tint ~--- 4 в x86-системах и 8 в x64-системах}
{Size of \Tint type variable is 4 in x86 systems and 8 in x64 systems}}

\IFRU{Это почти то же что и \IT{cdecl}, за исключением того что вызываемая функция сама возвращает \ESP 
в нормальное состояние, выполнив инструкцию \TT{RET x} вместо \RET, где 
\TT{x = количество\_аргументов * sizeof(int)\footnote{\SIZEOFINT}}.
Вызывающая функция не будет корректировать указатель стека при помощи инструкции \TT{add esp, x}.}
{Almost the same thing as \IT{cdecl}, with the exception that callee set \ESP to initial state executing \TT{RET x} instruction instead of \RET, where
\TT{x = arguments number * sizeof(int)\footnote{\SIZEOFINT}}.
Caller will not adjust stack pointer by \TT{add esp, x} instruction.}

\begin{lstlisting}
push arg3
push arg2
push arg1
call function

function:
... do something ...
ret 12
\end{lstlisting}

\IFRU{Этот способ используется почти везде в системных библиотеках win32, но не в win64 (о win64 смотрите ниже).}
{This method is ubiquitous in win32 standard libraries, but not in win64 (see below about win64).}

\subsubsection{\IFRU{Функции с переменным количеством аргументов}{Variable arguments number functions}}

\IFRU{Функции вроде \printf, должно быть, единственный случай функций в \CCpp с переменным количеством аргументов,
но с их помощью можно легко проследить очень важную разницу между \IT{cdecl} и \IT{stdcall}.
Начнем с того, что компилятор знает сколько аргументов было у \printf.}
{\printf-like functions are, probably, the only case of variable arguments functions in \CCpp,
but it's easy to illustrate an important difference between \IT{cdecl} and \IT{stdcall} with help of it.
Let's start with the idea that compiler knows argument count of each \printf function calling.}
\IFRU{Однако, вызываемая функция \printf, которая уже давно скомпилена 
и находится в системной библиотеке MSVCRT.DLL (если говорить о Windows), 
не знает сколько аргументов ей передали, хотя может установить их количество по строке формата.}
{However, called \printf, which is already compiled and located in MSVCRT.DLL (if to talk about Windows),
do not have information about how much arguments were passed, however it can determine it from format string.}
\IFRU{Таким образом, если бы \printf была \IT{stdcall}-функцией и возвращала указатель стека 
в первоначальное состояние 
подсчитав количество аргументов в строке формата, это была бы потенциально опасная ситуация, 
когда одна опечатка программиста могла бы вызывать неожиданные падения программы. 
Таким образом, для таких функций \IT{stdcall} явно не подходит, а подходит \IT{cdecl}.}
{Thus, if \printf would be \IT{stdcall}-function and restored stack pointer to its initial state by counting
number of arguments in format string, this could be dangerous situation, when one programmer's typo may
provoke sudden program crash.
Thus it's not suitable for such functions to use \IT{stdcall}, \IT{cdecl} is better.}

\subsection{fastcall}
\label{fastcall}

\IFRU{Это общее название для передачи некоторых аргументов через регистры а всех остальных ~--- через стек. 
На более старых процессорах, это работало потенциально быстрее чем \IT{cdecl}/\IT{stdcall}.}
{That's general naming for a method for passing some of arguments via registers and all others ~--- via stack.
It worked faster than \IT{cdecl}/\IT{stdcall} on older CPUs.}
\IFRU{Это не стандартизированый способ, поэтому разные компиляторы делают это по-своему. 
Разумеется, если у вас есть, скажем, две DLL, одна использует другую, и обе они собраны с \IT{fastcall}
но разными компиляторами, возможно будут проблемы.}
{It's not a standardized way, so, different compilers may do it differently.
Of course, if you have two DLLs, one use another, and they are built by different compilers with \IT{fastcall}
calling conventions, there will be a problems.}

\IFRU{MSVC и GCC передает первый и второй аргумент через \ECX и \EDX а остальные аргументы через стек. 
Вызываемая функция возвращает указатель стека в первоначальное состояние.}
{Both MSVC and GCC passing first and second argument via \ECX and \EDX and other arguments via stack.
Caller should restore stack pointer into initial state.}

\IFRU{Указатель стека должен быть возвращен в первоначальное состояние вызываемой функцией, 
как в случае \IT{stdcall}.}
{Stack pointer should be restored to initial state by callee, like in \IT{stdcall}.}

\begin{lstlisting}
push arg3
mov edx, arg2
mov ecx, arg1
call function

function:
.. do something ..
ret 4
\end{lstlisting}

\subsubsection{GCC regparm}

\newcommand{\URLREGPARMM}{\url{http://www.ohse.de/uwe/articles/gcc-attributes.html\#func-regparm}}

\IFRU{Это в некотором роде, развитие \IT{fastcall}\footnote{\URLREGPARMM}. 
Опцией \TT{-mregparm=x} можно указывать, 
сколько аргументов компилятор будет передавать через регистры. Максимально 3. 
В этом случае будут задействованы регистры \EAX, \EDX и \ECX.}
{It's \IT{fastcall} evolution\footnote{\URLREGPARMM} is some sense.
With the \TT{-mregparm} option it's possible to set, how many arguments will be passed via registers. 
3 at maximum.
Thus, \EAX, \EDX and \ECX registers will be used.}

\IFRU{Разумеется, если аргументов у функции меньше трех, то будет задействована только часть регистров.}
{Of course, if number of arguments is less then 3, not all registers 3 will be used.}

\IFRU{Вызывающая функция возвращает указатель стека в первоначальное состояние.}
{Caller restores stack pointer to its initial state.}

% TODO: example

\subsection{thiscall}
\label{thiscall}

\IFRU{В С++, это передача в функцию-метод указателя \IT{this} на объект.}
{In C++, it's a \IT{this} pointer to object passing into function-method.}

\IFRU{В MSVC указатель \IT{this} обычно передается в регистре \ECX.}
{In MSVC, \IT{this} is usually passed in \ECX register.}

\IFRU{В GCC указатель \IT{this} обычно передается как самый первый аргумент. 
Таким образом, внутри будет видно, что у всех функций-методов на один аргумент больше.}
{In GCC, \IT{this} pointer is passed as a first function-method argument.
Thus it will be seen that internally all function-methods has extra argument for it.}

% TODO: example

\subsection{x86-64}

\subsubsection{win64}

\IFRU{В win64 метод передачи всех параметров немного похож на \TT{fastcall}. 
Первые 4 аргумента записываются в регистры \TT{RCX}, \TT{RDX}, \TT{R8}, \TT{R9}, а остальные ~--- в стек. 
Вызывающая функция также должна подготовить место из 32 байт или для четырех 64-битных значений, 
чтобы вызываемая функция могла сохранить там первые 4 аргумента. 
Короткие функции могут использовать переменные прямо из регистров, 
но б\'{о}льшие могут сохранять их значения на будущее.}
{The method of arguments passing in Win64 is somewhat resembling to \TT{fastcall}.
First 4 arguments are passed via \TT{RCX}, \TT{RDX}, \TT{R8}, \TT{R9}, others ~--- via stack.
Caller also must prepare a place for 32 bytes or 4 64-bit values,
so then callee can save there first 4 arguments.
Short functions may use argument values just from registers,
but larger may save its values for further use.}

\IFRU{Вызывающая функция должна вернуть указатель стека в первоначальное состояние.}
{Caller also should return stack pointer into initial state.}

\IFRU{Это же соглашение используется и в системных библиотеках Windows x86-64 (вместо \IT{stdcall} в win32).}
{This calling convention is also used in Windows x86-64 system DLLs (instead if \IT{stdcall} in win32).}

% TODO: example

\subsection{\IFRU{Возвращение переменных типа \Tfloat, \Tdouble}{Returning values of \Tfloat and \Tdouble type}}

\IFRU{Во всех соглашениях кроме Win64, переменная типа \Tfloat или \Tdouble возвращается через регистр FPU \STZERO.}
{In all conventions except of Win64, values of type \Tfloat or \Tdouble returning via FPU register \STZERO.}

\IFRU{В Win64 переменные типа \Tfloat и \Tdouble возвращаются в регистре \XMMZERO вместо \STZERO.}
{In Win64, return values of \Tfloat and \Tdouble types are returned in \XMMZERO register instead of \STZERO.}



% done

\chapter{\IFRU{Поиск в коде того что нужно}{Finding important/interesting stuff in the code}}

\IFRU{Современный софт, в общем-то, минимализмом не отличается.}{Minimalism it's not a significant feature
of modern software.}

\IFRU{Но не потому, что программисты слишком много написали, 
а потому что к исполняемым файлам обыкновенно прикомпиливают все подряд библиотеки. 
Если бы все вспомогательные библиотеки всегда выносили во внешние DLL, мир был бы иным.}
{But not because programmers wrote a lot, but because all libraries are usually linked statically
to executable files.
If all external libraries were shifted into external DLL files, the world would be different.}

\newcommand{\FOOTNOTEBOOST}{\footnote{\url{http://www.boost.org/}}}
\newcommand{\FOOTNOTELIBPNG}{\footnote{\url{http://www.libpng.org/pub/png/libpng.html}}}

\IFRU{Таким образом, очень полезно сразу понимать, какая функция из стандартной библиотеки или 
более-менее известной (как Boost\FOOTNOTEBOOST, libpng\FOOTNOTELIBPNG), 
а какая ~--- имеет отношение к тому что мы пытаемся найти в коде.}
{Thus, it's very important to determine origin of some function, if it's from standard library or 
well-known library (like Boost\FOOTNOTEBOOST, libpng\FOOTNOTELIBPNG),
and which one ~--- is related to what we are trying to find in the code.}

\IFRU{Переписывать весь код на \CCpp, чтобы разобраться в нем, безусловно, не имеет никакого смысла.}
{It's just absurdly to rewrite all code to \CCpp to find what we looking for.}

\IFRU{Одна из задач reverse engineer-а это быстрый поиск в коде того что собственно его интересует.}
{One of the primary reverse engineer's task is to find quickly in the code what is needed.}

\IFRU{Дизассемблер \IDA позволяет делать поиск как минимум строк, последовательностей байт, констант.
Можно даже сделать экспорт кода в текстовый файл .lst или .asm и затем натравить на него \TT{grep}, \TT{awk}, итд.}
{\IDA disassembler can search among text strings, byte sequences, constants.
It's even possible to export the code into .lst or .asm text file and then use \TT{grep}, \TT{awk}, etc.}

\IFRU{Когда вы пытаетесь понять, что делает тот или иной код, это запросто может быть какая-то 
опенсорсная библиотека вроде libpng. Поэтому когда находите константы, или текстовые строки которые 
выглядят явно знакомыми, всегда полезно их погуглить.
А если вы найдете искомый опенсорсный проект где это используется, 
то тогда будет достаточно будет просто сравнить вашу функцию с ней. 
Это решит часть проблем.}
{When you try to understand what some code is doing, this easily could be some open-source library like libpng.
So when you see some constants or text strings looks familiar, it's always worth to google it.
And if you find the opensource project where it's used, 
then it will be enough just to compare the functions. It may solve some part of problem.}

\IFRU{К примеру, однажды я пытался разобраться как происходит компрессия/декомпрессия сетевых пакетов в SAP 6.0. 
Это очень большая программа, но к ней идет подробный .PDB-файл с отладочной информацией, и это очень удобно. 
Я в конце концов пришел к тому что одна из функций декомпрессирующая пакеты называется CsDecomprLZC(). 
Не сильно раздумывая, я решил погуглить и оказалось что функция с таким же названием имеется в MaxDB
(это опен-сорсный проект SAP)\footnote{Больше об этом в соответствующей секции~\ref{SAPGUI}}.}
{For example, once upon a time I tried to understand how SAP 6.0 network packets compression/decompression 
is working.
It's a huge software, but a detailed .PDB with debugging information is present, and that's cosily.
I finally came to idea that one of the functions doing decompressing of network packet called CsDecomprLZC().
Immediately I tried to google its name and I quickly found that the function named as the same is used in MaxDB
(it's open-source SAP project)\footnote{More about it in releval section~\ref{SAPGUI}}.}

\url{http://www.google.com/search?q=CsDecomprLZC}

\IFRU{Каково же было мое удивление, когда оказалось, что в MaxDB используется точно такой же алгоритм, 
скорее всего, с таким же исходником.}
{Astoundingly, MaxDB and SAP 6.0 software shared the same code for network packets compression/decompression.}

\section{\IFRU{Связь с внешним миром}{Communication with the outer world}}

\IFRU{Первое на что нужно обратить внимание, это какие функции из API операционной 
системы и какие функции стандартных библиотек используются.}
{First what to look on is which functions from operation system API and standard libraries are used.}

\IFRU{Если программа поделена на главный исполняемый файл и группу DLL-файлов, 
то имена функций в этих DLL, бывает так, могут помочь.}
{If the program is divided into main executable file and a group of DLL-files, sometimes,
these function's names may be helpful.}

\IFRU{Если нас интересует, что именно приводит к вызову \TT{MessageBox()} с определенным текстом, 
то первое что можно попробовать сделать: найти в сегменте данных этот текст, найти ссылки на него, и найти, 
откуда может передаться управление к интересующему нас вызову \TT{MessageBox()}.}
{If we are interesting, what exactly may lead to \TT{MessageBox()} call with specific text,
first what we can try to do: find this text in data segment, find references to it and find the points
from which a control may be passed to \TT{MessageBox()} call we're interesting in.}

\IFRU{Если речь идет об игре, и нам интересно какие события в ней более-менее случайны, 
мы можем найти функцию \rand или её заменитель (как алгоритм Mersenne twister), и посмотреть, 
из каких мест эта функция вызывается и что самое главное: как используется результат этой функции.}
{If we are talking about some game and we're interesting, which events are more or less random in it,
we may try to find \rand function or its replacement (like Mersenne twister algorithm) and find a places
from which this function called and most important: how the results are used.}

\IFRU{Но если это не игра, а \rand используется, то также весьма любопытно, зачем. 
Бывают неожиданные случаи вроде использования \rand в алгоритме для сжатия данных (для имитации шифрования):}
{But if it's not a game, but \rand is used, it's also interesing, why.
There are cases of unexpected \rand usage in data compression algorithm (for encryption imitation):}
\url{http://blogs.conus.info/node/44}.

\section{\IFRU{Строки}{String}}

\IFRU{Очень сильно помогают отладочные сообщения, если они имеются. В некотором смысле, отладочные сообщения, 
это отчет о том, что сейчас происходит в программе. Зачастую, это \printf-подобные функции, 
которые пишут куда-нибудь в лог, а бывает так что и не пишут ничего, но вызовы остались, так как эта сборка ~--- 
не отладочная, а release.}
{Debugging messages are often very helpful if present. In some sense, debugging messages are reporting
about what's going on in program right now. Often these are \printf-like functions,
which writes to log-files, and sometimes, not writing anything but calls are still present, because this build
is not debug build but release one.}
\IFRU{Если в отладочных сообщениях дампятся значения некоторых локальных или глобальных переменных, 
это тоже может помочь, как минимум, узнать их имена. 
Например, в Oracle RDBMS одна из таких функций: \TT{ksdwrt()}.}
{If local or global variables are dumped in debugging messages, it might be helpful as well because it's 
possible to get variable names at least.
For example, one of such functions in Oracle RDBMS is \TT{ksdwrt()}.}

\IFRU{Может также помочь наличие \TT{assert()} в коде: обычно этот макрос оставляет название файла-исходника, 
номер строки, и условие.}
{Sometimes \TT{assert()} macro presence is useful too: usually, this macro leave in code source file name, 
line number and condition.}

\newcommand{\CONUSONE}{http://blogs.conus.info/node/32}
\newcommand{\CONUSTWO}{http://blogs.conus.info/node/43}

\IFRU{Осмысленные текстовые строки вообще очень сильно могут помочь. 
Дизассемблер \IDA может сразу указать, из какой функции и из какого её места используется эта строка. 
Попадаются и \href{\CONUSONE}{смешные случаи}.}
{Meaningful text strings are often helpful.
\IDA disassembler may show from which function and from which point this specific string is used.
Funny cases \href{\CONUSONE}{sometimes happen}.}

\IFRU{Парадоксально, но сообщения об ошибках также могут помочь найти то что нужно. 
В Oracle RDBMS сигнализация об ошибках проходит при помощи вызова некоторой группы функций. 
\href{\CONUSTWO}{Тут еще немного об этом}.}
{Paradoxically, but error messages may help us as well.
In Oracle RDBMS, errors are reporting using group of functions.
\href{\CONUSTWO}{More about it}.}

\IFRU{Можно довольно быстро найти, какие функции сообщают о каких ошибках, и при каких условиях.}
{It's possible to find very quickly, which functions reporting about errors and in which conditions.}
\IFRU{Это, кстати, одна из причин, почему в защите софта от копирования, 
бывает так, что сообщение об ошибке заменяется 
невнятным кодом или номером ошибки. Мало кому приятно, если взломщик быстро поймет, 
из за чего именно срабатывает защита от копирования, просто по сообщению об ошибке.}
{By the way, it's often a reason why copy-protection systems has inarticulate cryptic error messages 
or just error numbers. No one happy when software cracker quickly understand why copy-protection
is triggered just by error message.}

\section{\IFRU{Константы}{Constants}}

\IFRU{Некоторые алгоритмы, особенно криптографические, используют хорошо различимые константы, 
которые при помощи \IDA легко находить в коде.}
{Some algorithms, especially cryptographical, use distinct constants, which is easy to find
in code using \IDA.}

\newcommand{\URLMD}{\IFRU{http://ru.wikipedia.org/wiki/MD5}{http://en.wikipedia.org/wiki/MD5}}

\IFRU{Например алгоритм MD5\footnote{\url{\URLMD}} инициализирует свои внутренние переменные так:}
{For example, MD5\footnote{\url{\URLMD}} algorithm initializes its own internal variables like:}

\begin{verbatim}
var int h0 := 0x67452301
var int h1 := 0xEFCDAB89
var int h2 := 0x98BADCFE
var int h3 := 0x10325476
\end{verbatim}

\IFRU{Если в коде найти использование этих четырех констант подряд ~--- 
очень высокая вероятность что эта функция имеет отношение к MD5.}
{If you find these four constants usage in the code in a row ~---
it's very high probability this function is related to MD5.}

\subsection{Magic numbers}

\newcommand{\FNURLMAGIC}{\footnote{\url{http://en.wikipedia.org/wiki/Magic_number_(programming)}}}

\IFRU{Немало форматов файлов определяет стандартный заголовок файла где используются \IT{magic numbers}\FNURLMAGIC.}
{A lot of file formats defining a standard file header where \IT{magic number}\FNURLMAGIC is used.}

\IFRU{Скажем, все исполняемые файлы для Win32 и MS-DOS начинаются с двух символов}
{For example, all Win32 and MS-DOS executables are started with two characters} "MZ"\footnote{\url{http://en.wikipedia.org/wiki/DOS_MZ_executable}}.

\IFRU{В начале MIDI-файла должно быть "MThd". Если у нас есть использующая для чего-нибудь MIDI-файлы программа
очень вероятно, что она будет проверять MIDI-файлы на правильность хотя бы проверяя первые 4 байта.}
{At the MIDI-file beginning "MThd" signature must be present. 
If we have a program that using MIDI-files for something,
very likely, it will check MIDI-files for validity by checking at least first 4 bytes.}

\IFRU{Это можно сделать при помощи:}{This could be done like:}

\IFRU{(\IT{buf} указывает на начало загруженного в память файла)}
{(\IT{buf} pointing to the beginning of loaded file into memory)}

\begin{verbatim}
cmp [buf], 0x6468544D ; "MThd"
jnz _error_not_a_MIDI_file
\end{verbatim}

\IFRU{... либо вызвав функцию сравнения блоков памяти \TT{memcmp()} или любой аналогичный код, 
вплоть до инструкции \TT{CMPSB}.}
{... or by calling function for comparing memory blocks \TT{memcmp()} or any other equivalent code
up to \TT{CMPSB} instruction.}
% TODO: CMPSB description

\IFRU{Найдя такое место мы получаем как минимум информацию о том, где начинается загрузка MIDI-файла, во вторых, 
мы можем увидеть где располагается буфер с содержимым файла, и что еще оттуда берется, и как используется.}
{When you find such place you already may say where MIDI-file loading is beginning, also, we could see a location
of MIDI-file contents buffer and what is used from that buffer and how.}

\subsubsection{DHCP}

\IFRU{Это касается также и сетевых протоколов. 
Например, сетевые пакеты протокола DHCP содержат так называемую \IT{magic cookie}: 0x63538263. 
Какой-либо код генерирующий пакеты по протоколу DHCP где-то и как-то должен внедрять в пакет также и эту константу. 
Найдя её в коде мы сможем найти место где происходит это и не только это. 
\IT{Что-либо} что получает пакеты по DHCP должно где-то как-то проверять \IT{magic cookie}, 
сравнивая это поле пакета с константой.}
{This applies to network protocols as well.
For example, DHCP protocol network packets contains so called \IT{magic cookie}: 0x63538263.
Any code generating DHCP protocol packets somewhere and somehow should embed this constant into packet.
If we find it in the code we may find where it happen and not only this.
\IT{Something} that received DHCP packet should check \IT{magic cookie}, comparing it with the constant.}

\IFRU{Например, берем файл dhcpcore.dll из Windows 7 x64 и ищем эту константу. 
И находим, два раза: оказывается, эта константа используется в функциях с красноречивыми 
названиями}
{For example, let's take dhcpcore.dll file from Windows 7 x64 and search for the constant.
And we found it, two times: it seems, that constant is used in two functions eloquently 
named as} \TT{DhcpExtractOptionsForValidation()} \IFRU{и}{and} \TT{DhcpExtractFullOptions()}:

\begin{lstlisting}
.rdata:000007FF6483CBE8 dword_7FF6483CBE8 dd 63538263h          ; DATA XREF: DhcpExtractOptionsForValidation+79
.rdata:000007FF6483CBEC dword_7FF6483CBEC dd 63538263h          ; DATA XREF: DhcpExtractFullOptions+97
\end{lstlisting}

\IFRU{А вот те места в функциях где происходит обращение к константам:}
{And the places where these constants accessed:}

\begin{lstlisting}
.text:000007FF6480875F                 mov     eax, [rsi]
.text:000007FF64808761                 cmp     eax, cs:dword_7FF6483CBE8
.text:000007FF64808767                 jnz     loc_7FF64817179
\end{lstlisting}

\IFRU{И:}{And:}

\begin{lstlisting}
.text:000007FF648082C7                 mov     eax, [r12]
.text:000007FF648082CB                 cmp     eax, cs:dword_7FF6483CBEC
.text:000007FF648082D1                 jnz     loc_7FF648173AF
\end{lstlisting}

\section{\IFRU{Поиск нужных инструкций}{Finding the right instructions}}

\IFRU{Если программа использует инструкции сопроцессора, и их не очень много, 
то можно попробовать проверить отладчиком какую-то из них.}
{If the program is using FPU instructions and there are very few of them in a code,
one can try to check each by debugger.}

\IFRU{К примеру, нас может заинтересовать, при помощи чего Microsoft Excel считает 
результаты формул введенных пользователем. Например, операция деления.}
{For example, we may be interesting, how Microsoft Excel calculating formulae entered by user.
For example, division operation.}

\IFRU{Если загрузить excel.exe (из Office 2010) версии 14.0.4756.1000 в \IDA, затем сделать полный листинг 
и найти все инструкции \FDIV (но кроме тех, которые в качестве второго операнда используют константы ~--- они, 
очевидно, не подходят нам):}
{If to load excel.exe (from Office 2010) version 14.0.4756.1000 into \IDA, then make a full listing
and to find each \FDIV instructions (except ones which use constants as a second 
operand ~--- obviously, it's not suits us):}

\begin{lstlisting}
cat EXCEL.lst | grep fdiv | grep -v dbl_ > EXCEL.fdiv
\end{lstlisting}

\IFRU{... то окажется, что их всего 144.}
{... then we realizing they are just 144.}

\IFRU{Мы можем вводить в Excel строку вроде \TT{"=(1/3)"} и проверить все эти инструкции.}
{We can enter string like \TT{"=(1/3)"} in Excel and check each instruction.}

\IFRU{Проверяя каждую инструкцию в отладчике или tracer~\ref{tracer} 
(проверять эти инструкции можно по 4 за раз), 
окажется, что нам везет и срабатывает всего-лишь 14-я по счету:}
{Checking each instruction in debugger or tracer~\ref{tracer}
(one may check 4 instruction at a time),
it seems, we are lucky here and sought-for instruction is just 14th:}

\begin{lstlisting}
.text:3011E919 DC 33                                fdiv    qword ptr [ebx]
\end{lstlisting}

\begin{lstlisting}
PID=13944|TID=28744|(0) 0x2f64e919 (Excel.exe!BASE+0x11e919)
EAX=0x02088006 EBX=0x02088018 ECX=0x00000001 EDX=0x00000001
ESI=0x02088000 EDI=0x00544804 EBP=0x0274FA3C ESP=0x0274F9F8
EIP=0x2F64E919
FLAGS=PF IF
FPU ControlWord=IC RC=NEAR PC=64bits PM UM OM ZM DM IM 
FPU StatusWord=
FPU ST(0): 1.000000
\end{lstlisting}

\IFRU{В \STZERO содержится первый аргумент (1), второй содержится в}
{\STZERO holding first argument (1) and second one is in} \TT{[ebx]}.

\IFRU{Следующая за \FDIV инструкция записывает результат в память:}
{Next instruction after \FDIV writes result into memory:}

\begin{lstlisting}
.text:3011E91B DD 1E                                fstp    qword ptr [esi]
\end{lstlisting}

\IFRU{Если поставить breakpoint на ней, то мы можем видеть результат:}
{If to set breakpoint on it, we may see result:}

\begin{lstlisting}
PID=32852|TID=36488|(0) 0x2f40e91b (Excel.exe!BASE+0x11e91b)
EAX=0x00598006 EBX=0x00598018 ECX=0x00000001 EDX=0x00000001
ESI=0x00598000 EDI=0x00294804 EBP=0x026CF93C ESP=0x026CF8F8
EIP=0x2F40E91B
FLAGS=PF IF
FPU ControlWord=IC RC=NEAR PC=64bits PM UM OM ZM DM IM 
FPU StatusWord=C1 P 
FPU ST(0): 0.333333
\end{lstlisting}

\IFRU{А также, в рамках пранка\footnote{practical joke}, модифицировать его на лету:}
{Also, as a practical joke, we can modify it on-fly:}

\begin{lstlisting}
gt -l:excel.exe bpx=excel.exe!base+0x11E91B,set(st0,666)
\end{lstlisting}

\begin{lstlisting}
PID=36540|TID=24056|(0) 0x2f40e91b (Excel.exe!BASE+0x11e91b)
EAX=0x00680006 EBX=0x00680018 ECX=0x00000001 EDX=0x00000001
ESI=0x00680000 EDI=0x00395404 EBP=0x0290FD9C ESP=0x0290FD58
EIP=0x2F40E91B
FLAGS=PF IF
FPU ControlWord=IC RC=NEAR PC=64bits PM UM OM ZM DM IM 
FPU StatusWord=C1 P 
FPU ST(0): 0.333333
Set ST0 register to 666.000000
\end{lstlisting}

\IFRU{Excel показывает в этой ячейке 666, что окончательно убеждает нас в том что мы нашли нужное место.}
{Excel showing 666 in that cell what finally convincing us we find the right place.}

\begin{figure}[ht!]
\centering
\includegraphics[scale=0.66]{digging_into_code/Excel_prank.png}
\caption{\IFRU{Пранк сработал}{Practical joke worked}}
\end{figure}

\IFRU{Если попробовать ту же версию Excel, только x64, то окажется что там инструкций \FDIV всего 12, 
причем нужная нам ~--- третья по счету.}
{If to try the same Excel version, but x64, we'll see there are only 12 \FDIV instructions,
and the one we looking for ~--- third.}

\begin{lstlisting}
gt64.exe -l:excel.exe bpx=excel.exe!base+0x1B7FCC,set(st0,666)
\end{lstlisting}

\IFRU{Видимо, все дело в том что много операций деления переменных типов \Tfloat и \Tdouble 
компилятор заменил на SSE-инструкции вроде \TT{DIVSD}, 
коих здесь теперь действительно много (\TT{DIVSD} присутствует в количестве 268 инструкций).}
{It seems, a lot of division operations of \Tfloat and \Tdouble types, compiler replaced by SSE-instructions
like \TT{DIVSD} (\TT{DIVSD} present here 268 in total).}

\section{\IFRU{Подозрительные паттерны кода}{Suspicious code patterns}}

\IFRU{Современные компиляторы не генерируют инструкции \TT{LOOP} и \TT{RCL}. 
С другой стороны, эти инструкции хорошо знакомы кодерам предпочитающим писать прямо на ассемблере. 
Если такие инструкции встретились, можно сказать с какой-то вероятностью, что этот кусок кода написан вручную.}
{Modern compilers do not emit \TT{LOOP} and \TT{RCL} instructions.
On the other hand, these instructions are well-known to coders who like to code in straight assembler.
If you spot these, it can be said, with a big probability, this piece of code is hand-written.}
% TODO: also, function prologue/epilogue isn't present


% done

\chapter{\IFRU{Задачи}{Tasks}}

\IFRU{Почти для всех задач, если не указано иное, два вопроса:}
{There are two questions almost for every task, if otherwise isn't specified:}

1) \IFRU{Что делает эта функция? Ответ должен состоять из одной фразы.}
{What this function does? Answer in one-sentence form.}

2) \IFRU{Перепишите эту функцию на \CCpp}{Rewrite this function into \CCpp}.

\IFRU{Подсказки и ответы собраны в приложении к этой брошюре.}{Hints and solutions are in the appendix of
this brochure.}

\section{\IFRU{Легкий уровень}{Easy level}}

\subsection{\Task 1.1}

\IFRU{Это стандартная функция из библиотек Си. Исходник взят из OpenWatcom. Скомпилировано в MSVC 2010.}
{This is standard C library function. Source code taken from OpenWatcom. Compiled in MSVC 2010.}

\lstinputlisting{tasks/tasks_1_1_msvc.asm}

\IFRU{Это он же скомпилирован при помощи GCC 4.4.1 с опцией \TT{-O3} (максимальная оптимизация)}
{It is the same code compiled by GCC 4.4.1 with \TT{-O3} option (maximum optimization)}:

\lstinputlisting{tasks/tasks_1_1_gcc.asm}

\subsection{\Task 1.2}

\IFRU{Это также стандартная функция из библиотек Си. Исходник взят из OpenWatcom и немного переделан. 
Скомпилировано в MSVC 2010 с флагом (\Ox).}
{This is also standard C library function. Source code is taken from OpenWatcom and modified slightly.
Compiled in MSVC 2010 with \Ox optimization flag.}

\IFRU{Эта функция использует стандартные функции Си:}
{This function also use these standard C functions:} isspace() \IFRU{и}{and} isdigit().

\lstinputlisting{tasks/tasks_1_2_msvc.asm}

\IFRU{То же скомпилировано в GCC 4.4.1. Задача немного усложняется тем, что GCC представил isspace() и isdigit() 
как inline-функции и вставил их тела прямо в код.}
{Same code compiled in GCC 4.4.1. This task is sligthly harder because GCC compiled isspace() and isdigit() 
functions like inline-functions and inserted their bodies right into code.}

\lstinputlisting{tasks/tasks_1_2_gcc.asm}

\subsection{\Task 1.3}

\IFRU{Это также стандартная функция из библиотек Си, а вернее, две функции, работающие в паре. 
Исходник взят из MSVC 2010 и немного переделан.}
{This is standard C function too, actually, two functions working in pair.
Source code taken from MSVC 2010 and modified sligthly.}

\IFRU{Суть переделки в том, что эта функция может корректно работать в мульти-тредовой среде, 
а я, для упрощения (или запутывания) убрал поддержку этого.}
{The matter of modification is that this function can work properly in multi-threaded environment,
and I removed its support for simplification (or for confusion).}

\IFRU{Скомпилировано в MSVC 2010 с флагом (\Ox)}{Compiled in MSVC 2010 with \Ox flag}.

\lstinputlisting{tasks/tasks_1_3_msvc.asm}

\IFRU{То же скомпилировано при помощи GCC 4.4.1}{Same code compiled in GCC 4.4.1}:

\lstinputlisting{tasks/tasks_1_3_gcc.asm}

\subsection{\Task 1.4}

\IFRU{Это стандартная функция из библиотек Си. Исходник взят из MSVC 2010. Скомпилировано в MSVC 2010 с флагом \Ox.}
{This is standard C library function. Source code taken from MSVC 2010. Compiled in MSVC 2010 with \Ox flag.}

\lstinputlisting{tasks/tasks_1_4_msvc.asm}

\IFRU{То же скомпилировано при помощи GCC 4.4.1}
{Same code compiled in GCC 4.4.1}:

\lstinputlisting{tasks/tasks_1_4_gcc.asm}

\subsection{\Task 1.5}

\IFRU{Задача, скорее, на эрудицию, нежели на чтение кода.}
{This task is rather on knowledge than on reading code.}

\IFRU{Функция взята из OpenWatcom. Скомпилировано в MSVC 2010 с флагом \Ox.}
{The function is taken from OpenWatcom. Compiled in MSVC 2010 with \Ox flag.}

\lstinputlisting{tasks/tasks_1_5_msvc.asm}

\subsection{\Task 1.6}

\IFRU{Скомпилировано в MSVC 2010 с ключом \Ox.}
{Compiled in MSVC 2010 with \Ox option.}

\lstinputlisting{tasks/tasks_1_6_msvc.asm}

\subsection{\Task 1.7}

\IFRU{Это взята функция из ядра Linux 2.6.}{This function is taken from Linux 2.6 kernel.}

\IFRU{Скомпилировано в MSVC 2010 с опцией \Ox:}{Compiled in MSVC 2010 with \Ox option:}

\lstinputlisting{tasks/tasks_1_7_msvc.asm}

\subsection{\Task 1.8}

\IFRU{Скомпилировано в MSVC 2010 с опцией \TT{/O1}\footnote{/O1: оптимизация по размеру кода}:}
{Compiled in MSVC 2010 with \TT{/O1} option\footnote{/O1: minimize space}:}

\lstinputlisting{tasks/tasks_1_8_msvc.asm}

\subsection{\Task 1.9}

\IFRU{Скомпилировано в MSVC 2010 с опцией \TT{/O1}:}
{Compiled in MSVC 2010 with \TT{/O1} option:}

\lstinputlisting{tasks/tasks_1_9_msvc.asm}

\subsection{\Task 1.10}

\IFRU{Если это скомпилировать и запустить, появится некоторое число. Откуда оно берется? 
Откуда оно берется если скомпилировать в MSVC с оптимизациями (\Ox)?}
{If to compile this piece of code and run, some number will be printed. Where it came from?
Where it came from if to compile it in MSVC with optimization (\Ox)?}

\begin{lstlisting}
#include <stdio.h>

int main()
{
	printf ("%d\n");

	return 0;
};
\end{lstlisting}

\section{\IFRU{Средний уровень}{Middle level}}

\subsection{\Task 2.1}

\IFRU{Довольно известный алгоритм, также включен в стандартную библиотеку Си. Исходник взят из glibc 2.11.1. 
Скомпилирован в GCC 4.4.1 с ключом \TT{-Os} (оптимизация по размеру кода). 
Листинг сделан дизассемблером IDA 4.9 из ELF-файла созданным GCC и линкером.}
{Well-known algorithm, also included in standard C library. Source code was taken from glibc 2.11.1.
Compiled in GCC 4.4.1 with \TT{-Os} option (code size optimization).
Listing was done by IDA 4.9 disassembler from ELF-file generated by GCC and linker.}

\IFRU{Для тех кто хочет использовать IDA в процессе изучения, вот здесь лежат .elf и .idb файлы, 
.idb можно открыть при помощи бесплатой IDA 4.9:}
{For those who wants use IDA while learning, here you may find .elf and .idb files,
.idb can be opened with freeware IDA 4.9:}

\url{http://conus.info/RE-tasks/middle/1/}

\lstinputlisting{tasks/tasks_2_1_gcc.asm}

\section{crackme / keygenme}

/IFRU{Несколько моих keygenme\footnote{программа имитирующая защиту вымышленной программы, 
для которой нужно сделать генератор ключей/лицензий.}:}
{Couple of my keygenmes\footnote{program which imitates fictional software protection, 
for which one need to make a keys/licenses generator}:}

\url{http://crackmes.de/users/yonkie/}



\chapter{\IFRU{Инструменты}{Tools}}

\begin{itemize}

\item
\label{IDA}
\IFRU{IDA как дизассемблер. Старая бесплатная версия доступна для скачивания}
{IDA as disassembler. Older freeware version is available for downloading}: 
\url{http://www.hex-rays.com/idapro/idadownfreeware.htm}.

\item
Microsoft Visual Studio Express\footnote{\url{http://www.microsoft.com/express/Downloads/}}:
\IFRU{Усеченная версия Visual Studio, пригодная для простых экспериментов}{Stripped-down Visual Studio version,
convenient for simple expreiments}.

\item
Hiew\footnote{\url{http://www.hiew.ru/}} /IFRU{для мелкой модификации кода в исполняемых файлах}
{for small modifications of code in binary files}.
\end{itemize}

\subsection{\IFRU{Отладчик}{Debugger}}

\label{tracer}
\IT{tracer}\footnote{\url{http://conus.info/gt/}} \IFRU{вместо отладчика}{instead of debugger}.

\IFRU{Со временем я отказался использовать отладчик, потому что все что мне нужно от него: это иногда подсмотреть 
какие-либо аргументы какой-либо функции во время исполнения или состояние регистров в определенном месте. 
Каждый раз загружать отладчик для этого это слишком, поэтому я написал очень простую утилиту \IT{tracer}. 
Она консольная, запускается из командной строки, позволяет перехватывать исполнение функций, 
ставить брякпоинты на произвольные места, смотреть состояние регистров, модифицировать их, и так далее.}
{I stopped to use debugger eventually, because all I need from it is to spot some function's arguments while
execution, or registers' state at some point.
To load debugger each time is too much, so I wrote a small utility \IT{tracer}.
It has console-interface, working from command-line, allow to intercept function execution,
set breakpoints at arbitrary places, spot registers' state, modify it, etc.}

\IFRU{Но для учебы, очень полезно трассировать код руками в отладчике, наблюдать как меняются значения регистров 
(например, как минимум классический SoftICE, OllyDbg, WinDbg подсвечивают измененные регистры), 
флагов, данные, менять их самому, смотреть реакцию, итд.}
{However, as for learning, it's highly advisable to trace code in debugger manually, watch how register's state
changing (for example, classic SoftICE, OllyDbg, WinDbg highlighting changed registers), flags, data, change them
manually, watch reaction, etc.}


% done

\chapter{\IFRU{Что стоит почитать}{Books/blogs worth reading}}

\section{\IFRU{Книги}{Books}}

\subsection{Windows}

\begin{itemize}
\item
Windows® Internals (Mark E. Russinovich and David A. Solomon with Alex Ionescu)\footnote{\url{http://www.microsoft.com/learning/en/us/book.aspx?ID=12069&locale=en-us}}
\end{itemize}

\subsection{\CCpp}

\begin{itemize}
\item
\IFRU{Стандарт языка Си++}{C++ language standard}: ISO/IEC 14882:2003\footnote{\url{http://www.iso.org/iso/catalogue_detail.htm?csnumber=38110}}
\end{itemize}

\subsection{x86 / x86-64}

\begin{itemize}
\item
\IFRU{Документация от Intel}{Intel manuals}: \url{http://www.intel.com/products/processor/manuals/}
\item
\IFRU{Документация от AMD}{AMD manuals}: \url{http://developer.amd.com/documentation/guides/Pages/default.aspx#manuals}
\end{itemize}

\section{\IFRU{Блоги}{Blogs}}

\subsection{Windows}

\begin{itemize}
\item
\href{http://blogs.msdn.com/oldnewthing/}{Microsoft: Raymond Chen}
\item
\url{http://www.nynaeve.net/}
\end{itemize}



\chapter{\IFRU{Еще примеры}{More examples}}

\section{\IFRU{"QR9": Любительская криптосистема вдохновленная кубиком Рубика}
{"QR9": Rubik's cube inspired amateur crypto-algorithm}}

\IFRU{Любительские криптосистемы иногда попадаются довольно странные.}
{Sometimes amateur cryptosystems appear to be pretty bizarre.}

\IFRU{Однажды меня попросили разобраться с одним таким любительским криптоалгоритмом встроенным в 
утилиту для шифрования, исходный код которой был утерян\footnote{Я также получил разрешение от 
клиента на публикацию деталей алгоритма}.}
{I was asked to revese engineer an amateur cryptoalgorithm of some data crypting utility, 
source code of which was lost\footnote{I also got permit from customer to publish the algorithm details}.}

\IFRU{Вот листинг этой утилиты для шифрования, полученный при помощи \IDA}
{Here is also \IDA exported listing from original crypting utility}:

\lstinputlisting{qr9/qr9_original.lst}

\IFRU{Все имена функций и меток даны мною в процессе анализа.}
{All function and label names are given by me while analysis.}

\IFRU{Я начал с самого верха. Вот функция, берущая на вход два имени файла и пароль.}
{I started from top. Here is a function taking two file names and password.}

\begin{lstlisting}
.text:00541320 ; int __cdecl crypt_file(int Str, char *Filename, int password)
.text:00541320 crypt_file      proc near
.text:00541320
.text:00541320 Str             = dword ptr  4
.text:00541320 Filename        = dword ptr  8
.text:00541320 password        = dword ptr  0Ch
.text:00541320
\end{lstlisting}

\IFRU{Открыть файл и сообщить об ошибке в случае ошибки:}{Open file and report error in case of error:}

\begin{lstlisting}
.text:00541320                 mov     eax, [esp+Str]
.text:00541324                 push    ebp
.text:00541325                 push    offset Mode     ; "rb"
.text:0054132A                 push    eax             ; Filename
.text:0054132B                 call    _fopen          ; open file
.text:00541330                 mov     ebp, eax
.text:00541332                 add     esp, 8
.text:00541335                 test    ebp, ebp
.text:00541337                 jnz     short loc_541348
.text:00541339                 push    offset Format   ; "Cannot open input file!\n"
.text:0054133E                 call    _printf
.text:00541343                 add     esp, 4
.text:00541346                 pop     ebp
.text:00541347                 retn
.text:00541348 ; ---------------------------------------------------------------------------
.text:00541348
.text:00541348 loc_541348:
\end{lstlisting}

\IFRU{Узнать размер файла используя}{Get file size via} \TT{fseek()}/\TT{ftell()}:

\lstinputlisting{\IFRU{qr9/1ru.asm}{qr9/1en.asm}}

\IFRU{Этот кусок кода вычисляет длину файла выровненную по 64-байтной границе.
Это потому что этот алгоритм шифрования работает только с блоками размерами 64 байта.
Работает очень просто: разделить длину файла на 64, забыть об остатке, прибавить 1,
умножить на 64.
Следующий код удаляет остаток от деления как если бы это значение уже было разделено 
на 64 и добавляет 64. Это почти то же самое.}
{This piece of code calculates file size aligned to 64-byte border. 
This is because this cryptoalgorithm works with only 64-byte blocks. 
Its operation is pretty simple: divide file size by 64, forget about remainder and add 1, 
then multiple by 64. 
The following code removes remainder as if value was already divided by 64 and adds 64. 
It is almost the same.}

\lstinputlisting{\IFRU{qr9/2ru.asm}{qr9/2en.asm}}

\IFRU{Выделить буфер с выровненным размером:}{Allocate buffer with aligned size:}

\begin{lstlisting}
.text:00541373                 push    esi             ; Size
.text:00541374                 call    _malloc
\end{lstlisting}

\IFRU{Вызвать memset(), т.е., очистить выделенный буфер\footnote{malloc() + memset() можно было бы 
заменить на calloc()}.}{Call memset(), e,g, clear allocated buffer\footnote{malloc() + memset() could 
be replaced by calloc()}.}

\lstinputlisting{\IFRU{qr9/3ru.asm}{qr9/3en.asm}}

\IFRU{Чтение файла используя стандартную функцию Си}{Read file via standard C function} \TT{fread()}.

\begin{lstlisting}
.text:00541392                 mov     eax, [esp+38h+Str]
.text:00541396                 push    eax             ; ElementSize
.text:00541397                 push    ebx             ; DstBuf
.text:00541398                 call    _fread          ; read file
.text:0054139D                 push    ebp             ; File
.text:0054139E                 call    _fclose
\end{lstlisting}

\IFRU{Вызов \TT{crypt()}. Эта функция берет на вход буфер, длину буфера (выровненную) и строку пароля.}
{Call \TT{crypt()}. This function takes buffer, buffer size (aligned) and password string.}

\begin{lstlisting}
.text:005413A3                 mov     ecx, [esp+44h+password]
.text:005413A7                 push    ecx             ; password
.text:005413A8                 push    esi             ; aligned size
.text:005413A9                 push    ebx             ; buffer
.text:005413AA                 call    crypt           ; do crypt
\end{lstlisting}

\IFRU{Создать выходной файл. Кстати, разработчик забыл вставить проверку, создался ли файл успешно!
Результат открытия файла, впрочем, проверяется.}
{Create output file. By the way, developer forgot to check if it is was created correctly! 
File opening result is being checked though.}

\begin{lstlisting}
.text:005413AF                 mov     edx, [esp+50h+Filename]
.text:005413B3                 add     esp, 40h
.text:005413B6                 push    offset aWb      ; "wb"
.text:005413BB                 push    edx             ; Filename
.text:005413BC                 call    _fopen
.text:005413C1                 mov     edi, eax
\end{lstlisting}

\IFRU{Теперь хэндл созданного файла в регистре \EDI. Зписываем сигнатуру "QR9".}
{Newly created file handle is in \EDI register now. Write signature "QR9".}

\begin{lstlisting}
.text:005413C3                 push    edi             ; File
.text:005413C4                 push    1               ; Count
.text:005413C6                 push    3               ; Size
.text:005413C8                 push    offset aQr9     ; "QR9"
.text:005413CD                 call    _fwrite         ; write file signature
\end{lstlisting}

\IFRU{Записываем настоящую длину файла (не выровненную)}{Write actual file size (not aligned)}:

\begin{lstlisting}
.text:005413D2                 push    edi             ; File
.text:005413D3                 push    1               ; Count
.text:005413D5                 lea     eax, [esp+30h+Str]
.text:005413D9                 push    4               ; Size
.text:005413DB                 push    eax             ; Str
.text:005413DC                 call    _fwrite         ; write original file size
\end{lstlisting}

\IFRU{Записываем шифрованный буфер}{Write crypted buffer}:

\begin{lstlisting}
.text:005413E1                 push    edi             ; File
.text:005413E2                 push    1               ; Count
.text:005413E4                 push    esi             ; Size
.text:005413E5                 push    ebx             ; Str
.text:005413E6                 call    _fwrite         ; write crypted file
\end{lstlisting}

\IFRU{Закрыть файл и освободить выделенный буфер}{Close file and free allocated buffer}:

\begin{lstlisting}
.text:005413EB                 push    edi             ; File
.text:005413EC                 call    _fclose
.text:005413F1                 push    ebx             ; Memory
.text:005413F2                 call    _free
.text:005413F7                 add     esp, 40h
.text:005413FA                 pop     edi
.text:005413FB                 pop     esi
.text:005413FC                 pop     ebx
.text:005413FD                 pop     ebp
.text:005413FE                 retn
.text:005413FE crypt_file      endp
\end{lstlisting}

\IFRU{Переписанный на Си код}{Here is reconstructed C-code}:

\begin{lstlisting}
void crypt_file(char *fin, char* fout, char *pw)
{
	FILE *f;
	int flen, flen_aligned;
	BYTE *buf;

	f=fopen(fin, "rb");
	
	if (f==NULL)
	{
		printf ("Cannot open input file!\n");
		return;
	};

	fseek (f, 0, SEEK_END);
	flen=ftell (f);
	fseek (f, 0, SEEK_SET);

	flen_aligned=(flen&0xFFFFFFC0)+0x40;

	buf=(BYTE*)malloc (flen_aligned);
	memset (buf, 0, flen_aligned);

	fread (buf, flen, 1, f);

	fclose (f);

	crypt (buf, flen_aligned, pw);
	
	f=fopen(fout, "wb");

	fwrite ("QR9", 3, 1, f);
	fwrite (&flen, 4, 1, f);
	fwrite (buf, flen_aligned, 1, f);

	fclose (f);

	free (buf);
};
\end{lstlisting}

\IFRU{Процедура дешифрования почти такая же}{Decrypting procedure is almost the same}:

\begin{lstlisting}
.text:00541400 ; int __cdecl decrypt_file(char *Filename, int, void *Src)
.text:00541400 decrypt_file    proc near
.text:00541400
.text:00541400 Filename        = dword ptr  4
.text:00541400 arg_4           = dword ptr  8
.text:00541400 Src             = dword ptr  0Ch
.text:00541400
.text:00541400                 mov     eax, [esp+Filename]
.text:00541404                 push    ebx
.text:00541405                 push    ebp
.text:00541406                 push    esi
.text:00541407                 push    edi
.text:00541408                 push    offset aRb      ; "rb"
.text:0054140D                 push    eax             ; Filename
.text:0054140E                 call    _fopen
.text:00541413                 mov     esi, eax
.text:00541415                 add     esp, 8
.text:00541418                 test    esi, esi
.text:0054141A                 jnz     short loc_54142E
.text:0054141C                 push    offset aCannotOpenIn_0 ; "Cannot open input file!\n"
.text:00541421                 call    _printf
.text:00541426                 add     esp, 4
.text:00541429                 pop     edi
.text:0054142A                 pop     esi
.text:0054142B                 pop     ebp
.text:0054142C                 pop     ebx
.text:0054142D                 retn
.text:0054142E ; ---------------------------------------------------------------------------
.text:0054142E
.text:0054142E loc_54142E:
.text:0054142E                 push    2               ; Origin
.text:00541430                 push    0               ; Offset
.text:00541432                 push    esi             ; File
.text:00541433                 call    _fseek
.text:00541438                 push    esi             ; File
.text:00541439                 call    _ftell
.text:0054143E                 push    0               ; Origin
.text:00541440                 push    0               ; Offset
.text:00541442                 push    esi             ; File
.text:00541443                 mov     ebp, eax
.text:00541445                 call    _fseek
.text:0054144A                 push    ebp             ; Size
.text:0054144B                 call    _malloc
.text:00541450                 push    esi             ; File
.text:00541451                 mov     ebx, eax
.text:00541453                 push    1               ; Count
.text:00541455                 push    ebp             ; ElementSize
.text:00541456                 push    ebx             ; DstBuf
.text:00541457                 call    _fread
.text:0054145C                 push    esi             ; File
.text:0054145D                 call    _fclose
\end{lstlisting}

\IFRU{Проверяем сигнатуру (первые 3 байта)}{Check signature (first 3 bytes)}:

\begin{lstlisting}
.text:00541462                 add     esp, 34h
.text:00541465                 mov     ecx, 3
.text:0054146A                 mov     edi, offset aQr9_0 ; "QR9"
.text:0054146F                 mov     esi, ebx
.text:00541471                 xor     edx, edx
.text:00541473                 repe cmpsb
.text:00541475                 jz      short loc_541489
\end{lstlisting}

\IFRU{Сообщить об ошибке если сигнатура отсутствует}{Report an error if signature is absent}:

\begin{lstlisting}
.text:00541477                 push    offset aFileIsNotCrypt ; "File is not crypted!\n"
.text:0054147C                 call    _printf
.text:00541481                 add     esp, 4
.text:00541484                 pop     edi
.text:00541485                 pop     esi
.text:00541486                 pop     ebp
.text:00541487                 pop     ebx
.text:00541488                 retn
.text:00541489 ; ---------------------------------------------------------------------------
.text:00541489
.text:00541489 loc_541489:
\end{lstlisting}

\IFRU{Вызвать}{Call} \TT{decrypt()}.

\begin{lstlisting}
.text:00541489                 mov     eax, [esp+10h+Src]
.text:0054148D                 mov     edi, [ebx+3]
.text:00541490                 add     ebp, 0FFFFFFF9h
.text:00541493                 lea     esi, [ebx+7]
.text:00541496                 push    eax             ; Src
.text:00541497                 push    ebp             ; int
.text:00541498                 push    esi             ; int
.text:00541499                 call    decrypt
.text:0054149E                 mov     ecx, [esp+1Ch+arg_4]
.text:005414A2                 push    offset aWb_0    ; "wb"
.text:005414A7                 push    ecx             ; Filename
.text:005414A8                 call    _fopen
.text:005414AD                 mov     ebp, eax
.text:005414AF                 push    ebp             ; File
.text:005414B0                 push    1               ; Count
.text:005414B2                 push    edi             ; Size
.text:005414B3                 push    esi             ; Str
.text:005414B4                 call    _fwrite
.text:005414B9                 push    ebp             ; File
.text:005414BA                 call    _fclose
.text:005414BF                 push    ebx             ; Memory
.text:005414C0                 call    _free
.text:005414C5                 add     esp, 2Ch
.text:005414C8                 pop     edi
.text:005414C9                 pop     esi
.text:005414CA                 pop     ebp
.text:005414CB                 pop     ebx
.text:005414CC                 retn
.text:005414CC decrypt_file    endp
\end{lstlisting}

\IFRU{Переписанный на Си код}{Here is reconstructed C-code}:

\begin{lstlisting}
void decrypt_file(char *fin, char* fout, char *pw)
{
	FILE *f;
	int real_flen, flen;
	BYTE *buf;

	f=fopen(fin, "rb");
	
	if (f==NULL)
	{
		printf ("Cannot open input file!\n");
		return;
	};

	fseek (f, 0, SEEK_END);
	flen=ftell (f);
	fseek (f, 0, SEEK_SET);

	buf=(BYTE*)malloc (flen);

	fread (buf, flen, 1, f);

	fclose (f);

	if (memcmp (buf, "QR9", 3)!=0)
	{
		printf ("File is not crypted!\n");
		return;
	};

	memcpy (&real_flen, buf+3, 4);

	decrypt (buf+(3+4), flen-(3+4), pw);
	
	f=fopen(fout, "wb");

	fwrite (buf+(3+4), real_flen, 1, f);

	fclose (f);

	free (buf);
};
\end{lstlisting}

\IFRU{OK, посмотрим глубже}{OK, now let's go deeper}.

\IFRU{Функция}{Function} \TT{crypt()}:

\begin{lstlisting}
.text:00541260 crypt           proc near
.text:00541260
.text:00541260 arg_0           = dword ptr  4
.text:00541260 arg_4           = dword ptr  8
.text:00541260 arg_8           = dword ptr  0Ch
.text:00541260
.text:00541260                 push    ebx
.text:00541261                 mov     ebx, [esp+4+arg_0]
.text:00541265                 push    ebp
.text:00541266                 push    esi
.text:00541267                 push    edi
.text:00541268                 xor     ebp, ebp
.text:0054126A
.text:0054126A loc_54126A:
\end{lstlisting}

\IFRU{Этот кусок кода копирует часть входного буфера во внутренний буфер, который я поже назвал "cube64".}
{This piece of code copies part of input buffer to internal array I named later "cube64".}
\IFRU{Длина в регистре \ECX. \TT{MOVSD} означает "скопировать 32-битное слово", так что, 16 32-битных слов
это как раз 64 байта.}{The size is in \ECX register. \TT{MOVSD} means "move 32-bit dword", so, 
16 of 32-bit dwords are exactly 64 bytes.}

\begin{lstlisting}
.text:0054126A                 mov     eax, [esp+10h+arg_8]
.text:0054126E                 mov     ecx, 10h
.text:00541273                 mov     esi, ebx   ; EBX is pointer within input buffer
.text:00541275                 mov     edi, offset cube64
.text:0054127A                 push    1
.text:0054127C                 push    eax
.text:0054127D                 rep movsd
\end{lstlisting}

\IFRU{Вызвать}{Call} \TT{rotate\_all\_with\_password()}:

\begin{lstlisting}
.text:0054127F                 call    rotate_all_with_password
\end{lstlisting}

\IFRU{Скопировать зашифрованное содержимое из "cube64" назад в буфер}
{Copy crypted contents back from "cube64" to buffer}:

\begin{lstlisting}
.text:00541284                 mov     eax, [esp+18h+arg_4]
.text:00541288                 mov     edi, ebx
.text:0054128A                 add     ebp, 40h
.text:0054128D                 add     esp, 8
.text:00541290                 mov     ecx, 10h
.text:00541295                 mov     esi, offset cube64
.text:0054129A                 add     ebx, 40h  ; add 64 to input buffer pointer
.text:0054129D                 cmp     ebp, eax  ; EBP contain ammount of crypted data.
.text:0054129F                 rep movsd
\end{lstlisting}

\IFRU{Если \EBP не больше чем длина во входном аргументе, тогда переходим к следующему блоку.}
{If \EBP is not bigger that input argument size, then continue to next block.}

\begin{lstlisting}
.text:005412A1                 jl      short loc_54126A
.text:005412A3                 pop     edi
.text:005412A4                 pop     esi
.text:005412A5                 pop     ebp
.text:005412A6                 pop     ebx
.text:005412A7                 retn
.text:005412A7 crypt           endp
\end{lstlisting}

\IFRU{Реконструированная функция \TT{crypt()}}{Reconstructed \TT{crypt()} function}:

\begin{lstlisting}
void crypt (BYTE *buf, int sz, char *pw)
{
	int i=0;
	
	do
	{
		memcpy (cube, buf+i, 8*8);
		rotate_all (pw, 1);
		memcpy (buf+i, cube, 8*8);
		i+=64;
	}
	while (i<sz);
};
\end{lstlisting}

\IFRU{OK, углубимся в функцию \TT{rotate\_all\_with\_password()}. Она берет на вход два аргумента: 
строку пароля и число.}{OK, now let's go deeper into function \TT{rotate\_all\_with\_password()}. 
It takes two arguments: password string and number.}
\IFRU{В функции \TT{crypt()}, число 1 используется и в \TT{decrypt()} (где \TT{rotate\_all\_with\_password()}
функция вызывается также), число 3.}
{In \TT{crypt()}, number 1 is used, and in \TT{decrypt()} (where \TT{rotate\_all\_with\_password()} function 
is called too), number is 3.}

\begin{lstlisting}
.text:005411B0 rotate_all_with_password proc near
.text:005411B0
.text:005411B0 arg_0           = dword ptr  4
.text:005411B0 arg_4           = dword ptr  8
.text:005411B0
.text:005411B0                 mov     eax, [esp+arg_0]
.text:005411B4                 push    ebp
.text:005411B5                 mov     ebp, eax
\end{lstlisting}

\IFRU{Проверяем символы в пароле. Если это ноль, выходим:}{Check for character in password. If it is zero, exit:}

\begin{lstlisting}
.text:005411B7                 cmp     byte ptr [eax], 0
.text:005411BA                 jz      exit
.text:005411C0                 push    ebx
.text:005411C1                 mov     ebx, [esp+8+arg_4]
.text:005411C5                 push    esi
.text:005411C6                 push    edi
.text:005411C7
.text:005411C7 loop_begin:
\end{lstlisting}

\IFRU{Вызываем \TT{tolower()}, стандартную функцию Си.}{Call \TT{tolower()}, standard C function.}

\begin{lstlisting}
.text:005411C7                 movsx   eax, byte ptr [ebp+0]
.text:005411CB                 push    eax             ; C
.text:005411CC                 call    _tolower
.text:005411D1                 add     esp, 4
\end{lstlisting}

\IFRU{Хмм, если пароль содержит символ не из латинского алфавита, он пропускается!
Действительно, если мы запускаем утилиту для шифрования используя символы не латинского алфавита, 
похоже, они просто игнорируются.}
{Hmm, if password contains non-alphabetical latin character, it is skipped! 
Indeed, if we run crypting utility and try non-alphabetical latin characters in password, 
they seem to be ignored.}

\begin{lstlisting}
.text:005411D4                 cmp     al, 'a'
.text:005411D6                 jl      short next_character_in_password
.text:005411D8                 cmp     al, 'z'
.text:005411DA                 jg      short next_character_in_password
.text:005411DC                 movsx   ecx, al
\end{lstlisting}

\IFRU{Отнимем значение "a" (97) от символа.}{Subtract "a" value (97) from character.}

\begin{lstlisting}
.text:005411DF                 sub     ecx, 'a'  ; 97
\end{lstlisting}

\IFRU{После вычитания, тут будет 0 для "a", 1 для "b", и так далее. И 25 для "z".}
{After subtracting, we'll get 0 for "a" here, 1 for "b", etc. And 25 for "z".}

\begin{lstlisting}
.text:005411E2                 cmp     ecx, 24
.text:005411E5                 jle     short skip_subtracting
.text:005411E7                 sub     ecx, 24
\end{lstlisting}

\IFRU{Похоже, символы "y" и "z" также исключительные.
После этого куска кода, "y" становится 0, а "z" ~--- 1.
Это значит что 26 латинских букв становятся значениями в интервале 0..23, (всего 24).}
{It seems, "y" and "z" are exceptional characters too. 
After that piece of code, "y" becomes 0 and "z" ~--- 1. 
This means, 26 latin alphabet symbols will become values in range 0..23, (24 in total).}

\begin{lstlisting}
.text:005411EA
.text:005411EA skip_subtracting:                       ; CODE XREF: rotate_all_with_password+35
\end{lstlisting}

\IFRU{Это, на самом деле, деление через умножение.
Читайте об этом больше в секции "\DivisionByNineSectionName"~\ref{sec:divisionbynine}.}
{This is actually division via multiplication. 
Read more about it in "\DivisionByNineSectionName" section~\ref{sec:divisionbynine}.}

\IFRU{Это код, на самом деле, делит значение символа пароля на 3.}
{The code actually divides password character value by 3.}

\begin{lstlisting}
.text:005411EA                 mov     eax, 55555556h
.text:005411EF                 imul    ecx
.text:005411F1                 mov     eax, edx
.text:005411F3                 shr     eax, 1Fh
.text:005411F6                 add     edx, eax
.text:005411F8                 mov     eax, ecx
.text:005411FA                 mov     esi, edx
.text:005411FC                 mov     ecx, 3
.text:00541201                 cdq
.text:00541202                 idiv    ecx
\end{lstlisting}

\IFRU{\EDX ~--- остаток от деления.}{\EDX is the remainder of division.}

\lstinputlisting{\IFRU{qr9/4ru.asm}{qr9/4en.asm}}

\IFRU{Если остаток 2, вызываем \TT{rotate3()}. 
\EDX это второй аргумент функции \TT{rotate\_all\_with\_password()}. 
Как я уже писал, 1 это для шифрования, 3 для дешифрования.
Так что здесь цикл, функции rotate1/2/3 будут вызываться столько же раз, сколько значение переменной
в первом аргументе.}
{If remainder is 2, call \TT{rotate3()}. 
\EDI is a second argument of \TT{rotate\_all\_with\_password()}. 
As I already wrote, 1 is for crypting operations and 3 is for decrypting. 
So, here is a loop. When crypting, rotate1/2/3 will be called the same number of times as 
given in the first argument.}

\begin{lstlisting}
.text:00541215 call_rotate3:
.text:00541215                 push    esi
.text:00541216                 call    rotate3
.text:0054121B                 add     esp, 4
.text:0054121E                 dec     edi
.text:0054121F                 jnz     short call_rotate3
.text:00541221                 jmp     short next_character_in_password
.text:00541223
.text:00541223 call_rotate2:
.text:00541223                 test    ebx, ebx
.text:00541225                 jle     short next_character_in_password
.text:00541227                 mov     edi, ebx
.text:00541229
.text:00541229 loc_541229:
.text:00541229                 push    esi
.text:0054122A                 call    rotate2
.text:0054122F                 add     esp, 4
.text:00541232                 dec     edi
.text:00541233                 jnz     short loc_541229
.text:00541235                 jmp     short next_character_in_password
.text:00541237
.text:00541237 call_rotate1:
.text:00541237                 test    ebx, ebx
.text:00541239                 jle     short next_character_in_password
.text:0054123B                 mov     edi, ebx
.text:0054123D
.text:0054123D loc_54123D:
.text:0054123D                 push    esi
.text:0054123E                 call    rotate1
.text:00541243                 add     esp, 4
.text:00541246                 dec     edi
.text:00541247                 jnz     short loc_54123D
.text:00541249
\end{lstlisting}

\IFRU{Достать следующий символ из строки пароля.}{Fetch next character from password string.}

\begin{lstlisting}
.text:00541249 next_character_in_password:
.text:00541249                 mov     al, [ebp+1]
\end{lstlisting}

\IFRU{Инкремент указателя на символ в строке пароля:}{Increment character pointer within password string:}

\begin{lstlisting}
.text:0054124C                 inc     ebp
.text:0054124D                 test    al, al
.text:0054124F                 jnz     loop_begin
.text:00541255                 pop     edi
.text:00541256                 pop     esi
.text:00541257                 pop     ebx
.text:00541258
.text:00541258 exit:
.text:00541258                 pop     ebp
.text:00541259                 retn
.text:00541259 rotate_all_with_password endp
\end{lstlisting}

\IFRU{Реконструированный код на Си:}{Here is reconstructed C code:}

\begin{lstlisting}
void rotate_all (char *pwd, int v)
{
	char *p=pwd;

	while (*p)
	{
		char c=*p;
		int q;

		c=tolower (c);

		if (c>='a' && c<='z')
		{
			q=c-'a';
			if (q>24)
				q-=24;

			int quotient=q/3;
			int remainder=q % 3;

			switch (remainder)
			{
			case 0: for (int i=0; i<v; i++) rotate1 (quotient); break;
			case 1: for (int i=0; i<v; i++) rotate2 (quotient); break;
			case 2: for (int i=0; i<v; i++) rotate3 (quotient); break;
			};
		};

		p++;
	};
};
\end{lstlisting}

\IFRU{Углубимся еще дальше и исследуем функции rotate1/2/3.
Каждая функция вызывает еще две.
В итоге я назвал их \TT{set\_bit()} и \TT{get\_bit()}.}
{Now let's go deeper and investigate rotate1/2/3 functions. 
Each function calls two another functions. 
I eventually gave them names \TT{set\_bit()} and \TT{get\_bit()}.}

\IFRU{Начнем с \TT{get\_bit()}:}{Let's start with \TT{get\_bit()}:}

\begin{lstlisting}
.text:00541050 get_bit         proc near
.text:00541050
.text:00541050 arg_0           = dword ptr  4
.text:00541050 arg_4           = dword ptr  8
.text:00541050 arg_8           = byte ptr  0Ch
.text:00541050
.text:00541050                 mov     eax, [esp+arg_4]
.text:00541054                 mov     ecx, [esp+arg_0]
.text:00541058                 mov     al, cube64[eax+ecx*8]
.text:0054105F                 mov     cl, [esp+arg_8]
.text:00541063                 shr     al, cl
.text:00541065                 and     al, 1
.text:00541067                 retn
.text:00541067 get_bit         endp
\end{lstlisting}

\IFRU{... иными словами: подсчитать индекс в массиве cube64}{... in other words: calculate an index in 
the array cube64}: \IT{arg\_4 + arg\_0 * 8}.
\IFRU{Затем сдвинуть байт из массива вправо на количество бит заданных в arg\_8. 
Изолировать самый младший бит и вернуть его}{Then shift a byte from an array by arg\_8 bits right. 
Isolate lowest bit and return it.}

\IFRU{Посмотрим другую функцию}{Let's see another function}, \TT{set\_bit()}:

\begin{lstlisting}
.text:00541000 set_bit         proc near
.text:00541000
.text:00541000 arg_0           = dword ptr  4
.text:00541000 arg_4           = dword ptr  8
.text:00541000 arg_8           = dword ptr  0Ch
.text:00541000 arg_C           = byte ptr  10h
.text:00541000
.text:00541000                 mov     al, [esp+arg_C]
.text:00541004                 mov     ecx, [esp+arg_8]
.text:00541008                 push    esi
.text:00541009                 mov     esi, [esp+4+arg_0]
.text:0054100D                 test    al, al
.text:0054100F                 mov     eax, [esp+4+arg_4]
.text:00541013                 mov     dl, 1
.text:00541015                 jz      short loc_54102B
\end{lstlisting}

\IFRU{DL тут равно 1. Сдвигаем эту единицу на количество указанное в arg\_8. Например, если в arg\_8 число 4,
тогда значение в DL станет 0x10 или 1000 в двоичной системе счисления.}
{DL is 1 here. Shift left it by arg\_8. For example, if arg\_8 is 4, DL register value became 
0x10 or 1000 in binary form.}

\begin{lstlisting}
.text:00541017                 shl     dl, cl
.text:00541019                 mov     cl, cube64[eax+esi*8]
\end{lstlisting}

\IFRU{Вытащить бит из массива и явно выставить его.}{Get bit from array and explicitly set one.} % TODO: rewrite

\begin{lstlisting}
.text:00541020                 or      cl, dl
\end{lstlisting}

\IFRU{Сохранить его назад:}{Store it back:} % TODO: rewrite

\begin{lstlisting}
.text:00541022                 mov     cube64[eax+esi*8], cl
.text:00541029                 pop     esi
.text:0054102A                 retn
.text:0054102B ; ---------------------------------------------------------------------------
.text:0054102B
.text:0054102B loc_54102B:
.text:0054102B                 shl     dl, cl
\end{lstlisting}

\IFRU{Если arg\_C не ноль...}{If arg\_C is not zero...}

\begin{lstlisting}
.text:0054102D                 mov     cl, cube64[eax+esi*8]
\end{lstlisting}

\IFRU{... инвертировать DL. Например, если состояние DL после сдвига 0x10 или 1000 в двоичной системе,
здесь будет 0xEF после инструкции \NOT или 11101111 в двоичной системе.}
{... invert DL. For example, if DL state after shift was 0x10 or 1000 in binary form, 
there will be 0xEF after \NOT instruction or 11101111 in binary form.}

\begin{lstlisting}
.text:00541034                 not     dl
\end{lstlisting}

\IFRU{Эта инструкция сбрасывает бит, иными словами, она сохраняет все биты в CL которые так же
выставлены в DL кроме тех в DL, что были сброшены. Это значит что если в DL, например,
11101111 в двоичной системе, все биты будут сохранены кроме пятого (считая с младшего бита).}
{This instruction clears bit, in other words, it saves all bits in CL which are also set in 
DL except those in DL which are cleared. This means that if DL is, for example, 
11101111 in binary form, all bits will be saved except 5th (counting from lowest bit).}

\begin{lstlisting}
.text:00541036                 and     cl, dl
\end{lstlisting}

\IFRU{Сохранить его назад}{Store it back:}

\begin{lstlisting}
.text:00541038                 mov     cube64[eax+esi*8], cl
.text:0054103F                 pop     esi
.text:00541040                 retn
.text:00541040 set_bit         endp
\end{lstlisting}

\IFRU{Это почти то же самое что и \TT{get\_bit()} кроме того что если arg\_C ноль, тогда функция сбрасывает
указанный бит в массиве, либо же, в противном случае, выставляет его в 1.}
{It is almost the same as \TT{get\_bit()}, except, if arg\_C is zero, the function clears specific bit in array, 
or sets it otherwise.}

\IFRU{Мы так же знаем что размер массива 64. Первые два аргумента и у \TT{set\_bit()} и у \TT{get\_bit()}
могут быть представлены как двумерные координаты. Таким образом, массив это матрица 8*8.}
{We also know that array size is 64. First two arguments both in \TT{set\_bit()} and \TT{get\_bit()} 
could be seen as 2D cooridnates. Then array will be 8*8 matrix.}

\IFRU{Представление на Си всего того, что мы уже знаем:}{Here is C representation of what we already know:}

\begin{lstlisting}
#define IS_SET(flag, bit)       ((flag) & (bit))
#define SET_BIT(var, bit)       ((var) |= (bit))
#define REMOVE_BIT(var, bit)    ((var) &= ~(bit))

char cube[8][8];

void set_bit (int x, int y, int shift, int bit)
{
	if (bit)
		SET_BIT (cube[x][y], 1<<shift);
	else
		REMOVE_BIT (cube[x][y], 1<<shift);
};

int get_bit (int x, int y, int shift)
{
	if ((cube[x][y]>>shift)&1==1)
		return 1;
	return 0;
};
\end{lstlisting}

\IFRU{Теперь вернемся к функциям rotate1/2/3.}{Now let's get back to rotate1/2/3 functions.}

\begin{lstlisting}
.text:00541070 rotate1         proc near
.text:00541070
\end{lstlisting}

\IFRU{Выделение внутреннего массива размером 64 байта в локальном стеке:}
{Internal array allocation in local stack, its size 64 bytes:}

\begin{lstlisting}
.text:00541070 internal_array_64= byte ptr -40h
.text:00541070 arg_0           = dword ptr  4
.text:00541070
.text:00541070                 sub     esp, 40h
.text:00541073                 push    ebx
.text:00541074                 push    ebp
.text:00541075                 mov     ebp, [esp+48h+arg_0]
.text:00541079                 push    esi
.text:0054107A                 push    edi
.text:0054107B                 xor     edi, edi        ; EDI is loop1 counter
\end{lstlisting}

\EBX \IFRU{указывает на внутренний массив}{is a pointer to internal array:}

\begin{lstlisting}
.text:0054107D                 lea     ebx, [esp+50h+internal_array_64]
.text:00541081
\end{lstlisting}

\IFRU{Здесь два вложенных цикла:}{Two nested loops are here:}

\lstinputlisting{\IFRU{qr9/5ru.asm}{qr9/5ru.asm}}

\IFRU{Мы видим что оба счетчика циклов в интервале 0..7. 
Также, они используются как первый и второй аргумент \TT{get\_bit()}.
Третий аргумент \TT{get\_bit()} это единственный аргумент \TT{rotate1()}. 
То что возвращает \TT{get\_bit()} будет сохранено во внутреннем массиве.}
{... we see that both loop counters are in range 0..7. 
Also, they are used as first and second arguments of \TT{get\_bit()}. 
Third argument of \TT{get\_bit()} is the only argument of \TT{rotate1()}. 
What \TT{get\_bit()} returns, is being placed into internal array.}

\IFRU{Снова приготовить указатель на внутренний массив:}{Prepare pointer to internal array again:}

\lstinputlisting{\IFRU{qr9/6ru.asm}{qr9/6en.asm}}

\IFRU{... этот код кладет содержимое из внутреннего массива в глобальный массив cube используя функцию 
\TT{set\_bit()}, \IT{но}, в обратном порядке!
Теперь счетчик первого цикла в интервале 7 до 0, уменьшается на 1 на каждой итерации!}
{... this code is placing contents from internal array to cube global array via \TT{set\_bit()} function, 
\IT{but}, in different order!
Now loop 1 counter is in range 7 to 0, decrementing at each iteration!}

\IFRU{Представление кода на Си выглядит так:}{C code representation looks like:}

\begin{lstlisting}
void rotate1 (int v)
{
	bool tmp[8][8]; // internal array
	int i, j;

	for (i=0; i<8; i++)
		for (j=0; j<8; j++)
			tmp[i][j]=get_bit (i, j, v);

	for (i=0; i<8; i++)
		for (j=0; j<8; j++)
			set_bit (j, 7-i, v, tmp[x][y]);
};
\end{lstlisting}

\IFRU{Не очень понятно, но если мы посмотрим в функцию \TT{rotate2()}:}
{Not very understandable, but if we will take a look at \TT{rotate2()} function:}

\lstinputlisting{\IFRU{qr9/7ru.asm}{qr9/7en.asm}}

\IFRU{\IT{Почти} то же самое, за исключением порядка аргументов в \TT{get\_bit()} и \TT{set\_bit()}.
Перепишем это на Си-подобный код:}
{It is \IT{almost} the same, except of order of arguments of \TT{get\_bit()} and \TT{set\_bit()}. 
Let's rewrite it in C-like code:}

\begin{lstlisting}
void rotate2 (int v)
{
	bool tmp[8][8]; // internal array
	int i, j;

	for (i=0; i<8; i++)
		for (j=0; j<8; j++)
			tmp[i][j]=get_bit (v, i, j);

	for (i=0; i<8; i++)
		for (j=0; j<8; j++)
			set_bit (v, j, 7-i, tmp[i][j]);
};
\end{lstlisting}

\IFRU{Перепишем также функцию \TT{rotate3()}:}{Let's also rewrite \TT{rotate3()} function:}

\begin{lstlisting}
void rotate3 (int v)
{
	bool tmp[8][8];
	int i, j;

	for (i=0; i<8; i++)
		for (j=0; j<8; j++)
			tmp[i][j]=get_bit (i, v, j);

	for (i=0; i<8; i++)
		for (j=0; j<8; j++)
			set_bit (7-j, v, i, tmp[i][j]);
};
\end{lstlisting}

\IFRU{Теперь всё проще. Если мы представим cube64 как трехмерный куб 8*8*8, где каждый элемент это бит,
то \TT{get\_bit()} и \TT{set\_bit()} просто берут на вход координаты бита.}
{Well, now things are simpler. If we consider cube64 as 3D cube 8*8*8, where each element is bit, 
\TT{get\_bit()} and \TT{set\_bit()} take just coordinates of bit on input.}

\IFRU{Функции rotate1/2/3 просто поворачивают все биты на определенной плоскости.
Три функции, каждая на каждую сторону куба и аргумент v выставляет плоскость в интервале 0..7}
{rotate1/2/3 functions are in fact rotating all bits in specific plane. 
Three functions are each for each cube side and v argument is setting plane in range 0..7.}

\newcommand{\URLWPRU}{http://en.wikipedia.org/wiki/Rubik's_Cube}

\IFRU{Может быть, автор алгоритма думал о \href{\URLWPRU}{кубике Рубика} 8*8*8?!}
{Maybe, algorithm's author was thinking of 8*8*8 \href{\URLWPRU}{Rubik's cube}?!}

\IFRU{Да, действительно.}{Yes, indeed.}

\IFRU{Рассмотрим функцию \TT{decrypt()}, я переписал её:}{Let's get closer into \TT{decrypt()} function, 
I rewrote it here:}

\begin{lstlisting}
void decrypt (BYTE *buf, int sz, char *pw)
{
	char *p=strdup (pw);
	strrev (p);
	int i=0;

	do
	{
		memcpy (cube, buf+i, 8*8);
		rotate_all (p, 3);
		memcpy (buf+i, cube, 8*8);
		i+=64;
	}
	while (i<sz);
	
	free (p);
};
\end{lstlisting}

\newcommand{\URLMSDNSTRREV}{http://msdn.microsoft.com/en-us/library/9hby7w40(VS.80).aspx}

\IFRU{Почти то же самое что и crypt(), \IT{но} строка пароля разворачивается стандартной функцией Си
\href{\URLMSDNSTRREV}{strrev()} и \TT{rotate\_all()} вызывается с аргументом 3.}
{It is almost the same excert of \TT{crypt()}, \IT{but} password string is reversed by 
\href{\URLMSDNSTRREV}{strrev()} standard C function and \TT{rotate\_all()} is called with argument 3.} 

\IFRU{Это значит что, в случае дешифровки, rotate1/2/3 будут вызываться трижды.}
{This means, that in case of decryption, each corresponding rotate1/2/3 call will be performed thrice.}

\IFRU{Это почти кубик Рубика!
Если вы хотите вернуть его состояние назад, делайте то же самое в обратном порядке и направлении!
Чтобы вернуть эффект от поворота плоскости по часовой стрелке, нужно повернуть её же против 
часовой стрелки трижды.}
{This is almost as in Rubik'c cube! 
If you want to get back, do the same in reverse order and direction! 
If you need to undo effect of rotating one place in clockwise direction, rotate it thrice
in counter-clockwise direction.}

\IFRU{\TT{rotate1()}, вероятно, поворот "лицевой" плоскости. \TT{rotate2()}, вероятно, поворот "верхней" плоскости.
\TT{rotate3()}, вероятно, поворот "левой" плоскости.}
{\TT{rotate1()} is probably for rotating "front" plane. \TT{rotate2()} is probably for rotating "top" plane. 
\TT{rotate3()} is probably for rotating "left" plane.}

\IFRU{Вернемся к ядру функции \TT{rotate\_all()}}{Let's get back to core of \TT{rotate\_all()} function:}

\begin{lstlisting}
q=c-'a';
if (q>24)
	q-=24;

int quotient=q/3; // in range 0..7
int remainder=q % 3;

switch (remainder)
{
    case 0: for (int i=0; i<v; i++) rotate1 (quotient); break; // front
    case 1: for (int i=0; i<v; i++) rotate2 (quotient); break; // top
    case 2: for (int i=0; i<v; i++) rotate3 (quotient); break; // left
};
\end{lstlisting}

\IFRU{Так понять проще: каждый символ пароля определяет сторону (одну из трех) и плоскость (одну из восьми).
3*8 = 24, вот почему два последних символа латинского алфавита переопределяются так чтобы алфавит состоял
из 24-х элементов.}
{Now it is much simpler to understand: each password character defines side (one of three) and plane (one of 8). 
3*8 = 24, that's why two last characters of latin alphabet are remapped to fit an alphabet of exactly 
24 elements.}

\IFRU{Алгоритм очевидно слаб: в случае коротких паролей, в бинарном редакторе файлов можно будет увидеть, 
что в зашифрованных файлах остались незашифрованные символы.}
{The algorithm is clearly weak: in case of short passwords, one can see, that in crypted file there are 
some original bytes of original file in binary files editor.}

\IFRU{Весь исходный код в реконструированном виде:}{Here is reconstructed whole source code:}

\lstinputlisting{qr9/qr9.cpp}




\section{\IFRU{Случай сжимания сетевого траффика в клиенте SAP}{SAP client network traffic compression case}}
\label{sec:SAPGUI}

\newcommand{\TDWNC}{TDW\_NOCOMPRESS\xspace}

\IFRU{(Трассировка связи между переменной окружения \TDWNC{} SAPGUI\footnote{GUI-клиент от SAP}
до ``надоедливого всплывающего окна'' и самой функции сжатия данных.)}
{(Tracing connection between \TDWNC{} SAPGUI\footnote{SAP GUI client} environment variable to 
nagging pop-up window and actual data compression routine.)}

\IFRU{Известно что сетевой траффик между SAPGUI и SAP по умолчанию не шифруется а сжимается} 
{It's known that network traffic between SAPGUI and SAP is not crypted by default, it's rather compressed}
(\IFRU{чиайте}{read} \href{http://blogs.conus.info/node/44}{\IFRU{здесь}{here}} \IFRU{и}{and} \href{http://blogs.conus.info/node/47}{\IFRU{здесь}{here}}). 

\IFRU{Известно также что если установить переменную окружения \IT{\TDWNC} в 1, можно выключить сжатие сетевых пакетов.}
{It's also known that by setting environment variable \IT{\TDWNC} to 1, it's possible to turn network packets compression off.}

\IFRU{Но вы увидите окно, которое нельзя будет закрыть}{But you will see a nagging pop-up windows that cannot be closed}:

\begin{figure}[ht!]
\centering
\includegraphics[scale=0.66]{sapgui/sapgui720.png}
\caption{\IFRU{Скриншот}{Screenshot}}
\end{figure}

\IFRU{Посмотрим, сможем ли мы как-то убрать это окно}{Let's see, if we can remove that window somehow}.

\IFRU{Но в начале давайте посмотрим, что мы уже знаем}{But before this, let's see what we already know}.
\IFRU{Первое: мы знаем что переменна окружения \IT{\TDWNC} проверяется где-то внутри клиента SAPGUI.}
{First: we know that environment variable \IT{\TDWNC} is checked somewhere inside of SAPGUI client.}
\IFRU{Второе: строка вроде ``data compression switched off'' так же должна где-то присутствовать.}
{Second: string like ``data compression switched off'' must be present somewhere too.}
\IFRU{При помощи файлового менджера FAR я нашел обе эти строки в файле SAPguilib.dll.}
{With the help of FAR file manager I found that both of these strings are stored in the SAPguilib.dll file.}

\IFRU{Так что давайте откроем файл SAPguilib.dll в \IDA и поищем там строку \IT{``\TDWNC''}.
Да, она присутствует и имеется только одна ссылка на эту строку.}
{So let's open SAPguilib.dll in \IDA and search for \IT{``\TDWNC''} string. 
Yes, it is present and there is only one reference to it.}

\IFRU{Мы увидим такой кусок кода
(все смещения верны для версии SAPGUI 720 win32, SAPguilib.dll версия файла 7200,1,0,9009)}
{We see the following piece of code 
(all file offsets are valid for SAPGUI 720 win32, SAPguilib.dll file version 7200,1,0,9009)}:

\begin{lstlisting}
.text:6440D51B                 lea     eax, [ebp+2108h+var_211C]
.text:6440D51E                 push    eax             ; int
.text:6440D51F                 push    offset aTdw_nocompress ; "TDW_NOCOMPRESS"
.text:6440D524                 mov     byte ptr [edi+15h], 0
.text:6440D528                 call    chk_env
.text:6440D52D                 pop     ecx
.text:6440D52E                 pop     ecx
.text:6440D52F                 push    offset byte_64443AF8
.text:6440D534                 lea     ecx, [ebp+2108h+var_211C]

; demangled name: int ATL::CStringT::Compare(char const *)const
.text:6440D537                 call    ds:mfc90_1603
.text:6440D53D                 test    eax, eax
.text:6440D53F                 jz      short loc_6440D55A
.text:6440D541                 lea     ecx, [ebp+2108h+var_211C]

; demangled name: const char* ATL::CSimpleStringT::operator PCXSTR 
.text:6440D544                 call    ds:mfc90_910
.text:6440D54A                 push    eax             ; Str
.text:6440D54B                 call    ds:atoi
.text:6440D551                 test    eax, eax
.text:6440D553                 setnz   al
.text:6440D556                 pop     ecx
.text:6440D557                 mov     [edi+15h], al
\end{lstlisting}

\IFRU{Строка возвращаемая функцией \TT{chk\_env()} через второй аргумент, обрабатывается далее строковыми
функциями MFC, затем вызывается \TT{atoi()}\footnote{Стандартная функция Си, конвертирующая число в строке в число}.
После этого, число сохраняется в \TT{edi+15h}}
{String returned by \TT{chk\_env()} via second argument is then handled by MFC string functions and then 
\TT{atoi()}\footnote{standard C library function, coverting number in string into number} is called. 
After that, numerical value is stored to \TT{edi+15h}}.

\IFRU{Обратите так же внимание на функцию \TT{chk\_env} (это я так назвал её)}
{Also, take a look onto \TT{chk\_env()} function (I gave a name to it)}:

\begin{lstlisting}
.text:64413F20 ; int __cdecl chk_env(char *VarName, int)
.text:64413F20 chk_env         proc near
.text:64413F20
.text:64413F20 DstSize         = dword ptr -0Ch
.text:64413F20 var_8           = dword ptr -8
.text:64413F20 DstBuf          = dword ptr -4
.text:64413F20 VarName         = dword ptr  8
.text:64413F20 arg_4           = dword ptr  0Ch
.text:64413F20
.text:64413F20                 push    ebp
.text:64413F21                 mov     ebp, esp
.text:64413F23                 sub     esp, 0Ch
.text:64413F26                 mov     [ebp+DstSize], 0
.text:64413F2D                 mov     [ebp+DstBuf], 0
.text:64413F34                 push    offset unk_6444C88C
.text:64413F39                 mov     ecx, [ebp+arg_4]

; (demangled name) ATL::CStringT::operator=(char const *)
.text:64413F3C                 call    ds:mfc90_820 
.text:64413F42                 mov     eax, [ebp+VarName]
.text:64413F45                 push    eax             ; VarName
.text:64413F46                 mov     ecx, [ebp+DstSize]
.text:64413F49                 push    ecx             ; DstSize
.text:64413F4A                 mov     edx, [ebp+DstBuf]
.text:64413F4D                 push    edx             ; DstBuf
.text:64413F4E                 lea     eax, [ebp+DstSize]
.text:64413F51                 push    eax             ; ReturnSize
.text:64413F52                 call    ds:getenv_s
.text:64413F58                 add     esp, 10h
.text:64413F5B                 mov     [ebp+var_8], eax
.text:64413F5E                 cmp     [ebp+var_8], 0
.text:64413F62                 jz      short loc_64413F68
.text:64413F64                 xor     eax, eax
.text:64413F66                 jmp     short loc_64413FBC
.text:64413F68 ; ---------------------------------------------------------------------------
.text:64413F68
.text:64413F68 loc_64413F68:
.text:64413F68                 cmp     [ebp+DstSize], 0
.text:64413F6C                 jnz     short loc_64413F72
.text:64413F6E                 xor     eax, eax
.text:64413F70                 jmp     short loc_64413FBC
.text:64413F72 ; ---------------------------------------------------------------------------
.text:64413F72
.text:64413F72 loc_64413F72:
.text:64413F72                 mov     ecx, [ebp+DstSize]
.text:64413F75                 push    ecx
.text:64413F76                 mov     ecx, [ebp+arg_4]

; demangled name: ATL::CSimpleStringT<char, 1>::Preallocate(int)
.text:64413F79                 call    ds:mfc90_2691
.text:64413F7F                 mov     [ebp+DstBuf], eax
.text:64413F82                 mov     edx, [ebp+VarName]
.text:64413F85                 push    edx             ; VarName
.text:64413F86                 mov     eax, [ebp+DstSize]
.text:64413F89                 push    eax             ; DstSize
.text:64413F8A                 mov     ecx, [ebp+DstBuf]
.text:64413F8D                 push    ecx             ; DstBuf
.text:64413F8E                 lea     edx, [ebp+DstSize]
.text:64413F91                 push    edx             ; ReturnSize
.text:64413F92                 call    ds:getenv_s
.text:64413F98                 add     esp, 10h
.text:64413F9B                 mov     [ebp+var_8], eax
.text:64413F9E                 push    0FFFFFFFFh
.text:64413FA0                 mov     ecx, [ebp+arg_4]

; demangled name: ATL::CSimpleStringT::ReleaseBuffer(int)
.text:64413FA3                 call    ds:mfc90_5835
.text:64413FA9                 cmp     [ebp+var_8], 0
.text:64413FAD                 jz      short loc_64413FB3
.text:64413FAF                 xor     eax, eax
.text:64413FB1                 jmp     short loc_64413FBC
.text:64413FB3 ; ---------------------------------------------------------------------------
.text:64413FB3
.text:64413FB3 loc_64413FB3:
.text:64413FB3                 mov     ecx, [ebp+arg_4]

; demangled name: const char* ATL::CSimpleStringT::operator PCXSTR 
.text:64413FB6                 call    ds:mfc90_910
.text:64413FBC
.text:64413FBC loc_64413FBC:
.text:64413FBC
.text:64413FBC                 mov     esp, ebp
.text:64413FBE                 pop     ebp
.text:64413FBF                 retn
.text:64413FBF chk_env         endp
\end{lstlisting}

\IFRU{Да}{Yes}. \IFRU{Функция}{} \TT{getenv\_s()}\footnote{\url{http://msdn.microsoft.com/en-us/library/tb2sfw2z(VS.80).aspx}} 
\IFRU{это \IT{безопасная} версия функции \TT{getenv()}\footnote{Стандартная функция Си возвращающая значение переменной окружения} в MSVC}
{function is Microsoft security-enhanced version of \TT{getenv()}\footnote{Standard C library returning environment variable}}.

\IFRU{Тут так же имеются манипуляции со строками при помощи функций из MFC}{There are also some MFC string manipulations}.

\IFRU{Множество других переменных окружения также проверяются. Здесь список всех переменных проверяемых SAPGUI 
а так же сообщение записываемое им в лог-файл, если переменная включена}{Lots of other environment variables are checked as well. 
Here is a list of all variables being checked and what SAPGUI could write to trace log when logging is turned on}:

\begin{center}
\begin{tabular}{ | l | l | }
\hline                        
DPTRACE                  & ``GUI-OPTION: Trace set to \%d'' \\
TDW\_HEXDUMP             & ``GUI-OPTION: Hexdump enabled'' \\
TDW\_WORKDIR             & ``GUI-OPTION: working directory `\%s\''' \\
TDW\_SPLASHSRCEENOFF     & ``GUI-OPTION: Splash Screen Off'' / ``GUI-OPTION: Splash Screen On'' \\
TDW\_REPLYTIMEOUT        & ``GUI-OPTION: reply timeout \%d milliseconds'' \\
TDW\_PLAYBACKTIMEOUT     & ``GUI-OPTION: PlaybackTimeout  set to \%d milliseconds'' \\ 
TDW\_NOCOMPRESS          & ``GUI-OPTION: no compression read'' \\
TDW\_EXPERT              & ``GUI-OPTION: expert mode'' \\
TDW\_PLAYBACKPROGRESS    & ``GUI-OPTION: PlaybackProgress'' \\
TDW\_PLAYBACKNETTRAFFIC  & ``GUI-OPTION: PlaybackNetTraffic'' \\
TDW\_PLAYLOG             & ``GUI-OPTION: /PlayLog is YES, file \%s'' \\
TDW\_PLAYTIME            & ``GUI-OPTION: /PlayTime set to \%d milliseconds'' \\
TDW\_LOGFILE             & ``GUI-OPTION: TDW\_LOGFILE `\%s\''' \\
TDW\_WAN                 & ``GUI-OPTION: WAN - low speed connection enabled'' \\
TDW\_FULLMENU            & ``GUI-OPTION: FullMenu enabled'' \\
SAP\_CP / SAP\_CODEPAGE  & ``GUI-OPTION: SAP\_CODEPAGE `\%d\''' \\
UPDOWNLOAD\_CP           & ``GUI-OPTION: UPDOWNLOAD\_CP `\%d\''' \\
SNC\_PARTNERNAME         & ``GUI-OPTION: SNC name `\%s\''' \\
SNC\_QOP                 & ``GUI-OPTION: SNC\_QOP `\%s\''' \\
SNC\_LIB                 & ``GUI-OPTION: SNC is set to: \%s'' \\ 
SAPGUI\_INPLACE          & ``GUI-OPTION: environment variable SAPGUI\_INPLACE is on'' \\
\hline  
\end{tabular}
\end{center}

\IFRU{Настройки для каждой переменной записываются в массив через указатель в регистре \EDI. \EDI выставляется перед вызовом функции}
{Settings for each variable are written to the array via pointer in \EDI register. \EDI is being set before that function call}:

\begin{lstlisting}
.text:6440EE00                 lea     edi, [ebp+2884h+var_2884] ; options here like +0x15...
.text:6440EE03                 lea     ecx, [esi+24h]
.text:6440EE06                 call    load_command_line
.text:6440EE0B                 mov     edi, eax
.text:6440EE0D                 xor     ebx, ebx
.text:6440EE0F                 cmp     edi, ebx
.text:6440EE11                 jz      short loc_6440EE42
.text:6440EE13                 push    edi
.text:6440EE14                 push    offset aSapguiStoppedA ; "Sapgui stopped after commandline interp"...
.text:6440EE19                 push    dword_644F93E8
.text:6440EE1F                 call    FEWTraceError
\end{lstlisting}

\IFRU{А теперь, можем ли мы найти строку \IT{``data record mode switched on''}?}{Now, can we find \IT{``data record mode switched on''} string?}
\IFRU{Да, и есть только одна ссылка на эту строку в функции}{Yes, and here is the only reference in function} \TT{CDwsGui::PrepareInfoWindow()}.
\IFRU{Откуда я узнал имена классов/методов? Здесь много специальных отладочных вызовов пишущих в лог-файл вроде}
{How do I know class/method names? There is a lot of special debugging calls writing to log-files like}:

\begin{lstlisting}
.text:64405160                 push    dword ptr [esi+2854h]
.text:64405166                 push    offset aCdwsguiPrepare ; "\nCDwsGui::PrepareInfoWindow: sapgui env"...
.text:6440516B                 push    dword ptr [esi+2848h]
.text:64405171                 call    dbg
.text:64405176                 add     esp, 0Ch
\end{lstlisting}

... \IFRU{или}{or}:

\begin{lstlisting}
.text:6440237A                 push    eax
.text:6440237B                 push    offset aCclientStart_6 ; "CClient::Start: set shortcut user to '\%"...
.text:64402380                 push    dword ptr [edi+4]
.text:64402383                 call    dbg
.text:64402388                 add     esp, 0Ch
\end{lstlisting}

\IFRU{Они \textbf{очень} полезны}{It's \textbf{very} useful}.

\IFRU{Посмотрим содержимое функции ``надоедливого всплывающего окна''}{So let's see contents of that nagging pop-up window function}:

\begin{lstlisting}
.text:64404F4F CDwsGui__PrepareInfoWindow proc near
.text:64404F4F
.text:64404F4F pvParam         = byte ptr -3Ch
.text:64404F4F var_38          = dword ptr -38h
.text:64404F4F var_34          = dword ptr -34h
.text:64404F4F rc              = tagRECT ptr -2Ch
.text:64404F4F cy              = dword ptr -1Ch
.text:64404F4F h               = dword ptr -18h
.text:64404F4F var_14          = dword ptr -14h
.text:64404F4F var_10          = dword ptr -10h
.text:64404F4F var_4           = dword ptr -4
.text:64404F4F
.text:64404F4F                 push    30h
.text:64404F51                 mov     eax, offset loc_64438E00
.text:64404F56                 call    __EH_prolog3
.text:64404F5B                 mov     esi, ecx        ; ECX is pointer to object
.text:64404F5D                 xor     ebx, ebx
.text:64404F5F                 lea     ecx, [ebp+var_14]
.text:64404F62                 mov     [ebp+var_10], ebx

; demangled name: ATL::CStringT(void)
.text:64404F65                 call    ds:mfc90_316    
.text:64404F6B                 mov     [ebp+var_4], ebx
.text:64404F6E                 lea     edi, [esi+2854h]
.text:64404F74                 push    offset aEnvironmentInf ; "Environment information:\n"
.text:64404F79                 mov     ecx, edi

; demangled name: ATL::CStringT::operator=(char const *)
.text:64404F7B                 call    ds:mfc90_820
.text:64404F81                 cmp     [esi+38h], ebx
.text:64404F84                 mov     ebx, ds:mfc90_2539
.text:64404F8A                 jbe     short loc_64404FA9
.text:64404F8C                 push    dword ptr [esi+34h]
.text:64404F8F                 lea     eax, [ebp+var_14]
.text:64404F92                 push    offset aWorkingDirecto ; "working directory: '\%s'\n"
.text:64404F97                 push    eax

; demangled name: ATL::CStringT::Format(char const *,...)
.text:64404F98                 call    ebx ; mfc90_2539
.text:64404F9A                 add     esp, 0Ch
.text:64404F9D                 lea     eax, [ebp+var_14]
.text:64404FA0                 push    eax
.text:64404FA1                 mov     ecx, edi

; demangled name: ATL::CStringT::operator+=(class ATL::CSimpleStringT<char, 1> const &)
.text:64404FA3                 call    ds:mfc90_941
.text:64404FA9
.text:64404FA9 loc_64404FA9:
.text:64404FA9                 mov     eax, [esi+38h]
.text:64404FAC                 test    eax, eax
.text:64404FAE                 jbe     short loc_64404FD3
.text:64404FB0                 push    eax
.text:64404FB1                 lea     eax, [ebp+var_14]
.text:64404FB4                 push    offset aTraceLevelDAct ; "trace level \%d activated\n"
.text:64404FB9                 push    eax

; demangled name: ATL::CStringT::Format(char const *,...)
.text:64404FBA                 call    ebx ; mfc90_2539
.text:64404FBC                 add     esp, 0Ch
.text:64404FBF                 lea     eax, [ebp+var_14]
.text:64404FC2                 push    eax
.text:64404FC3                 mov     ecx, edi

; demangled name: ATL::CStringT::operator+=(class ATL::CSimpleStringT<char, 1> const &)
.text:64404FC5                 call    ds:mfc90_941
.text:64404FCB                 xor     ebx, ebx
.text:64404FCD                 inc     ebx
.text:64404FCE                 mov     [ebp+var_10], ebx
.text:64404FD1                 jmp     short loc_64404FD6
.text:64404FD3 ; ---------------------------------------------------------------------------
.text:64404FD3
.text:64404FD3 loc_64404FD3:
.text:64404FD3                 xor     ebx, ebx
.text:64404FD5                 inc     ebx
.text:64404FD6
.text:64404FD6 loc_64404FD6:
.text:64404FD6                 cmp     [esi+38h], ebx
.text:64404FD9                 jbe     short loc_64404FF1
.text:64404FDB                 cmp     dword ptr [esi+2978h], 0
.text:64404FE2                 jz      short loc_64404FF1
.text:64404FE4                 push    offset aHexdumpInTrace ; "hexdump in trace activated\n"
.text:64404FE9                 mov     ecx, edi

; demangled name: ATL::CStringT::operator+=(char const *)
.text:64404FEB                 call    ds:mfc90_945
.text:64404FF1
.text:64404FF1 loc_64404FF1:
.text:64404FF1
.text:64404FF1                 cmp     byte ptr [esi+78h], 0
.text:64404FF5                 jz      short loc_64405007
.text:64404FF7                 push    offset aLoggingActivat ; "logging activated\n"
.text:64404FFC                 mov     ecx, edi

; demangled name: ATL::CStringT::operator+=(char const *)
.text:64404FFE                 call    ds:mfc90_945
.text:64405004                 mov     [ebp+var_10], ebx
.text:64405007
.text:64405007 loc_64405007:
.text:64405007                 cmp     byte ptr [esi+3Dh], 0
.text:6440500B                 jz      short bypass
.text:6440500D                 push    offset aDataCompressio ; "data compression switched off\n"
.text:64405012                 mov     ecx, edi

; demangled name: ATL::CStringT::operator+=(char const *)
.text:64405014                 call    ds:mfc90_945
.text:6440501A                 mov     [ebp+var_10], ebx
.text:6440501D
.text:6440501D bypass:
.text:6440501D                 mov     eax, [esi+20h]
.text:64405020                 test    eax, eax
.text:64405022                 jz      short loc_6440503A
.text:64405024                 cmp     dword ptr [eax+28h], 0
.text:64405028                 jz      short loc_6440503A
.text:6440502A                 push    offset aDataRecordMode ; "data record mode switched on\n"
.text:6440502F                 mov     ecx, edi

; demangled name: ATL::CStringT::operator+=(char const *)
.text:64405031                 call    ds:mfc90_945
.text:64405037                 mov     [ebp+var_10], ebx
.text:6440503A
.text:6440503A loc_6440503A:
.text:6440503A
.text:6440503A                 mov     ecx, edi
.text:6440503C                 cmp     [ebp+var_10], ebx
.text:6440503F                 jnz     loc_64405142
.text:64405045                 push    offset aForMaximumData ; "\nFor maximum data security delete\nthe s"...

; demangled name: ATL::CStringT::operator+=(char const *)
.text:6440504A                 call    ds:mfc90_945
.text:64405050                 xor     edi, edi
.text:64405052                 push    edi             ; fWinIni
.text:64405053                 lea     eax, [ebp+pvParam]
.text:64405056                 push    eax             ; pvParam
.text:64405057                 push    edi             ; uiParam
.text:64405058                 push    30h             ; uiAction
.text:6440505A                 call    ds:SystemParametersInfoA
.text:64405060                 mov     eax, [ebp+var_34]
.text:64405063                 cmp     eax, 1600
.text:64405068                 jle     short loc_64405072
.text:6440506A                 cdq
.text:6440506B                 sub     eax, edx
.text:6440506D                 sar     eax, 1
.text:6440506F                 mov     [ebp+var_34], eax
.text:64405072
.text:64405072 loc_64405072:
.text:64405072                 push    edi             ; hWnd
.text:64405073                 mov     [ebp+cy], 0A0h
.text:6440507A                 call    ds:GetDC
.text:64405080                 mov     [ebp+var_10], eax
.text:64405083                 mov     ebx, 12Ch
.text:64405088                 cmp     eax, edi
.text:6440508A                 jz      loc_64405113
.text:64405090                 push    11h             ; i
.text:64405092                 call    ds:GetStockObject
.text:64405098                 mov     edi, ds:SelectObject
.text:6440509E                 push    eax             ; h
.text:6440509F                 push    [ebp+var_10]    ; hdc
.text:644050A2                 call    edi ; SelectObject
.text:644050A4                 and     [ebp+rc.left], 0
.text:644050A8                 and     [ebp+rc.top], 0
.text:644050AC                 mov     [ebp+h], eax
.text:644050AF                 push    401h            ; format
.text:644050B4                 lea     eax, [ebp+rc]
.text:644050B7                 push    eax             ; lprc
.text:644050B8                 lea     ecx, [esi+2854h]
.text:644050BE                 mov     [ebp+rc.right], ebx
.text:644050C1                 mov     [ebp+rc.bottom], 0B4h

; demangled name: ATL::CSimpleStringT::GetLength(void)
.text:644050C8                 call    ds:mfc90_3178
.text:644050CE                 push    eax             ; cchText
.text:644050CF                 lea     ecx, [esi+2854h]

; demangled name: const char* ATL::CSimpleStringT::operator PCXSTR 
.text:644050D5                 call    ds:mfc90_910
.text:644050DB                 push    eax             ; lpchText
.text:644050DC                 push    [ebp+var_10]    ; hdc
.text:644050DF                 call    ds:DrawTextA
.text:644050E5                 push    4               ; nIndex
.text:644050E7                 call    ds:GetSystemMetrics
.text:644050ED                 mov     ecx, [ebp+rc.bottom]
.text:644050F0                 sub     ecx, [ebp+rc.top]
.text:644050F3                 cmp     [ebp+h], 0
.text:644050F7                 lea     eax, [eax+ecx+28h]
.text:644050FB                 mov     [ebp+cy], eax
.text:644050FE                 jz      short loc_64405108
.text:64405100                 push    [ebp+h]         ; h
.text:64405103                 push    [ebp+var_10]    ; hdc
.text:64405106                 call    edi ; SelectObject
.text:64405108
.text:64405108 loc_64405108:
.text:64405108                 push    [ebp+var_10]    ; hDC
.text:6440510B                 push    0               ; hWnd
.text:6440510D                 call    ds:ReleaseDC
.text:64405113
.text:64405113 loc_64405113:
.text:64405113                 mov     eax, [ebp+var_38]
.text:64405116                 push    80h             ; uFlags
.text:6440511B                 push    [ebp+cy]        ; cy
.text:6440511E                 inc     eax
.text:6440511F                 push    ebx             ; cx
.text:64405120                 push    eax             ; Y
.text:64405121                 mov     eax, [ebp+var_34]
.text:64405124                 add     eax, 0FFFFFED4h
.text:64405129                 cdq
.text:6440512A                 sub     eax, edx
.text:6440512C                 sar     eax, 1
.text:6440512E                 push    eax             ; X
.text:6440512F                 push    0               ; hWndInsertAfter
.text:64405131                 push    dword ptr [esi+285Ch] ; hWnd
.text:64405137                 call    ds:SetWindowPos
.text:6440513D                 xor     ebx, ebx
.text:6440513F                 inc     ebx
.text:64405140                 jmp     short loc_6440514D
.text:64405142 ; ---------------------------------------------------------------------------
.text:64405142
.text:64405142 loc_64405142:
.text:64405142                 push    offset byte_64443AF8

; demangled name: ATL::CStringT::operator=(char const *)
.text:64405147                 call    ds:mfc90_820
.text:6440514D
.text:6440514D loc_6440514D:
.text:6440514D                 cmp     dword_6450B970, ebx
.text:64405153                 jl      short loc_64405188
.text:64405155                 call    sub_6441C910
.text:6440515A                 mov     dword_644F858C, ebx
.text:64405160                 push    dword ptr [esi+2854h]
.text:64405166                 push    offset aCdwsguiPrepare ; "\nCDwsGui::PrepareInfoWindow: sapgui env"...
.text:6440516B                 push    dword ptr [esi+2848h]
.text:64405171                 call    dbg
.text:64405176                 add     esp, 0Ch
.text:64405179                 mov     dword_644F858C, 2
.text:64405183                 call    sub_6441C920
.text:64405188
.text:64405188 loc_64405188:
.text:64405188                 or      [ebp+var_4], 0FFFFFFFFh
.text:6440518C                 lea     ecx, [ebp+var_14]

; demangled name: ATL::CStringT::~CStringT()
.text:6440518F                 call    ds:mfc90_601
.text:64405195                 call    __EH_epilog3
.text:6440519A                 retn
.text:6440519A CDwsGui__PrepareInfoWindow endp
\end{lstlisting}

\IFRU{\ECX в начале функции содержит в себе указатель на объект (потому что это тип функции thiscall~\ref{thiscall}). 
В нашем случае, класс имеет тип, очевидно, \IT{CDwsGui}. В зависимости от включенных опций в объекте, 
разные сообщения добавляются к итоговому сообщению.}
{\ECX at function start gets pointer to object (because it is thiscall~\ref{thiscall}-type of function). 
In our case, that object obviously has class type \IT{CDwsGui}. 
Depends of option turned on in the object, specific message part will be concatenated to resulting message.}

\IFRU{Если переменная по адресу \TT{this+0x3D} не ноль, компрессия сетевых пакетов будет выключена}
{If value at \TT{this+0x3D} address is not zero, compression is off}:

\begin{lstlisting}
.text:64405007 loc_64405007:
.text:64405007                 cmp     byte ptr [esi+3Dh], 0
.text:6440500B                 jz      short bypass
.text:6440500D                 push    offset aDataCompressio ; "data compression switched off\n"
.text:64405012                 mov     ecx, edi

; demangled name: ATL::CStringT::operator+=(char const *)
.text:64405014                 call    ds:mfc90_945
.text:6440501A                 mov     [ebp+var_10], ebx
.text:6440501D
.text:6440501D bypass:
\end{lstlisting}

\IFRU{Интересно, что в итоге, состояние переменной \IT{var\_10} определяет, будет ли показано сообщение вообще}
{It is interesting, that finally, \IT{var\_10} variable state defines whether the message is to be shown at all}:

\lstinputlisting{\IFRU{sapgui/1ru.lst}{sapgui/1en.lst}}

\IFRU{Давайте проверим нашу теорию на практике}{Let's check our theory on practice}.

\JNZ \IFRU{в этой строке}{at this line} ...

\lstinputlisting{\IFRU{sapgui/2ru.lst}{sapgui/2en.lst}}

... \IFRU{заменим просто на \JMP и получим SAPGUI работающим без этого надоедливого всплывающего окна!}
{replace it with just \JMP, and get SAPGUI working without that nagging pop-up window appearing!}

\IFRU{Копнем немного глубже и проследим связь между смещением 0x15 в \TT{load\_command\_line()}
(Это я дал имя этой функции) и переменной \TT{this+0x3D} в \IT{CDwsGui::PrepareInfoWindow}. 
Уверены ли мы что это одна и та же переменная?}
{Now let's dig deeper and find connection between 0x15 offset in \TT{load\_command\_line()} 
(I gave that name to that function) and \TT{this+0x3D} variable in \IT{CDwsGui::PrepareInfoWindow}. 
Are we sure that the value is the same?}

\IFRU{Начинаю искать все места где в коде используется константа 0x15.
Для таких небольших программ как SAPGUI, это иногда срабатывает. Вот первое что я нашел:}
{I'm starting to search for all occurences of 0x15 value in code. 
For some small programs like SAPGUI, it sometimes works. Here is the first occurence I got:}

\begin{lstlisting}
.text:64404C19 sub_64404C19    proc near
.text:64404C19
.text:64404C19 arg_0           = dword ptr  4
.text:64404C19
.text:64404C19                 push    ebx
.text:64404C1A                 push    ebp
.text:64404C1B                 push    esi
.text:64404C1C                 push    edi
.text:64404C1D                 mov     edi, [esp+10h+arg_0]
.text:64404C21                 mov     eax, [edi]
.text:64404C23                 mov     esi, ecx ; ESI/ECX are pointers to some unknown object.
.text:64404C25                 mov     [esi], eax
.text:64404C27                 mov     eax, [edi+4]
.text:64404C2A                 mov     [esi+4], eax
.text:64404C2D                 mov     eax, [edi+8]
.text:64404C30                 mov     [esi+8], eax
.text:64404C33                 lea     eax, [edi+0Ch]
.text:64404C36                 push    eax
.text:64404C37                 lea     ecx, [esi+0Ch]

; demangled name:  ATL::CStringT::operator=(class ATL::CStringT ... &)
.text:64404C3A                 call    ds:mfc90_817 
.text:64404C40                 mov     eax, [edi+10h]
.text:64404C43                 mov     [esi+10h], eax
.text:64404C46                 mov     al, [edi+14h]
.text:64404C49                 mov     [esi+14h], al
.text:64404C4C                 mov     al, [edi+15h] ; copy byte from 0x15 offset
.text:64404C4F                 mov     [esi+15h], al ; to 0x15 offset in CDwsGui object
\end{lstlisting}

\IFRU{Эта функция вызывается из функции с названием \IT{CDwsGui::CopyOptions}! И снова спасибо отладочной информации.}
{That function was called from the function named \IT{CDwsGui::CopyOptions}! And thanks again for debugging information.}

\IFRU{Но настоящий ответ находится в функции}{But the real answer in the function} \IT{CDwsGui::Init()}:

\begin{lstlisting}
.text:6440B0BF loc_6440B0BF:
.text:6440B0BF                 mov     eax, [ebp+arg_0]
.text:6440B0C2                 push    [ebp+arg_4]
.text:6440B0C5                 mov     [esi+2844h], eax
.text:6440B0CB                 lea     eax, [esi+28h] ; ESI is pointer to CDwsGui object
.text:6440B0CE                 push    eax
.text:6440B0CF                 call    CDwsGui__CopyOptions
\end{lstlisting}

\IFRU{Теперь ясно: массив заполняемый в \TT{load\_command\_line()} на самом деле расположен в классе \IT{CDwsGui} но по адресу
\TT{this+0x28}. 0x15 + 0x28 это 0x3D. ОК, мы нашли место, куда наша переменная копируется.}
{Finally, we understand: array filled in \TT{load\_command\_line()} is actually placed in \IT{CDwsGui} class 
but on \TT{this+0x28} address. 0x15 + 0x28 is exactly 0x3D. OK, we found the place where the value is copied to.}

\IFRU{Посмотрим так же и другие места, где используется смещение 0x3D.
Одно из таких мест находится в функции \IT{CDwsGui::SapguiRun} (и снова спасибо отладочным вызовам):}
{Let's also find other places where 0x3D offset is used. 
Here is one of them in \IT{CDwsGui::SapguiRun} function (again, thanks to debugging calls):}

\begin{lstlisting}
.text:64409D58                 cmp     [esi+3Dh], bl   ; ESI is pointer to CDwsGui object
.text:64409D5B                 lea     ecx, [esi+2B8h]
.text:64409D61                 setz    al
.text:64409D64                 push    eax             ; arg_10 of CConnectionContext::CreateNetwork
.text:64409D65                 push    dword ptr [esi+64h]

; demangled name: const char* ATL::CSimpleStringT::operator PCXSTR 
.text:64409D68                 call    ds:mfc90_910
.text:64409D68                                         ; no arguments
.text:64409D6E                 push    eax
.text:64409D6F                 lea     ecx, [esi+2BCh]

; demangled name: const char* ATL::CSimpleStringT::operator PCXSTR 
.text:64409D75                 call    ds:mfc90_910
.text:64409D75                                         ; no arguments
.text:64409D7B                 push    eax
.text:64409D7C                 push    esi
.text:64409D7D                 lea     ecx, [esi+8]
.text:64409D80                 call    CConnectionContext__CreateNetwork
\end{lstlisting}

\IFRU{Проверим нашу идею. Заменяем \TT{setz al} здесь на \TT{xor eax, eax / nop}, убираем переменную окружения 
\TDWNC и запускаем SAPGUI. Wow! Надоедливого окна больше нет (как и ожидалось: ведь переменной окружении так же
нет), но в Wireshark мы видим что сетевые пакеты больше не сжимаются!
Очевидно, это то самое место где флаг отражающий сжатие пакетов выставляется в объекте \TT{CConnectionContext}.}
{Let's check our findings. Replace \TT{setz al} here to \TT{xor eax, eax / nop}, clear \TDWNC 
environment variable and run SAPGUI. Wow! There is no nagging window (just as expected, 
because variable is not set), but in Wireshark we can see that the network packets are not compressed anymore! 
Obviously, this is the place where compression flag is to be set in \IT{CConnectionContext} object.}

\IFRU{Так что, флаг сжатия передается в пятом аргументе функции \IT{CConnectionContext::CreateNetwork}. 
Внутри этой функции, вызывается еще одна:}
{So, compression flag is passed in the 5th argument of function \IT{CConnectionContext::CreateNetwork}. 
Inside that function, another one is called:}

\begin{lstlisting}
...
.text:64403476                 push    [ebp+compression]
.text:64403479                 push    [ebp+arg_C]
.text:6440347C                 push    [ebp+arg_8]
.text:6440347F                 push    [ebp+arg_4]
.text:64403482                 push    [ebp+arg_0]
.text:64403485                 call    CNetwork__CNetwork
\end{lstlisting}

\IFRU{Флаг отвечающий за сжатие здесь передается в пятом аргументе для конструктора \IT{CNetwork::CNetwork}.}
{Compression flag is passing here in the 5th argument to \IT{CNetwork::CNetwork} constructor.}

\IFRU{И вот как конструктор \IT{CNetwork} выставляет некоторые флаги в объекте \IT{CNetwork} в соответствии с пятым аргументом \textbf{и}
еще какую-то переменную, возможно, также отвечающую за сжатие сетевых пакетов.}
{And here is how \IT{CNetwork} constructor sets some flag in \IT{CNetwork} object according to the 5th argument \textbf{and}
some another variable which probably could affect network packets compression too.}

\begin{lstlisting}
.text:64411DF1                 cmp     [ebp+compression], esi
.text:64411DF7                 jz      short set_EAX_to_0
.text:64411DF9                 mov     al, [ebx+78h]   ; another value may affect compression?
.text:64411DFC                 cmp     al, '3'
.text:64411DFE                 jz      short set_EAX_to_1
.text:64411E00                 cmp     al, '4'
.text:64411E02                 jnz     short set_EAX_to_0
.text:64411E04
.text:64411E04 set_EAX_to_1:
.text:64411E04                 xor     eax, eax
.text:64411E06                 inc     eax             ; EAX -> 1
.text:64411E07                 jmp     short loc_64411E0B
.text:64411E09 ; ---------------------------------------------------------------------------
.text:64411E09
.text:64411E09 set_EAX_to_0:
.text:64411E09
.text:64411E09                 xor     eax, eax        ; EAX -> 0
.text:64411E0B
.text:64411E0B loc_64411E0B:
.text:64411E0B                 mov     [ebx+3A4h], eax ; EBX is pointer to CNetwork object
\end{lstlisting}

\IFRU{Теперь мы знаем что флаг отражающий сжатие данных сохраняется в классе \IT{CNetwork} по адресу \IT{this+0x3A4}.}
{At this point we know that compression flag is stored in \IT{CNetwork} class at \IT{this+0x3A4} address.}

\IFRU{Поищем теперь значение 0x3A4 в SAPguilib.dll. Находим второе упоминание этого значения в функции
\IT{CDwsGui::OnClientMessageWrite} (бесконечная благодарность отладочной информации):}
{Now let's dig across SAPguilib.dll for 0x3A4 value. And here is the second occurence in 
\IT{CDwsGui::OnClientMessageWrite} (endless thanks for debugging information):}

\begin{lstlisting}
.text:64406F76 loc_64406F76:
.text:64406F76                 mov     ecx, [ebp+7728h+var_7794]
.text:64406F79                 cmp     dword ptr [ecx+3A4h], 1
.text:64406F80                 jnz     compression_flag_is_zero
.text:64406F86                 mov     byte ptr [ebx+7], 1
.text:64406F8A                 mov     eax, [esi+18h]
.text:64406F8D                 mov     ecx, eax
.text:64406F8F                 test    eax, eax
.text:64406F91                 ja      short loc_64406FFF
.text:64406F93                 mov     ecx, [esi+14h]
.text:64406F96                 mov     eax, [esi+20h]
.text:64406F99
.text:64406F99 loc_64406F99:
.text:64406F99                 push    dword ptr [edi+2868h] ; int
.text:64406F9F                 lea     edx, [ebp+7728h+var_77A4]
.text:64406FA2                 push    edx             ; int
.text:64406FA3                 push    30000           ; int
.text:64406FA8                 lea     edx, [ebp+7728h+Dst]
.text:64406FAB                 push    edx             ; Dst
.text:64406FAC                 push    ecx             ; int
.text:64406FAD                 push    eax             ; Src
.text:64406FAE                 push    dword ptr [edi+28C0h] ; int
.text:64406FB4                 call    sub_644055C5       ; actual compression routine
.text:64406FB9                 add     esp, 1Ch
.text:64406FBC                 cmp     eax, 0FFFFFFF6h
.text:64406FBF                 jz      short loc_64407004
.text:64406FC1                 cmp     eax, 1
.text:64406FC4                 jz      loc_6440708C
.text:64406FCA                 cmp     eax, 2
.text:64406FCD                 jz      short loc_64407004
.text:64406FCF                 push    eax
.text:64406FD0                 push    offset aCompressionErr ; "compression error [rc = \%d]- program wi"...
.text:64406FD5                 push    offset aGui_err_compre ; "GUI_ERR_COMPRESS"
.text:64406FDA                 push    dword ptr [edi+28D0h]
.text:64406FE0                 call    SapPcTxtRead
\end{lstlisting}

\IFRU{Заглянем в функцию \IT{sub\_644055C5}. Всё что в ней мы находим это вызов memcpy() и еще какую-то функцию названную
\IDA \IT{sub\_64417440}.}
{Let's take a look into \IT{sub\_644055C5}. In it we can only see call to memcpy() and some other function named 
(by \IDA) \IT{sub\_64417440}.}

\IFRU{И теперь заглянем в \IT{sub\_64417440}. Увидим там:}
{And, let's take a look inside \IT{sub\_64417440}. What we see is:}

\begin{lstlisting}
.text:6441747C                 push    offset aErrorCsrcompre ; "\nERROR: CsRCompress: invalid handle"
.text:64417481                 call    eax ; dword_644F94C8
.text:64417483                 add     esp, 4
\end{lstlisting}

\newcommand{\URLSAPDECOMP}{http://conus.info/utils/SAP_pkt_decompr.txt}

Voilà! \IFRU{Мы находим функцию которая собственно и сжимает сетевые пакеты.}{We've found the function which actually compresses data.}
\IFRU{Как я уже \href{\URLSAPDECOMP}{разобрался}, эта функция используется в SAP и в опен-сорсном проекте MaxDB.
Так что эта функция доступна в виде исходников.}
{As \href{\URLSAPDECOMP}{I revealed in past}, this function is used in SAP and also open-source MaxDB project. 
So it is available in sources.}

\IFRU{Последняя проверка:}{Doing last check here:}

\begin{lstlisting}
.text:64406F79                 cmp     dword ptr [ecx+3A4h], 1
.text:64406F80                 jnz     compression_flag_is_zero
\end{lstlisting}

\IFRU{Заменим \JNZ на безусловный переход \JMP. Уберем переменную окружения \TDWNC.}
{Replace \JNZ here for unconditional \JMP. Remove environment variable \TDWNC.} Voilà! 
\IFRU{В Wireshark мы видим что сетевые пакеты исходящие от клиента не сжаты. Ответы сервера, впрочем, сжаты.}
{In Wireshark we see that client messages are not compressed. Server responses, however, are compressed.}

\IFRU{Так что мы нашли связь между переменной окружения и местом где функция сжатия данных вызывается, а так же
может быть отключена.}
{So we found exact connection between environment variable and the point where data compression 
routine may be called or may be bypassed.}



\chapter{\IFRU{Прочее}{Other things}}

\section{\IFRU{Аномалии компиляторов}{Compiler's anomalies}}

\IFRU{Intel C++ 10.1 которым скомпилирован Oracle RDBMS 11.2 Linux86, может сгенерировать два \JZ идущих подряд, 
причем на второй \JZ нет ссылки ниоткуда. Второй \JZ таким образом, не имеет никакого смысла.}
{Intel C++ 10.1, which was used for Oracle RDBMS 11.2 Linux86 compilation, may emit two \JZ in row,
and there are no references to the second \JZ. Second \JZ is thus senseless.}

\IFRU{Например, kdli.o из}{For example, kdli.o from} libserver11.a:

\begin{lstlisting}
.text:08114CF1                   loc_8114CF1:                            ; CODE XREF: __PGOSF539_kdlimemSer+89A
.text:08114CF1                                                           ; __PGOSF539_kdlimemSer+3994
.text:08114CF1 8B 45 08                          mov     eax, [ebp+arg_0]
.text:08114CF4 0F B6 50 14                       movzx   edx, byte ptr [eax+14h]
.text:08114CF8 F6 C2 01                          test    dl, 1
.text:08114CFB 0F 85 17 08 00 00                 jnz     loc_8115518
.text:08114D01 85 C9                             test    ecx, ecx
.text:08114D03 0F 84 8A 00 00 00                 jz      loc_8114D93
.text:08114D09 0F 84 09 08 00 00                 jz      loc_8115518
.text:08114D0F 8B 53 08                          mov     edx, [ebx+8]
.text:08114D12 89 55 FC                          mov     [ebp+var_4], edx
.text:08114D15 31 C0                             xor     eax, eax
.text:08114D17 89 45 F4                          mov     [ebp+var_C], eax
.text:08114D1A 50                                push    eax
.text:08114D1B 52                                push    edx
.text:08114D1C E8 03 54 00 00                    call    len2nbytes
.text:08114D21 83 C4 08                          add     esp, 8
\end{lstlisting}

\IFRU{Еще там же:}{From the same code:}

\begin{lstlisting}
.text:0811A2A5                   loc_811A2A5:                            ; CODE XREF: kdliSerLengths+11C
.text:0811A2A5                                                           ; kdliSerLengths+1C1
.text:0811A2A5 8B 7D 08                          mov     edi, [ebp+arg_0]
.text:0811A2A8 8B 7F 10                          mov     edi, [edi+10h]
.text:0811A2AB 0F B6 57 14                       movzx   edx, byte ptr [edi+14h]
.text:0811A2AF F6 C2 01                          test    dl, 1
.text:0811A2B2 75 3E                             jnz     short loc_811A2F2
.text:0811A2B4 83 E0 01                          and     eax, 1
.text:0811A2B7 74 1F                             jz      short loc_811A2D8
.text:0811A2B9 74 37                             jz      short loc_811A2F2
.text:0811A2BB 6A 00                             push    0
.text:0811A2BD FF 71 08                          push    dword ptr [ecx+8]
.text:0811A2C0 E8 5F FE FF FF                    call    len2nbytes
\end{lstlisting}

\IFRU{Возможно, это ошибка его кодегенератора, не выявленная тестами 
(ведь результирующий код и так работает нормально).}
{It's probably code generator bug wasn't found by tests, because, 
resulting code is working correctly anyway.}

% done

\chapter{\IFRU{Ответы на задачи}{Tasks solutions}}

\section{\IFRU{Легкий уровень}{Easy level}}

\subsection{\Task 1.1}

\IFRU{Решение}{Solution}: \TT{toupper()}.

\IFRU{Исходник на Си}{C source code}:

\begin{lstlisting}
char toupper ( char c )
{
    if( c >= 'a' && c <= 'z' ) {
        c = c - 'a' + 'A';
    }
    return( c );
}
\end{lstlisting}

\subsection{\Task 1.2}

\IFRU{Ответ}{Solution}: \TT{atoi()}

\IFRU{Исходник на Си}{C source code}:

\begin{lstlisting}
#include <stdio.h>
#include <string.h>
#include <ctype.h>

int atoi ( const *p )  /* convert ASCII string to integer */
{
    int i;
    char s;

    while( isspace ( *p ) )
        ++p;
    s = *p;
    if( s == '+' || s == '-' )
        ++p;
    i = 0;
    while( isdigit(*p) ) {
        i = i * 10 + *p - '0';
        ++p;
    }
    if( s == '-' )
        i = - i;
    return( i );
}
\end{lstlisting}

\subsection{\Task 1.3}

\IFRU{Ответ}{Solution}: \TT{srand()} / \TT{rand()}.

\IFRU{Исходник на Си}{C source code}:

\begin{lstlisting}
static unsigned int v;

void srand (unsigned int s)
{
        v = s;
}

int rand ()
{
        return( ((v = v * 214013L
            + 2531011L) >> 16) & 0x7fff );
}
\end{lstlisting}

\subsection{\Task 1.4}

\IFRU{Ответ}{Solution}: \TT{strstr()}.

\IFRU{Исходник на Си}{C source code}:

\begin{lstlisting}
char * strstr (
        const char * str1,
        const char * str2
        )
{
        char *cp = (char *) str1;
        char *s1, *s2;

        if ( !*str2 )
            return((char *)str1);

        while (*cp)
        {
                s1 = cp;
                s2 = (char *) str2;

                while ( *s1 && *s2 && !(*s1-*s2) )
                        s1++, s2++;

                if (!*s2)
                        return(cp);

                cp++;
        }

        return(NULL);

}
\end{lstlisting}

\subsection{\Task 1.5}

\IFRU{Подсказка}{Hint} \#1: \IFRU{Не забывайте что}{Keep in mind that} \TT{\_\_v} ~--- 
\IFRU{глобальная переменная}{global variable}.

\IFRU{Подсказка}{Hint} \#2: \IFRU{Эта функция вызывается startup-кодом перед вызовом \main}
{That function is called in startup code, before \main execution}.

\IFRU{Ответ: это проверка на наличие FDIV-ошибки в ранних процессорах Pentium}
{Solution: early Pentium CPU FDIV bug checking}\footnote{\url{http://en.wikipedia.org/wiki/Pentium_FDIV_bug}}.

\IFRU{Исходник на Си}{C source code}:

\begin{lstlisting}
unsigned _v; // _v

enum e {
    PROB_P5_DIV = 0x0001
};

void f( void ) // __verify_pentium_fdiv_bug
{
    /*
        Verify we have got the Pentium FDIV problem.
        The volatiles are to scare the optimizer away.
    */
    volatile double     v1     = 4195835;
    volatile double     v2   = 3145727;

    if( (v1 - (v1/v2)*v2) > 1.0e-8 ) {
        _v |= PROB_P5_DIV;
    }
}
\end{lstlisting}

\subsection{\Task 1.6}

\IFRU{Подсказка: если погуглить применяемую здесь константу, это может помочь.}
{Hint: it might be helpful to google a constant used here.}

\IFRU{Ответ: шифрование алгоритмом TEA}{Solution: TEA encryption algorithm}\footnote{Tiny Encryption Algorithm}.

\IFRU{Исходник на Си}{C source code} (\IFRU{взято с}{taken from} \url{http://en.wikipedia.org/wiki/Tiny_Encryption_Algorithm}):

\begin{lstlisting}
void f (unsigned int* v, unsigned int* k) {
    unsigned int v0=v[0], v1=v[1], sum=0, i;           /* set up */
    unsigned int delta=0x9e3779b9;                     /* a key schedule constant */
    unsigned int k0=k[0], k1=k[1], k2=k[2], k3=k[3];   /* cache key */
    for (i=0; i < 32; i++) {                       /* basic cycle start */
        sum += delta;
        v0 += ((v1<<4) + k0) ^ (v1 + sum) ^ ((v1>>5) + k1);
        v1 += ((v0<<4) + k2) ^ (v0 + sum) ^ ((v0>>5) + k3);  
    }                                              /* end cycle */
    v[0]=v0; v[1]=v1;
}
\end{lstlisting}

\subsection{\Task 1.7}

\IFRU{Подсказка: таблица содержит зараннее вычисленные значения. 
Можно было бы обойтись и без нее, но тогда функция работала бы чуть медленнее.}
{Hint: the table contain pre-calculated values.
It's possible to implement the function without it, but it will work slower, though.}

\IFRU{Ответ: эта функция переставляет все биты во входном 32-битном слове наоборот. 
Это \TT{lib/bitrev.c} из ядра Linux.}
{Solution: this function reverse all bits in input 32-bit integer. 
It's \TT{lib/bitrev.c} from Linux kernel.}

\IFRU{Исходник на Си}{C source code}:

\begin{lstlisting}
const unsigned char byte_rev_table[256] = {
	0x00, 0x80, 0x40, 0xc0, 0x20, 0xa0, 0x60, 0xe0,
	0x10, 0x90, 0x50, 0xd0, 0x30, 0xb0, 0x70, 0xf0,
	0x08, 0x88, 0x48, 0xc8, 0x28, 0xa8, 0x68, 0xe8,
	0x18, 0x98, 0x58, 0xd8, 0x38, 0xb8, 0x78, 0xf8,
	0x04, 0x84, 0x44, 0xc4, 0x24, 0xa4, 0x64, 0xe4,
	0x14, 0x94, 0x54, 0xd4, 0x34, 0xb4, 0x74, 0xf4,
	0x0c, 0x8c, 0x4c, 0xcc, 0x2c, 0xac, 0x6c, 0xec,
	0x1c, 0x9c, 0x5c, 0xdc, 0x3c, 0xbc, 0x7c, 0xfc,
	0x02, 0x82, 0x42, 0xc2, 0x22, 0xa2, 0x62, 0xe2,
	0x12, 0x92, 0x52, 0xd2, 0x32, 0xb2, 0x72, 0xf2,
	0x0a, 0x8a, 0x4a, 0xca, 0x2a, 0xaa, 0x6a, 0xea,
	0x1a, 0x9a, 0x5a, 0xda, 0x3a, 0xba, 0x7a, 0xfa,
	0x06, 0x86, 0x46, 0xc6, 0x26, 0xa6, 0x66, 0xe6,
	0x16, 0x96, 0x56, 0xd6, 0x36, 0xb6, 0x76, 0xf6,
	0x0e, 0x8e, 0x4e, 0xce, 0x2e, 0xae, 0x6e, 0xee,
	0x1e, 0x9e, 0x5e, 0xde, 0x3e, 0xbe, 0x7e, 0xfe,
	0x01, 0x81, 0x41, 0xc1, 0x21, 0xa1, 0x61, 0xe1,
	0x11, 0x91, 0x51, 0xd1, 0x31, 0xb1, 0x71, 0xf1,
	0x09, 0x89, 0x49, 0xc9, 0x29, 0xa9, 0x69, 0xe9,
	0x19, 0x99, 0x59, 0xd9, 0x39, 0xb9, 0x79, 0xf9,
	0x05, 0x85, 0x45, 0xc5, 0x25, 0xa5, 0x65, 0xe5,
	0x15, 0x95, 0x55, 0xd5, 0x35, 0xb5, 0x75, 0xf5,
	0x0d, 0x8d, 0x4d, 0xcd, 0x2d, 0xad, 0x6d, 0xed,
	0x1d, 0x9d, 0x5d, 0xdd, 0x3d, 0xbd, 0x7d, 0xfd,
	0x03, 0x83, 0x43, 0xc3, 0x23, 0xa3, 0x63, 0xe3,
	0x13, 0x93, 0x53, 0xd3, 0x33, 0xb3, 0x73, 0xf3,
	0x0b, 0x8b, 0x4b, 0xcb, 0x2b, 0xab, 0x6b, 0xeb,
	0x1b, 0x9b, 0x5b, 0xdb, 0x3b, 0xbb, 0x7b, 0xfb,
	0x07, 0x87, 0x47, 0xc7, 0x27, 0xa7, 0x67, 0xe7,
	0x17, 0x97, 0x57, 0xd7, 0x37, 0xb7, 0x77, 0xf7,
	0x0f, 0x8f, 0x4f, 0xcf, 0x2f, 0xaf, 0x6f, 0xef,
	0x1f, 0x9f, 0x5f, 0xdf, 0x3f, 0xbf, 0x7f, 0xff,
};

unsigned char bitrev8(unsigned char byte)
{
	return byte_rev_table[byte];
}

unsigned short bitrev16(unsigned short x)
{
	return (bitrev8(x & 0xff) << 8) | bitrev8(x >> 8);
}

/**
 * bitrev32 - reverse the order of bits in a unsigned int value
 * @x: value to be bit-reversed
 */

unsigned int bitrev32(unsigned int x)
{
	return (bitrev16(x & 0xffff) << 16) | bitrev16(x >> 16);
}
\end{lstlisting}

\subsection{\Task 1.8}

\IFRU{Ответ: сложение двух матриц размером 100 на 200 элементов типа \Tdouble.}
{Solution: two 100*200 matrices of \Tdouble type addition.}

\IFRU{Исходник на \CCpp}{\CCpp source code}:

\begin{lstlisting}
#define M    100
#define N    200

void s(double *a, double *b, double *c)
{
  for(int i=0;i<N;i++)
    for(int j=0;j<M;j++)
      *(c+i*M+j)=*(a+i*M+j) + *(b+i*M+j);
};
\end{lstlisting}

\subsection{\Task 1.9}

\IFRU{Ответ: умножение двух матриц размерами 100*200 и 100*300 элементов типа \Tdouble, результат: матрица 100*300.}
{Solution: two matrices (one is 100*200, second is 100*300) of \Tdouble type multiplication, result: 100*300
matrix.}

\IFRU{Исходник на \CCpp}{\CCpp source code}:

\begin{lstlisting}
#define M     100
#define N     200
#define P     300

void m(double *a, double *b, double *c)
{
  for(int i=0;i<M;i++)
    for(int j=0;j<P;j++)
    {
      *(c+i*M+j)=0;
      for (int k=0;k<N;k++) *(c+i*M+j)+=*(a+i*M+j) * *(b+i*M+j);
    }
};
\end{lstlisting}

\section{\IFRU{Средний уровень}{Middle level}}

\subsection{\Task 2.1}

\IFRU{Подсказка \#1: В этом коде есть одна особенность, по которой можно значительно сузить поиск функции в glibc.}
{Hint \#1: The code has one characteristic thing, considering it, it may help narrowing search of right function 
among glibc functions}.

\IFRU{Ответ: особенность ~--- это вызов callback-функции}{Solution: characteristic ~--- is callback-function
calling}~\ref{sec:pointerstofunctions}, 
\IFRU{указатель на которую передается в четвертом аргументе}{pointer to which is passed in 4th
argument}. \IFRU{Это}{It's} \TT{quicksort()}.

\IFRU{Исходник на Си:}{C source code:}

\lstinputlisting{tasks_answers/tasks_2_1.c}



\end{document}
